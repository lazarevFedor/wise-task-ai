{{Определение
|definition=
Пусть G — произвольный граф. Превратим каждое его ребро в дугу, придав ребру одно из двух возможных направлений. Полученный [[Основные определения теории графов#Ориентированные графы|орграф]] на том же самом множестве вершин будем называть '''ориентацией''' графа G. 
}}

{{Лемма
|statement=
Пусть K — [[Матрица Кирхгофа| матрица Кирхгофа]] графа G, I — [[Матрица инцидентности графа| матрица инцидентности]] G с некоторой ориентацией. Тогда 
 K = I \cdot I^T.

|proof=
При умножении i-й строки исходной матрицы I на j-й столбец транспонированной матрицы I^T перемножаются i-я и j-я строки исходной матрицы. При умножении i-й строки на саму себя на диагонали полученной матрицы получится сумма квадратов элементов i-й строки, которая равна, очевидно, \deg(v_i). Пусть теперь i \ne j. Если (v_i, v_j) \in E , то существует ровно одно ребро, соединяющее v_i и v_j , следовательно результат перемножения i-й и j-й строк равен -1, в противном случае он равен 0 в силу отсутствия ребра, инцидентного обеим вершинам. Определенная данными условиями матрица и является матрицей Кирхгофа.
}}
{|class="wikitable"
!Граф
!Матрица Кирхгофа
!Матрица инцидентности
|-
|[[Файл:Link_kirhgof_matrix_1.png|200px]]
|\left(\begin{array}{rrrrrr}
 2 & -1 & 0 & 0 & -1 & 0\\
-1 & 3 & -1 & 0 & -1 & 0\\
 0 & -1 & 2 & -1 & 0 & 0\\
 0 & 0 & -1 & 3 & -1 & -1\\
-1 & -1 & 0 & -1 & 3 & 0\\
 0 & 0 & 0 & -1 & 0 & 1\\
\end{array}\right)
|\begin{pmatrix}
1 & 0 & 0 & 0 & 1 & 0 & 0\\
1 & 1 & 0 & 0 & 0 & 1 & 0\\
0 & 1 & 1 & 0 & 0 & 0 & 0\\
0 & 0 & 1 & 1 & 0 & 0 & 1\\
0 & 0 & 0 & 1 & 1 & 1 & 0\\
0 & 0 & 0 & 0 & 0 & 0 & 1\\
\end{pmatrix}
|}

==См. также==
*[[Матрица Кирхгофа]]
*[[Подсчет числа остовных деревьев с помощью матрицы Кирхгофа]]
*[[Количество помеченных деревьев]]
*[[Коды Прюфера]]

==Источники информации==

*Асанов М., Баранский В., Расин В. {{---}} Дискретная математика: Графы, матроиды, алгоритмы — Ижевск: ННЦ "Регулярная и хаотическая динамика", 2001, 288 стр.

[[Категория: Алгоритмы и структуры данных]]
[[Категория: Остовные деревья ]]
[[Категория: Свойства остовных деревьев ]]