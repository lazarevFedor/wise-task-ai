== Случай неориентированного графа ==

{{Определение
|definition=
Две вершины u и v называются '''связанными''' ''(англ. adjacent)'', если в графе G существует [[Основные определения теории графов|путь]] из u в v (обозначение: u \rightsquigarrow v ).}}

{{Теорема
|statement=
Связность {{---}} '''[[Отношение_эквивалентности|отношение эквивалентности]]''' ''(англ. equivalence relation)''.
|proof=
'''[[Рефлексивное_отношение|Рефлексивность]]''': \forall a \in V a \rightsquigarrow a (очевидно).

'''[[Симметричное_отношение|Симметричность]]''': a\rightsquigarrow b \Rightarrow b\rightsquigarrow a (в силу неориентированности графа).

'''[[Транзитивное_отношение|Транзитивность]]''': a\rightsquigarrow b \land b\rightsquigarrow c \Rightarrow a\rightsquigarrow c. Действительно, сначала пройдем от a до b, затем от b до c, что и означает существования пути a \rightsquigarrow c.
}}

{{Определение
|id = def2
|definition=
'''Компонентой связности''' ''(англ. connected component)'' называется класс эквивалентности относительно связности.}}

{{Определение
|id = connected_graph
|definition=
Граф G=(V, E) называется '''связным''' ''(англ. connectivity graph)'', если он состоит из одной компоненты связности. В противном случае граф называется '''несвязным'''.}}

== Случай ориентированного графа ==
В общем случае для ориентированного графа существование пути — не симметричное отношение, поэтому вместо понятия связности различают понятие слабой и сильной связности.
=== Слабая связность ===
{{Определение
|definition=
Отношение \(R(v, u)\) называется отношением '''слабой связности''' ''(англ. weak connectivity)'', если вершины \(u\) и \(v\) связаны в неориентированном графе \(G'\) , полученном из графа \(G\) удалением ориентации с рёбер.
}}

{{Теорема
|statement=
Слабая связность '''является [[Отношение_эквивалентности|отношением эквивалентности]]'''.
|proof=
Аналогично доказательству соответствующей теоремы для неориентированного графа.
}}
[[Файл:components1.png|400px|thumb|left|Пример ориентированного графа с тремя компонентами слабой связности.]]

=== Сильная связность ===
{{Определение
|id=sc_def
|definition=
Отношение R(v, u) = v \rightsquigarrow u \land u \rightsquigarrow v на вершинах графа называется отношением '''сильной связности''' ''(англ. strong connectivity)''.
}}

{{Теорема
|statement=
Сильная связность {{---}} '''[[Отношение_эквивалентности|отношение эквивалентности]]'''.
|proof=
'''[[Рефлексивное_отношение|Рефлексивность]]''' и '''[[Симметричное_отношение|симметричность]]''' очевидны. Рассмотрим '''[[Транзитивное_отношение|транзитивность]]''': 
(a\rightsquigarrow b \land b\rightsquigarrow a) \land (b\rightsquigarrow c \land c\rightsquigarrow b)\Leftrightarrow (a\rightsquigarrow b \land b\rightsquigarrow c) \land (c\rightsquigarrow b \land b\rightsquigarrow a) \Leftrightarrow a\rightsquigarrow c \land c\rightsquigarrow a
}}

{{Определение
|definition=
Пусть G = (V, E) — [[Основные_определения_теории_графов|ориентированный граф]]. '''Компонентой сильной связности''' ''(англ. strongly connected component)'' называется класс эквивалентности множества вершин этого графа относительно сильной связности.}}
Компоненты сильной связности могут быть найдены [[Использование обхода в глубину для поиска компонент сильной связности|с помощью обхода в глубину]].
[[Файл:Components2.png|400px|thumb|left|Пример ориентированного графа с тремя компонентами сильной связности.]]
{{Определение
|definition=
[[Основные_определения_теории_графов|Ориентированный граф]] G = (V, E) называется '''сильно связным''' ''(англ. strongly connected)'', если он состоит из одной компоненты сильной связности.}}

==См. также==

*[[Отношение рёберной двусвязности]]
*[[Отношение вершинной двусвязности]]

==Источники информации==
* [http://xn--90abr5b.xn--p1ai/wiki/doku.php?id=examination:diskretka:question12 Отношения связности для вершин неорграфа на ivtb.ru]
* Харари Фрэнк '''Теория графов''': Пер. с англ./ Предисл. В. П. Козырева; Под ред. Г.П.Гаврилова. Изд. 4-е. — М.: Книжный дом "ЛИБРОКОМ", 2009. — 296 с. — ISBN 978-5-397-00622-4.

[[Категория:Алгоритмы и структуры данных]]
[[Категория:Связность в графах]]