{{Определение
|id = tree_barycenter
|definition = '''Барицентром дерева''' (англ. ''tree barycenter'') называется вершина x , у которой величина d(x) = \sum\limits_v dist(x, v) минимальна, где dist(x, v) - расстояние между вершинами x и v .
}}

== Основные свойства ==

{{Лемма
|id = lem1
|statement = Пусть существуют вершины y, z - соседи вершины x . Тогда 2d(x) .
|proof = Подвесим дерево за вершину x . Тогда дерево можно представить в виде объединения трёх непересекающихся множеств: Y, Z \ - поддеревья с корнями в вершинах y, z соответственно и X = V_G \setminus (Y \cup Z) . Заметим, что все эти множества не пустые, так как содержат вершины y, z, x соответственно. Найдём d(x) .
Простой путь до вершины t \in Y из x всегда единственный и представим следующим образом: x \rightarrow y \rightsquigarrow t. Значит, все вершины из множества Y находятся от x на одно ребро дальше, чем от y . Аналогично все вершины из множеств Z, X ближе на одно ребро к x , чем к y . Тогда d(x) = d(y) + |Y| - |Z| - |X| . Аналогично d(x) = d(z) + |Z| - |Y| - |X| . Сложим эти уравнения и получим: 2d(x) = d(y) + d(z) - 2|X| . При этом |X| > 0 . Таким образом, 2d(x) .
}}

{{Лемма
|id = lem2
|statement = Функция d(x) строго выпукла на любом пути в дереве.
|proof = Признак непрерывной строго выпуклой функции https://ru.wikipedia.org/wiki/Выпуклая_функция: \forall x, y \in \mathbb{R} (x \neq y) выполнено 2f(\displaystyle \frac{x+y}{2}) . Будем называть функцию строго выпуклой на множестве натуральных чисел, если предыдущее неравенство выполнено для всех x, y \in \mathbb{N} . Введём функцию g_p(x): \mathbb{N} \rightarrow V_G , которая по номеру вершины в пути p находит саму вершину. В дальнейшем d(a) будем считать как d(g_p(a)) для некоторого рассматриваемого пути p . Доопределим функцию d(x) так, чтобы в точке a + \displaystyle \frac{1}{2} , где a \in \mathbb{N} она принимала значение, строго меньшее \displaystyle \frac{d(a)+d(a+1)}{2} . Это нужно сделать, потому что не всегда \displaystyle \frac{a+b}{2} \in \mathbb{N}. Докажем, что такая функция строго выпукла на множестве натуральных чисел.
Есть два случая: когда расстояние между x, y четное и нечетное. Докажем первый случай, второй доказывается аналогично. Пусть в пути от x до y , кроме них есть только вершина z . Тогда по предыдущей лемме неравенство очевидно. Пусть есть другие вершины, кроме x, y, z , тогда их не меньше двух, так как dist(x, y) четно. Рассмотрим тогда в этом пути вершины, которые находятся от z на расстоянии не больше 2 (пусть они идут в порядке: a b z c d ), и докажем, что неравенство всё ещё сохраняется. 4d(z) . Так будем увеличивать расстояние от z и придём к вершинам x, y , сохраняя инвариант. 
}}

{{Теорема
|id = theor1
|about = о числе барицентров
|statement= В дереве не более 2 барицентов
|proof= Пусть в дереве есть хотя бы 3 барицентра: a, b, c . Тогда рассмотрим путь, начинающийся в a и заканчивающийся в b . Так как d(a) = d(b) = d_{min} , и функция d(x) строго выпукла, вершины a, b являются соседями. В противном случае, или в этом пути есть вершина v: d(v) , или для всех вершин u в пути d(u) = d_{min} . Первое предположение противоречит тому, что a, b \ - барицентры, а второе - тому, что функция d(x) строго выпукла. Таким образом, вершины a, b являются соседями. Аналогично доказывается, что вершины b, c и a, c \ - соседи. Но в таком случае в дереве образовался цикл, что противоречит определению дерева. Таким образом, более 2 барицентров в дереве быть не может.
}}

== Центр дерева ==

{{Определение
|id = tree_center
|definition = '''Центром дерева''' (англ. ''Tree center'') называется вершина x , для которой величина \max\limits_v dist(x, v) минимальна.
}}

{{Теорема
|id = theor2
|statement= Для любого числа k существует дерево, в котором расстояние между центром и барицентром дерева не меньше k 
|proof= Рассмотрим дерево, построенное следующим образом: к вершине дерева x подвесим n свободных вершин (в дереве они станут листьями) и бамбук, расстояние в котором от листа до x назовём числом l . Докажем, что существуют такие n, l , что расстояние между центром и барицентром не меньше k .
Для удобства будем считать, что центр один, для этого будем рассматривать только нечётные l. Назовём лист бамбука вершиной a , а центр дерева - \ c . Тогда dist(a, c) = \displaystyle \frac{l+1}{2} . Теперь будем искать, какое n стоит выбрать, чтобы барицентром оказалась вершина x . Найдём d(x) : d(x) = n + 1 + \dots + l = n + \displaystyle \frac{(l+1)l}{2} . Рассмотрим вершину v \neq x . d(v) > 2(n-1) , так как все вершины, кроме x удалены хотя бы на расстояние 2 от n-1 вершины. В таком случае, d(x) \displaystyle \frac{(l+1)l}{2} + 2 . Мы получили, что dist(c, x) = \displaystyle \frac{l-1}{2} , и x является барицентром. Найдём такие l , что \displaystyle \frac{l-1}{2} \geqslant k. Для этого можно взять любое l \geqslant 2k + 1 . Таким образом, искомые n, l существуют.
}}

'''Замечание''': доказательство существования такого n , чтобы вершина x была барицентром можно было проводить и менее строго, ведь очевидно, что при больших n у барицентра должно быть минимальное расстояние до n листьев. И такой вершиной и является x .

== Примечания ==

== См. также ==

* [[Выпуклые функции]]
* [[Дерево, эквивалентные определения]]

== Источники информации ==

* ''Melnikov O.''. «Exercises in Graph Theory» — «Springer», 2013 г. — 47-48 стр. — ISBN 978-94-017-1514-0

[[Категория: Дискретная математика и алгоритмы]]
[[Категория: Основные определения теории графов]]