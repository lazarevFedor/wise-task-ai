{{Определение
|neat = 1
|definition= '''Модель Эрдёша-Реньи''' (англ. ''Erdős–Rényi model'') {{---}} модель генерации случайных графов, в которой все графы с фиксированным набором вершин и фиксированным набором рёбер одинаково вероятны. Существует два тесно связанных варианта модели: ''биномиальная'' и ''равномерная''.
}}
{{Определение 
|neat = 1
|definition= '''Биномиальная модель случайного графа''' (англ. ''binomial random graph model'') G(n, p) {{---}} модель, в которой каждое ребро входит в случайный граф независимо от остальных ребер с вероятностью p. G(n, p) = (\Omega_n, F_n, P_{n, p}) {{---}} [[ Вероятностное пространство, элементарный исход, событие | вероятностное пространство ]]. |\Omega_n| = 2^{C^2_n}, P_{n, p}(G) = p^m(1 - p)^{C^2_n - m}, где m {{---}} число ребер в графе.
}}
{{Определение 
|neat = 1
|definition= '''Равномерная модель случайного графа''' (англ. ''uniform random graph model'') G(n, m) {{---}} модель, в которой все графы с m ребрами равновероятны. G(n, m) = (\Omega_n, F_n, P_{n, m}) {{---}} вероятностное пространство. |\Omega_n| = m, P_{n, m}(G) = \dfrac{1}{C^m_n}.
}}
{{Определение
|definition= Свойство A '''асимптотически почти наверное истинно''', если \lim\limits_{n \rightarrow \infty} p(n) = 1, где p(n) {{---}} вероятность графа G(n, p) обладать этим свойством.
}}
{{Определение
|definition= Свойство A '''асимптотически почти наверное ложно''', если \lim\limits_{n \rightarrow \infty} p(n) = 0, где p(n) {{---}} вероятность графа G(n, p) обладать этим свойством.
}}

== Существование треугольников в случайном графе ==
{{Теорема
|statement=Если p(n) = o(\dfrac{1}{n}), то G(n, p) асимптотически почти наверное (далее а.п.н) не содержит треугольников.
|proof=
Пусть T {{---}} число треугольников в графе, T_{i,j,k} {{---}} индикаторная случайная величина, равная 1, если вершины i, j и k образуют треугольник.

Воспользуемся [[Неравенство Маркова| неравенством Маркова]]:

P(T > 0) = P(T \geqslant 1) \leqslant E[T] = \sum\limits_{i, j, k}T_{i, j, k}p^3 = C^3_np^3 \sim \dfrac{n^3p^3}{6} \rightarrow 0, при n \rightarrow \infty.
}}

{{Теорема
|statement=Если p(n) = \omega(\dfrac{1}{n}), то G(n, p) а.п.н содержит треугольник.

|proof=
Пусть T {{---}} число треугольников в графе, T_{i,j,k} {{---}} индикаторная случайная величина, равная 1, если вершины i, j и k образуют треугольник.

Воспользуемся [[Неравенство Маркова#thCheb| неравенством Чебышева]]:

P(T = 0) = P(T \leqslant 0) = P(E[T] - T \geqslant E[T]) \leqslant P(|E[T] - T| \geqslant E[T]) \leqslant \dfrac{D[T]}{(E[T])^2}.

Найдем E[T^2]:

E[T^2] = E[(\sum\limits_{i, j, k}T_{i, j, k})^2]= E[\sum\limits_{i, j, k}T_{i, j, k}^2] + E[\sum\limits_{i, j, k, a, b, c}T_{i, j, k}T_{a, b, c}] =

= E[T] + (C^3_nC^3_{n - 3} + C^3_nC^2_{n - 3})p^6 + 3C^3_n(n - 3)p^5 \sim \dfrac{n^3p^3}{6} + (\dfrac{n^6}{36} + \dfrac{n^5}{4})p^6 + \dfrac{n^4}{2}p^5 \sim \dfrac{n^3p^3}{6} + \dfrac{n^6p^6}{36} + \dfrac{n^4p^5}{2} \sim \dfrac{n^3p^3}{6} + \dfrac{n^6p^6}{36}

D[T] = E[T^2] - (E[T])^2 \sim \dfrac{n^3p^3}{6} + \dfrac{n^6p^6}{36} - \dfrac{n^6p^6}{36} = \dfrac{n^3p^3}{6} 

P(T = 0) \leqslant \dfrac{\dfrac{n^3p^3}{6}}{\dfrac{n^6p^6}{36}} = \dfrac{6}{p^3n^3} \rightarrow 0, при n \rightarrow \infty
}}

== Связность графа ==

{{Лемма
|id=lemma1
|statement=Если c \geqslant 3, n \geqslant 100, p = \dfrac{c\ln n}{n}. Тогда P(G - связен) \rightarrow 1.
|proof= 
Пусть X {{---}} индикаторная величина, равная нулю, если G связен, и k, если G содержит k компонент связности.

X_i {{---}} число компонент связности размера i.

X_{a_1,a_2, \dots , a_i} = 1, если a_1,a_2, \dots , a_i {{---}} компонента связности.

X_i = \sum\limits_{a_1,a_2, \dots , a_i} X_{a_1,a_2, \dots , a_i}

E[X_i] = \sum\limits_{a_1,a_2, \dots , a_i} E[X_{a_1,a_2, \dots , a_i}] = C_n^iEX_{1, 2, \dots, i} = C_n^i P(1, 2, \dots, i - комп.связности) \leqslant C_n^i (1 - p)^{i(n - i)}.

E[X] \sum\limits_{i = 1}^{n - 1} E[X_i] \leqslant \sum\limits_{i = 1}^{n - 1} C_n^i(1 - p)^{i(n - i)}

Последняя сумма симметрична (слагаемые при i = k и i = n - k равны), кроме того слагаемое при i = 1 {{---}} наибольшее (для доказательства достаточно рассмотреть отношения слагаемых при i \leqslant \dfrac{n}{8} и \dfrac{n}{8} ).

Оценим сверху первое слагаемое n(1 - p)^{n - 1}:

n(1 - p)^{n - 1} \leqslant ne^{-p(n - 1)} \leqslant ne^{\frac{-3 (n - 1) \ln n}{n}}

n \geqslant 100, поэтому \dfrac{n - 1}{n} > 0.9.

ne^{\frac{-3 (n - 1) \ln n}{n}} 

\sum\limits_{i = 1}^{n - 1} C_n^i(1 - p)^{i(n - i)} \leqslant \sum\limits_{i = 1}^{n - 1}\dfrac{1}{n^{2.7}} , при n \rightarrow \infty

}}

{{Лемма
|id=lemma2
|statement=Если c \geqslant 3, n \geqslant 100, p = \dfrac{c\ln n}{n}. Тогда P(G - связен) > 1 - \dfrac{1}{n}.
}}

{{Теорема 
|statement=p = \dfrac{c\ln n}{n}, тогда при c граф а.п.н связен, при c > 1 граф а.п.н не связен.
}}

== Распределение степеней вершин ==
{{Определение
|id=def_degree_dist
|definition='''Распределение степеней вершин случайного графа''' - это функция P(x), определённая на \mathbb{R} как P(\xi=x), то есть выражающая вероятность того, что вершина \xi имеет степень x. Другими словами, распределение степеней P(k) графа определяется как доля узлов, имеющих степень k.
}}
{{Пример
|id=example_1
|example=Если есть в общей сложности n узлов в графе и из них n_k имеют степень k, то P(k) = \frac{n_k}{n}. Другими словами, P(k) равно вероятности того, что отдельно взятая вершина имеет степень k.
}}

{{Утверждение
|statement=Дан случайный граф G(n, p) в биноминальной модели. Тогда для него распределение степеней вершин

\begin{equation*}
P(k) = {n-1 \choose k} p^k(1-p)^{n-1-k}
\end{equation*}

|proof=Действительно, если вероятность появления ребра p, то вероятность появления ровно k рёбер у вершины равна p^k(1-p)^{n-1-k}([[схема Бернулли]]). Таких наборов рёбер у одной вершины всего {n-1 \choose k}, откуда получаем искомое распределение.
}}

== Распределение максимальной степени вершин ==
{{Определение
|id=def_max_degree_dist
|definition='''Распределение максимальной степени вершин случайного графа''' - это функция Q(x), определённая на \mathbb{R} как P(\xi=x), то есть выражающая вероятность того, что максимальная степень вершины \xi равна x.
}}
{{Утверждение
|statement=Q(k) = P(k) \cdot (1 - \sum_{x=k+1}^{n} P(x))
|proof=Будем выводить формулу для Q(k) через распределение степеней вершин P(k).

Максимальная степень вершины равна k тогда и только тогда, когда не существует вершины степенью больше k. Таким образом, нужно посчитать вероятность события A: \exists v\in G: \; deg(v) = k \;\&\; !\exists v\in G: \; deg(v) > x. 

P(\exists v: \; deg(v) = k) = P(k)

P(k) - вероятность того, что вершина имеет степень k. Тогда вероятность того, что имеет одну из степеней 1...k - \sum_{x=1}^{k}P(x). Нам нужно обратное событие, при наступлении которого вершина имеет степень больше k. Его вероятность равна 1 - \sum_{x=1}^{k} P(x).

P(!\exists v: \; deg(v) > k) = 1 - \sum_{x=1}^{k} P(x)

События независимы, поэтому получаем: Q(k) = P(k) \cdot (1 - \sum_{x=1}^{k} P(x))
}}

== Теоремы о связи вероятности и матожидания ==
{{Теорема
|id=th1 
|statement= Пусть N_z {{---}} число объектов в графе G(n, p). A = \{G | N_z(G) > 0 \} {{---}} свойство. Тогда, если E[N_z] \rightarrow 0, при n \rightarrow \infty, то A а.п.н ложно.
|proof=
Воспользуемся [[Неравенство Маркова | неравенством Маркова]]:

P(N_z > 0) = P(N_z \geqslant 1) \leqslant E[N_z] \rightarrow 0, при n \rightarrow \infty.
}}

{{Теорема
|id=th2 
|statement= Пусть N_z {{---}} число объектов в графе G(n, p). A = \{G | N_z(G) > 0 \} {{---}} свойство. Тогда, если E[N_z] \rightarrow \infty, при n \rightarrow \infty, и E[N_z^2] \leqslant (E[N_z])^2(1 + o(1)) то A а.п.н истинно.
|proof=
Воспользуемся [[Неравенство Маркова#thCheb | неравенством Чебышева]]:

P(N_z = 0) = P(N_z \leqslant 0) = P(E[N_z] - N_z \geqslant E[N_z]) \leqslant P(|E[N_z] - N_z| \geqslant E[N_z]) \leqslant \dfrac{D[N_z]}{(E[N_z])^2} \rightarrow 0, при n \rightarrow \infty.
}}

== Графы, имеющие диаметр два ==
{{Определение
|definition=A {{---}} некоторое свойство случайного графа. p называется '''пороговой функцией''' (англ. ''threshold function''), если граф G(n, cp) при c а.п.н не имеет такого свойства, а при c > 1 а.п.н имеет.
}}
{{Теорема
|statement=Пусть рассматривается свойство графа иметь диаметр два. Тогда p = \sqrt{2} \sqrt{\dfrac{\ln n}{n}} {{---}} пороговая функция.
|proof=
Назовем вершины u и v плохой парой, если кратчайшее расстояние между u и v больше двух. B_{i, j} {{---}} индикаторная величина, равная 1, если i и j являются плохой парой.
N_z = \sum\limits_{i, j} B_{i,j}
P(B_{i, j}) = (1 - p)(1 - p^2)^{n - 2}

Сначала докажем, что при c > sqrt{2}, граф а.п.н не имеет диаметр, равный двум. Для этого оценим матожидание N_z.
EN_z = C_n^2(1 - p)(1 - p^2)^{n - 2} \approx \dfrac{n^2}{2}(1 - c\sqrt{\dfrac{\ln n}{n}})(1 - \dfrac{c^2\ln n}{n})^{n - 2} \leqslant \dfrac{n^2}{2}e^{-c^2\ln n} = \dfrac{n^{2 - c^2}}{2}

При c > \sqrt{2} последнее выражение стремится к 0, по [[#th1 | вышедоказанному ]] граф а.п.н. не имеет диаметр, равный двум.

Рассмотрим c :

EN_z^2 = E(\sum B_{i, j})^2 = E\sum B_{i,j}^2 + E\sum B_{i,j}B_{k,l} = EN_z + \sum EB_{i,j}B_{k,l}

Рассмотрим сумму \sum EB_{i,j}B_{k,l}:

Если i, j, k и k различны, то EB_{i,j}B_{k,l} \leqslant (1 - p^2)^{2(n - 4)} \leqslant n^{-2c^2}(1 + o(1)).

\sum EB_{i,j}B_{k,l} \leqslant n^{4 - 2c^2}(1 + o(1))

EB_{i,j}B_{i,l} = (1 - p + p(1 - p)^2)^{n - 3} \approx (1 - 2p^2)^{n - 3} = (1 - 2c^2\dfrac{\ln n}{n})^{n - 3} \approx e^{-2c^2 \ln n} = n^{-2c^2}

\sum EB_{i,j}B_{i,l} \leqslant n^{3 - 2c}

В итоге: EN_z^2 \leqslant n^{2 - c^2} + n^{4 - 2c^2} + n^{3 - 2c^2}. Из этого следует, что EN_z \leqslant (EN_z)^2(1 + o(1)), а значит граф а.п.н имеет диаметр, равный двум при c > \sqrt{2}.
}}

== См. также ==
* [[Дискретная случайная величина]]
* [[Дисперсия случайной величины]]
* [[Математическое ожидание случайной величины]]

== Источники информации ==
* [https://www.coursera.org/learn/sluchajnye-graphy/ Coursera {{---}} Онлайн-курс]
* [https://en.wikipedia.org/wiki/Chernoff_bound Wikipedia {{---}} Random graphs]
* Avrim Blum, John Hopcroft, and Ravindran Kannan. «Foundations of Data Science» {{---}} «Cambridge University Press», 2013 г. {{---}} 245-260 стр. {{---}} ISBN 978-1108485067

[[Категория: Дискретная математика и алгоритмы]][[Категория: Теория графов]]