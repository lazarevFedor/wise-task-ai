== Дополнительный граф ==

{{Определение
|definition =
Пусть дан граф G \langle V, E \rangle. '''Дополнительным графом''' (англ. ''complement graph'') к \(G\) называется граф G \langle V, \overline E \rangle то есть граф с вершинами из \(V\) и теми и только теми ребрами из \(E\) , которые не вошли в \(G\) .
}}
{|class="wikitable" border="1" style="border-collapse:collapse; border:noborder"
|+
|colspan="2"|Пример графа с 6-ю вершинами и его дополнение.
|-
| [[Файл:допграф1.png|200px|link=]]
| [[Файл:допграф2.png|200px|link=]]
|}

{{Теорема
|statement=
Дополнительный [[Основные_определения_теории_графов|граф]] к дополнительному графу \(G\) есть граф \(G\) .
|proof=
\overline{\overline {G \langle V, E \rangle}} = \overline{G_1 \langle V, \overline{E} \rangle} = G_2 \langle V, \overline{\overline{E}} = G_2 \langle V, E \rangle = G
}}

{{Теорема
|statement=
В дополнительном графе к G \langle V, E \rangle. количество ребер равняется \frac{\left\vert V \right\vert \cdot \left ( \left\vert V \right\vert - 1 \right )}{2} - \left\vert E \right\vert.
|proof=

Так как множества ребер в \(G\) и \(\overline{G}\) дизъюнктны, то \(\left\vert E \right\vert + \left\vert \overline{E} \right\vert =\) \frac{\left\vert V \right\vert \cdot \left ( \left\vert V \right\vert - 1 \right )}{2}, из чего следует утверждение теоремы.
}}

{{Теорема
|statement=
Дополнительный граф к [[Отношение связности, компоненты связности|несвязному]] графу связен.
|proof=

Для графа с одной вершиной утверждение очевидно. Докажем его для остальных графов.

Пусть \(G\) {{---}} данный граф. Рассмотрим произвольные вершины \(v\) и \(u\) из \(G\) . Возможны два случая: \(v\) и \(u\) лежат в одной или в разных компонентах связности.

*Пусть \(v\) и \(u\) лежат в разных компонентах связности \(G\) . 
:Тогда ребро \((u, v) \notin G \Rightarrow (u, v) \in \overline{G} \Rightarrow u\) и \(v\) лежат в одной компоненте связности \(\overline{G}\) . 
 
[[Файл:допграф3.png|500px|слева]]

*Пусть \(v\) и \(u\) лежат в одной компоненте связности \(G\) . \(G\) {{---}} несвязный \(\Rightarrow \exists w \in G\) , не лежащая в одной компоненте связности с \(v\) и \(u\) . 
:Тогда по предыдущему пункту \((v, w) \in \overline{G}\) и \((u, w) \in \overline{G} \Rightarrow v\) и \(u\) лежат в одной компоненте связности \(\overline{G}\) . 
 
[[Файл:допграф4.png|500px|слева]]

То есть \(\forall u, v \in V\)  \(u\) и \(v\) лежат в одной компоненте связности \(\overline{G}\) , то есть \(\overline{G}\) связен.
}}

== Самодополнительный граф ==

{{Определение
|definition =
'''Самодополнительным графом''' (англ. ''self-complement'') называется граф, [[Основные определения теории графов|изоморфный]] своему дополнительному.
}}

{{Теорема
|statement=
Любой самодополнительный граф имеет \(4k\) или \(4k + 1\) вершину.
|proof=

Обозначим \(\left\vert V \right\vert\) за \(n\) , \(\left\vert E \right\vert\) за \(a\) .

Граф самодополнителен \(\Rightarrow\) количество его ребер равно количеству ребер в его дополнении.

 Но по одной из предыдущих теорем, \frac{n \cdot \left ( n - 1 \right )}{2} \(- a = \left\vert \overline{E} \right\vert = a \Rightarrow 4a = n \cdot \left ( n - 1 \right )\) , из чего следует утверждение теоремы.

}}

{{Теорема
|statement=
Для любых \(k > 0\) существует самодополнительный граф с \(4k\) или \(4k + 1\) вершиной.
|proof=

Будем доказывать по индукции. Для \(k = 1\) утверждение справедливо.

[[Файл:допграф7.png|400px|link=]]

Пусть у нас есть самодополнительный граф \(G\) с \(n\) вершинами, построим самодополнительный граф с \(n + 4\) вершинами.
Добавим к \(G\) 4 вершины \(v_1, v_2, v_3, v_4\) и ребра:

*Добавим ребра \((v_1, v_2), (v_2, v_3), (v_3, v_4)\) *Если \(u\) была вершиной в \(G\) , добавим ребра \((u, v_1), (u, v_4)\) Назовем полученный граф \(A\) . Докажем, что \(A\) самодополнителен. 

Рассмотрим биекцию на множестве вершин \(A\) и \(\overline{A}\) :
*Среди всех вершин, принадлежавших \(G\) биекция будет такая же, как и у \(G\) с \(\overline{G}\) ;
* \(v_1 \rightarrow v_2, v_2 \rightarrow v_4, v_3 \rightarrow v_1, v_4 \rightarrow v_3\) . 

Тогда между ребрами \(A\) и \(\overline{A}\) тоже будет биекция.
[[Файл:допграф9.png|400px|]]
}}

== См. также ==
* [[Основные определения теории графов]]
* [[Отношение связности, компоненты связности]]

== Источники информации ==
* ''Харари Ф.'' Теория графов. /пер. с англ. — изд. 2-е — М.: Едиториал УРСС, 2003. — 296 с. — ISBN 5-354-00301-6 
* [https://ru.wikipedia.org/wiki/Дополнение_графа Википедия {{---}} дополнение графа]
* [https://ru.wikipedia.org/wiki//Самодополнительный_граф Википедия {{---}} самодополнительный граф]

[[Категория: Алгоритмы и структуры данных]]
[[Категория: Основные определения теории графов]]