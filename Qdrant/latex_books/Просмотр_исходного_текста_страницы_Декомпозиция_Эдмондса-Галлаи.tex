В этом направлении много усилий приложили Вильям Томас '''Татт''' (''William Thomas Tutte''), Клод '''Берж''' (''Claude Berge''), Джек '''Эдмондс''' (''Jack Edmonds'') и Тибор '''Галлаи''' (''Tibor Gallai'').
{{Определение 
|id = deficit
|definition=
'''Дефицитом''' (англ. ''deficit'') графа G мы будем называть величину: 
\mathrm{def}(G) = |V| - 2\alpha (G), 
где \alpha (G) {{---}} размер [[Теорема о максимальном паросочетании и дополняющих цепях#theorem1|максимального паросочетания]] в G, а 
V(G) {{---}} множество вершин графа G. 
}}
{{Определение
|definition =\mathrm{odd}({G}) {{---}} число нечетных компонент связности в графе {G}, где '''нечетная компонента''' (англ. ''odd component'') {{---}} это [[Отношение связности, компоненты связности#def2|компонента связности]], содержащая нечетное число вершин. 
}}

{{Лемма
|statement= (n + |S| + odd(G \setminus S)) \; \equiv \; 0 \; ( mod \; 2) \; , где G {{---}} граф с n вершинами, S \subset {V}_{G}
|proof= 
Удалим из графа G множество S, получим t компонент связности, содержащих k_1, k_2 ... k_t вершин соответственно.
|S|\; + \; \sum_{i=1}^{k}k_i \; = \; n \; , так как в сумме это все вершины исходного графа G. 
Возьмем данное равенство по модулю два: (|S|\; + \; \sum_{i=1}^{k}k_i) \; \equiv \; n \; (mod \; 2)
В сумме \sum_{i=1}^{k}(k_i \; mod \; 2) число единиц равно числу нечетных компонент odd(G \setminus S). Таким образом, \forall S \in V : \; (odd(G \setminus S) + |S|) \; \equiv \; n \; (mod \; 2) \;. 
}}

{{Теорема
|id = Th_Berge
|about=Бержа
|statement=
Для любого графа G выполняется:
\mathrm{def}(G) = \max\limits_{S \subset V(G)} \{\mathrm{odd}(G - S) - |S|\}. 
|proof= 
 \forall S \in V : \; (odd(G \setminus S) + |S|) \; \equiv \; n ( mod \; 2) \;

Рассмотрим несколько случаев:

1. Если \max\limits_{S \in V}(odd(G \setminus S) \; - \; |S|) \; = 0 \; , тогда для любых S \in V: \; odd(G \setminus S) \leq |S| \; , следовательно выполнено условие [[Теорема Татта о существовании полного паросочетания|теоремы Татта]], значит, в графе есть совершенное паросочетание, то есть его дефицит равен нулю. 

2. Если \max\limits_{S \in V}(odd(G \setminus S) - |S|) = k \; , тогда рассмотрим исходный граф G и полный граф K_k с k вершинами, W - вершины K_k. Каждую вершину K_k соединим с каждой вершиной G. Получим граф H \; = \; K_k + G \;, докажем, что для него выполнено условие теоремы Татта. Докажем, что для любых S \in V_{H}: odd(H \setminus S) \; \leq \; |S| \; . 
Рассмотрим S \; \subset \; V_H\;:

* Если W \not\subset S, тогда поскольку граф K_k полный и все его вершины связаны с каждой вершиной графа G, то граф H связный и odd(H \setminus S) \; = \; 0 \; или odd(G \setminus S) \; = \; 1 \;. 
** В случае odd(H \setminus S) \; = \; 0 \; условие очевидно выполняется, так как для любых S \in G : 0 \; \leq \; |S| \;. 
** Рассмотрим случай odd(H \setminus S) \; = \; 1 \;, |V_H| \; = \; n \; + \; k \; = \; n \; + \; odd(G \setminus A) \; - \; |A| \; , где A \; = \; arg \max\limits_{S \in V}(odd(G \setminus S) \; - \; |S|) \; . Разность odd(G \setminus A) \; - \; |A| \; имеет ту же четность, что и n, и odd(H \setminus S) \; = \; 1 \;, поэтому |V_H| четно, значит, по лемме, мощность S нечетна, следовательно она не равна нулю, значит, 1 \leq |S| .

* Если W \subset S \;, то odd(H \setminus S) \; = \; odd(G \setminus (S \cap V)) \; = odd(G \setminus (S \cap V)) \; - \; |S \cap V| \; + \; |S \cap V| \; \leq \; |S \cap V| \; + \; k \leq |S| \; , так как \max\limits_{S \in V}(odd(G \setminus S) - |S|) = k \; . Таким образом, для графа H выполнено условие теоремы Татта, следовательно в нём есть полное паросочетание. Рассмотрим полное паросочетание в графе H, удалим вершины W из графа H. Количество непокрытых вершин после удаления не больше, чем количество удаленных вершин k, значит, def(G) \; \leq \; k. Удалим множество вершин A \; = \; arg \max\limits_{S \in V}(odd(H \setminus S) \; - \; |S|) \; из графа G\;. Заметим, что после удаления в графе осталось odd(G \setminus A)\; нечетных компонент и образовались новые непокрытые вершины, но при этом число нечетных компонент больше числа удаленных на k. Значит, хотя бы k нечетных компонент содержали исходно непокрытую вершину, следовательно def(G) \; \geq \; k \; . Из def(G) \; \leq \; k и def(G) \; \geq \; k \; следует def(G) \; = \; k \; .
}}

{{Теорема
|id = theorem_Tatt_Berge
|about=Татта-Бержа
|statement=
Дан граф G, размер максимального паросочетания в нем равен:
\mathrm{\alpha}(G) = \min\limits_{U \in V} \{\dfrac{1}{2}(|V|+|U|-\mathrm{odd}(G - U)\}. 
|proof=
Предположим G {{---}} связный, иначе мы можем применить индукцию к компонентам G. Приведем доказательство по индукции по числу вершин в графе. 
 ''База индукции:'' 
Очевидно, для n = 1 утверждение верно. 
 ''Индукционный переход:'' 
Рассмотрим два случая:
# G {{---}} содержит вершину v покрытую всеми максимальными паросочетаниями (например средняя вершина)
#: Тогда \mathrm{\alpha}(G - v) = \mathrm{\alpha}(G) - 1 .
#: По индукции, формула Татта-Берджа содержит G - v для некоторого множества U'. Пусть U = U' \bigcup v. Тогда:
#: \mathrm{\alpha}(G) = \mathrm{\alpha}(G - v) + 1 = \dfrac{1}{2}(|V - v|+|U - v| - \mathrm{odd}(G - v - (U - v))) + 1 = 
#: = \dfrac{1}{2}(|V| - 1 + |U|- 1 - \mathrm{odd}(G - U)) + 1 = \dfrac{1}{2}(|V|+|U| - \mathrm{odd}(G - U)). 
#:
# Для каждой вершины v есть максимальное паросочетание M которое не покрывает v (например C_3)
#:
#: Покажем, что существует паросочетание размера \dfrac{1}{2}(|V| - 1) , из которого следует теорема (при U = \emptyset ).
#: ''От противного:''
#: Предположим что любое максимальная паросочетание M не покрывает, по крайней мере, две различные вершины u и v . Среди всех таких (M, u, v) выберем их так, что \mathrm{d}(u, u) в G {{---}} минимально.
#: Если \mathrm{d}(u, u) = 1 , то u и v являются смежными, и, следовательно, мы можем увеличить M , что противоречит его максимальности.
#: Значит \mathrm{d}(u, u) \geqslant 2 , и, следовательно, мы можем выбрать промежуточную вершину t на пути u-v и N максимальное паросочетание, такое что симметрическая разность с M минимальна. Так как (M, u, v) минимально, то N должно охватывать u и v так, что есть другая вершина x , покрытая только в M .
#: Пусть y будет вершиной покрытой с x в M и заметим y \neq t (иначе можно было бы добавить к N ). Пусть z будет вершиной покрытой с y в N и заметим z \neq x (так как x не покрыто в N ). Тогда N - yz + xy {{---}} паросочетание, которое имеет с M меньшую симметрическую разность, что противоречит выбору N .
}}
{{Определение 
|id=barrier
|definition=
Множество S \subset V (G), для которого \mathrm{odd}(G - S) - |S| = \mathrm{def}(G) , называется '''барьером''' (англ. ''barrier'').
}}
{{Определение
|definition=Пусть X \subset V . '''Множeство соседей''' (англ. ''neighbors'') X определим формулой: N(X)= \{ y \in V:(x,y) \in E \}
}}

==Структурная теорема Эдмондса-Галлаи==
{{Определение 
|neat = 1
|definition=
Структурные единицы декопозиции:
# D(G) = \{v \in V \mid существует [[Теорема о максимальном паросочетании и дополняющих цепях|максимальное паросочетание]], не покрывающее v\}
# A(G) = N(D(G)) \setminus D(G)
# C(G) = V \setminus(D(G) \bigcup A(G))
# \alpha (G) {{---}} размер максимального паросочетания в G. 
}}
[[Файл: EG_red.png|300px|thumb|right|Пример. Рёбра из паросочетания выделены красным]]
{{Определение 
|definition=
Граф G называется '''фактор-критическим''' (англ. ''factor-critical graph''), если для любой вершины v \in G в графе G \setminus {v} существует [[Теорема Холла#def1|совершенное паросочетание]].
}}

{{Теорема
|id = theorem_Gallai
|about=Галлаи
|statement=
G {{---}} фактор-критический граф \Leftrightarrow 
G {{---}} связен и для любой вершины u \in V(G) выполняется равенство \alpha (G - u) = \alpha (G).
}}

{{Лемма
|id = stability_lemma
|about= Галлаи, о стабильности (англ. ''stability lemma'')
|statement=
Пусть a \in A(G). Тогда: 
# D(G - a) = D(G) 
# A(G - a) = A(G) \setminus \{a\}
# C(G - a) = C(G) 
# \alpha (G - a) = \alpha (G) - 1.
|proof=
Для начала докажем, что D(G - a) = D(G). 
[[Файл: Gallai-lema-a.png|150px|thumb|right|Случай '''а''']]
[[Файл: Gallai-lema-b.png|150px|thumb|right|Случай '''b''']]
[[Файл: Gallai-lema-с.png|150px|thumb|right|Случай '''c''']]
# Покажем, что D(G - a) \supset D(G) : 
#:Пусть u \in D(G). Тогда существует [[Теорема о максимальном паросочетании и дополняющих цепях|максимальное паросочетание]] M_u графа G, не покрывающее u. Поскольку любое максимальное паросочетание графа G покрывает a, то \alpha (G - a) = \alpha (G) - 1 и более того, если, для некоторой вершины x \in D(G), ax \in M_u, то M_u \setminus {ax} {{---}} максимальное паросочетание графа G - a , не покрывающее u . Таким образом, D(G - a) \supset D(G) . 
#покажем, что D(G - a) \subset D(G): 
Предположим, что существует максимальное паросочетание M' графа G - a, не покрывающее вершину v \notin D(G). Пусть w \in D(G) {{---}} смежная с a \in A(G) вершина, а M_w {{---}} максимальное паросочетание графа G , не покрывающее w . Так как v \notin D(G) , максимальное паросочетание M_w покрывает вершину v. Рассмотрим граф H = G(M_w \bigcup M') {{---}} очевидно, он является объединением нескольких путей и чётных циклов. Пусть U {{---}} компонента связности графа H , содержащая v. Так как deg_H(v) = 1 (степень вершины), то P = H(U) {{---}} путь с началом в вершине v. В пути P чередуются рёбра из M_w и M' , причём начинается путь ребром из M_w . Так как deg_H(a) = 1 , то вершина a либо не принадлежит пути P, либо является её концом (в этом случае последнее ребро пути принадлежит паросочетанию M_w). Рассмотрим несколько случаев: 

'''a.''' Путь P кончается ребром из M' (см. рисунок)
Рассмотрим паросочетание M_v = M_w \oplus E(P) (симметрическая разность
 M_w и E(P). то есть, рёбра, входящие ровно в одно из двух множеств).
Очевидно, M_v {{---}} максимальное паросочетание графа G, не покрывающее v, поэтому v \in D(G), противоречие. 

'''b.''' Путь P кончается ребром из M_w, вершина a {{---}} конец пути P. (см.рисунок)
Рассмотрим паросочетание M_v∗ = (M_w \oplus E(P)) \bigcup \{aw\} . Тогда M_v∗ {{---}} максимальное паросочетание графа G , не покрывающее v , поэтому v \in D(G) , противоречие.

'''c.''' Путь P кончается ребром из M_w, a \in V(P) (см. рисунок)
Рассмотрим паросочетание M'' = M \oplus E(P) . Тогда |M''| = |M'| + 1 , причём M'' \subset E(G - a). Противоречие с максимальностью паросочетания M'.

Таким образом, наше предположение невозможно и D(G - a) \subset D(G).
А значит, D(G - a) = D(G).

Так как D(G - a) = D(G), то все вершины, которые были соседями D(G), таковыми и остались. Однако, по условию a \in A(G), значит A(G - a) = A(G) \setminus \{a\}.

Так же заметим, что C(G - a) = V(G - a) \setminus (D(G - a) \cup A(G - a)) = V(G - a) \setminus (D(G) \cup (A(G) \setminus \{a\})) = V(G) \setminus (D(G) \cup A(G)) = C(G)

Наконец, так как a \in A(G), то все максимальные паросочетания в G включали a. Следовательно, \alpha (G - a) . Заметим, что, взяв любое максимальное паросочетания в G и удалив ребро инцидентное a, мы получим паросочетание M', которое на 1 меньше исходного, при этом M' \in E(G - a). В свою очередь, это самое большое паросочетание, которое мы могли теоретически получить в G - a. Следовательно, \alpha (G - a) = \alpha (G) - 1.
}}

{{Теорема
|id = theorem_Gallai_Edmonds
|about = Галлаи, Эдмондс
|statement=
Пусть G {{---}} граф, U_1\ldots U_n {{---}} компоненты связности графа G(D(G)), D_i = G(U_i), C = G(C(G)). Тогда:
# Граф C имеет совершенное паросочетание.
# Графы D_1\ldots D_n {{---}} фактор-критические. 
# Любое максимальное паросочетание M графа G состоит из совершенного паросочетания графа C , почти совершенных паросочетаний графов D_1\ldots D_n и покрывает все вершины множества A(G) рёбрами с концами в различных компонентах связности U_1\ldots U_n. 
# \mathrm{def}(G) = n - |A(G)|. 
# 2\mathrm{\alpha}(G) = v(G) + |A(G)| - n.
|proof=
[[Файл: Edmonds-Gallai_2.png|300px|thumb|right|Пример]]
# Последовательно удаляя вершины множества A = A(G), по лемме о стабильности мы получим:
#:* D(G - A) = D(G), 
#:* A(G - A) = \O, 
#:* C(G - A) = C(G),
#:* \alpha (G - A) = \alpha (G) - |A|.
#:
#:Это означает, что не существует рёбер, соединяющих вершины из C(G - A) и D(G - A). Каждое максимальное паросочетание M' графа G - A покрывает все вершины множества C(G), поэтому M' содержит совершенное паросочетание графа C. Тем самым, мы доказали пункт 1).
#:
# Из формулы \alpha(G - A) = \alpha (G) - |A| следует, что U_1\ldots U_n {{---}} компоненты связности графа G - A. Для любой вершины u \in U_i существует максимальное паросочетание M_u графа G - A, не содержащее u. Так как U_i {{---}} компонента связности графа G - A, паросочетание M_u содержит максимальное паросочетание графа D_i (разумеется, не покрывающее вершину u). Следовательно, \alpha (D_i) = \alpha (D_i - u) и по теореме Галлаи (мы получаем, что граф D_i {{---}} фактор-критический.
#:
# Пусть M {{---}} максимальное паросочетание графа G, а M' получено из M удалением всех рёбер, инцидентных вершинам множества A. Тогда |M'| \geqslant |M| - |A| и по формуле \alpha (G - A) = \alpha (G) - |A| понятно, что M' {{---}} максимальное паросочетание графа G - A. Более того, из \alpha (G - A) = \alpha (G) - |A| следует |M'| = |M| - |A|, а значит, все вершины множества A покрыты в M различными рёбрами. Так как M' {{---}} максимальное паросочетание графа G - A, то по пунктам 1) и 2) очевидно, что M' содержит совершенное паросочетание графа C и почти совершенные паросочетания фактор-критических графов D_1\ldots D_n. Значит, рёбра паросочетания M соединяют вершины A с непокрытыми M' вершинами различных компонент связности из U_1\ldots U_n. 
# Из пункта 3) сразу же следуют равенства пункта 4) и 5).
}}

{{Утверждение
|about=следствие из теоремы
|statement=
A(G) {{---}} '''барьер''' графа G
}}

{{Лемма
|id = barier_struct1
|about = о связи барьера с D(G)
|statement= Для любого барьера B графа G верно, что B\cap D(G) = \varnothing
|proof= Рассмотрим U_{1}, U_{2}, \ldots U_{n} {{---}} нечётные компоненты связанности G \setminus B, \ M {{---}} максимальное паросочетание в G. \forall\ U_{i}\ \exists x \in U_{i}: x не покрыта \ M или xv \in M \land v \in B. Всего графе не покрыто хотя бы odd(G\setminus B) - |B| вершин. Однако, так как B {{---}} барьер, непокрыто '''ровно''' столько вершин. Следовательно, любое максимальное паросочетание не покрывает только вершины из G \setminus B, а значит каждая вершина барьера покрыта в любом максимальном паросочетании. Отсюда получаем, что ни одна вершина из D(G) не могла оказаться в барьере.
}}

{{Утверждение
|id = barier_struct1a
|about=Следствие из леммы
|statement=В любом максимальном паросочетании все вершины барьера соединены соединены с вершинами G \setminus B
|proof=Так как для барьера B верно, что odd(G\setminus B) - |B|=def(G) \geqslant 0, то ровно |B| вершин из нечётных компонент G \setminus B покрыты рёбрами xv \in M \land v \in B
}}

{{Лемма
|id = barier_struct2
|about = о дополнении барьера
|statement= Пусть x\in A(G)\cup C(G),\ G'=G\setminus x,\ B' {{---}} барьер графа G'. Тогда B=B'\cup x {{---}} барьер графа G
|proof= Так как x \notin D(G), то для любого максимального паросочетания M: x \in M. Следовательно, |M'| = |M| - 1, где M' {{---}} максимальное паросочетание в G'.

def(G') = (|V| - 1)- 2 \cdot |M'| = |V| - 2 \cdot |M| + 1 = def(G) + 1 

odd(G - (B'\cup x)) = odd(G' - B') = |B'| + def(G') = |B'| + 1 + def(G) = |B'\cup x| + def(G)
Отсюда следует, что B {{---}} барьер графа G.
 }}

{{Теорема
|id=barier_struct3 
|about=о структуре барьера
|statement=Любой барьер графа состоит только из вершин A(G)\cup C(G), причём каждая вершина из этого множества входит в какой-то барьер
|proof=По лемме о связи барьера с D(G) мы знаем, что в барьере нет вершин вершин из D(G). По лемме о дополнение барьера мы можем взять любую вершину из A(G)\cup C(G), удалить из графа, и с помощью барьера нового графа получить барьер исходного, включающий данную вершину.
}}

== См. также ==
* [[Теорема Татта о существовании полного паросочетания]]
* [[Лапы и минимальные по включению барьеры в графе]]
* [[Пересечение всех максимальных по включению барьеров]]

== Источники информации==
*[http://www.people.vcu.edu/~dcranston/691/edmonds-gallai.pdf Edmonds-Gallai Decomposition and Factor-Critical Graphs]
*[http://immorlica.com/combOpt/lec2.pdf Edmonds-Gallai Decomposition, Edmonds’ Algorithm]
*[https://www.youtube.com/watch?v=1KggxCJZFRg {{---}} Лекция А.С. Станкевича]

[[Категория:Алгоритмы и структуры данных]]
[[Категория:Задача о паросочетании]]