{{Теорема
|statement=
Если f {{---}} некоторый [[Определение_сети,_потока|поток]] в сети G = (V, E) с источником s и стоком t, то следующие утверждения эквивалентны:
# Поток f максимален.
# В G_f не существует пути s \leadsto t. 
# |f| = c(S, T) для некоторого разреза (S, T) сети G. 
|proof=
 (1)\Rightarrow (2)

Докажем от противного. Предположим, что в G_f существует какой-нибудь путь p = s \leadsto t.
Тогда рассмотрим f + f_p . По [[Лемма_о_сложении_потоков|лемме о сумме потоков]] f + f_p тоже является потоком в сети G , 
и причем |f + f_p| = |f| + |f_p| > |f| , что приводит нас к противоречию, что f максимальный поток.

 (2) \Rightarrow (3) 

Рассмотрим множество S = \lbrace v \in V : \exists\, s \leadsto v \text{ in } G_f \rbrace и T = V \setminus S. 
Разбиение \langle S, T \rangle является разрезом, так как по (2) в G_f не существует s \leadsto t.
По [[Разрез,_лемма_о_потоке_через_разрез|лемме о потоке через разрез]] f(S, T) = |f| . Также \forall u \in S, v \in T известно, что f(u, v) = c(u, v) , так как иначе вершина v 
должна была бы принадлежать множеству S . Поэтому c(S, T) = f(S, T) = |f| .

 (3) \Rightarrow (1) 

Так как существует разрез, такой что |f| = c(S, T) , то согласно [[Разрез,_лемма_о_потоке_через_разрез|следствию леммы о слабой двойственности потока и разреза]] |f| \leqslant c(S, T), поэтому f максимален.
}}

== См. также ==
* [[Определение_сети,_потока|Определение сети, потока]]
* [[Алгоритм_Форда-Фалкерсона,_реализация_с_помощью_поиска_в_глубину|Алгоритм Форда-Фалкерсона, реализация с помощью поиска в глубину]]

== Источники информации ==
* ''Кормен, Томас Х., Лейзерсон, Чарльз И., Ривест, Рональд Л., Штайн Клиффорд'' '''Алгоритмы: построение и анализ''', 2-е издание. Пер. с англ. — М.:Издательский дом "Вильямс", 2010. — 1296 с.: ил. — Парал. тит. англ. — ISBN 978-5-8459-0857-5 (рус.)

[[Категория: Алгоритмы и структуры данных]]
[[Категория: Задача о максимальном потоке ]]