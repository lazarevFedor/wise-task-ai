'''Многочлен Татта''' — наиболее общая характеристика, описывающая комбинаторные свойства графа.

==Основное определение==
{{Определение|definition=
Рассмотрим граф G , возможно c петлями и кратными рёбрами. Определим '''многочлен Татта''' (англ. ''Tutte polynomial'') T_G (x, y) следующими рекурсивными соотношениями:
# Если граф G не имеет рёбер, то T_G (x, y) = 1 ;
# Если ребро e является мостом, то T_G (x, y) = xT_{G\backslash e} (x, y) ;
# Если ребро e является петлей, то T_G (x, y) = yT_{G/e} (x, y) ;
# Если ребро e не является ни мостом, ни петлей то T_G (x, y) = T_{G\backslash e} (x, y) + T_{G/e} (x, y) ;
}}

Из этого определения не очевидна корректность: почему полученная функция не зависит от порядка выкидывания рёбер? Однако, если определение корректно, T_G , очевидно, является многочленом от двух переменных с целыми неотрицательными коэффициентами. Корректность мы докажем, связав многочлен Татта с другим многочленом — ранговым многочленом Уитни (''Whitney rank polynomial'').

==Корректность определения, связь с ранговым многочленом==
{{Определение|definition=
Пусть G = (V,E) — некоторый граф. Для множества A \subset E через G(A) будем обозначать граф (V, A) . Через c(G) будем обозначать '''число компонент связности''' графа G . '''Рангом''' множества A будем называть число \rho(A) = |V| - c(G(A)) .
}}

{{Утверждение
|statement=
Ранг множества A равен количеству рёбер в любом остовном лесе графа G(A) . (под остовным лесом здесь понимается объединение остовных деревьев всех компонент связности, т.е. такой ациклический граф G(B) , что B \subset A и c(G(B)) = c(G(A)) )
|proof=
Действительно, в каждой компоненте связности остовного леса рёбер на одно меньше чем вершин, а общее число вершин равно |V| .
}}

Теперь определим сам ранговый многочлен:

{{Определение|definition=
'''Ранговый многочлен''' (англ. ''Rank polynomial'') графа G есть многочлен от двух переменных, определяемый формулой: 
 R_G(u, v) = \sum\limits_{A \subset E} u^{\rho (E) - \rho (A)}v^{|A| - \rho (A)} 
}}

Показатели в формуле раногового многочлена тоже имеют некоторый смысл. Величина \rho (E) - \rho (A) равна c(G(A)) - c(G) , т.е. приросту числа компонент связности за счёт перехода к множеству рёбер A . Мы будем обозначать эту величину через \rho ^{*}(A) и называть числом ''важных'' для A рёбер. (Их важно добавить к A , чтобы получилось столько же компонент связности, сколько было изначально). 
Величину |A| - \rho (A) будем называть числом ''лишних'' ребёр: именно столько рёбер можно выкинуть из множества A , не меняя число компонент связности. Обозначать эту величину будем через \overline{\rho} (A).

Далее докажем следующую техническую лемму:

{{Лемма
|statement=
Пусть фиксировано некоторое ребро e \in E и множество A \subset E\backslash {e}. Обозначим через \rho _1(A), \rho ^{*}_{1} (A), \overline {\rho _1}(A) ранги множества A в графе G/e , а через \rho _2(A), \rho ^{*}_{2}(A), \overline {\rho _2}(A) — ранги в графе G\backslash e . Тогда для множества A' = A\cup {e} выполняются следующие соотношения:
# Если e не петля, то \rho ^{*}(A') = \rho ^{*}_{1} (A) и \overline{\rho} (A') = \overline {\rho _{1}} (A) ;
# Если e не мост, то \rho ^{*}(A') = \rho ^{*}_{2} (A) и \overline{\rho} (A') = \overline {\rho _{2}} (A) ;
# Если e мост, то \rho ^{*}(A') = \rho ^{*} (A) - 1 и \overline{\rho} (A') = \overline {\rho} (A) ;
# Если e петля, то \rho ^{*}(A') = \rho ^{*} (A) и \overline{\rho} (A') = 1 + \overline {\rho} (A) .

|proof=
# Стягивание ребра e в любом случае не меняет числа компонент связности, поэтому \rho ^{*}(A') = \rho ^{*}_{1} (A) . Если e не петля, то стягивание также не меняет числа лишних рёбер, откуда \overline{\rho} (A') = \overline {\rho _{1}} (A) .
# Если e не мост, то удаление ребра e не меняет числа компонент связности, откуда \rho (A) = \rho _2(A) и \rho (E) = \rho _2 (E \backslash {e}) . Подставляя эти равенства в формулы для \rho ^{*} и \overline {\rho} , получаем \rho ^{*}(A') = \rho ^{*}_{2} (A) и \overline{\rho} (A') = \overline {\rho _{2}} (A) , что и требовалось.
# Если e мост, то в графе G(A') на одну компоненту связности меньше, чем в G(A) , откуда \rho ^{*}(A') = \rho ^{*} (A) - 1 . При этом ребро e не будет лишним A' , поэтому \overline{\rho} (A') = \overline {\rho} (A) .
# Если e петля, то её исключение не меняет числа компонент связности, поэтому \rho ^{*}(A') = \rho ^{*} (A) . По той же причине e является лишним, откуда \overline{\rho} (A') = 1 + \overline {\rho} (A) .
}}

Теперь, собственно, докажем связь многочлена Татта с ранговым, откуда будет следовать корректность определения для многочлена Татта:

{{Теорема
|about=
Татта
|statement=
Для любого графа G выполнено равенство T_G(u + 1, v + 1) = R_G(u, v)

|proof=
Если граф G пуст, то единственным подмножеством E является пустое множество, для которого нет важных и лишних рёбер. Поэтому \rho^*(\emptyset ) = \overline {\rho} (\emptyset) = 0 и R_G(u, v) = 1 = T_G(u + 1, v + 1) .
Пусть граф G не пуст. Докажем, что для рангового многочлена выполняются соотношения Татта (из определения многочлена Татта). Выберем некоторое ребро e \in E и разобьём все подмножества E на пары вида (A, A') , где e \not\in A и A' = A \cup {e} . Тогда 
R_G(u, v) = \sum\limits_{A \subset {E \backslash {e}}} ( u^{\rho^* (A)}v^{\overline {\rho}(A)} + u^{\rho^* (A')}v^{\overline {\rho} (A')} )

Далее, разберём несколько случаев:

# Пусть e петля. Тогда \rho ^{*}(A') = \rho ^{*} (A) и \overline{\rho} (A') = 1 + \overline {\rho} (A) . Тогда u^{\rho^* (A')}v^{\overline {\rho} (A')} = u^{\rho^* (A)}v^{1 + \overline {\rho} (A)} = vu^{\rho^* (A)}v^{\overline {\rho} (A)} , откуда u^{\rho^* (A)}v^{\overline {\rho}(A)} + u^{\rho^* (A')}v^{\overline {\rho} (A')} = (v + 1)u^{\rho^* (A)}v^{\overline {\rho} (A)} . Вынося (v + 1) за скобки, получаем R_G(u, v) = (v + 1)\sum\limits_{A \subset {E \backslash {e}}} u^{\rho^* (A)}v^{\overline {\rho}(A)} = (v + 1) R_{G \backslash e}(u, v). Это соответствует первому соотношению Татта.
# Пусть e мост. Тогда \rho ^{*}(A) = \rho ^{*} (A') + 1 = \rho ^{*}_{1} (A') и \overline{\rho} (A) = \overline {\rho} (A') = \overline {\rho _1} (A) . Отсюда u^{\rho^* (A)}v^{\overline {\rho}(A)} + u^{\rho^* (A')}v^{\overline {\rho}(A')} =
u^{\rho^{*}_{1} (A) + 1}v^{\overline {\rho _1}(A)} + u^{\rho ^{*}_{1} (A)}v^{\overline{\rho _{1}}(A)} =
(u + 1)R_{G \backslash e}(u, v) . Это второе соотношение Татта.
# Наконец, пусть e не мост и не петля. Тогда u^{\rho^* (A)}v^{\overline {\rho}(A)} + u^{\rho^* (A')}v^{\overline {\rho}(A')} = u^{\rho ^{*}_{2} (A)}v^{\overline {\rho _2}(A)} + u^{\rho ^{*}_{1} (A)}v^{\overline {\rho _1}(A)} , откуда 
R_{G}(u, v) = \sum\limits_{A \subset {E \backslash {e}}} u^{\rho ^{*}_{2} (A)}v^{\overline {\rho _2}(A)} + \sum\limits_{A \subset {E \backslash {e}}} u^{\rho ^{*}_{1} (A)}v^{\overline {\rho _1}(A)} = R_{G \backslash e}(u, v) + R_{G / e}(u, v) . Это третье соотношение Татта.
Таким образом, многочлен R_{G}(u + 1, v + 1) удовлетворяет определению многочлена Татта, что и требовалось.

}}

==Многочлен Татта дерева==
Пусть G — дерево c n вершинами. Тогда T_G(x, y) = x^{n - 1} . Этот факт можно легко показать по индукции: в дереве любое ребро является мостом, после стягивания которого получается опять дерево с n - 1 вершинами.

==Многочлен Татта цикла==
Пусть G = Z_n — цикл из n вершин. Тогда для произвольного ребра e , граф G \backslash e — цепочка L_n из n , а G/e = Z_n . По свойству 4, T_{Z_n}(x, y) = T_{L_n}(x, y) + T_{Z_{n - 1} }(x, y) = x^{n - 1} + T_{Z_{n - 1}}(x, y) — верно для всех n > 1 . При этом граф Z_1 — петля, так что T_{Z_1} = y по свойствам 1 и 3. Следовательно, T_{Z_{n}}(x, y) = y + x + ... + x^{n - 1}

==Многочлен Татта полного графа==

{{Определение
|definition=
Пусть G = K_{n + 1} = (V, E) , причём V = \{0, 1, 2,...,n\} . Определим лексикографический порядок \prec на множестве рёбер E следующим образом: (i, j) \prec (i', j') , если i или i = i', j = j' .
}}

{{Определение
|definition=
Обозначим за S_n множество остовных деревьев T графа G . Будем говорить, что ребро p \in T '''внутренне активно''' (англ. ''internally active'') в T , если p \prec q для всех q \in E \backslash t , таких что T \backslash p \cup {q} \in S_n. Аналогичным образом, будем говорить, что ребро p \in T '''внешне активно''' (англ. ''externally active'') в T , если p \prec q для всех q \in E \backslash T , таких что T \backslash q \cup {p} \in S_n. Величиной внутренней (внешней) активности будем называть число внутренне (внешне) активных элементов в T ; эти величины будем обозначать i(T) и e(T) соответственно. 
}}

Также приведём без доказательства теорему, которая связывает многочлен Татта и понятие [[Остовные деревья: определения, лемма о безопасном ребре|остовного дерева]]:

{{Теорема
|about=
Татта
|statement=
Пусть на G с множеством остовных деревьев S . Тогда 
T_G(x, y) = \sum\limits_{T \in S} x^{i(T)}y^{e(T)}

}}

'''Обозначение:''' Для простоты обозначим многочлен Татта для полного графа G_{K_{n + 1}}(x, y) за F_n(x, y) . Тогда имеет место следующая теорема:
{{Теорема
|about=
Многочлен Татта полного графа
|statement=

F_{n}(x, y) = \sum \limits_{k = 1}^n {n - 1 \choose k - 1} (x + y + y^2 + ... + y^{k - 1}) F_{k - 1}(1, y)F_{n - k} (x, y)

|proof=
Зафиксируем остовное дерево T \in S_n . Рассмотрим ребро (0, k) \in T , которое разбивает T на поддеревья T' и T'' , и при этом вершина 0 лежит в T'' . Пусть a = |\{j|j \in T \& j . Тогда докажем следующие два утверждения:
# i(T) = i(T') + \delta _{a, 0} , где \delta _{a, 0} — символ Кронекера
# e(T) = e(T') + e(T'') + a 
Понятно, что ребро (j_1, j_2) \in T не может быть ''внутренне'' активным, так как (0, j_1) \prec (j_1, j_2) , (0, j_2) \prec (j_1, j_2) и T \backslash (j_1, j_2) \cup {(0, j_1)} \in S_n, T \backslash (j_1, j_2) \cup {(0, j_2)} \in S_n. Также ребро (0, k) \in T внутренне активно в T \Leftrightarrow a = 0 , потому как если существует такая вершина j \in T'' , такая что j , то (0, j) \prec (0, k) и T \backslash (0, k) \cup {(0, j)} \in S_n. Таким образом равенство (1) доказано. 
Рассмотрим (j_1, j_2) , где j_1 \in T' , j > 0 и j_2 \in T'' . Тогда (j_1, j_2) не может быть ''внешне'' активным, так как (0, k) \prec (j_1, j_2) и T \backslash (j_1, j_2) \cup {(0, k)} \in S_n . Аналогично, пусть j \in T'' , тогда ребро (0, j) — внешне активно \Leftrightarrow j . Таким образом мы доказали и равенство (2). Теперь необходимое тождество для полинома Татта полного графа может быть получено при подстановке равенств (1) и (2) в 
F_n(x, y) = \sum\limits_{T \in S} x^{i(T)}y^{e(T)}
 и суммировании по всем парам поддеревьев T', T'' и всем рёбрам типа (0, k) .
}}

==Универсальное свойство многочлена Татта==
{{Теорема
|statement=
Пусть числовая функция на графах f(G) обладает следующими свойствами для некоторых констант a, b, x_0, y_0 :
# Если в G нет рёбер, то f(G) = 1 
# Если ребро e является мостом, то f(G) = x_0f(G/e)
# Если ребро e является петлёй, то f(G) = y_0f(G \backslash e)
# Если ребро e не является ни мостом, ни петлёй, то f(G) = af(G/e) + bf(G \backslash e) .
Тогда f(G) = a^{\rho (E)}b^{|E| - \rho (E)}T_G(\frac {x_0}{a}, \frac {y_0}{b}) .

|proof=
Для доказательства проведём индукцию по количеству рёбер. Поскольку для пустого графа |E| = \rho(E) = 0 , а T_G = 1 , то база индукции верна. Докажем переход. 

Пусть e является мостом. Тогда \rho _1 (E \backslash {e}) = \rho (E) - 1 , так как стягивание e не меняет число компонент связности и уменьшает число вершин на одну. Тогда f(G) = x_0f(G/e) = x_0 a^{\rho _1 (E \backslash {e})} b^{|E| - 1 - \rho _1 (E \backslash {e})} T_{G/e}(\frac {x_0}{a}, \frac {y_0}{b}) = x_0 a^{\rho (E) - 1} b^{|E| - \rho (E)} T_{G/e}(\frac {x_0}{a}, \frac {y_0}{b}) = \frac {x_0}{a} a^{\rho (E)} b^{|E| - \rho (E)} T_{G/e}(\frac {x_0}{a}, \frac {y_0}{b}) = a^{\rho (E)}b^{|E| - \rho (E)}T_G(\frac {x_0}{a}, \frac {y_0}{b}) . 

Пусть e является петлёй. Тогда \rho _2 (E \backslash {e}) = \rho (E) , так как удаление e не меняет ни числа вершин, ни числа компонент связности. Тогда f(G) = y_0f(G \backslash e) = y_0 a^{\rho _2 (E \backslash {e})} b^{|E| - 1 - \rho _2 (E \backslash {e})} T_ {G \backslash e} (\frac {x_0}{a}, \frac {y_0}{b}) = y_0 a^{\rho _2 (E \backslash {e})} b^{|E| - 1 - \rho _2 (E \backslash {e})} T_ {G \backslash e} (\frac {x_0}{a}, \frac {y_0}{b}) = \frac {y_0}{b} a^{\rho (E)} b^{|E| - \rho (E)} T_ {G \backslash e} (\frac {x_0}{a}, \frac {y_0}{b}) = a^{\rho (E)}b^{|E| - \rho (E)}T_G(\frac {x_0}{a}, \frac {y_0}{b}) .

Пусть e не является ни мостом, ни петлёй. Тогда \rho _1 (E \backslash {e}) = \rho (E) - 1 и \rho _2 (E \backslash {e}) = \rho (E) . Тогда f(G) = a f(G/e) + b f(G \backslash e) = a\cdot a^{\rho _1 (E \backslash {e})} b^{|E| - 1 - \rho _1 (E \backslash {e})} T_ {G / e} (\frac {x_0}{a}, \frac {y_0}{b}) + b\cdot a^{\rho _2 (E \backslash {e})} b^{|E| - 1 - \rho _2 (E \backslash {e})} T_ {G \backslash e} (\frac {x_0}{a}, \frac {y_0}{b}) = a^{\rho (E)}b^{|E| - \rho (E)} (T_ {G / e} (\frac {x_0}{a}, \frac {y_0}{b}) + T_ {G \backslash e} (\frac {x_0}{a}, \frac {y_0}{b})) = a^{\rho (E)}b^{|E| - \rho (E)}T_G(\frac {x_0}{a}, \frac {y_0}{b}) . 
Таким образом, все случаи разобраны, и теорема доказана.

}}

==Связь с хроматическим многочленом==

{{Теорема
|statement=
Для графа G и k \in N выполняется соотношение \chi _G (k) = (-1)^{|V| - c(G)}k^{c(G)}T_G(1 - k, 0) .
|proof=
Воспользуемся универсальным свойством многочлена Татта для функции P_G(k) = \frac {\chi _G (k)}{k^{|V|}} . Проверим условие теоремы. 
Пусть ребро e является мостом. Тогда множество вершин V разбивается на два непересекающихся подмножества: V_1 и V_2 . Обозначим через G_1 и G_2 соответствующие подграфы. Их раскраски не связаны друг другом, поэтому \chi_{G \backslash e} (k) = \chi_{G_1} (k) \cdot \chi_{G_2} (k) . Далее, правильная раскраска G/e получается из правильных раскрасок G_1 и G_2 , где цвета склеиваемых вершин совпадают. Можно взять любую правильную раскраску G_1 , для чего есть \chi_{G_1} (k) , а из правильных раскрасок G_2 годится только доля \frac {1}{k} , где цвет склеиваемой вершины нужный. Таким образом, \chi _{G/e}(k) = \frac {1}{k} \chi _{G_1}(k) \chi _{G_2}(k) . Далее, по рекуррентному свойству [[Хроматический многочлен|хроматического многочлена]] \chi _{G}(k) = \chi _{G \backslash e}(k) - \chi _{G / e}(k) = (1 - \frac {1}{k})\chi _{G_1}(k) \cdot \chi _{G_2}(k) = (k - 1)\chi _{G / e}(k) . Значит, P_G (k) = \frac {\chi _{G}(k)}{k^{|V|}} = \frac {(k - 1)\chi _{G / e}(k)}{k^{|V|}} = \frac {k - 1}{k} P_{G / e} (k) , то есть первое условие выполнено для x_0 = \frac {k - 1}{k} . 
Пусть ребро e является петлёй. Тогда правильных раскрасок нет, то есть P_G (k) = 0 . Значит второе условие выполнено для y_0 = 0 .
Пусть ребро e не является ни мостом, ни петлёй. Опять же, в силу рекуррентного свойства хроматического многочлена \chi _{G}(k) = \chi _{G \backslash e}(k) + \chi _{G / e}(k) . Поделив на k^{|V|} , получим P_G(k) = -\frac {1}{k} P_{G / e} (k) + P_{G \backslash e} (k) . Значит, третье соотношение выполнено для a = \frac {1}{k}, b = 1 . 
Согласно универсальному свойству многочлена Татта получаем P_G (k) = (-\frac {1}{k})^{\rho (E)} T_G(1 - k, 0) . Значит, \chi _G (k) = (-1)^{\rho (E)}k^{|V| - \rho (E)}T_G(1 - k, 0) . Поскольку \rho (E) = |V| - c(G) , получаем \chi _G (k) = (-1)^{|V| - c(G)}k^{c(G)}T_G(1 - k, 0) .
}}

==Значения многочлена Татта==
{{Теорема
|statement=
Для любого графа G верно, что:
# T_G (1, 1) равно количеству остовных лесов;
# T_G (1, 2) равно количеству подграфов G , имеющих столько же компонент связности, что и G ;
# T_G (2, 1) равно количеству ациклических подграфов G .
|proof=
Заметим, что \overline {\rho} (A) = 0 тогда и только тогда, когда G(A) не содержит циклов, а \rho ^*(A) = 0 тогда и только тогда, когда G(A) имеет столько же компонент связности, что и G . 
Далее, воспользуемся теоремой о связи с ранговым многочленом:
# T_G(1, 1) = R_G(0, 0) . Учитывая, что 0^0 = 1 и 0^k = 0 при k > 0 , ненулевыми (а именно единичными) будут только те слагаемые, где \rho^*(A) = 0 и \overline {\rho} (A) = 0 . Это означает, что G(A) не содержит циклов и содержит столько же компонент связности, сколько и G , то есть является остовным лесом. Суммируя единицы для каждого остовного леса, получаем число остовных лесов.
# T_G(1, 2) = R_G(0, 1) . Здесь мы суммируем единицы для тех A , где \rho^*(A) = 0 , то есть для подграфов имеющих столько же компонент связности, сколько и G .
# T_G(2, 1) = R_G(1, 0) . Здесь мы суммируем единицы для тех A , где \overline {\rho}(A) = 0 , то есть для ациклических подграфов.

}}

==Источники информации==
* [http://logic.pdmi.ras.ru/~dvk/211/graphs_dk.pdf Карпов Д.В. — Теория Графов]
* [http://www.mathnet.ru/links/f39ceb5fdeb743a27ea1e43da2218762/mp215.pdf Бурман Ю.М. — Многочлен Татта и модель случайных кластеров]
* [http://www.math.ucla.edu/~pak/papers/Pak_Computation_Tutte_polynomial_complete_graphs.pdf Igor M. Pak — Computations of Tutte polynomial of complete graphs]

[[Категория: Алгоритмы и структуры данных]]
[[Категория: Раскраски графов]]