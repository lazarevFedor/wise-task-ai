{{Определение
|definition=
'''Точка сочленения''' [[Основные определения: граф, ребро, вершина, степень, петля, путь, цикл|графа]] G {{---}} вершина, принадлежащая как минимум двум [[Отношение вершинной двусвязности#Блоки|блокам]] G. (1)
}}
{{Определение
|definition=
'''Точка сочленения''' графа G {{---}} вершина, при удалении которой в G увеличивается число [[Отношение связности, компоненты связности|компонент связности]]. (2)
}}
[[Файл:Cut_vertex_1.png|thumb|left|335px|Вершины a_1, a_2, a_3 - точки сочленения графа G.]]

{{Лемма
|statement=
Определения (1) и (2) эквивалентны.

|proof=
1 \Rightarrow 2 

Пусть вершина v принадлежит некоторым блокам A и B. Вершине v инцидентны некоторые ребра e=uv \in A и f=wv \in B. Ребра e и f находятся в различных блоках, поэтому не существует двух непересекающихся путей между их концами. Учитывая, что один из путей между концами - путь из v в эту же вершину, получаем, что любой путь, соединяющий u и w, пройдет через v. При удалении v между u и w не останется путей, и одна из компонент связности распадется на две.

2 \Rightarrow 1 

Пусть v принадлежала только одному блоку C. Все вершины u_1...u_n, смежные с v, также лежат в C (в силу рефлексивности отношения вершинной двусвязности). Между каждой парой u_i, u_j вершин из u_1...u_n существует как минимум два вершинно непересекающихся пути. Теперь удалим v. Это разрушит путь u_{i}vu_{j}, но не разрушит любой оставшийся, так как иначе он тоже прошел бы через v.
Рассмотрим D {{---}} компоненту связности, в которой лежала v. Пусть между вершинами u, w \in D существовал путь, проходящий через v. Но он проходил также через некоторые вершины из u_1...u_n, связность которых не нарушилась, поэтому есть как минимум еще один путь, отличный от удаленного. Противоречие: число компонент связности не увеличилось.
}}

{{Теорема
|statement=
Следующие утверждения эквивалентны:
# v {{---}} точка сочленения графа G;
# существуют такие вершины u и w, отличные от v, что v принадлежит любому простому пути из u в w;
# существует разбиение множества вершин V \setminus \{v\} на такие два подмножества U и W, что для любых вершин u \in U и w \in W вершина v принадлежит любому простому пути из u в w.

|proof=
1 \Rightarrow 3 Так как v {{---}} точка сочленения графа G, то граф G \setminus v не связен и имеет по крайней мере две компоненты. Образуем разбиение V \setminus \{v\}, отнеся к U вершины одной из этих компонент, а к W {{---}} вершины всех остальных компонент. Тогда любые две вершины u \in U и w \in W лежат в разных компонентах графа G \setminus v. Следовательно, любой простой путь из u в w графа G содержит v.

3 \Rightarrow 2 Следует из того, что (2) - частный случай (3).

2 \Rightarrow 1 Если v принадлежит любому простому пути в G, соединяющему u и w, то в G нет простого пути, соединяющего эти вершины в G \setminus \{v\}. Поскольку G \setminus \{v\} не связен, то v {{---}} точка сочленения графа G.
}}

{{Теорема
|statement=
Пусть G {{---}} связный граф с не менее чем тремя вершинами. Следующие утверждения эквивалентны:
# G {{---}} блок ;
# любые две вершины графа G принадлежат некоторому общему простому циклу;
# любая вершина и любое ребро графа G принадлежат некоторому общему простому циклу;
# любые два ребра графа G принадлежат некоторому общему простому циклу;
# для любых двух вершин и любого ребра графа G существует простая цепь, соединяющая эти вершины и включающая данное ребро;
# для любых трех различных вершин графа G существует простая цепь, соединяющая две из них и проходящая через третью;
# для каждых трех различных вершин графа G существует простая цепь, соединяющая две из них и не проходящая через третью.

|proof=
1 \Rightarrow 2 Пусть u,v - различные вершины графа G, а U - множество вершин, отличных от u, которые лежат на простом цикле, содержащем u. Поскольку в G по крайней мере три вершины и нет точек сочленения, то в G нет также мостов. Значит, каждая вершина, смежная с u, принадлежит U, т.е. U не пусто. Предположим, что u не принадлежит U. Пусть w - вершина в U, для которой расстояние d(w-u)-цепь минимально. Пусть P_0 - кратчайшая простая (w-u)- цепь, а P_1 и P_2 - две простые (u-w)-цепи цикла, содержащего u и w. Так как w не является точкой сочленения, то существует простая (u-v)-цепь P', не содержащая w. Обозначим через w' ближайшую к u вершину, принадлежащую P', которая также принадлежит P_0, и через u' последнюю вершину (u-w')-подцепи в P', которая принадлежит или P_1, или P_2. Не теряя общности, предположим, что u' принадлежит  P_1.
Пусть Q_1 - простая (u-w')-цепь, содержащая (u-u')-подцепь цепи P_1 и (u'-w')-подцепь цепи P', а Q_2 - простая (u-w')-подцепь, содержащая P_2 вслед за (w-w')-подцепью цепи P_0. Ясно, что Q_1 и Q_2 - непересекающиеся простые (u-w')-цепи. Вместе они образуют простой цикл, так что w' принадлежит U. Поскольку w' принадлежит кратчайшей цепи, d(w',u)d(w,u). Это противоречит выбору w и, следовательно, доказывает, что u и v лежат на одном простом цикле.

3 \Rightarrow 2 Следует из того, что (2) - частный случай (3).

2 \Rightarrow 1 Если v принадлежит любому простому пути в G, соединяющему u и w, то в G нет простого пути, соединяющего эти вершины в G \setminus \{v\}. Поскольку G \setminus \{v\} не связен, то v {{---}} точка сочленения графа G.
}}

==Источники информации==
* Харари, Ф. Теория графов. — М.: Книжный дом «ЛИБРОКОМ», 2009

[[Категория:Алгоритмы и структуры данных]]
[[Категория:Связность в графах]]