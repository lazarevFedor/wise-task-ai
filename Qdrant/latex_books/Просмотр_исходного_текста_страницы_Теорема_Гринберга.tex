== Базовые определения ==
{{Определение
|definition=
'''Подграф''' (англ. ''subgraph'') исходного графа {{---}} граф, содержащий некое подмножество вершин данного графа и некое подмножество инцидентных им рёбер. По отношению к подграфу исходный граф называется суперграфом.
}}

{{Определение
|definition=
'''Бонд''' (англ. ''bond'') графа {{---}} это минимальный (по включению) непустой [[Разрез,_лемма_о_потоке_через_разрез | разрез графа]] G. 
}}

{{Определение
|definition=
'''Минимальный (по включению)''' (англ. ''minimal by inclusion'') разрез графа G {{---}} разрез, из которого нельзя выделить разрезы с меньшим количеством ребер. 
}}

{{Лемма
|statement=
Разрез E(V_1, V_2) связного графа G является '''бондом''', если и только если оба графа G(V_1) и G(V_2) связны.
|proof=
Для удобства примем E = E(V_1, V_2). 

\Rightarrow. Пусть E {{---}} бонд. Докажем, что для любого ребра e \in E граф G - E + e связен. Действительно, пусть этот граф несвязен и имеет, скажем, компоненты связности U_1 и U_2. Тогда E \supsetneq E(U_1, U_2), а из связности графа G следует, что E(U_1, U_2) \neq \varnothing. Противоречие с минимальностью E. 

Теперь докажем, что подграфы G(V_1) \text{ и } G(V_2) связны. Рассмотрим отдельно подграф G(V_1), если он не связный, то имеет как минимум 2 компоненты связности, назовем их O_1 \text{ и } O_2. 

e \in E можно также представить как e = (u, v) \text{ при этом } u \in G(V_1), v \in G(V_2), то есть u \in O_1 \mid u \in O_2, и граф G - E + e состоит из 2 компонент {{---}} (O_1 \cup G(V_2), O_2) \mid (O_2 \cup G(V_2), O_1), что противоречит условию связности. Так же доказывается связность G(V_2).

\Leftarrow. Если оба графа G(V_1) и G(V_2) — связны, то добавление любого ребра из E даст нам связный подграф графа G, содержащий все его вершины. Значит, в этом случае разрез E минимален по включению. В силу связности G этот разрез непуст, то есть, является бондом.
}}

{{Определение
|definition=
Подграфы V_1 и V_2 из предыдущей леммы называются '''торцевыми графами''' (англ. ''end graph'').
}}
Также стоит отметить, что если граф G несвязен, то его '''бонд''' определим как бонд какой-либо его компоненты, а всякий [[Мост,_эквивалентные_определения | мост]] графа образует однореберный бонд. Торцевые графы моста являются торцевыми графами соответствующего бонда.

{{Определение
|definition=
'''Гамильтоновым бондом''' (англ. ''hamiltonian bond'') называется бонд графа G , торцевыми графами которого являются деревья.
}}

== Теорема Гринберга ==
{{Теорема
|author=Гринберг
|statement=
Пусть связный граф G имеет гамильтонов бонд H с торцевыми графами X и Y . Пусть f_n^{X} и f_n^{Y} {{---}} число вершин в графов X и Y соответственно, имеющих в G степень n ~ (n = 1, ~ 2, ~ 3, ~ \ldots) . Тогда:
 \sum\limits_{n=1}^{\infty} (n - 2) (f_n^{X} - f_n^{Y}) = 0 ~~~ \bf{(1)} . 
|proof=
Так как торцевые графы являются деревьями, то количество их вершин на единицу больше количества ребер:
 \sum\limits_{n=1}^{\infty} f_n^{X} = |V(X)| = |E(X)| + 1 ~~~ \textbf{(2)} . 
Посчитаем \sum\limits_{n=1}^{\infty} n f_n^{X} , то есть количество всех исходящих ребер из X. По [[Лемма_о_рукопожатиях | лемме о рукопожатиях]] ребер, с обоих сторон прикрепленных к X, будет 2|E(X)|. Количество ребер, прикрепленных и к X, и к Y, по определению бонда {{---}} количество ребер в бонде H, то есть |E(H)|. Отсюда:
 \sum\limits_{n=1}^{\infty} n f_n^{X} = |E(H)| + 2|E(X)| ~~~ \textbf{(3)} . 
Вычитаем дважды из формулы \textbf{(3)} формулу \textbf{(2)} и получаем:
 \sum\limits_{n=1}^{\infty} (n - 2) f_n^{X} = |E(H)| - 2 ~~~ \textbf{(4)} . 
Полученная формула в правой части не зависит от подграфа, поэтому вычитая вариант для Y из \textbf{(4)}, приходим к \textbf{(1)}.
}}

== Использование теоремы ==

* Сам Гринберг использовал свою теорему для того, чтобы искать негамильтоновы кубические (все вершины имеют степень 3) полиэдральные графы[https://ru.wikipedia.org/wiki/%D0%9F%D0%BE%D0%BB%D0%B8%D1%8D%D0%B4%D1%80%D0%B0%D0%BB%D1%8C%D0%BD%D1%8B%D0%B9_%D0%B3%D1%80%D0%B0%D1%84 Википедия {{---}} Полиэдральный граф] с высокой циклической реберной связностью. Циклическая рёберная связность графа {{---}} это наименьшее число рёбер, которое можно удалить так, чтобы оставшийся граф содержал более чем одну циклическую компоненту. Например он нашел граф с 46 вершинами, 25 гранями и циклической рёберной связностью пять, показанный на рисунке 1.

{|align="center"
 |[[Файл: Гамильтонов граф.png|300px|center|thumb|Рис. 1]]
 |[[Файл: Новый гамильтонов_бонд.png|500x300px|thumb|Рис. 2]]
 |}

* Теорему Гринберга можно иногда использовать для доказательства отсутствия гамильтонова бонда в графе. Пусть, например, все вершины связного графа G , кроме одной, имеют степени, сравнимые с 2 по модулю 3. Тогда левая часть формулы \textbf{(1)} не делится на 3 и, следовательно, гамильтонова бонда в графе G не существует. Рисунок 2 иллюстрирует этот простой пример.
* Чтобы планарный граф существовал и содержал гамильтонов цикл, необходимо выполнение теоремы Гринберга.Grinberg, È. Ja. (1968), "Plane homogeneous graphs of degree three without Hamiltonian circuits", Latvian Math. Yearbook 4 (in Russian), Riga: Izdat. “Zinatne”, pp. 51–58, MR 0238732. English translation by Dainis Zeps, [https://arxiv.org/abs/0908.2563v1 arXiv:0908.2563.]
* Теорема Гринберга используется также для поиска планарных гипогамильтоновых графов[https://ru.wikipedia.org/wiki/%D0%93%D0%B8%D0%BF%D0%BE%D0%B3%D0%B0%D0%BC%D0%B8%D0%BB%D1%8C%D1%82%D0%BE%D0%BD%D0%BE%D0%B2_%D0%B3%D1%80%D0%B0%D1%84 Википедия {{---}} Гипогамильтонов граф] путём построения графа, в котором все грани имеют число рёбер, сравнимых с 2 по модулю 3.

== См. также ==
* [[Гамильтоновы графы]]
* [[Разрез, лемма о потоке через разрез]]
* [[Лемма о рукопожатиях]]
* [[Дерево, эквивалентные определения]]

== Примечания ==

== Источники информации ==
* У. Татт. Теория графов. М.: "Мир", 1988. с. 304. ISBN 5-03-001001-7
* [https://logic.pdmi.ras.ru/~dvk/graphs_dk.pdf Д.В. Карпов. Теория графов. c. 301]

[[Категория:Алгоритмы и структуры данных]]
[[Категория:Обходы графов]]
[[Категория:Гамильтоновы графы]]