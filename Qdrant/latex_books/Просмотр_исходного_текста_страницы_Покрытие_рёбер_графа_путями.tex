Следующее утверждение являются следствием из [[Эйлеров_цикл,_Эйлеров_путь,_Эйлеровы_графы,_Эйлеровость_орграфов|критерия Эйлеровости]] [[Основные определения теории графов|графа]]:
{{Теорема|statement=
Пусть G {{---}} связный граф, в котором 2N вершин имеют нечётную [[Основные определения теории графов|степень]]. Тогда множество рёбер G можно покрыть N [[Основные определения теории графов|рёберно-простыми]] путями.
|proof=

Рассмотрим связный граф G, который содержит 2N вершин, имеющих нечётную степень. Докажем, что его можно покрыть N рёберно-простыми путями. 

Добавим в граф N рёбер, соединив попарно вершины, имеющие нечётные степени, и получим связный граф G', все вершины которого имеют чётную степень. Такой граф удовлетворяет [[Эйлеровость_графов#.D0.9A.D1.80.D0.B8.D1.82.D0.B5.D1.80.D0.B8.D0.B9_.D1.8D.D0.B9.D0.BB.D0.B5.D1.80.D0.BE.D0.B2.D0.BE.D1.81.D1.82.D0.B8|критерию эйлеровости]] и содержит эйлеров цикл. Рассмотрим этот цикл и удалим из него N добавленных ребер G' \backslash G. Цикл распадётся на N путей, которые являются простыми, так как рассматриваемый цикл эйлеров, и покрывают весь граф, поэтому полученное разбиение является искомым.
}}

==См. также==
* [[Эйлеров_цикл,_Эйлеров_путь,_Эйлеровы_графы,_Эйлеровость_орграфов|Эйлеровость графов]]

==Источники информации==
* Харари Фрэнк '''Теория графов''' = Graph theory/Пер. с англ. и предисл. В. П. Козырева. Под ред. Г.П.Гаврилова. Изд. 2-е. — М.: Едиториал УРСС, 2003. — 296 с. — ISBN 5-354-00301-6

[[Категория: Алгоритмы и структуры данных]]
[[Категория: Обходы графов]]
[[Категория: Эйлеровы графы]]