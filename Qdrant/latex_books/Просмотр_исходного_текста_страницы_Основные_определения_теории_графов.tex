==Ориентированные графы==

{{Определение
|id = oriented_grath
|definition =
'''Ориентированным графом''' (англ. ''directed graph'') G называется пара G = (V, E), где V {{---}} множество вершин (англ. ''vertices''), а E \subset V \times V {{---}} множество рёбер.
}}

{{Определение
|id = finite_graph
|definition =
'''Конечным графом''' (англ. ''finite graph'') G называется граф, в котором множества V и E {{---}} конечны. Следует заметить, что большинство рассматриваевых нами графов {{---}} конечны.
}}

{{Определение
|id = def_graph_edge_1
|definition =
'''Ребром''' (англ. ''edge'', дугой (англ. ''arc''), линией (англ. ''line'')) ориентированного графа называют упорядоченную пару вершин (v, u) \in E .
}}

{{Определение
|id = isomorphic_graphs
|definition=
'''Изоморфные графы''' (англ. ''isomorphic graphs'') {{---}} два графа A и B называются изоморфными, если можно установить биекцию между их вершинами и соответствующими им рёбрами.
}}

В графе ребро, концы которого совпадают, то есть e=(v, v), называется петлей (англ. ''loop'').

Два ребра, имеющие общую концевую вершину, то есть e_1=(v, u_1) и e_2=(v, u_2), называются '''смежными''' (англ. ''adjacent''). 

Если имеется ребро (v, u) \in E , то говорят:
* v {{---}} '''предок''' (англ. ''direct predecessor'') u .
* u и v {{---}} '''смежные'''.
* Вершина u '''инцидентна''' ребру (v, u) .
* Вершина v '''инцидентна''' ребру (v, u) .

'''Инцидентность''' (англ. ''incidence'') {{---}} понятие, используемое только в отношении ребра и вершины. Две вершины или два ребра не могут быть инцидентны.

Граф с p вершинами и q рёбрами называют (p, q) -графом. (1, 0) -граф называют тривиальным.

Заметим, что по определению ориентированного графа, данному выше, любые две вершины u,~v нельзя соединить более чем одним ребром (u, v).
Поэтому часто используют другое определение. 
{{Определение
|id = def1
|definition =
'''Ориентированным графом''' G называется четверка G = (V, E, \operatorname{beg}, \operatorname{end}) , где V и E {{---}} некоторые множества, а \operatorname{beg}, \operatorname{end} : E \rightarrow V. 
}} 
Данное определение разрешает соединять вершины более чем одним ребром. Такие рёбра называются '''кратными''' (иначе {{---}} '''параллельные''', англ. ''multi-edge'', ''parallel edge''). Граф с кратными рёбрами принято называть '''мультиграфом''' (англ. ''multigraph''). Если в мультиграфе присутствуют петли, то такой граф называют '''псевдографом''' (англ. ''pseudograph'').
{|border="0" cellpadding="5" width=30% align=center
|[[Файл: Graph_definition_1.png|thumb|210px|center|Красным выделено кратное ребро (6, 2)Синим обозначена петля (6, 6)]]
|[[Файл: Multi_graph.png|thumb|150px|center|Мультиграф]]
|[[Файл: Pseudo_graph.png|thumb|150px|center|Псевдограф]]
|
|}

{{Определение
|definition=
Для ориентированных графов определяют '''полустепень исхода вершины''' (англ. ''outdegree'') \operatorname{deg}^+v_i = |\{e \mid \operatorname{beg(e)} = v_i\}| и '''полустепень захода вершины''' (англ. ''indegree'') \operatorname{deg}^-v_i = |\{e \mid \operatorname{end(e)} = v_i\}|.
}}

Стоит отметить, что для ориентированного графа справедлива [[Лемма о рукопожатиях|лемма о рукопожатиях]], связывающая количество рёбер с суммой [[Основные определения теории графов#Степень вершины|степеней вершин]].

==Неориентированные графы==
{{Определение
|id = def_undirected_graph_1
|definition =
'''Неориентированным графом''' (англ. ''undirected graph'') G называется пара G = (V, E), где V {{---}} множество вершин, а E \subset \{\{v, u\}: v, u \in V\} {{---}} множество рёбер.
}}
{{Определение
|id=def_edge_und
|definition =
'''Ребром''' в неориентированном графе называют неупорядоченную пару вершин \{v, u\} \in E .
}}
[[Файл: Graph_definition_2.png|thumb|210px|center|Неориентированный граф]]
Иное определение:
{{Определение
|id = def_undirected_graph_2
|definition =
'''Неориентированным графом''' G называется тройка G = (V, E, \operatorname{ends}) , где V {{---}} множество вершин, E {{---}} множество рёбер, а \operatorname{ends} : E \to \{\{u, v\}, u, v \in V\}. Это определение, в отличие от предыдущего, позволяет задавать графы с кратными рёбрами.
}}

{{Определение
|id = def_simple_graph
|definition =
'''Простым графом''' G называется граф, в котором нет петель и кратных рёбер.
}}

{{Определение
|id = def_graph_degree_1
|definition =
'''Степенью''' (англ. ''degree'', ''valency'') вершины \operatorname{deg} v_i в неориентированном графе называют число рёбер, инцидентных v_i.
}}
Будем считать, что петли добавляют к степени вершины 2.

{{Определение
|id = isolated_vertex
|definition =
'''Изолированной вершиной''' (англ. ''isolated vertex'') в неориентированном графе называют вершину степени 0 
}}

Остальные определения в неориентированном графе совпадают с аналогичными определениями в ориентированном графе.

== Представление графов ==

=== Матрица и списки смежности ===

Граф можно представить в виде [[Матрица смежности графа|матрицы смежности]] (англ. ''adjacency matrix''), где graph[v][u] = true \Leftrightarrow (v, u) \in E. Также в ячейке матрицы можно хранить вес ребра или их количество (если в графе разрешены параллельные рёбра).
Для матрицы смежности существует [[Связь степени матрицы смежности и количества путей|теорема]], позволяющая связать степень матрицы и количество путей из вершины v в вершину u.

Если граф '''разрежен''' (англ. ''sparse graph''), |E| \ll |V^2|, то есть, неформально говоря, в нем не очень много рёбер. Формально говорить не получается, потому что везде разреженные графы определяются по-разному, его лучше представить в виде списков смежности, где список для вершины v будет содержать вершины u: (v, u) \in E. Данный способ позволит сэкономить память, так как не придется хранить много нулей.

=== Пути в графах ===
{{Определение
|id = path
|definition =
'''Путём''' (маршрутом,англ. ''path'') в графе называется последовательность вида v_0 e_1 v_1 ... e_k v_k, где e_i \in E,~e_i = (v_{i-1}, v_i), k {{---}} '''длина''' (англ. ''length'') пути.
}}

{{Определение
|definition=
'''Длина пути''' {{---}} количество [[Основные определения теории графов|рёбер]], входящих в последовательность, задающую этот путь.
}}

{{Определение
|definition =
'''Циклическим путём''' (англ. ''closed walk'') в ''ориентированном графе'' называется путь, в котором v_0 = v_k.
}}

{{Определение
|id = def_no_graph_path
|definition =
'''Циклическим путём''' в ''неориентированном графе'' называется путь, в котором v_0 = v_k, а также e_i \ne e_{i \bmod k + 1}.
}}

{{Определение
|id = def_graph_cycle_1
|definition =
'''Цикл''' (англ. ''integral cycle'') {{---}} это [[Отношение эквивалентности#Классы эквивалентности|класс эквивалентности]] циклических путей на отношении эквивалентности таком, что два пути эквивалентны, если \exists j \forall i : e_{(i \mod k)} = e'_{(i + j) \bmod k}; где e и e' {{---}} это две последовательности рёбер в циклическом пути.
}}

{{Определение
|definition=
'''Простой (вершинно-простой) путь''' (англ. ''simple path'') {{---}} путь, в котором каждая из вершин графа встречается не более одного раза.
}}
{{Определение
|definition=
'''Рёберно-простой путь''' {{---}} путь, в котором каждое из рёбер графа встречается не более одного раза.
}}

== Часто используемые графы ==
{{Определение
|id = defFullGraph
|definition=
'''Полный граф''' (англ. ''complete graph'', ''clique'') {{---}} граф, в котором каждая пара различных вершин смежна. Полный граф с n вершинами имеет n(n-1)/2 рёбер и обозначается K_n.
}}

{{Определение
|id = defBiparateGraph
|definition=
'''Двудольный граф''' или '''биграф''' (англ. ''bipartite graph'') {{---}} граф, множество вершин которого можно разбить на две части таким образом, что каждое ребро графа соединяет какую-то вершину из одной части с какой-то вершиной другой части, то есть не существует ребра, соединяющего две вершины из одной и той же части. Двудольный граф с n вершинами в одной доле и m во второй обозначается K_{n,m}.
}}

{{Определение
|id = defRegularGraph
|definition=
'''Регулярный граф''' (англ. ''regular graph'') {{---}} граф, степени всех вершин которого равны, то есть каждая вершина имеет одинаковое количество соседей. Регулярный граф с вершинами степени k называется k‑регулярным, или регулярным графом степени k.
}}

{{main|Дерево, эквивалентные определения}}
{{Определение
|id=defTree
|definition='''Дерево''' (англ. ''tree'') {{---}} связный ациклический граф.
}}

{{main|Эйлеров цикл, Эйлеров путь, Эйлеровы графы, Эйлеровость орграфов}}
{{Определение
|definition=
Граф называется '''эйлеровым''' (англ. ''eulerian graph''), если он содержит эйлеров цикл. 
}}

{{main|Гамильтоновы графы}}
{{Определение
|definition=
Граф называется '''гамильтоновым''' (англ. ''hamiltonian graph''), если он содержит гамильтонов цикл.
}}

{{main|Укладка графа на плоскости}}
{{Определение
|definition=
Граф называется '''планарным''' (англ. ''planar graph''), если он обладает укладкой на плоскости. '''Плоским''' (англ. ''plane graph'', ''planar embedding of the graph'') называется граф уже уложенный на плоскости.
}}

{{main|Лемма о безопасном ребре}}
{{Определение
|definition=
'''Остовное дерево''' (англ. ''spanning tree'') {{---}} ациклический связный подграф данного связного неориентированного графа, в который входят все его вершины.
}}

==См. также==
* [[Лемма о рукопожатиях]]
* [[Матрица смежности графа]]
* [[Связь степени матрицы смежности и количества путей]]

==Источники информации==
* [[wikipedia:ru:Граф_(математика) | Википедия {{---}} Граф]]
* [[wikipedia:Graph_(mathematics) | Wikipedia {{---}} Graph]]
* [http://mathworld.wolfram.com/Graph.html Wolfram Mathworld: Graph]
* Харари Фрэнк '''Теория графов''' = Graph theory/Пер. с англ. и предисл. В. П. Козырева. Под ред. Г.П.Гаврилова. Изд. 2-е. — М.: Едиториал УРСС, 2003. — 296 с. — ISBN 5-354-00301-6
* Асанов М. О., Баранский В. А., Расин В. В. '''Дискретная математика: графы, матроиды, алгоритмы''' — НИЦ РХД, 2001. — 288 с. — ISBN 5-93972-076-5
* ''Кормен, Томас Х., Лейзерсон, Чарльз И., Ривест, Рональд Л., Штайн Клиффорд'' '''Алгоритмы: построение и анализ''', 2-е издание. Пер. с англ. — М.:Издательский дом "Вильямс", 2010. — 1296 с.: ил. — Парал. тит. англ. — ISBN 978-5-8459-0857-5 (рус.)

[[Категория: Алгоритмы и структуры данных]]
[[Категория: Основные определения теории графов]]