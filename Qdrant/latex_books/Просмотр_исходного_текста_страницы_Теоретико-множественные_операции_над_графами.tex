__TOC__

Пусть [[Основные_определения_теории_графов|графы]] G_1 и G_2 имеют непересекающиеся множества вершин V_1 и V_2 и непересекающиеся множества ребер X_1 и X_2.
{{Определение
|id = obedinenie
|definition =
'''Объединением''' (англ. ''union'') G_1 \cup G_2 называется граф, множеством вершин которого является V=V_1 \cup V_2, а множество ребер X=X_1 \cup X_2.
}}
{{Определение
|id = soedinenie
|definition =
'''Соединением''' (англ. ''graph join'') G_1 + G_2 называется граф, который состоит из G_1 \cup G_2 и всех ребер, соединяющих V_1 и V_2.
}}
[[Файл:соединение.png|thumb|1100px|center|Соединение G_1 и G_2]]
{{Определение
|id = proizvedenie
|definition =
'''Произведением''' (англ. ''cartesian product'') G_1 \times G_2 называется граф с множеством вершин V равным декартовому произведению V_1 \times V_2. Множество ребер X определяется следующим образом:
* рассмотрим любые две вершины u=(u_1, u_2) и v=(v_1, v_2) из V=V_1 \times V_2,
* вершины u и v [[Основные_определения_теории_графов|смежны]] в G=G_1 + G_2 тогда и только тогда, когда (u_1 = v_1, а u_2 и v_2 — смежные) или (u_2 = v_2, а u_1 и v_1 — смежные).
}}
[[Файл:произведение.png|thumb|1100px|center|Произведение G_1 и G_2]]
{{Определение
|id = compozicia
|definition =
'''Композицией''' (англ. ''lexicographical product'') G_1[G_2] называется граф с множеством вершин V равным декартовому произведению V_1 \times V_2. Множество ребер X определяется следующим образом:
* так же рассмотрим любые две вершины u=(u_1, u_2) и v=(v_1, v_2) из V=V_1 \times V_2,
* вершины u и v смежны в G=G_1 + G_2 тогда и только тогда, когда (u_1 и v_1 — смежные) или (u_1 = v_1, а u_2 и v_2 — смежные).
}}
[[Файл:композиция.png|thumb|1100px|center|Композиция G_1 и G_2]]

{{Лемма
|about=
о произведении регулярных графов
|statement=
G_1 и G_2 — [[Основные_определения_теории_графов|регулярные]] графы. Тогда G = G_1 \times G_2 — регулярный граф.
|proof=
Пусть степень графов G_1 и G_2 будут k_1 и k_2 соответственно.
Рассмотрим любую вершину графа G: у нее k_1 + k_2 смежных вершин. Значит граф G регулярный.
}}

{{Лемма
|about=
о композиции регулярных графов
|statement=
G_1 и G_2 — регулярные графы. Тогда G = G_1[G_2] — регулярный граф.
|proof=
Пусть степень графов G_1 и G_2 будут k_1 и k_2 соответственно.
Рассмотрим любую вершину графа G: у нее |V_2| \cdot k_1 + k_2 смежных вершин. Значит граф G регулярный.
}}

{{Лемма
|about=
о произведении двудольных графов
|statement=
G_1 и G_2 — [[Основные_определения_теории_графов|двудольные]] графы. Тогда G = G_1 \times G_2 — двудольный граф.
|proof=
Пусть цвет c левых долей G_1 и G_2 будет 0, а правых 1.
А цвет каждой вершины v = (v_1, v_2) графа G будет равен c(v) = (c(v_1) + c(v_2)) \bmod 2.

Рассмотрим любую пару смежных вершин u = (u_1, u_2) и v = (v_1, v_2) из графа G, два случая:

# u_1 = v_1, u_2 и v_2 — смежные, значит c(u_1) = c(v_1) и с(u_2) \ne c(v_2), из этого следует c(u) \ne c(v),
# u_2 = v_2, u_1 и v_1 — смежные, аналогично следует c(u) \ne c(v).
Следовательно каждое ребро графа G соединяет вершины разного цвета, значит G двудольный. 
}}

==См. также==
* [[Дополнительный, самодополнительный граф]]
* [[Дерево, эквивалентные определения]]

== Источники информации ==
* Харари Ф. Теория графов / пер. с англ. — изд. 1-ое, с.35

[[Категория: Алгоритмы и структуры данных]]
[[Категория: Основные определения теории графов]]