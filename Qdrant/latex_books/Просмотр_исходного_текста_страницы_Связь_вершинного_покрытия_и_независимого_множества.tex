==Независимое множество==
{{Определение|definition=
'''Независимым множеством вершин''' ''(англ. independent vertex set)'' графа G=(V,E) называется такое подмножество S множества вершин графа V, что
 \forall u, v \in S uv \notin E.
}}
{{Определение|definition=
'''Максимальным независимым множеством''' ''(англ. maximum independent set)'' называется независимое множество вершин максимальной мощности.
}}

[[Файл:Independent_set_graph.gif|thumb|left|300px|Множество вершин синего цвета — максимальное независимое множество.]]

==Связь вершинного покрытия и независимого множества==
{{Теорема|statement=
Дополнение минимального вершинного покрытия является максимальным независимым множеством.
|proof=
Пусть M произвольное максимальное независимое множество вершин графа G=(V,E), а S его минимальное вершинное покрытие. Из определения следует, что любое ребро соединяет либо вершину из M и V \backslash M, либо вершины множества V \backslash M. Таким образом, каждое
ребро инцидентно некоторой вершине множества V \backslash M, то есть V \backslash M является некоторым вершинным покрытием. Тогда |S| \leqslant |V \backslash M| или |S| + |M| \leqslant |V|.

Рассмотрим произвольное минимальное вершинное покрытие графа S. Так как каждое ребро инцидентно хотя бы одной вершине из S, то V \backslash S является независимым множеством. Тогда |V \backslash S| \leqslant |M| или |V| \leqslant |S| + |M|.

Значит, |V| = |M| + |S|, и V \backslash S является максимальным независимым множеством, а V \backslash M — минимальным вершинным покрытием.
}}

==См. также ==
*[[Связь_максимального_паросочетания_и_минимального_вершинного_покрытия_в_двудольных_графах|Связь максимального паросочетания и минимального вершинного покрытия в двудольных графах]].

==Источники информации==
* [http://en.wikipedia.org/wiki/Vertex_cover_problem Wikipedia {{---}} Vertex cover]
* [http://en.wikipedia.org/wiki/Independent_set_(graph_theory) Wikipedia {{---}} Independent set]

[[Категория: Алгоритмы и структуры данных]]
[[Категория: Задача о паросочетании]]