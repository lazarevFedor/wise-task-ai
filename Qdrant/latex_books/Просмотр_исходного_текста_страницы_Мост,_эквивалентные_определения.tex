Пусть G {{---}} связный граф.
{{Определение
|definition=
'''Мост''' ''(англ. bridge)'' графа G {{---}} ребро, соединяющее две компоненты [[Отношение рёберной двусвязности|реберной двусвязности]] G. (1)
}}

[[Файл:Bridge_1.png|left|thumb|240px|Граф G]]

Пример графа с тремя мостами

==Эквивалентные определения==

{{Определение
|definition=
Мост графа G {{---}} ребро, при удалении которого граф G становится несвязным. (2)
}}

{{Определение
|definition=
Ребро x является мостом графа G, если в G существуют такие вершины u и v, что любой простой путь между этими вершинами проходит через ребро x. (3)
}}

{{Определение
|definition=
Ребро x является мостом графа G, если существует разбиение множества вершин V на такие множества U и W, что \forall u \in U и \forall w \in W ребро x принадлежит любому простому пути u \rightsquigarrow w. (4)
}}

{{Теорема
|statement = Определения (1), (2), (3) и (4) эквивалентны.
|proof = 

(1) \Rightarrow (2) Пусть ребро x соединяет вершины a и b. Пусть граф G - {x} {{---}} связный. Тогда между вершинами a и b существует еще один путь, т.е. между вершинами a и b существуют два реберно-непересекающихся пути. Но тогда ребро x не является мостом графа G. Противоречие.

(2) \Rightarrow (4) В условиях определения (4) пусть существуют такие вершины u и w, что между ними существует простой путь P: x \notin P. Но тогда граф G - {x} {{---}} связный. Противоречие.

(4) \Rightarrow (3) Возьмем \forall u \in U и \forall w \in W . Тогда \forall простой путь u \rightsquigarrow w содержит ребро x. Утверждение доказано

(3) \Rightarrow (1) Пусть (a, b) = x. Пусть ребро x не является мостом по определению (1).
Тогда между вершинами a и b есть простой путь P : P \cap x = \varnothing. Составим такой путь Q, что Q = ((u \rightsquigarrow a ) \circ P \circ (b \rightsquigarrow w)) . Сделаем путь Q [[Теорема о существовании простого пути в случае существования пути|простым]]. Получим простой путь (u \rightsquigarrow w), не проходящий по ребру x. Противоречие.
}}

== См.также ==
* [[Точка сочленения, эквивалентные определения]]
* [[Использование обхода в глубину для поиска мостов]]

==Источники информации==
* Харари Ф. Теория графов. М.: Мир, 1973. (Изд. 3, М.: КомКнига, 2006. — 296 с.)

[[Категория:Алгоритмы и структуры данных]]
[[Категория:Связность в графах]]