{{Определение
|id=def_1
|definition=
'''Независимым множеством''' (англ. ''Independent set'') в графе G = (V, E) называется непустое множество S \subset V: \forall v,u \in S ребро (v,u) \notin E.
}}
{{Теорема
|about=
Зыкова
|statement=
Для [[Хроматический многочлен|хроматического многочлена]] графа G верна формула:
P(G,x)=\sum\limits_{i=1}^n pt(G,i)x^{\underline{i}}, где pt(G,i) — число способов разбить вершины G на i независимых множеств, n = |V|, а x^{\underline i} = x \cdot (x - 1) \cdot \ldots \cdot (x - i + 1) {{---}} нисходящая факториальная степень.
|proof=
В правильной раскраске вершины, имеющие одинаковый цвет, не смежны, поэтому все такие вершины могут быть объединены в одно независимое множество. Перебрав все возможные разбиения на независимые множества с последующей их всевозможной покраской x доступными цветами получим искомое число способов раскраски графа G в x цветов.

Теперь проделаем это более формально. Подсчитаем число раскрасок графа G, в которых используется точно i цветов, для этого его нужно разбить на i независимых множеств и вершины в каждом таком классе покрасить в один из i цветов, отличный от всех других множеств, так как мы не делаем никаких предположений о связи между классами. 

Рассмотрим случай, где 1 \leqslant i \leqslant x. Чтобы получить такую раскраску зафиксируем какое-нибудь разбиение множества вершин графа G на i независимых множеств, затем берем один из классов в разбиении и 
раскрашиваем его в один из x цветов, потом берем следующий класс и окрашиваем его вершины в одинаковый цвет любой из x - 1 оставшихся красок и т.д. Всего таких способов разбиения существует pt(G,i).
Следовательно, перебрав все возможные разбиения на i независимых множеств, получим, что число интересующих нас раскрасок графа G равно pt(G,i) \cdot x \cdot (x - 1) \cdot \ldots \cdot (x - i + 1) = pt(G,i) \cdot x^{\underline i}.

Заметим теперь, что при i > x число x-раскрасок, в которых используется точно i цветов, равно 0 и при этом x^{\underline i} тоже равно 0. 

Суммирование по i от 1 до n даст полное число способов.
}}
'''Примечание''': в такой формулировке задача о поиске хроматического многочлена сводится к отысканию количества способов разбить граф на независимые множества, что в свою очередь также [[Примеры_NP-полных_языков#NP-полнота поиска максимального независимого множества | не разрешимо за полиномиальное время]]. 

==См. также==
*[[Формула Уитни]]

==Источники информации==
* ''Асанов М. О., Баранский В. А., Расин В. В.'' Дискретная математика: Графы, матроиды, алгоритмы. {{---}} Ижевск: НИЦ «РХД», 2001. С. 140-141. {{---}} '''ISBN 5-93972-076-5'''

[[Категория: Алгоритмы и структуры данных]]
[[Категория: Раскраски графов]]