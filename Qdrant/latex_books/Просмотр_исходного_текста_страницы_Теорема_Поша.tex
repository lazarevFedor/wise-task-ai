==Теорема==

{{Теорема
|id = Th1 
|about = 
Поша

|statement = Пусть граф G имеет n \geqslant 3 вершин и выполнены следующие два условия:

* \forall k,\, 1 \leqslant k , число вершин со степенями, не превосходящими k, меньше чем k
*для нечетного n число вершин степени (n-1)/2 не превосходит (n-1)/2

тогда G {{---}} [[Гамильтоновы графы|гамильтонов]] граф.

|proof = 
[[Файл: Graph-Posha.png|500px|thumb|right|Гамильтонов цикл v_{1} v_{2} \dotsc v_{{i}_{j-1}} v_{n} v_{n-1} \dotsc v_{{i}_{j}} v_{1} ]]
Предположим, что теорема неверна. Пусть G {{---}} максимальный негамильтонов граф с n вершинами, удовлетворяющий условиям теоремы.

Легко видеть, что добавление любого ребра в граф, обладающий указанными свойствами, приводит к графу, который также обладает этими свойствами. Таким образом, поскольку добавление к G произвольного ребра приводит к гамильтонову графу, любые две несмежные вершины соединимы простым гамильтоновым путем.

Покажем сначала, что всякая вершина, степень которой не меньше (n-1)/2 , смежна с каждой вершиной со степенью, большей чем (n-1)/2 . Не умаляя общности, допустим, что \deg v_{1} \geqslant (n-1)/2 и \deg v_{n} \geqslant n/2 , но вершины v_{1} и v_{n} не смежны. Тогда существует простой гамильтонов путь v_{1} v_{2} \dotsc v_{n} , соединяющий v_{1} и v_{n} . Обозначим вершины, смежные с v_{1} , через v_{{i}_{1}}, \dotsc,v_{{i}_{k}} , где k = \deg v_{1} и 2=i_{1} . Ясно, что вершина v_{n} не может быть смежной ни с одной вершиной из G вида v_{{i}_{j-1}} , поскольку тогда в G был бы гамильтонов цикл v_{1} v_{2} \dotsc v_{{i}_{j-1}} v_{n} v_{n-1} \dotsc v_{{i}_{j}} v_{1} .

Далее, так как k \geqslant (n-1)/2 , то n/2 \leqslant \deg v_{n} \leqslant n-1-k , что невозможно. Поэтому v_{1} и v_{n} должны быть смежны.

Отсюда следует, что если \deg v \geqslant n/2 для всех вершин v , то G {{---}} гамильтонов граф. В силу изложенного выше каждая пара вершин графа G смежна, т.е. G {{---}} полный граф. Мы пришли к противоречию, поскольку K_{n} {{---}} гамильтонов граф для всех n \geqslant 3 .

Таким образом, в G есть вершина v с \deg v . Обозначим через m наибольшую среди степеней всех таких вершин. Выберем такую вершину v_{1} , что \deg v_{1} = m . По принятому предположению число вершин со степенями, не превосходящими m , не больше чем m , поэтому должно быть более чем m вершин со степенями, превосходящими m , и, следовательно, не меньшими чем n/2 . В результате найдется некоторая вершина, скажем v_{n} , степени по крайней мере n/2 , не смежная с v_{1} . Так как v_{1} и v_{n} не смежны, то существует простой гамильтонов путь v_{1} \dotsc v_{n} . Как и выше, обозначим через v_{{i}_{1}}, \dotsc, v_{{i}_{m}} вершины графа G , смежные с v_{1} , и заметим, что вершина v_{n} не может быть смежной ни с одной из m вершин v_{{i}_{j-1}} для 1 \leqslant j \leqslant m . Но поскольку v_{1} и v_{n} не смежны, а v_{n} имеет степень не меньше n/2 , то, как было показано в первой части доказательства, m должно быть меньше чем (n-1)/2 . Так как по предположению число вершин со степенями, не превосходящими m , меньше чем m , то хотя бы одна из m вершин v_{{i}_{j-1}} , скажем v' , должна иметь степень не меньше n/2 . Итак, мы установили, что степени двух несмежных вершин v_{n} и v' не меньше n/2 . Полученное противоречие завершает доказательство теоремы. 
}}

==Замечания==
[[Файл: Graph-Posha-Cubic.png|250px|thumb|right|Кубический гамильтонов граф]]
*Приведенное достаточное условие не является необходимым. Изображенный на рисунке кубический граф {{---}} гамильтонов, хотя ясно, что он не удовлетворяет условиям теоремы.
*Условия теоремы нельзя улучшить, так как при их ослаблении новое условие уже не будет достаточным для гамильтоновости графа.

==Следствия==

Ограничивая условия теоремы Поша, получаем более простые, но менее сильные достаточные условия, найденные [[Теорема Оре|Оре]] и [[Теорема Дирака|Дираком]] соответственно:

{{Теорема
|id = Th2
|about = Следствие 1
|statement = 
Если n \geqslant 3 и \deg u + \deg v \geqslant n для любой пары u и v несмежных вершин графа G , то G {{---}} гамильтонов граф. 
}}

{{Теорема
|id = Th3
|about = Следствие 2
|statement = 
Если n > 3 и \deg v \geqslant n/2 для любой вершины v графа G , то G {{---}} гамильтонов граф. 
}}

==См. также==
*[[Теорема Оре]]
*[[Теорема Дирака]]
*[[Теорема Хватала]]
*[[Теорема Гринберга]]
*[[Теорема Редеи-Камиона]]

==Источники информации==
*Харари Ф. Теория графов: Пер. с англ. / Предисл. В. П. Козырева; Под ред. Г.П.Гаврилова. Изд. 4-е. — М.: Книжный дом "ЛИБРОКОМ", 2009. — 60 с.

[[Категория: Алгоритмы и структуры данных]]
[[Категория: Обходы графов]]
[[Категория: Гамильтоновы графы]]