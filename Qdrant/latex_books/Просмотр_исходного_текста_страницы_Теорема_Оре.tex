{{Теорема
|statement=
Если n \geqslant 3 и \operatorname{deg} u + \operatorname{deg} v \geqslant n для любых двух различных несмежных вершин u и v [[Основные определения теории графов#Неориентированные графы|неориентированного графа]] G, то G {{---}} [[Гамильтоновы графы | гамильтонов граф]].

|proof=

Пусть, от противного, существует граф G, который удовлетворяет условию теоремы, но не является гамильтоновым графом.
Будем добавлять к нему новые ребра до тех пор, пока не получим максимальный негамильтонов граф G'. В силу того, что мы только добавляли ребра, условие теоремы не нарушилось.

Пусть u,v несмежные вершины в полученном графе G'. Если добавить ребро uv, появится гамильтонов цикл. Тогда путь (u,v) {{---}} гамильтонов.

Для вершин u,v выполнено \operatorname{deg} u + \operatorname{deg} v \geqslant n. 

По принципу Дирихле всегда найдутся две смежные вершины t_1,t_2 на пути (u,v), т.е. u \dots t_1t_2 \dots v , такие, что существует ребро ut_2 и ребро t_1v.

Действительно, пусть S= \{ i \mid e_i=ut_{i+1} \in EG \} и T = \{ i \mid f_i=t_iv \in EG \} 

Имеем: \left\vert S \right\vert + \left\vert T \right\vert = \operatorname{deg} u + \operatorname{deg} v \geqslant n , но \left\vert S + T \right\vert 

Тогда \left\vert S\cap T \right\vert = \left\vert S \right\vert + \left\vert T \right\vert - \left\vert S+T \right\vert > 0, т. е. \exists i: ut_{i+1}\in EG и t_iv \in EG.
Получили противоречие, т. к. u \dots t_1v \dots t_2u {{---}} гамильтонов цикл.
}}
== См. также ==
* [[Эйлеров цикл, Эйлеров путь, Эйлеровы графы, Эйлеровость орграфов]]
* [[Алгоритм нахождения Гамильтонова цикла в условиях теорем Дирака и Оре]]
* [[Теорема Дирака]]
== Источники информации ==
* Асанов М. О., Баранский В. А., Расин В. В. {{---}} Дискретная математика: Графы, матроиды, алгоритмы. '''ISBN 978-5-8114-1068-2'''
* Харари {{---}} Теория графов. '''ISBN 978-5-397-00622-4'''

[[Категория: Алгоритмы и структуры данных]]
[[Категория: Обходы графов]]
[[Категория: Гамильтоновы графы]]