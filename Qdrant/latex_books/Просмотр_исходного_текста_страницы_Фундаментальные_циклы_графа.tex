{{Определение
|definition=
'''Фундаментальный цикл графа G относительно остова T''' (''англ. fundamental cycle'') {{---}} простой цикл C, полученный путем добавления к [[Остовные деревья: определения, лемма о безопасном ребре|остову]] T ребра e_1e_2 \notin T.}}
[[Файл:Fundomential.png|380px|центр|thumb|Пример фундаментального цикла в графе. Красным выделен фундаментальный цикл, полученный добавлением ребра (3, 4)]]

{{Теорема
|statement =
Множество всех фундаментальных циклов относительно любого остова T графа G образует базис [[Циклическое пространство графа|циклического пространства]] этого графа.
|proof =

Рассмотрим остов T графа G и фундаментальные циклы C_1 \ldots C_s относительно остова T. В каждом цикле есть ребро e_i, которое принадлежит ровно одному из C_1 \ldots C_{s} . Поэтому сумма различных фундаментальных циклов относительно остова T не является пустым графом, из чего следует, что C_1 \ldots C_s линейно независимы.

Докажем, что любой цикл из циклического пространства графа G является суммой фундаментальных циклов. Пусть Z — цикл циклического пространства графа G, e_1 \ldots e_{k} ребра принадлежащие Z и не принадлежащие T. Рассмотрим граф F = Z \oplus C_1 \oplus \ldots \oplus C_{k} . Каждое из ребер e_{t} , t = 1,\ldots ,k встречается ровно в двух слагаемых — Z и C_{k}. Значит F содержит только ребра из T. Так как C_1 \ldots C_{k} простые циклы, то степени всех их вершин четны, степени вершин Z тоже четны по [[Циклическое пространство графа#lemma1|лемме]], значит степени всех вершин F четны. Если F непустой граф, то в F есть цикл, значит цикл есть и в T. Значит F пустой граф, откуда следует что Z = C_1 \oplus \ldots \oplus C_{k} .
}}

==См. также==
* [[Остовные деревья: определения, лемма о безопасном ребре|Остовные деревья]]
* [[Циклическое пространство графа|Циклическое пространство графа]]

==Источники информации==
* Харари Фрэнк '''Теория графов''' = Graph theory/Пер. с англ. и предисл. В. П. Козырева. Под ред. Г.П.Гаврилова. Изд. 2-е. — М.: Едиториал УРСС, 2003. — 296 с. — ISBN 5-354-00301-6

[[Категория: Алгоритмы и структуры данных]]
[[Категория: Основные определения теории графов]]