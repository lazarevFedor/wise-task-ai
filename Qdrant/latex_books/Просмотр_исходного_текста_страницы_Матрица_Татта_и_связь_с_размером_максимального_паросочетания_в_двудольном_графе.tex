==Матрица Татта==
{{Определение
|definition = '''Матрицей Татта''' (англ. ''Tutte matrix'') для графа G с n вершинами называется матрица размера n \times n

A_{ij} = \begin{cases}
 E_{ij}, & \mathrm{edge}\;(i,j)\;exists\;\mathrm{and}\;ij\\
 0, & \mathrm{otherwise} \end{cases},
где E_{ij} {{---}} независимые переменные (E_{ij} не зависят друг от друга, и могут принимать произвольные значения)
}}

{{Теорема
|statement = В графе G существует [[Паросочетания:_основные_определения,_теорема_о_максимальном_паросочетании_и_дополняющих_цепях | совершенное паросочетание]] тогда и только тогда, когда определитель матрицы Татта для G не равен нулю тождественно.
|proof = 
\det(A) = \sum\limits_\varphi \operatorname{sign}(\varphi)A_{1\varphi(1)}A_{2\varphi(2)} \ldots A_{n\varphi(n)}

Пусть \Phi = \{\forall\varphi | A_{1\varphi(1)}A_{2\varphi(2)} \ldots A_{n\varphi(n)} \ne 0\}

Любой перестановке \chi \in \Phi соответствует орграф G_{\chi}, для любой вершины которого \deg^+=\deg^-=1

Если \exists G_\chi : все циклы в нём чётной длины, то совершенное паросочетание в G найдено.

В противном случае в \forall G_\chi \exists цикл нечётной длины. Рассмотрим G'_\chi, полученный из G обратной ориентацией дуг в каком-нибудь цикле нечётной длины. Заметим, что \forall G'_\chi соответствует \chi' \in \Phi. При этом \operatorname{sign}(\chi) = \operatorname{sign}(\chi'), так как одна получается из другой за чётное число транспозиций. Однако \sum\limits_\chi A_{1\chi(1)}A_{2\chi(2)} \ldots A_{n\chi(n)} = - \sum\limits_{\chi'} A_{1\chi'(1)}A_{2\chi'(2)} \ldots A_{n\chi'(n)}, так как перенаправлено было нечётное число рёбер.
Таким образом, для \forall \chi,\chi' слагаемые, соответствующие им в выражении для \det(A) сократятся. А так как в нём все слагаемые либо нулевые, либо \in \Phi, то \det(A) = 0
}}

==Матрица Эдмондса==
Для случая, когда G {{---}} [[Основные_определения_теории_графов#.D0.A7.D0.B0.D1.81.D1.82.D0.BE_.D0.B8.D1.81.D0.BF.D0.BE.D0.BB.D1.8C.D0.B7.D1.83.D0.B5.D0.BC.D1.8B.D0.B5_.D0.B3.D1.80.D0.B0.D1.84.D1.8B | двудольный]], существует более простая матрица, аналогичная матрице Татта.
{{Определение
|definition = '''Матрицей Эдмондса''' (англ. ''Edmonds matrix'') для двудольного графа G с размерами долей n,m называется матрица размера n \times m
 D_{ij} = \begin{cases}
 E_{ij}, & \mathrm{edge}\;(i,j)\;exists\\
 0, & \mathrm{otherwise}
\end{cases}, 
где E_{ij} {{---}} независимые переменные
}}

{{Теорема
|statement = Ранг матрицы Эдмондса для графа G совпадает с размером [[Паросочетания:_основные_определения,_теорема_о_максимальном_паросочетании_и_дополняющих_цепях | максимального паросочетания]] в этом графе
|proof = Рангом матрицы называется количество линейно независимых строчек/столбцов в ней. Или, что эквивалентно, размер наибольшего ненулевого минора. 
Рассмотрим этот максимальный минор A_M. На нём матрицу Эдмондса легко дополнить до матрицы Татта, причём её определитель, очевидно, останется ненулевым. По ранее доказанной теореме, в графе, соответствующем A_M существует совершенное паросочетание, то есть покрывающее все его вершины. То есть мощности, равной размеру A_M.
Предположим, что существует паросочетание большей мощности. Однако тогда и соответствующий ему ненулевой (по теореме о матрице Татта) минор большего размера, чем A_M, что невозможно в силу выбора A_M максимальным.
}}

==См. также==
* [[Теорема Татта о существовании полного паросочетания]]
* [[Паросочетания: основные определения, теорема о максимальном паросочетании и дополняющих цепях]]
* [[Декомпозиция Эдмондса-Галлаи]]

==Источники информации==
*[https://en.wikipedia.org/wiki/Tutte_matrix Wikipedia {{---}} Tutte matrix]
*[https://en.wikipedia.org/wiki/Edmonds_matrix Wikipedia {{---}} Edmonds matrix]
*[http://e-maxx.ru/algo/tutte_matrix MAXimal::algo::Матрица Татта]

[[Категория: Алгоритмы и структуры данных]]
[[Категория: Задача о паросочетании]]