{{Лемма
|statement=Наличие двух различных рёберно простых путей между какими-либо двумя вершинами неориентированного [[Основные определения теории графов|графа]] G равносильно наличию цикла в этом графе.
|proof=
\Rightarrow

Предположим, что в графе G существует два различных рёберно простых пути между вершинами u и v. Пусть это будут пути p = u e_1 v_1\ldots v_{n-1} e_n v и p' = u e'_1 v'_1\ldots v'_{n-1} e'_n v. Пусть их наибольший общий префикс заканчивается в вершине w = v_k = v'_k. Заметим, что w \neq v, т.к. пути различны. Рассмотрим суффиксы путей p и p': s = w e_{k+1} \ldots v и s' = w e'_{k+1} \ldots v соответственно. Найдём первую совпадающую вершину w' в s и s', не равную w. Осталось заметить, что замкнутый путь c, полученный объединением w \leadsto w' части пути s вместе с w' \leadsto w частью цепи s', является циклическим путем. Действительно, в путях s и s' двух одинаковых рёбер подряд не бывает, т.к. это рёберно простые пути, а рёбра, смежные с w и w', не совпадают по построению. Циклический путь c является представителем некоторого цикла в графе G.

\Leftarrow

Предположим, что в графе G существует цикл и пусть циклический путь c = v_0 e_1 v_1 \ldots e_n v_0 {{---}} его представитель. Найдём первую точку w = v_k = v_l (l > k) пересечения c с самим собой. Такая точка существует, т.к. путь замкнутый. Рассмотрим циклический путь v_k e_{k+1} \ldots e_l v_l: он простой, т. к. если это неверно и существует вершина v_j = v_j' (k , то в c вершина v_j' повторяется раньше, чем v_l. Теперь элементарно взяв две вершины v_k и v_{k+1} легко заметить, что существует два различных рёберно непересекающихся пути между ними: v_k e_{k+1} v_{k+1} и v_k e_l v_{l - 1} \ldots v_k.
 }}
[[Файл:2_paths_and_a_cycle.png|600px|thumb|center|Иллюстрация к лемме: пути отмечены соответственно красным и синим (их общий префикс отмечен пунктиром), а циклический путь c проходит вдоль чёрных стрелок]]

{{Теорема
|statement=
Если в неориентированном графе существует цикл, то в этом графе существует простой цикл.
|proof=
Выберем в графе минимальный по количеству рёбер цикл (он существует, потому что количество рёбер в любом цикле — натуральное число [[Натуральные и целые числа#.D0.A1.D1.83.D1.89.D0.B5.D1.81.D1.82.D0.B2.D0.BE.D0.B2.D0.B0.D0.BD.D0.B8.D0.B5_.D0.BD.D0.B0.D0.B8.D0.BC.D0.B5.D0.BD.D1.8C.D1.88.D0.B5.D0.B3.D0.BE_.D1.8D.D0.BB.D0.B5.D0.BC.D0.B5.D0.BD.D1.82.D0.B0|Существование наименьшего элемента в любом подмножестве \Bbb N]]). Предположим, что он не простой. Но тогда он содержит дважды одну и ту же вершину, т. е. содержит в себе цикл меньшего размера, что противоречит тому, что наш цикл минимальный. Таким образом, этот цикл — простой.}}

[[Файл:Simple cycle.png|thumb|580px|center|В графе минимальный цикл включает в себя три ребра — например, [2 - 5 - 6] (выделен красным). Согласно теореме, он является простым.]]

== Замечания ==
* Так как вершинно-простой путь всегда является рёберно-простым, первая теорема справедлива и для вершинно-простых путей (усиление условия).
* Так как вершинно-простой цикл всегда является рёберно-простым, первая теорема справедлива и для рёберно-простого цикла (ослабление результата).
{{Утверждение
|about=неверное
|statement=''Если две вершины графа лежат на цикле, то они лежат на простом цикле.''
|proof=
В общем случае неверно, так как эти вершины могут лежать в разных компонентах вершинной или рёберной двусвязности: все пути из одной вершины в другую будут содержать одну и ту же [[Точка сочленения, эквивалентные определения|точку сочленения]] или один и тот же [[Мост, эквивалентные определения|мост]].
}}

== Примечания ==

== См. также ==
* [[Теорема о существовании простого пути в случае существования пути]]
* [[Отношение рёберной двусвязности]]
* [[Отношение вершинной двусвязности]]

[[Категория: Алгоритмы и структуры данных]]
[[Категория: Основные определения теории графов]]