==Задача о потоке минимальной стоимости==

{{Определение
|definition=Пусть дана сеть G(V,E). S, T \in V {{---}} источник и сток. Ребра (u,v) \in E имееют пропускную способность c(u, v), поток f(u,v) и цену за единицу потока a(u, v) . Тогда '''общая стоимость потока''' из S в T:
:p(u,v) = \sum\limits_{u,v \in V, f(u,v)>0} a(u,v) \cdot f(u,v)
}}
===Свойства сети===
* Поток не может превысить пропускную способность. 
:f(u,v) \leqslant c(u,v)
* Поток из u в v должен быть противоположным потоку из v в u. 
:f(u, v)=-f(v, u)
* Сохранение потока. Для каждой вершины, сумма входящего и исходящего потоков равно 0.
: \sum\limits_{w \in V} f(u,w) = 0

{{Задача
|definition = Дана сеть G(V,E). S, T \in V {{---}} источник и сток. Ребра (u,v) \in E имееют пропускную способность c(u, v), поток f(u,v) {{---}}
и цену за единицу потока a(u, v) . Требуется найти максимальный поток, суммарная стоимость которого минимальна.
}}

== Алгоритмы решения ==
===Метод устранения отрицательных циклов в остаточной сети===
Воспользуемся [[Лемма об эквивалентности свойства потока быть минимальной стоимости и отсутствии отрицательных циклов в остаточной сети|леммой об эквивалентности свойства потока быть минимальной стоимости и отсутствии отрицательных циклов в остаточной сети]]. Получим следующий алгоритм:
====Алгоритм====
* '''Начало.'''
* '''Шаг 1'''. Определим для каждого прямого ребра (u,v) обратное ребро (v,u). Определим его характеристики: c(v,u)=0, f(v,u)=-f(u,v), a(v,u)=-a(u,v).
* '''Шаг 2'''. Для каждого ребра зададим поток равный 0.
* '''Шаг 3'''. Найдем произвольный максимальный поток(любым алгоритмом нахождения максимального потока), построим для него остаточную сеть, где каждому ребру будет соответствовать величина a(u,v) \cdot (c(u,v) - f(u,v)).
* '''Шаг 4'''. При помощи [[Алгоритм Форда-Беллмана| алгоритма Форда-Беллмана]] найдем отрицательный цикл в построенной сети (отрицательный цикл ищется по стоимости ребра, т.е. ребра имеют вес a(u,v)). Если отрицательный цикл не нашелся {{---}} перейдем к '''шагу 6'''.
* '''Шаг 5'''. Избавимся от отрицательного цикла, для этого пустим по нему максимально возможный поток. Величина потока равна минимальной остаточной пропускной способности в цикле. Перейдем к '''шагу 4'''.
* '''Шаг 6'''. Отрицательных циклов в остаточной сети нет, значит, максимальный поток минимальной стоимости найден.
* '''Конец.'''

====Асимптотика====
Алгоритм Форда-Беллмана работает за O(VE), улучшение цикла за O(E). Если обозначить максимальную стоимость потока как C, а максимальную пропускную способность как U, то алгоритм совершит O(ECU) итераций поиска цикла, если каждое улучшение цикла будет улучшать его на 1. В итоге имеем O(V E^2 C U + maxFlow), где maxFlow - асимптотика поиска максимального потока.

===Метод дополнения потока вдоль путей минимальной стоимости===
{{main|Поиск потока минимальной стоимости методом дополнения вдоль путей минимальной стоимости}}

===Использование потенциалов Джонсона===
{{main|Использование потенциалов Джонсона при поиске потока минимальной стоимости}}

== См. также ==
* [[Сведение задачи о назначениях к задаче о потоке минимальной стоимости|Сведение задачи о назначениях к задаче о потоке минимальной стоимости]]
* [[Венгерский алгоритм решения задачи о назначениях|Венгерский алгоритм решения задачи о назначениях]]

== Источники информации ==
*[http://ru.wikipedia.org/wiki/Поток_минимальной_стоимости Википедия {{---}} Поток минимальной стоимости]
*[http://rain.ifmo.ru/cat/view.php/vis/graph-flow-match/min-cost-max-flow-2009 Визуализатор алгоритма нахождения максимального потока минимальной стоимости]
*[http://habrahabr.ru/blogs/algorithm/61884/ Хабрахабр {{---}} Максимальный поток минимальной стоимости]
* ''Кормен, Томас Х., Лейзерсон, Чарльз И., Ривест, Рональд Л., Штайн Клиффорд'' '''Алгоритмы: построение и анализ''', 2-е издание. Пер. с англ. {{---}} М.:Издательский дом "Вильямс", 2010. {{---}} 1296 с.: ил. {{---}} Парал. тит. англ. {{---}} ISBN 978-5-8459-0857-5 (рус.)

[[Категория:Алгоритмы и структуры данных]]
[[Категория: Задача о потоке минимальной стоимости]]