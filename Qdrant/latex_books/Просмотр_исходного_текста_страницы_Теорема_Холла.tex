==Определения==

Пусть G(V,E) {{---}} [[Основные_определения_теории_графов#Двудольный_граф |двудольный граф]]. L {{---}} множество вершин левой доли. R {{---}} множество вершин правой доли.
{{Определение
|id=def1. 
|nеat=1
|definition='''Полным (совершенным)''' паросочетанием ''(англ. perfect matching)'' называется паросочетание, в которое входят все вершины.
}}
{{Определение
|id=def2.
|nеat=1
|definition=Пусть X \subset V . '''Множeство соседей''' X ''(англ. neighborhood)'' определим формулой: N(X)= \{ y \in V \mid (x,y) \in E , x \in X\}
}}

==Теорема==
{{Теорема
|id=th1. 
|author=Холл 
|statement=Полное паросочетание существует тогда и только тогда, когда для любого A \subset L выполнено |A| \leqslant |N(A)|.
|proof=
\Rightarrow 
Очевидно, что если существует полное паросочетание, то для любого A \subset L выполнено |A| \leqslant |N(A)|. У любого подмножества вершин есть по крайней мере столько же ''соседей'' (''соседи по паросочетанию'').

\Leftarrow 
В обратную сторону докажем по индукции (будем добавлять в изначально пустое паросочетание P по одному ребру и доказывать, что мы можем это сделать, если P не полное). Таким образом, в конце получим что P — полное паросочетание.
 
'''''База индукции'''''

Вершина из L соединена хотя бы с одной вершиной из R. Следовательно база верна.

'''''Индукционный переход'''''

Пусть после k шагов построено паросочетание P. Докажем, что в P можно добавить вершину x из L, не насыщенную паросочетанием P. Рассмотрим множество вершин H — все вершины, достижимые из x, если можно ходить из R в L только по ребрам из P, а из L в R по любым ребрам из G. Тогда в H найдется вершина y из R, не насыщенная паросочетанием P, иначе, если рассмотреть вершины H_L (вершины из H принадлежащие L), то для них не будет выполнено условие: |H_L| \leqslant |N(H_L)|. Тогда существует путь из x в y, который будет удлиняющим для паросочетания P (т.к из R в L мы проходили по ребрам паросочетания P). Увеличив паросочетание P вдоль этого пути, получаем искомое паросочетание. Следовательно предположение индукции верно. 

}}

==Пояснения к доказательству==
[[Файл:aba.gif|600px|thumb|right|Пример]]

Пусть было построено паросочетание размером 3 (синие ребра).

Добавляем вершину с номером 4.

Во множество H вошли вершины с номерами 1, 3, 4, 5, 7, 8.

Ненасыщенная вершина из правой доли всегда найдется (в примере вершина с номером 8), т.к иначе получаем противоречие:
# В H_R входят только насыщенные вершины.
# N(H_L) = H_R 
# В H_L по крайней мере H_R+1 вершин (''соседи'' по паросочетанию для каждой вершины из H_R и ещё одна вершина, которую пытаемся добавить).
Цепь {4, 7, 3, 8} является удлиняющей для текущего паросочетания.

Увеличив текущее парасочетание вдоль этой цепи, мы насытим вершину с номером 4.

==См. также==
* [[Паросочетания: основные определения, теорема о максимальном паросочетании и дополняющих цепях]]
* [[Связь максимального паросочетания и минимального вершинного покрытия в двудольных графах]]
* [[Связь вершинного покрытия и независимого множества]]

==Примечания==

Также теорема обобщается на граф, имеющий произвольное множество долей.
Иногда теорему называют теоремой о свадьбах.

==Источники информации==
* [http://ru.wikipedia.org/wiki/%D0%A2%D0%B5%D0%BE%D1%80%D0%B5%D0%BC%D0%B0_%D0%A5%D0%BE%D0%BB%D0%BB%D0%B0 Википедия {{---}} Теорема Холла]
* [https://en.wikipedia.org/wiki/Hall%27s_marriage_theorem Wikipedia {{---}} Hall's marriage theorem]

[[Категория: Алгоритмы и структуры данных]]
[[Категория: Задача о паросочетании ]]