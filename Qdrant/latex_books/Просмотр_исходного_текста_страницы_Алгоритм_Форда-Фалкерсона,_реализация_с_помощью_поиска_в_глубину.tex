'''Алгоритм Форда-Фалкерсона''' — алгоритм, решающий задачу нахождения максимального [[Определение сети, потока #Определение потока | потока]] в транспортной сети.

== Идея ==
Идея алгоритма заключается в следующем. Изначально величине потока присваивается значение 0: f(u,v) = 0 для всех u, v из V . Затем величина потока итеративно увеличивается посредством поиска увеличивающего пути (путь от источника s к стоку t, вдоль которого можно послать ненулевой поток). В данной статье рассматривается алгоритм, осуществляющий этот поиск с помощью [[Обход в глубину, цвета вершин|обхода в глубину (dfs)]]. Процесс повторяется, пока можно найти увеличивающий путь.

== Реализация ==
 '''int''' dfs('''int''' u, '''int''' Cmin): // Cmin {{---}} пропускная способность в текущем подпотоке
 '''if''' u = t
 '''return''' Cmin
 visited[u] = ''true'' 
 '''for''' v '''in''' u.children
 '''auto''' uv = edge(u, v)
 '''if''' '''not''' visited[v] '''and''' uv.f 0
 uv.f += delta
 uv.backEdge.f -= delta
 '''return''' delta
 '''return''' 0

== Оценка производительности ==
Добавляя поток увеличивающего пути к уже имеющемуся потоку, максимальный поток будет получен, когда нельзя будет найти увеличивающий путь. Тем не менее, если величина пропускной способности — иррациональное число, то алгоритм может работать бесконечно. В целых числах таких проблем не возникает и время работы ограничено O(|E|f), где E — число рёбер в графе, f — максимальный поток в графе, так как каждый увеличивающий путь может быть найден за O(E) и увеличивает поток как минимум на 1.

=== Пример несходящегося алгоритма ===
[[Файл:F-f.5.png|300px|thumb|right|Рис. 1]]
Рассмотрим приведённую справа сеть с источником \ s, стоком \ t, пропускными способностями рёбер \ e_1, \ e_2 и \ e_3 соответственно \ 1, r=\dfrac{\sqrt{5}-1}{2} и \ 1 и пропускной способностью всех остальных рёбер, равной целому числу M \geqslant 2. Константа \ r выбрана так, что \ r^2 = 1 - r. Мы используем пути из остаточного графа, приведённые в таблице, причём \ p_1 = \{ s, v_4, v_3, v_2, v_1, t \}, \ p_2 = \{ s, v_2, v_3, v_4, t \} и \ p_3 = \{ s, v_1, v_2, v_3, t \}.

{| class="wikitable" style="text-align: center"
! valign="top" rowspan=2 | Шаг !! valign="top" rowspan=2 | Найденный путь !! valign="top" rowspan=2 | Добавленный поток !! colspan=3 | Остаточные пропускные способности
|-
! e_1 !! e_2 !! e_3
|-
| 0 || - || - || r^0=1 || r || 1
|-
| 1 || \{ s, v_2, v_3, t \} || 1 || r^0 || r^1 || 0
|-
| 2 || p_1 || r^1 || r^2 || 0 || r^1
|-
| 3 || p_2 || r^1 || r^2 || r^1 || 0
|-
| 4 || p_1 || r^2 || 0 || r^3 || r^2
|-
| 5 || p_3 || r^2 || r^2 || r^3 || 0
|}

Заметим, что после шага 1, как и после шага 5, остаточные способности рёбер e_1, e_2 и e_3 имеют форму r^n, r^{n+1} и 0, соответственно, для какого-то натурального n. Это значит, что мы можем использовать увеличивающие пути p_1, p_2, p_1 и p_3 бесконечно много раз, и остаточные пропускные способности этих рёбер всегда будут в той же форме. Полный поток после шага 5 равен 1 + 2(r^1 + r^2). За бесконечное время полный поток сойдётся к \textstyle 1 + 2\sum\limits_{i=1}^\infty r^i = 3 + 2r, тогда как максимальный поток равен 2M + 1. Таким образом, алгоритм не только работает бесконечно долго, но даже и не сходится к оптимальному решению.

=== Пример медленной работы алгоритма Форда-Фалкерсона с использованием поиска в глубину по сравнению с реализацией, использующей поиск в ширину ===
При использовании поиска в ширину алгоритму потребуется всего лишь два шага.
Дана сеть (Рис. 2).
[[Файл:F-f.1.png|thumb|300px|center|Рис. 2]]
Благодаря двум итерациям (Рис. 3 и Рис. 4)
[[Файл:F-f.2.png|thumb|300px|center|Рис. 3]]
[[Файл:F-f.3.png|thumb|300px|center|Рис. 4]]
рёбра AB, AC, BD, CD насытились лишь на 1.
Конечная сеть будет получена ещё через 1998 итераций (Рис. 5).
[[Файл:F-f.4.png|thumb|300px|center|Рис. 5]]

== См. также ==
* [[Теорема Форда-Фалкерсона]]
* [[Алгоритм Эдмондса-Карпа]]

== Источники информации ==
* [http://ru.wikipedia.org/wiki/Алгоритм_Форда_—_Фалкерсона Википедия: Алгоритм Форда — Фалкерсона] 
* Томас Х. Кормен и др. Алгоритмы: построение и анализ = INTRODUCTION TO ALGORITHMS. — 2-е изд. — М.: «Вильямс», 2006. — С. 1296. — ISBN 0-07-013151-1

[[Категория: Алгоритмы и структуры данных]]
[[Категория: Задача о максимальном потоке ]]