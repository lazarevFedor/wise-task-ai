== Неориентированный граф ==

{{Лемма
|statement=
Сумма степеней всех вершин графа (или мультиграфа без петель) — чётное число, равное удвоенному числу рёбер:
 \sum\limits_{v\in V(G)} deg\ v=2\cdot|E(G)|
|proof=
Возьмем пустой граф. Сумма степеней вершин такого графа равна нулю. При добавлении ребра, связывающего любые две вершины, сумма всех степеней увеличивается на 2 единицы. Таким образом, сумма всех степеней вершин чётна и равна удвоенному числу рёбер.
}}
Например, для следующего графа выполнено: deg(1)+\ldots+deg(6)=16=2\cdot|E|

[[Файл:undir_grap.png]]

'''Следствие 1.''' В любом графе число вершин нечётной степени чётно.

'''Следствие 2.''' Число рёбер в полном графе \frac{n\cdot(n-1)}{2} .

== Ориентированный граф ==

{{Лемма
|statement=
Сумма входящих и исходящих степеней всех вершин ориентированного графа — чётное число, равное удвоенному числу рёбер:
 \sum\limits_{v\in V(G)} deg^{-}\ v \; + \sum\limits_{v\in V(G)} deg^{+}\ v=2\cdot |E(G)| 

|proof=
[[Файл:dir_grap.png|thumb|300px| deg^{-}+deg^{+}=10=2\cdot |E|]]
Аналогично доказательству леммы о рукопожатиях неориентированном графе. 
То есть возьмем пустой граф и будем добавлять в него рёбра. При этом каждое добавление ребра увеличивает на единицу сумму входящих и на единицу сумму исходящих степеней. Таким образом, сумма входящих и исходящих степеней всех вершин ориентированного графа чётна и равна удвоенному числу рёбер.
}}

== Бесконечный граф ==
[[Файл:inf_grap.png|thumb|300px|right|Пример бесконечного графа, в котором не выполняется лемма]]

В бесконечном графе лемма не работает, даже в случае с конечным числом вершин нечётной степени. Покажем это на примере.

При выборе бесконечного пути из вершины V (см. рисунок справа) имеем путь, в котором все вершины кроме стартовой имеют чётную степень, что противоречит следствию из леммы.

== Регулярный граф ==
{{Определение
|definition=
Граф называется '''регулярным''', если степени всех его вершин равны.
}}
{{Утверждение
|statement=
В регулярном графе с n вершинами ровно \frac{k\cdot n}{2} рёбер.

}}

{{Утверждение
|statement=Если степень каждой вершины нечётна и равна k, то количество рёбер кратно k .
|proof= [[Файл:reg_grap.png|thumb|300px|right|Регулярный граф с \frac{k\cdot n}{2} = \frac{3\cdot 6}{2}=9 рёбрами ]]
Действительно, так как степень каждой вершины нечётна, то число вершин в графе чётно(так сумма степеней всех вершин чётна). Пусть n = 2\cdot r , то равенство принимает вид |E| =\frac{k\cdot n}{2} = \frac{2\cdot k\cdot r}{2}=k\cdot r , то есть количество рёбер кратно k.
}}

== Источники информации ==
* Lecture Notes on Graph Theory By Tero Harju, Department of Mathematics University of Turku, 2011 — с. 7-8
* [http://en.wikipedia.org/wiki/Handshaking_lemma Handshaking lemma — Wikipedia]
[[Категория: Алгоритмы и структуры данных]]
[[Категория: Основные определения теории графов]]