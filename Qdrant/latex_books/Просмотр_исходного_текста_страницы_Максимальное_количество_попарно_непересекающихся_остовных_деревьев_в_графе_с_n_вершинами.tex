==Попарно непересекающиеся остовные деревья==
{{Утверждение
|id = max_spanning_tree 
|statement=Максимальное количество попарно непересекающихся [[Остовные деревья: определения, лемма о безопасном ребре#spanning_tree| остовных деревьев]] в графе с n вершинами не более \left \lfloor {\dfrac{n}{2}}\right \rfloor 
|proof = 
Очевидно, что среди графов с n вершинами наибольшее количество непересекающихся остовных деревьев может быть только в полном графе. Количество ребер в таком графе равно \dfrac{n(n - 1)}{2}, а в каждом дереве n -
 1 ребро. Значит, в полном графе мы сможем построить не более \left \lfloor {\dfrac{n(n - 1)}{2(n - 1)}}\right \rfloor = \left \lfloor {\dfrac{n}{2}}\right \rfloor остовных деревьев. 
}}

==Построение== 
===Описание алгоритма=== 
Расположим вершины на окружности так, чтобы они являлись вершинами правильного многоугольника, и выберем начальную вершину (рис.1). Для \left \lfloor {\dfrac{n}{2}}\right \rfloor вершин по часовой стрелке, начиная с этой вершины, будем строить остовные деревья. Для i-ой вершины строим такой путь :V_i V_{i+1} V_{i-1} V_{i+2} V_{i-2}\ldots, {{---}} до тех пор, пока не соединим все вершины. Это и будет остовным деревом. (рис.2-3) 
{| cellpadding="10" 
|- 
|[[Файл:Max spanning tree1.png|thumb|300px|center|Рис.1 Стрелкой указана начальная вершина]] || [[Файл:Max spanning tree2.png|thumb|339px|center|Рис.2 Красным цветом выделено первое построенное остовное дерево]] || [[Файл:Max spanning tree6.png|thumb|270px|center|Рис.3 Все остовные деревья]] 
|}

===Доказательство корректности=== 
[[Файл:Max spanning tree3.png|thumb|250px|right|Рис.4 Черным цветом выделено рассматриваемое ребро, красным - все его повороты]] [[Файл:Max spanning tree4.png|thumb|250px|right|Рис.5 Черным цветом выделены рассматриваемые ребра, красным - остальные ребра остовного дерева]]Докажем, что построенные с помощью такого алгоритма остовные деревья будут попарно непересекающимися. Для этого докажем, что никакие ребра не совпадут. Ребра могут совпасть только в том случае, если дуги, на которые эти ребра опираются, будут одинаковой длины. Заметим, что при построении каждого последующего дерева его ребра получаются из поворотов ребер предыдущего на длину \dfrac{l}{n}, где l {{---}} длина окружности. Рассмотрим первое построенное остовное дерево.(рис.3) В нем не более 2-х ребер имеют одинаковую длину дуги (длина дуги у ребра, расположенного на диаметре окружности, не совпадает с длиной дуги любого другого ребра данного остовного дерева). Значит, повороты только этих ребер могут совпасть между собой. 
#Докажем, что повороты ребра, расположенного на диаметре окружности, не совпадут друг с другом (если n нечетно, то такого ребра не будет). Чтобы хоть какой-то поворот совпал, мы должны повернуть ребро на 180 ^{\circ}. Каждый раз мы поворачиваем ребро на \dfrac{360 ^{\circ}}{n}. А так как мы поворачиваем ребро не более чем \dfrac{n}{2} - 1 раз, то в сумме мы повернем его на \dfrac{360 ^{\circ}}{n} \cdot (\dfrac{n}{2} - 1 ) =180 ^{\circ} - \dfrac{360 ^{\circ}}{n} . А это значит, что никакие ребра не совпадут друг с другом. (рис.4) 
#Докажем для остальных ребер. (рис.5) Возьмем ребро, которое не лежит на диаметре окружности. В данном остовном дереве есть ребро, которое имеет такую же длину дуги. Ориентируем данные ребра в сторону часовой стрелки. Чтобы повороты этих ребер совпали, нужно, чтобы совпали их начала и концы. Покажем, что их начала никогда не совпадут. Чтобы начало первого ребра совпало с началом второго, нужно первое ребро повернуть хотя бы на половину длины окружности, то есть на \dfrac{l}{2}. Для этого нам нужно сделать \dfrac{n}{2} поворотов: \dfrac{l}{n} \cdot \dfrac{n}{2} = \dfrac{l}{2}. Но мы делаем только \dfrac{n}{2} - 1 поворот. Аналогично с поворотом второго ребра. Для нечетных n граф будет неполным, поэтому даже \dfrac{n}{2} поворотов может не хватить для совпадения ребер.

==См. также== 
*[[Остовные деревья: определения, лемма о безопасном ребре]] 
*[[Остовное дерево в планарном графе]] 
*[[Минимально узкое остовное дерево]] 

==Источники информации==
*Карпов Д. В. {{---}} Теория графов, стр 297
 
[[Категория: Алгоритмы и структуры данных]] 
[[Категория: Остовные деревья]] 
[[Категория: Построение остовных деревьев]]