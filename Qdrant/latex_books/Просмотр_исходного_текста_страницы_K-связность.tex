k-cвязность {{---}} одна из топологических характеристик графа.
{{Определение
|id=def_1
|definition=
Граф называется '''вершинно k-связным''', если удаление любых (k - 1) вершин оставляет граф связным.
}}

[[Вершинная, реберная связность, связь между ними и минимальной степенью вершины|Вершинной связностью]] графа называется
 \varkappa (G) = \max \{ k \mid G вершинно k-связен \} , при этом для полного графа полагаем \varkappa (K_n) = n - 1 .

{{Определение
|id=def_2
|definition=
Граф называется '''реберно l-связным''', если удаление любых (l - 1) ребер оставляет граф связным. 
}}

[[Вершинная, реберная связность, связь между ними и минимальной степенью вершины|Реберной связностью]] графа называется \lambda(G) = \max \{ l \mid G реберно l-связен \} , для тривиального графа считаем \lambda (K_1) = 0 . 

==k-связность и непересекающиеся пути между вершинами==

Рассмотрим граф G и вершины u и v .

Пусть S {{---}} множество вершин/ребер/вершин и ребер.

 S разделяет u и v , если u и v принадлежат разным компонентам связности графа G \setminus S , который получается удалением элементов множества S из G .

Из теоремы [[Теорема Менгера, альтернативное доказательство|теоремы Менгера для вершинной k-связности]] имеем, что наименьшее число вершин, разделяющих две несмежные вершины u и v , равно наибольшему числу простых путей, не имеющих общих вершин, соединяющих u и v .

Отсюда непосредственно следует:

{{Утверждение
|statement=
Граф G является '''вершинно k-связным ''' \Leftrightarrow любая пара его вершин соединена по крайней мере k вершинно непересекающимися путями.
}}

Подобная теорема справедлива и для реберной связности. То есть из [[Теорема Менгера, альтернативное доказательство|''теоремы Менгера для реберной k-связности'']] следует:

{{Утверждение
|statement=
Граф   G является '''реберно l-связным''' \Leftrightarrow любая пара его вершин соединена по крайней мере l-реберно непересекающимися путями.
}}

==См. также==
* [[Теорема Менгера]]
* [[Теорема Менгера, альтернативное доказательство]]

==Источники информации==

* Харари Ф. Теория графов.[1] — М.: Мир, 1973. (Изд. 3, М.: КомКнига, 2006. — 296 с.)
* Форд Л., Фалкерсон Д., Потоки в сетях, пер. с англ., М., 1966
[[Категория:Алгоритмы и структуры данных]]
[[Категория:Связность в графах]]
{{Заголовок со строчной буквы}}