{{Определение
|id = de_bruijn_graph
|definition = '''Графом де Брюина''' (англ. ''De Bruijn graph'') с параметром l для n-буквенного алфавита называется ориентированный граф G(V, E), где V - множество всех слов длины l в заданном алфавите, и (u, v) \in E \Leftrightarrow \exists слово L длины l+1 в заданном алфавите такое, что u = prefix(L) и v = suffix(L) . Обозначается как B(n, l) 
}}

== Основные свойства ==

{{Лемма
|id = lem1
|about = об эйлеровости графа
|statement= B(n, l)\ - эйлеров
|proof= Ориентированный граф является эйлеровым, если число входящих рёбер равно числу исходящих. Докажем, что это верно для B(n, l) . А именно, что \forall v \in V верно, что deg_{out}(v) = n = deg_{in}(v). Докажем первое равенство, второе аналогично. Существует ровно n символов алфавита, которые можно добавить в конец слова \alpha, соответствующему вершине v . Получим ровно n различных слов. И у всех этих слов различные суффиксы длины l . Таким образом, из вершины v выходит ровно n рёбер и входит тоже n рёбер. Значит, граф де Брюина - эйлеров.
}}

{{Лемма
|id = lem2
|about = о количестве вершин и рёбер в графе
|statement= В B(n, l) \ |V| = n^l, |E| = n^{l+1}
|proof= Число вершин очевидно находится из определения графа и равно n^l . Число рёбер следует из доказательства предыдущей леммы: каждой вершине инцидентно ровно 2n ребер. Значит, |E| = n^{l+1} по [[Лемма о рукопожатиях | лемме о рукопожатиях]].
}}

{{Лемма
|id = lem3
|about = о равносильном определении
|statement= В B(n, l) \ (u, v) \in E \Leftrightarrow prefix_{l-1}(v) = suffix_{l-1}(u) 
|proof= \Leftarrow 
Составим слово длины l+1 \ L = a \gamma b , тогда \gamma b = suffix(L), a \gamma = prefix(L) . Если выбрать a, b \ как первый и последний символ слов u, v соответственно, и взять \gamma = prefix_{l-1}(v) = suffix_{l-1}(u) , то ребро между этими вершинами есть по определению. 

 \Rightarrow Возьмём подстроку слова L (из определения) без крайних символов (её длина l - 1 ). Так же из определения следует, что это будет суффиксом строки, соответствующей вершине u , и префиксом для строки, соответствующей v .
}}

{{Лемма
|id = lem4
|about = о графе с l = 1 
|statement= B(n, 1)\ - полный граф.
|proof = Действительно, для любых (необязательно различных) вершин u, v \ \exists L = \alpha \beta , где \alpha, \beta - слова (символы), соответствующие вершинам u, v . И тогда очевидно, что существует ребро (u, v)\ \forall u, v \in V .
}}

== Алгоритм построения ==

{{Задача
|definition =
Дан алфавит длины n и длина слов l . Построить по ним граф де Брюина.
}}
'''Алгоритм: '''

1. Создаём пустой граф из n^l вершин. Установим в алфавите отношение порядка и будем рассматривать его символы как цифры в n -значной системе счисления.

2. Генерируем минимальное в лексикографическом порядке слово длины l+1 , которое ещё не было использовано (порядок может быть любым, главное перебрать все такие слова без повторений).

3. Считаем префикс pref и суффикс suff длины l для текущего.

4. Проводим ребро (pref, suff) в графе. Переходим к пункту 2 , пока не будут перебраны все слова длины l+1 .

'''Корректность''': перебраны все слова длины l+1 , следовательно, были рассмотрены все возможные пары вершин, между которыми проведено ребро.

'''Время работы''': O(n^{l+1} \cdot substring) = O(|E| \cdot l) , где substring - время генерации слова, а так же поиска префикса и суффикса в нём.

== Применение графов де Брюина ==

{{Задача
|definition =
Известно, что пароль имеет длину l , и состоит из цифр от 1 до n . Требуется вывести кратчайшую последовательность цифр, которая гарантированно содержит пароль как подстроку.
}}

[[Файл: De_brujin_binary_graph.png‎|справа|400px|thumb|Построенный граф Де Брюина для двоичного алфавита со словами L (из определения) на рёбрах]]

'''Решение''':

1. Составим граф де Брюина (n, l-1) .

2. Найдём в построенном графе эйлеров цикл. Он существует, так как граф де Брюина эйлеров по первой лемме.

3. Слово, соответствующее первой вершине цикла, возьмём полностью ( l - 1 символов), затем будем последовательно добавлять в конец строки последний символ слова вершины, в которую был осуществлён переход. Так как рёбер n^l , получим последовательность длиной n^l + l - 1 .

'''Корректность''':

1. Очевидно, что последовательность меньшей длины составить нельзя: в полученной последовательности ровно n^l подстрок длины l , и именно столько чисел можно составить из цифр от 1 до n .

2. Двум разным рёбрам (u_1, v_1), (u_2, v_2) соответствует два ''различных'' слова L длины l . Иначе u_{1} = prefix(L) = u_{2} и v_{1} = suffix(L) = v_{2} . То есть это одно и то же ребро, при этом кратных рёбер в графе нет.

'''Вывод''':

Из доказательства корректности следует, что в последовательности содержится n^l различных подстрок длины l . И короче последовательность получить нельзя. Мы получили ответ за O((|E| \cdot (l-1)) + (|E| + |V|)) = O(|E| \cdot (l-1)) , то есть за время построения графа де Брюина (n, l-1) .

{{Задача
|definition = 
Даны неповторяющиеся последовательности ''нуклеотидов'' (''риды'') длины l . Известно, что все подстроки генома длины l входят в данное множество ридов. Построить возможный геном.
}}

[[Файл: De Brujin Graph In Science.png‎|справа|400px|thumb|Подграф графа Де Брюина для ридов, указанных на рёбрах. Искомый геном: '''ACGTACTAT''']]

'''Пояснение''':

Геном (как и его части - риды) является словом из 4-символьного алфавита \{A, G, C, T\} , где символы так же называются нуклеотидами. В реальности длины ридов находятся обычно в диапазоне 100 - 1000 нуклеотидов, а геном может содержать от 10^6 нуклеотидов у простейших организмов. При этом учёные могут получать информацию только о ридах (в силу размера последовательностей) физическим путем (''метод секвенирования'').

'''Решение''':

Решение этой задачи очень простое после решения предыдущей задачи. Построим граф G , где вершинами будут являться суффиксы и префиксы длины l - 1 всех ридов. Получили подграф графа де Брюина (4, l-1) (подграф, так как в нём есть не обязательно все 4^{l - 1} вершины), где каждому ребру соответствует рид. Найдём в нём эйлеров путь. Он существует, так как на геном было наложено условие о том, что все его подстроки длины l входят в наше множество ридов. Этот путь является возможным ответом. Очевидно, что единственно верный ответ (коим является реальный геном реального существа) получить можно не всегда, так как не всегда в графе есть единственный эйлеров путь.

'''Комментарий''':

К сожалению, кроме того, что алгоритм работает за O(4^l \cdot l) , в реальности есть немало технических проблем:

1. Как можно догадаться из пояснения: едва ли риды обязательно будут неповторяющимися. 

2. Данные о ридах могут быть получены лишь с некоторой вероятностью (как правило, ошибка в нуклеотиде имеет вероятность около 0.001 , но вблизи "края" генома она может достигать и 0.3 ).

3. Риды не могут иметь фиксированную длину в силу особенностей метода секвенирования.

'''Вывод''':

Несмотря на то, что задача не решается в общем случае приведённым алгоритмом, граф де Брюина действительно используется в ''ассемблерах'' (программах, собирающих геном или его большие части из ридов), но с заметными усложнениями.

== См. также ==

* [[Основные определения, связанные со строками]]
* [[Эйлеровость графов]]
* [[Алгоритм построения Эйлерова цикла]]

== Источники информации ==

* [https://en.wikipedia.org/wiki/De_Bruijn_graph Wikipedia {{---}} De Bruijn graph]
* [http://se.math.spbu.ru/SE/diploma/2011/Nurk%20Sergej%20-%20text.pdf Нурк Сергей {{---}} Разработка алгоритмов обработки графа де Брюина в задаче геномного ассемблирования]

[[Категория: Дискретная математика и алгоритмы]]
[[Категория: Обходы графов]]