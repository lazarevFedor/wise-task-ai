== Определения для ориентированного и неориентированного графов ==

{{Определение
|definition=
'''Матрицей инцидентности''' (инциденций) ''(англ. Incidence matrix)'' неориентированного графа называется матрица I (|V| \times |E|), для которой I_{i,j} = 1, если вершина v_i инцидентна ребру e_j, в противном случае I_{i,j} = 0.
}}

{{Определение
|definition=
'''Матрицей инцидентности''' (инциденций) ''(англ. Incidence matrix)'' ориентированного графа называется матрица I (|V| \times |E|), для которой I_{i,j} = 1, если вершина v_i является началом дуги e_j, I_{i,j} = -1, если v_i является концом дуги e_j, в остальных случаях I_{i,j} = 0.
}}

== Свойства ==
{{Утверждение
|statement=Для неориентированных графов без петель и кратных рёбер матрица инцидентности бинарна (состоит из нулей и единиц).
}}

{{Утверждение
|statement=Для ориентированных графов без петель и кратных рёбер матрица инцидентности состоит из нулей, единиц и -1.
}}

{{Утверждение
|about=о сумме элементов строки матрицы инцидентности для неориентированного графа
|statement=Сумма элементов i-й строки равна deg \; v_i.
}}

{{Утверждение
|about=о сумме элементов строки матрицы инцидентности для ориентированного графа
|statement=Сумма элементов i-й строки равна deg^+ v_i - deg^- v_i.
}}

== Пример ==
{| border="1" cellpadding="5" cellspacing="0" style="text-align:center"
!style="background:#f2f2f2"|Граф
!style="background:#f2f2f2"|Матрица инцидентности
!style="background:#f2f2f2"|Ориентированный граф
!style="background:#f2f2f2"|Матрица инцидентности
|-
|style="background:#f9f9f9"|[[Файл:incidence_matrix_undirected_graph.png|200px]]
|style="background:#f9f9f9"|\begin{pmatrix}
1 & 1 & 1 & 0 & 1 & 0\\
1 & 0 & 0 & 1 & 0 & 0\\
0 & 1 & 0 & 0 & 0 & 1\\
0 & 0 & 1 & 0 & 0 & 1\\
0 & 0 & 0 & 1 & 1 & 0\\
\end{pmatrix}
|style="background:#f9f9f9"|[[Файл:incidence_matrix_directed_graph.png|200px]]
|style="background:#f9f9f9"|\begin{pmatrix}
-1 & 1 & -1 & 0 & -1 & 0\\
1 & 0 & 0 & -1 & 0 & 0\\
0 & -1 & 0 & 0 & 0 & 1\\
0 & 0 & 1 & 0 & 0 & -1\\
0 & 0 & 0 & 1 & 1 & 0\\
\end{pmatrix}
|}

== См. также ==
* [[Матрица смежности графа]]

==Источники==
* Харари Фрэнк :'''Теория графов'''. Под ред. Л. Б. Штейнпресс. Изд. 2-е. — М.: Мир, 1973. — 180 с. — ISBN 5-354-00301-6
* Асанов М. О., Баранский В. А., Расин В. В.: '''Дискретная математика: графы, матроиды, алгоритмы''' — НИЦ РХД, 2001. — 288 с. — ISBN 5-93972-076-5
* [https://en.wikipedia.org/wiki/Incidence_matrix Википедия {{---}} Incidence matrix]

[[Категория: Алгоритмы и структуры данных]]
[[Категория: Основные определения теории графов]]