==Минимальное вершинное покрытие==
{{Определение|definition=
'''Вершинным покрытием''' ''(англ. vertex covering)'' графа G=(V,E) называется такое подмножество S множества вершин графа V, что любое ребро этого графа инцидентно хотя бы одной вершине из множества S.
}}
{{Определение|definition=
'''Минимальным вершинным покрытием''' ''(англ. minimum vertex covering)'' графа G=(V,E) называется вершинное покрытие, состоящее из наименьшего числа вершин. 
}}

[[Файл:Cover.jpg|left|thumb|300px|Множество вершин красного цвета — минимальное вершинное покрытие.]]

===Теорема о мощности минимального вершинного покрытия и максимального паросочетания===
{{Определение|definition=
'''Максимальным''' [[Теорема_о_максимальном_паросочетании_и_дополняющих_цепях|'''паросочетанием''']] ''(англ. maximum matching)'' в [[Двудольные графы и раскраска в 2 цвета|двудольном графе]] G называется паросочетание максимальной мощности.
}}

{{Теорема
|author=Кёниг
|neat = neat|statement=
В произвольном двудольном графе мощность максимального паросочетания равна мощности минимального вершинного покрытия.
|proof=
Пусть в G построено максимальное паросочетание. Ориентируем ребра паросочетания, чтобы они шли из правой доли в левую, ребра не из паросочетания — так, чтобы они шли из левой доли в правую. Запустим [[Обход_в_глубину,_цвета_вершин|обход в глубину]] из всех не насыщенных паросочетанием вершин левой доли. Разобьем вершины каждой доли графа на два множества: те, которые были посещены в процессе обхода, и те, которые не были посещены в процессе обхода.
Тогда L = L^+ \cup L^-, R = R^+ \cup R^-, где L, R — правая и левая доли соответственно, L^+, R^+ — вершины правой и левой доли, посещенные обходом, L^-, R^- — не посещенные обходом вершины.
Тогда в G могут быть следующие ребра:
[[Файл:bipartdfs_right.jpg|thumb|center|300px|Доли L^+, L^-, R^+, R^- и ребра между ними.]]
*Из вершин L^+ в вершины R^+ и из вершин R^+ в вершины L^+.
*Из вершин L^- в вершины R^- и из вершин R^- в вершины L^-. 
*Из вершин L^- в вершины R^+. 

Очевидно, что ребер из L^+ в R^- и из R^+ в L^- быть не может.
Ребер из R^- в L^+ быть не может, т.к. если такое ребро uv существует, то оно — ребро паросочетания. Тогда вершина v насыщена паросочетанием. Но т.к. v \in L^+, то в нее можно дойти из какой-то ненасыщенной вершины левой доли. Значит, существует ребро wv, w \in R^+. Но тогда v инцидентны два ребра из паросочетания. Противоречие. 

Заметим, что минимальным вершинным покрытием G является либо L, либо R, либо L^- \cup R^+.
В R^+ не насыщенных паросочетанием вершин быть не может, т.к. иначе в G существует дополняющая цепь, что противоречит максимальности построенного паросочетания.
В L^- свободных вершин быть не может, т.к. все они должны находиться в L^+. Тогда т.к. ребер из паросочетания между R^+
и L^- нет, то каждому ребру максимального паросочетания инцидентна ровно одна вершина из L^- \cup R^+. 
Тогда |L^- \cup R^+| равна мощности максимального паросочетания. Множество вершин L^- \cup R^+ является минимальным вершинным покрытием. Значит мощность максимального паросочетания равна мощности минимального вершинного покрытия.
}}

===Алгоритм построения минимального вершинного покрытия===
Из доказательства предыдущей теоремы следует алгоритм поиска минимального вершинного покрытия графа:
#Построить максимальное паросочетание.
#Ориентировать ребра:
#*Из паросочетания — из правой доли в левую.
#*Не из паросочетания — из левой доли в правую.
#Запустить обход в глубину из всех свободных вершин левой доли, построить множества L^+,L^-,R^+,R^-.
#В качестве результата взять L^- \cup R^+.

==См. также ==
*[[Теорема_о_максимальном_паросочетании_и_дополняющих_цепях|Теорема о максимальном паросочетании и дополняющих цепях]]
*[[Связь_вершинного_покрытия_и_независимого_множества|Связь вершинного покрытия и независимого множества]]

==Источники информации==
* [http://en.wikipedia.org/wiki/K%C3%B6nig's_theorem_(graph_theory) Википедия {{---}} Теорема Кёнига]

[[Категория: Алгоритмы и структуры данных]]
[[Категория: Задача о паросочетании]]