Рассмотрим [[Случайные графы | биномиальную модель]] случайного графа \(G(n,p)\) .

{{Определение
|definition =
Подмножество \(\mathcal{A}\) всех графов на \(n\) вершинах называется '''свойством''' (англ. ''graph property'').
}}

{{Определение
|definition =
Свойство называется '''нетривиальным''' (англ. ''non-trivial property''), если существуют графы, как удовлетворяющие ему, так и нет.
}}

{{Определение
|definition =
Свойство \(\mathcal{A}\) называется '''монотонным''' (англ. ''monotone property''), если для любой пары \(G_1,G_2\in G(n,p),\,E(G_1)\subset E(G_2)\) выполнено следствие: \(G_1\in \mathcal{A}\Rightarrow G_2\in\mathcal{A}\) }}

{{Определение
|definition =
Функция \(p_0(n)\) называется '''пороговой''' для монотонного свойства \(\mathcal{A}\) (англ. ''threshold''), если для любой функции вероятности \(p(n)\) выполнено:
* \(P(G(n,p(n))\in\mathcal{A})\xrightarrow[n \to \infty]{} 0\) , если \(p(n)/p_0(n)\xrightarrow[n \to \infty]{} 0\) * \(P(G(n,p(n))\in\mathcal{A})\xrightarrow[n \to \infty]{} 1\) , если \(p(n)/p_0(n)\xrightarrow[n \to \infty]{} \infty\) }}
Мы уже знакомы с некоторыми [[Случайные графы | пороговыми функциями]].

{{Лемма
|about=о монотонности вероятности
|statement=Для любого монотонного свойства \(\mathcal{A}\) верно следствие: \(p_2\geqslant p_1\Rightarrow P(G(n,p_2)\in \mathcal{A})\geqslant P(G(n,p_1)\in \mathcal{A})\) |proof=
Пусть \(p=\dfrac{p_2-p_1}{1-p_1}\in[0,1]\) . Выберем случайно граф \(G_1\) из \(G(n,p_1)\) , затем \(G\) из \(G(n,p)\) и рассмотрим \(G_1\cup G\) .
В нем с вероятностью \((1-p)(1-p_1)=(1-p_2)\) не будет фиксированного ребра, как и в графе \(G_2\) . Мы смогли представить выбор \(G_2\) как объединение, один из которых {{---}} выбор графа \(G(n,p_1)\) , значит \(P(G(n,p_2))\geqslant P(G(n,p_1))\) }}

{{Теорема
|author=Bollob́as-Thomason
|statement=Любое нетривиальное монотонное свойство имеет пороговую функцию.
|proof=
Сначала найдем эту пороговую функцию. Зафиксируем \(n\) .
* Заметим, что функция \(f(p)=P(G(n,p)\in\mathcal{A})\) непрерывна. На самом деле это многочлен, у которого степень оценивается как количество ребер в полном графе.
:* Вероятность получить конкретный граф равна \(p^\alpha\cdot(1-p)^\beta\) , где \(\alpha\) и \(\beta\) {{---}} количество присутствующих и отсутствующих ребер соответственно ( \(\alpha+\beta=C_n^2\) ).
:* Чтобы получить \(f(p)\) , нужно просуммировать такие многочлены по всем графам из \(\mathcal{A}\) .
* \(f(0)=0\) , \(f(1)=1\) , так как свойство нетривиальное и монотонное (то есть пустой граф точно не удовлетворяет ему, тогда как полный должен удовлетворять).
* По теореме Больцано-Коши найдется такое \(p_0\) , что \(P(G(n,p_0)\in\mathcal{A})=1/2\) .
* Мы по \(n\) научились находить такую вероятность \(p_0\) , что \(P(G(n,p_0)\in\mathcal{A})=1/2\) . Теперь проделаем это для каждого \(n\in\mathbb{N}\) и получим функцию \(p_0(n)\) . Она окажется пороговой для свойства \(\mathcal{A}\) .

Докажем это.

Пусть \(\dfrac{p(n)}{p_0(n)}\xrightarrow[n \to \infty]{} \infty\) . Докажем, что \(P(G(n,p(n))\in\mathcal{A})\xrightarrow[n \to \infty]{} 1\) .

Докажем, что неравенство \(P(G(n,p(n))\in\mathcal{A})>P(G(n,1-(1-p_0(n))^m)\in\mathcal{A})\) верно при достаточно больших \(n\) .
* Для этого по лемме о монотонности вероятности достаточно установить: \(p(n)>1-(1-p_0(n))^m\) .
:* \(p(n)>m p_0(n)\) . Неравенство верно с некоторого момента, так как \(p\gg p_0\) по предположению.
:* \(m p_0(n)\geqslant 1-(1-p_0(n))^m\) . Это неравенство Бернулли. Оно верно при \((-p_0(n))>-1\) и \(m\notin[0,1]\) . Первое ограничение соблюдено, далее выберем \(m\) с учетом второго ограничения.

Выберем графы \(G_1,\ldots,G_m\in G(n,p_0(n))\) и рассмотрим \(H=G_1\cup G_2\cup\ldots\cup G_m\) .

Оказывается, что \(P(G(n,1-(1-p_0(n))^m)\in\mathcal{A})=P(H\in\mathcal{A})\) .
* Действительно, посмотрим, с какой вероятностью в \(H\) не окажется фиксированного ребра. Это будет тогда, когда во всех графах \(G_i\) не будет его, то есть \(P(\text{в }H\text{ нет ребра})=(1-p_0(n))^m\) . Тогда в \(H\) будет ребро с вероятностью \(1-(1-p_0(n))^m\) . Мы получили даже, что \(H\in G(n,1-(1-p_0(n))^m)\) .

Оценим вероятность принадлежности \(H\) свойству \(\mathcal{A}\) : \(P(H\in\mathcal{A})\geqslant 1-(1-P(G(n,p_0(n))\in\mathcal{A}))^m\) . 
* Перепишем неравенство с учетом \(P(H\in\mathcal{A})=1-P(H\notin\mathcal{A})\) : \((1-P(G(n,p_0(n))\in\mathcal{A}))^m\geqslant P(H\notin\mathcal{A})\) .
* Если мы покажем, что из правого события следует левое, то тогда докажем само неравенство.
* Справа {{---}} вероятность того, что граф \(H\) не попал в \(\mathcal{A}\) . Тогда (в силу монотонности свойства) и все его подграфы (в том числе и \(G_i\) ) тоже не попали в \(\mathcal{A}\) .
* А слева как раз и есть вероятность того, что все графы \(G_i\) не попали в \(\mathcal{A}\) .

По построению \(p_0(n)\) правую часть последнего неравенства можно легко посчитать: \(1-(1-P(G(n,p_0(n))\in\mathcal{A}))^m=1-(1-1/2)^m=1-1/2^m\) Совершим последнюю оценку: \(1-1/2^m>1-\varepsilon\) .
* Это равносильно \(m\geqslant \log_2{1/\varepsilon}\) . Положим \(m=\log_2{1/\varepsilon}+2\) . Тогда исходное неравенство верно, а также ограничение для неравенства Бернулли выполнено.

За несколько шагов мы показали, что неравенство \(P(G(n,p(n))\in\mathcal{A})>1-\varepsilon\) выполняется с некоторого момента. Это и значит \(P(G(n,p(n))\in\mathcal{A})\xrightarrow[n \to \infty]{} 1\) Теперь пусть \(\dfrac{p(n)}{p_0(n)}\xrightarrow[n \to \infty]{} 0\) . Докажем, что \(P(G(n,p(n))\in\mathcal{A})\xrightarrow[n \to \infty]{} 0\) .

Зафиксируем \(\varepsilon>0\) .
* (1) Так как \(1-\varepsilonm p(n)\geqslant 1-(1-p(n))^m\) .

Выберем \(m\) графов из \(G(n,p(n))\)  \(G_1,\ldots,G_m\) и рассмотрим \(H=G_1\cup G_2\cup\ldots\cup G_m\) . Как мы уже знаем, \(H\in G(n,1-(1-p(n))^m)\) . \((1-P(G(n,p(n))\in\mathcal{A}))^m=P(\forall\,i\colon G_i\notin\mathcal{A})\geqslant P(H\notin\mathcal{A})\geqslant1/2\) .
* Из нового только последнее неравенство, остальное уже доказано.
* Оно следует из $P(H\in\mathcal{A})=P(G(n,1-(1-p(n))^m)\in\mathcal{A}) \(1/2>(1-\varepsilon)^m\) по (1). Тогда \((1-P(G(n,p(n))\in\mathcal{A}))^m>(1-\varepsilon)^m\Rightarrow P(G(n,p(n))\in\mathcal{A})<\varepsilon\) Мы по \(\varepsilon\) научились понимать, что \(P(G(n,p_0(n))\in\mathcal{A})<\varepsilon\) верно с некоторого момента, что и означает сходимость.
}}