{{Определение
|definition = 
 Пара \langle A, b\rangle называется '''неустойчивой''' (англ. ''unstable pair''), если:
# В паросочетании есть пары \langle A, a\rangle и \langle B, b\rangle (A женат на a, B женат на b);
# A предпочитает b элементу a;
# b предпочитает A элементу B.
}}
{{Определение
|definition='''Устойчивое паросочетание''' (англ. ''stable matching'') — [[Паросочетания: основные определения, теорема о максимальном паросочетании и дополняющих цепях| паросочетание]] без неустойчивых пар.
}}
{{Задача
|definition=
Найти полное устойчивое паросочетание между элементами двух множеств размера n, имеющими свои предпочтения.}}

== Основная задача ==

Есть n мужчин и n женщин. Они обладают следующими особенностями:
# Каждый человек оценивает лишь людей противоположного пола (все гетеросексуальны);
# Каждый мужчина может отсортировать женщин от ''наименее привлекательной'' к ''наиболее привлекательной'', причем его предпочтения не меняются (у каждого мужчины своя функция оценки);
# Каждая женщина может отсортировать мужчин от ''наименее привлекательного'' к ''наиболее привлекательному'', причем её предпочтения не меняются (у каждой женщины своя функция оценки).

Очевидным образом по такому определению строится [[Двудольные_графы| полный двудольный граф]] (левая доля — мужчины, правая — женщины), назовем его МЖ.

Рассмотрим некоторое [[Паросочетания: основные определения, теорема о максимальном паросочетании и дополняющих цепях| паросочетание]]
в МЖ.

== Алгоритм Гейла-Шепли ==

Решение задачи было описано в 1962 году математиками Девидом Гейлом (Университет Брауна) и Ллойдом Шепли (Принстонский университет) в статье «Поступление в колледж и стабильность браков» (College admissions and the stability of marriage) в журнале American Mathematical Monthly
https://ru.wikipedia.org/wiki/American_Mathematical_Monthly American Mathematical Monthly 69, 9-14, 1962.. Набор правил, следование которым всегда приводит к образованию стабильных пар, получил название алгоритма Гейла-Шепли или «алгоритма отложенного согласия» (алгоритм предложи-и-откажи).

=== Интуитивное описание ===

# Мужчины делают предложение наиболее предпочитаемой женщине.
# Каждая женщина из всех поступивших предложений выбирает наилучшее и отвечает на него «может быть» (помолвка), на все остальные отвечает «нет» (отказ).
# Мужчины, получившие отказ, обращаются к следующей женщине из своего списка предпочтений, мужчины, получившие ответ «может быть», ничего не делают.
# Если женщине пришло предложение лучше предыдущего, то она прежнему претенденту (которому ранее сказала «может быть») говорит «нет», а новому претенденту говорит «может быть».
# Шаги 1-4 повторяются, пока у всех мужчин не исчерпается список предложений, в этот момент женщины отвечают «да» на те предложения «может быть», которые у них есть в настоящий момент.

=== Описание в псевдокоде ===

 // Изначально все мужчины не женаты и все женщины незамужние.
 '''while''' существует свободный мужчина
 M = некоторый свободный мужчина
 w = первая женщина из текущего списка M
 '''if''' w свободна
 помечаем M и w помолвленными
 '''else if''' w предпочитает M своему текущему жениху M'
 помечаем M и w помолвленными
 вычёркиваем w из списка предпочтений M'
 помечаем M' свободным
 '''else'''
 вычёркиваем w из списка предпочтений M

Время работы алгоритма {{---}} O(n^2), так как количество итераций цикла \mathrm {while} не превосходит O(n^2), где n равно размеру каждого из данных множеств.

=== Доказательство корректности ===

{{Утверждение
|id=observation1
|about=Наблюдение 1
|statement=Мужчины делают предложения женщинам в порядке убывания симпатии.
}}

{{Утверждение
|id=observation2
|about=Наблюдение 2
|statement=Как только женщина была помолвлена, она не может стать непомолвленной, она может только улучшить свой выбор (сказать «может быть» более предпочтительному кандидату).
}}

Для начала покажем, что алгоритм завершит свою работу.

{{Лемма
|id=lemma1
|about=Лемма 1
|statement=
 Алгоритм завершается после максимум n^2 итераций цикла \mathrm{\mathbf{while}}.
|proof=
 На каждой итерации мужчина делает предложение очередной женщине. Но всего может быть не более n^2 предложений.
}}

Теперь покажем, что по завершении алгоритма задача будет решена.

{{Лемма
|id=lemma2
|about=Лемма 2
|statement=
 Все мужчины и женщины будут заняты.
|proof=
Предположим, что некоторый мужчина (A) не женат по завершении алгоритма. Тогда и некоторая женщина (a) незамужняя. По [[#observation2|наблюдению 2]], a не получала предложений. Но A сделал предложения всем женщинам, так как он остался не женат. Получаем противоречие. Таким образом, все мужчины заняты.

Аналогичные рассуждения применяем для женщин. Пусть какая-то женщина незамужняя. Значит, есть мужчина, который остался не женат. Но мы доказали, что по завершении алгоритма все мужчины заняты. Снова пришли к противоречию.
}}

{{Лемма
|id=lemma3
|about=Лемма 3
|statement=
 После завершения алгоритма не будет неустойчивых пар.
|proof=
Предположим \langle A, b\rangle (где A, B — мужчины; a, b — женщины; A женат на a, B женат на b) — нестабильная пара в паросочетании, найденном алгоритмом Гейла-Шепли. Возможны два случая:
# A не делал предложение b. Значит, A находит a более привлекательной, чем b. Но чтобы рассматриваемая пара была неустойчивой, необходимо, чтобы b для A была более привлекательна, чем a. Следовательно, \langle A, b\rangle — устойчивая пара.
# A делал предложение b. Тогда был такой момент, когда b отказала A, значит, b находит B более привлекательным, чем A. Снова получается, что \langle A, b\rangle — устойчивая пара.
 
}}

=== Анализ полученного алгоритмом паросочетания ===

Алгоритм Гейла-Шепли гарантирует, что будет найдено некоторое решение задачи. Но решений может быть более одного. Зададимся вопросом, какими свойствами обладает решение, найденное алгоритмом.

{{Лемма 
|id=lemma4
|about=man-optimality
|statement=
 Из всех возможных решений алгоритмом Гейла-Шепли будет найдено решение, наилучшее для мужчин (каждый мужчина получает в жены женщину, наилучшую из всех возможных при условии корректности решения).
|proof=
Докажем от противного, что для каждого мужчины не существует устойчивого паросочетания, в котором его супругой была бы более желанная для него женщина. 

Предположим, для мужчины A это свойство не выполняется. Так как он оказался женат не на лучшей из кандидатур, то существует женщина a, которая предпочла ему другого, более привлекательного мужчину B, при этом женщина a для мужчины B стоит на первом месте в его текущем списке. Предположим, существует устойчивое паросочетание, содержащее \langle A, a\rangle. По определению, в устойчивом паросочетании нет неустойчивых пар. Пара \langle B, a\rangle станет неустойчивой, если B будет предпочитать a своей супруге. Значит, B женат на ком-то, кто лучше, чем a. Но такое невозможно, так как a стоит для него на первом месте. Таким образом, если женщина a вычёркивается из списка предпочтений мужчины A, то любое паросочетание, содержащее \langle A, a\rangle, неустойчиво.

}}

{{Лемма 
|id=lemma5
|about=woman-pessimality
|statement=
 Из всех возможных решений алгоритмом Гейла-Шепли будет найдено решение, наихудшее для женщин.
|proof= 
Пусть A, B — мужчины; c, d — женщины; A женат на d, B женат на c.

Предположим, \langle A, c\rangle — стабильная пара в паросочетании S', найденном алгоритмом Гейла-Шепли, но A не самый худший выбор для c. Тогда существует стабильная пара в паросочетании S, в которой c замужем за B, который менее привлекателен, чем A. Тогда пусть мужем d будет A в паросочетании S. Получается A считает c более привлекательной, чем d. Соответственно \langle A, c\rangle {{---}} нестабильная пара в паросочетании S. То есть для c есть мужчина, который более привлекателен, чем её муж.

}}

== Обобщения задачи ==

Интересно, что данная задача не всегда имеет решение, если допустить однополые пары (устойчивого паросочетания может не быть) https://ru.wikipedia.org/wiki/Задача_о_соседях_по_комнате Задача о соседях по комнате.

Случай же, когда у нас есть N мужчин и M женщин (N \neq M) легко сводится к описанной выше задаче. Рассмотрим M > N (M аналогично). Добавим M - N фиктивных мужчин, которые являются наименее привлекательными с точки зрения каждой из женщин. Тогда если в найденном алгоритмом Гейла-Шепли паросочетании некоторая женщина будет замужем за таким фиктивным мужчиной, то это будет означать, что она на самом деле осталась без пары.

Также интересна задача о выборе учебного заведения: вместо множества мужчин введем множество университетов, а вместо множества женщин — множество кандидатов, подающих заявления на поступление. Причем в каждом университете есть квота на количество студентов, которое университет может принять. Задача очевидно сводится к основной добавлением (K-1) "филиалов" для каждого университета (K — квота). И добавлением фиктивного университета (поступление в который означает, что кандидату придется попробовать поступить через год).

== Применения в реальной жизни ==

Задача о нахождении устойчивого паросочетания и её решение имеют множество применений в реальной жизни, лишь некоторые из них:
* Распределение студентов по коллеждам в США
* Распределение интернов по больницам
* Распределение донорских органов по нуждающимся в них людям

Решение данной задачи было отмечено при вручении Нобелевской премии по экономике в 2012 году за «теорию стабильного распределения и практическое применение рыночных моделей». Её получили один из создателей алгоритма, Ллойд Шепли, а также Элвин Рот, во многом развивший исследования Ллойда Шепли и Дэвида Гейла. Сам Гейл не был удостоен премии, вероятно, лишь в силу того, что умер в 2008 году.

== Примечания ==

== См. также ==
* [[Паросочетания: основные определения, теорема о максимальном паросочетании и дополняющих цепях|Паросочетания]]
== Источники информации ==

* [http://www.cs.princeton.edu/courses/archive/spring05/cos423/lectures/01stable-matching.pdf Stable matching, Prinston lecture's presentation]
* [http://en.wikipedia.org/wiki/Stable_marriage_problem Stable marriage problem]
* [http://ru.wikipedia.org/wiki/%D0%97%D0%B0%D0%B4%D0%B0%D1%87%D0%B0_%D0%BE_%D0%BC%D0%B0%D1%80%D1%8C%D1%8F%D0%B6%D0%B5 Задача о марьяже]
* [http://ge.tt/api/1/files/4LU3zaD1/0/blob?download Устойчивость супружеских пар и другие комбинаторные задачи (Статья Дональда Кнута)]
* [http://kek.ksu.ru/eos/Lerner/KnuthRu.pdf Устойчивость супружеских пар и другие комбинаторные задачи - Казанский государственный университет]

[[Категория: Алгоритмы и структуры данных]]
[[Категория: Задача о паросочетании]]