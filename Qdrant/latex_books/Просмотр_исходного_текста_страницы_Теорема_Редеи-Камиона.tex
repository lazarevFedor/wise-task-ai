{{Теорема
|about=
Редеи-Камиона (для пути)
|statement=
В любом [[Турниры|турнире]] есть [[Гамильтоновы_графы#.D0.9E.D1.81.D0.BD.D0.BE.D0.B2.D0.BD.D1.8B.D0.B5_.D0.BE.D0.BF.D1.80.D0.B5.D0.B4.D0.B5.D0.BB.D0.B5.D0.BD.D0.B8.D1.8F|гамильтонов путь]].
|proof= 
Приведем доказательство по индукции по числу вершин в графе. Пусть n {{---}} количество вершин в графе.

 ''База индукции:'' 

Очевидно, для n = 3 утверждение верно.

 ''Индукционный переход:'' 

Пусть предположение верно для всех турниров с количеством вершин не более n . Рассмотрим турнир T с n + 1 вершинами.

Пусть u {{---}} произвольная вершина турнира T . Тогда турнир T - u имеет n вершин, значит, в нем есть гамильтонов путь P: (v_1 \rightarrow v_2 \rightarrow \ldots \rightarrow v_n) .
[[Файл: Redei_kamion_1.png|150px|thumb|center]]
 
Одно из рёбер (u, v_1) или (v_1, u) обязательно содержится в T .
Если ребро (u, v_1) \in ET , то путь (u \rightarrow P) {{---}} гамильтонов.
[[Файл: Redei_kamion_2.png|150px|thumb|center|Красным цветом выделен искомый путь]]

Пусть теперь ребро (u, v_1) \notin ET, v_i {{---}} первая вершина пути P , для которой ребро (u, v_i) \in T .
Если такая вершина существует, то в T существует ребро (v_{i - 1}, u) и путь (v_1 \rightarrow \ldots \rightarrow v_{i - 1} \rightarrow u \rightarrow v_i \rightarrow \ldots v_n) {{---}} гамильтонов.
[[Файл: Redei_kamion_3.png|180px|thumb|center|Красным цветом выделен искомый путь]]

Если такой вершины не существует, то путь (P \rightarrow u) {{---}} гамильтонов.
[[Файл: Redei_kamion_4.png|150px|thumb|center|Красным цветом выделен искомый путь]]

Значит, в любом случае в турнире существует гамильтонов путь, q.e.d.
}}

{{Теорема
|about=
Редеи-Камиона (для цикла)
|statement=
В любом [[Отношение_связности,_компоненты_связности#.D0.A1.D0.B8.D0.BB.D1.8C.D0.BD.D0.B0.D1.8F_.D1.81.D0.B2.D1.8F.D0.B7.D0.BD.D0.BE.D1.81.D1.82.D1.8C|сильно связанном]] турнире есть [[Гамильтоновы_графы#.D0.9E.D1.81.D0.BD.D0.BE.D0.B2.D0.BD.D1.8B.D0.B5_.D0.BE.D0.BF.D1.80.D0.B5.D0.B4.D0.B5.D0.BB.D0.B5.D0.BD.D0.B8.D1.8F|гамильтонов цикл]].
|proof= 
Приведем доказательство по индукции по числу вершин в цикле. Пусть n {{---}} количество вершин в графе.

 ''База индукции:'' 

{{Утверждение
|statement=
Cильно связанный турнир T из n \geqslant 3 вершин содержит [[Основные_определения_теории_графов#.D0.9E.D1.80.D0.B8.D0.B5.D0.BD.D1.82.D0.B8.D1.80.D0.BE.D0.B2.D0.B0.D0.BD.D0.BD.D1.8B.D0.B5_.D0.B3.D1.80.D0.B0.D1.84.D1.8B|цикл]] длины 3 .
|proof=
Пусть u {{---}} произвольная вершина турнира T . Множество вершин VT - u распадается на 2 непересекающихся множества:
* V_1 = \{ v_1 \in VT \mid (v_1, u) \in ET \} ,
* V_2 = \{ v_2 \in VT \mid (u, v_2) \in ET \} .
[[Файл: Redei_kamion_5.png|290px|thumb|center]]

 T сильно связен, следовательно:
# V_1 \neq \emptyset , иначе v {{---}} исток турнира
# V_2 \neq \emptyset , иначе v {{---}} сток турнира
# \exists e = (w_2, w_1) \in ET , иначе нет пути из V_2 в V_1 :
#* w_1 \in V_1 ,
#* w_2 \in V_2 .
[[Файл: Redei_kamion_6.png|290px|thumb|center|Красным цветом выделен цикл длины 3 ]]

Цикл S_3: (u \rightarrow w_2 \rightarrow w_1 \rightarrow u) {{---}} искомый цикл длины 3 , q.e.d.
}}

 ''Индукционный переход:'' 

{{Утверждение
|statement=
Если сильно связанный турнир T из n \geqslant 3 вершин содержит цикл S_k длины k, (k , то он содержит и цикл длины k + 1 .
|proof=
Пусть S_k = (v_1 \rightarrow v_2 \rightarrow \ldots \rightarrow v_k \rightarrow v_1) .

Пусть v_0 : v_0 \notin S_k и верно, что \exists u, w \in S_k :
* (v_0, u) \in ET ,
* (w, v_0) \in ET .

Рассмотрим два случая:
# существует такая вершина v_0 ,
# не существует такой вершины v_0 .
Заметим, что при k = n - 1 такая вершина необходимо существует, так как иначе вершина, не входящая в цикл, будет являться либо стоком, либо истоком.

 Первый случай: 

Перенумеруем вершины S_k так, чтобы ребро e = (v_1, v_0) \in ET для вершины v_1 \in S_k . Пусть v_i – первая вершина при обходе S_k из v_1 , для которой ребро f = (v_0, v_i) \in ET .
[[Файл: Redei_kamion_7.png|150px|thumb|center]]

Тогда ребро g = (v_{i - 1}, v_0) \in ET .
[[Файл: Redei_kamion_8.png|150px|thumb|center|Красным цветом выделен искомый цикл длины k + 1 ]]

Тогда S_{k + 1} = (v_1 \rightarrow v_2 \rightarrow \ldots \rightarrow v_{i - 1} \rightarrow v_0 \rightarrow v_i \rightarrow \ldots \rightarrow v_k \rightarrow v_1) – искомый цикл длины k + 1 .

 Второй случай: 

Пусть:
* V_1 = \{ u \in VT \mid u \notin S_k, e = (u, v_i) \in ET, \forall i = \overline{1, n} \} ,
* V_2 = \{ u \in VT \mid u \notin S_k, f = (v_i, u) \in ET, \forall i = \overline{1, n} \} .
Тогда V_1 \cap V_2 = \emptyset .
[[Файл: Redei_kamion_9.png|290px|thumb|center]]

Турнир сильно связен, следовательно:
* V_1 \neq \emptyset , иначе T не будет сильно связным, так как тогда нет простых путей с началом в V_2 и концом в S_k 
* V_2 \neq \emptyset , иначе T не будет сильно связным, так как тогда нет простых путей с началом в S_k и концом в V_1 
* \exists g = (w_2, w_1) \in ET , иначе T не будет сильно связным, так как тогда нет простых путей с началом в V_2 и концом в V_1 ):
** w_1 \in V_1 ,
** w_2 \in V_2 .
[[Файл: Redei_kamion_10.png|290px|thumb|center|Красным цветом выделен цикл длины k + 1 ]]
Тогда S_{k + 1} = (v_1 \rightarrow w_2 \rightarrow w_1 \rightarrow v_3 \rightarrow \ldots \rightarrow v_k \rightarrow v_1) – искомый цикл длины k + 1 .

В любом случае утверждение верно, q.e.d.

}}
Таким образом, любой сильно связанный турнир T с n \geqslant 3 вершинами содержит цикл длины n , то есть гамильтонов цикл, q.e.d.
}}

{{Теорема
|about=
Следствие
|statement=
Турнир является сильно связанным тогда и только тогда, когда он имеет гамильтонов цикл.
}}

==См. также==
* [[Гамильтоновы графы]]
* [[Турниры]]

== Источники информации ==
* Асанов М., Баранский В., Расин В.: ''Дискретная математика: Графы, матроиды, алгоритмы''
* Ф. Харари: ''Теория графов''

[[Категория: Алгоритмы и структуры данных]]
[[Категория: Обходы графов]]
[[Категория: Гамильтоновы графы]]