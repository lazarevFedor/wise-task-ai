== Определение сети ==

{{Определение
|id=flow_network
|definition=
'''Сеть''' (англ. ''flow network'') G=(V,E) представляет собой [[Основные определения теории графов#oriented_grath|ориентированный граф]], в котором каждое [[Основные определения теории графов#def_graph_edge_1|ребро]] (u,v)\in E имеет положительную '''пропускную способность''' (англ. ''capacity'') c(u,v)>0. Если (u,v)\notin E, предполагается что c(u,v)=0.
}}
В транспортной сети выделяются две вершины: '''исток''' s и '''сток''' t.

== Определение потока ==
{{Определение
|id=flow
|definition=
'''Потоком''' (англ. ''flow'') f в G является действительная функция f\colon V\times V\to R, удоволетворяющая условиям:
1) f(u,v)=-f(v,u) (антисимметричность);

2) f(u,v) \leqslant c(u,v) (ограничение пропускной способности), если ребра нет, то f(u,v)=0;

3) \sum\limits_v f(u,v)=0 для всех вершин u, кроме s и t (закон сохранения потока).
'''Величина''' потока f определяется как |f|=\sum\limits_{v\in V} f(s,v).
}}

Также существует альтернативное определение (по Асанову), не вводящее антисимметричность (зачастую, из-за этого с ним труднее работать):
{{Определение
|definition=
'''Потоком''' f в сети G=(V,E,c) называется функция f\colon E\to R, удоволетворяющая условиям:
1) 0 \leqslant f(e) \leqslant c(e) для всех e\in E;

2) f(v-) = f(v+) для всех v\in V, v\ne s, v\ne t, где f(v-)=\sum\limits_{w\in v-} f(w,v), f(v+)=\sum\limits_{w\in v+} f(v,w).
Здесь s {{ --- }} '''источник''', а t {{ --- }} '''сток''' сети G (s имеет нулевую степень захода, а t имеет нулевую степень исхода); через v+ обозначено множество вершин, к которым идут [[Основные определения теории графов#def_graph_edge_1|дуги]] из вершины v; через v- обозначено множество вершин, из которых идут дуги в вершину v; c(e) называется '''пропускной способностью''' дуги e и неотрицательно.
}}
Число f(v,w) можно интерпретировать, например, как количество жидкости, поступающей из v в w по дуге (v,w). С этой точки зрения значение f(v-) может быть интерпретировано как поток, втекающий в вершину v, а f(v+) {{---}} вытекающий из v .
Условие 1) называется условием ограничения по пропускной способности, а условие 2) {{---}} условием сохранения потока в вершинах; иными словами, поток, втекающий в вершину v , отличную от s или t , равен вытекающему из неё потоку.

== Пример ==
Пример сети с источником s и стоком t.

[[Файл:Flow-network.png|340px|center]]

Первое число означает величину потока, второе {{---}} пропускную способность ребра. Отрицательные величины потока не указаны (так как они мгновенно получаются из антисимметричности: f(u,v)=-f(v,u)). Сумма входящих рёбер везде (кроме источника и стока) равна сумме исходящих и на то, что в общем c(u,v) \neq c(v, u). Кроме того, величина потока на ребре никогда не превышает пропускную способность этого ребра.

Величина потока в этом примере равна 3 + 2 = 5 (считаем от вершины s).

== Источники информации ==
* ''Кормен, Томас Х., Лейзерсон, Чарльз И., Ривест, Рональд Л., Штайн Клиффорд'' '''Алгоритмы: построение и анализ''', 2-е издание. Пер. с англ. — М.:Издательский дом "Вильямс", 2010. — 1296 с.: ил. — Парал. тит. англ. — ISBN 978-5-8459-0857-5 (рус.)
* ''Асанов М. О., Баранский В. А., Расин В. В.'' — '''Дискретная математика: Графы, матроиды, алгоритмы: Учебное пособие.''' 2-е изд., испр. и доп. — СПб.: Издательство "Лань", 2010. — 368 с.: ил. — (Учебники для вузов. Специальная литература). ISBN 978-5-8114-1068-2
* [http://ru.wikipedia.org/wiki/Транспортная_сеть Википедия - Транспортная сеть]
* [http://en.wikipedia.org/wiki/Flow_network Wikipedia - Flow network]

[[Категория:Алгоритмы и структуры данных]]
[[Категория:Задача о максимальном потоке]]