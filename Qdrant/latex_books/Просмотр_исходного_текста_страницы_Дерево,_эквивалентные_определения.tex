{{Определение
|id = tree
|definition =
'''Дерево''' (англ. ''tree'') {{---}} связный ациклический [[Основные определения теории графов|граф]].

}}
[[Файл:tree_def_1.png|300px|Пример дерева]]

{{Определение
|definition =
'''Лес''' (англ. ''forest'') {{---}} [[Основные определения теории графов|граф]], являющийся набором непересекающихся деревьев.
}}
[[Файл:tree_def_2.png|400px|Примеры леса]]

==Определения==
Для графа G эквивалентны следующие утверждения:
# G — дерево.
# Любые две вершины графа G соединены единственным простым путем.
# G — связен и p = q + 1 , где p — количество вершин, а q количество ребер.
# G — ацикличен и p = q + 1 , где p — количество вершин, а q количество ребер.
# G — ацикличен и при добавлении любого ребра для [[Основные определения теории графов|несмежных вершин]] появляется один простой [[Основные определения теории графов|цикл]].
# G — связный граф, отличный от K_p для p > 3 , а также при добавлении любого ребра для несмежных вершин появляется один простой цикл.
# G — граф, отличный от K_3 \cup K_1 и K_3 \cup K_2 , а также p = q + 1 , где p — количество вершин, а q количество ребер, и при добавлении любого ребра для несмежных вершин появляется один простой цикл.

==Доказательство эквивалентности==
 1 \Rightarrow 2 
:Граф связен, поэтому любые две вершнины соединены путем. Граф ацикличен, значит путь единственен, а также [[Теорема о существовании простого пути в случае существования пути|прост]], поскольку никакой путь не может зайти в одну вершину два раза, потому что это противоречит ацикличности.

 2 \Rightarrow 3 
:Очевидно, что граф связен. Докажем по индукции, соотношение p = q + 1. Утверждение очевидно для связных графов с одной и двумя вершинами. Предположим, что оно верно для графов, имеющих меньше p вершин. Если же граф G имеет p вершин, то удаление из него любого ребра делает граф G несвязным в силу единственности простых цепей; более того, получаемый граф будет иметь в точности две компоненты. По предположению индукции в каждой компоненте число вершин на единицу больше числа ребер. Таким образом, p = q + 1 .

 3 \Rightarrow 4 
:Очевидно, что если граф связен и ребер на одно меньше, чем вершин, то он ацикличен. Преположим, что у нас есть p вершин, и мы добавляем ребра. Если мы добавили ребро для получения цикла, то добавили второй путь между парой вершин, а значит нам не хватит его на добавление вершины и мы получим не связный граф, что противоречит условию.

 4 \Rightarrow 5 
:G — ациклический граф, значит каждая компонента связности графа является деревом. Так как в каждой из них вершин на единицу больше чем ребер, то p = q + k , где k — число [[Отношение связности, компоненты связности|компонент связности]]. Поскольку p = q + k , то k = 1 , а значит G — связен. Таким образом наш граф — дерево, у которого между любой парой вершин есть единственный простой путь. Очевидно, при добавлении ребра появится второй путь между парой вершин, то есть мы получим цикл.

 5 \Rightarrow 6 
:Поскольку K_p для p > 3 содержит простой цикл, то G не может им являться. G связен, так как в ином случае можно было бы добавить ребро так, что граф остался бы ациклическим.

 6 \Rightarrow 7 
:Докажем, что любые две вершины графа соединены единственной простой цепью, а тогда поскольку 2 \Rightarrow 3 , получим p = q + 1 . Любые две вершины соединены простой цепью, так как G — связен. Если две вершины соединены более чем одной простой цепью, то мы получим цикл. Причем он должен являться K_3 , так как иначе добавив ребро, соединяющее две вершины цикла, мы получим более одного простого цикла, что противоречит условию. K_3 является собственным подграфом G, поскольку G не является K_p для p > 3 . G — связен, а значит есть вершина смежная с K_3 . Очевидно, можно добавить ребро так, что образуется более одного простого цикла. Если нельзя добавить ребра так, чтобы не нарушалось исходное условие, то граф G является K_p для p > 3 , и мы получаем противоречие с исходным условием. Значит, любые две вершины графа соединены единственной простой цепью, что и требовалось.

 7 \Rightarrow 1 
:Если G имеет простой цикл, то он является отдельной компонентой K_3 по ранее доказанному. Все остальные компоненты должны быть деревьями, но для выполнения соотношения p = q + 1 должно быть не более одной компоненты отличной от K_3, так как в K_3 p = q = 3 . Если это дерево содержит простой путь длины 2, то в G можно добавить ребро так, что образуются два простых цикла. Следовательно, этим деревом является K_1 или K_2. Значит G является K_3 \cup K_1 или K_3 \cup K_2, которые мы исключили из рассмотрения. Значит наш граф ацикличен. Если G ациклический и p = q + 1 , то из 4 \Rightarrow 5 и 5 \Rightarrow 6 верно, что G — связен. В итоге получаем, что G является деревом по определению.

==См. также==
* [[Алгоритмы на деревьях|Алгоритмы на деревьях]]
* [[Дерево поиска, наивная реализация|Двоичное дерево поиска]]

==Источники информации==

* ''Харари Ф.'' Теория графов. /пер. с англ. — изд. 2-е — М.: Едиториал УРСС, 2003. — 296 с. — ISBN 5-354-00301-6
* [http://en.wikipedia.org/wiki/Tree_(graph_theory) Википедия {{---}} дерево(теория графов)]

[[Категория: Алгоритмы и структуры данных]]
[[Категория: Основные определения теории графов]]