{{Определение
|definition=
'''Матрицей Кирхгофа''' [[Основные определения теории графов#def_simple_graph|простого графа]] G = (V,E) называется матрица K (|V| \times |V|) = \parallel k_{i,j} \parallel , элементы которой определяются равенством: 
k_{i,j} = 
\begin{cases}
\deg(v_i), \ i = j \\
-1, \ (v_i,v_j) \in E \\
0, \mbox{ otherwise}.
\end{cases}

}}
Иными словами, на главной диагонали матрицы Кирхгофа находятся степени вершин, а на пересечении i-й строки и j-го столбца (i \ne j) стоит -1, если вершины с номерами i и j смежны, и 0 в противном случае.

== Пример матрицы Кирхгофа==

{|class="wikitable"
!Граф
!Матрица Кирхгофа
|-
|[[Файл:Kirhgof_matrix_1.png|200px]]
|\left(\begin{array}{rrrrrr}
 2 & -1 & 0 & 0 & -1 & 0\\
-1 & 3 & -1 & 0 & -1 & 0\\
 0 & -1 & 2 & -1 & 0 & 0\\
 0 & 0 & -1 & 3 & -1 & -1\\
-1 & -1 & 0 & -1 & 3 & 0\\
 0 & 0 & 0 & -1 & 0 & 1\\
\end{array}\right)
|}

== Некоторые свойства ==

{{Утверждение
|statement=Сумма элементов каждой строки (столбца) матрицы Кирхгофа равна нулю:
: \ \sum_{i=1}^{|V|} k_{i,j} = 0.
}}

{{Утверждение
|statement=Определитель матрицы Кирхгофа равен нулю:
: \det K=0
|proof= \det K = 
\begin{vmatrix}
k_{1, 1} & k_{1, 2} & \cdots & k_{1, |V|} \\
k_{2, 1} & k_{2, 2} & \cdots & k_{2, |V|} \\ 
\vdots & \vdots & \ddots & \vdots \\
k_{|V|, 1} & k_{|V|, 2} & \cdots & k_{|V|, |V|}
\end{vmatrix}
 

Прибавим к первой строке все остальные строки (это не изменит значение определителя):

\begin{vmatrix}
k_{1, 1} + k_{2, 1} + \cdots + k_{|V|, 1} & k_{1, 2} + k_{2, 2} + \cdots + k_{|V|, 2} & \cdots & k_{1, |V|} + k_{2, |V|} + \cdots + k_{|V|, |V|} \\
k_{2, 1} & k_{2, 2} & \cdots & k_{2, |V|} \\ 
\vdots & \vdots & \ddots & \vdots \\
k_{|V|, 1} & k_{|V|, 2} & \cdots & k_{|V|, |V|}
\end{vmatrix}

Так как сумма элементов каждого столбца равна 0, получим:

\det K = \begin{vmatrix}
0 & 0 & \cdots & 0 \\
k_{2, 1} & k_{2, 2} & \cdots & k_{2, |V|} \\ 
\vdots & \vdots & \ddots & \vdots \\
k_{|V|, 1} & k_{|V|, 2} & \cdots & k_{|V|, |V|}
\end{vmatrix} = 0

}}

{{Утверждение
|statement=Матрица Кирхгофа простого графа симметрична:
: \ k_{i,j} = k_{j,i}\quad i,j=1, \ldots, |V|.
}}

{{Утверждение
|statement=Связь с [[Матрица смежности графа|матрицей смежности]]: 
: K = 
\begin{pmatrix}
\mathrm{deg}(v_1) & 0 & \cdots & 0 \\
0 & \mathrm{deg}(v_2) & \cdots & 0 \\ 
\vdots & \vdots & \ddots & \vdots \\
0 & 0 & \cdots & \mathrm{deg}(v_n)
\end{pmatrix} - A,
 
где A — матрица смежности графа G.
}}

{{Утверждение
|statement=[[Связь матрицы Кирхгофа и матрицы инцидентности|Связь с матрицей инцидентности]]: 
: K = I \cdot I^T, где I — матрица инцидентности некоторой ориентации графа.
}}

{{Утверждение
|statement=0 является [[Собственные векторы и собственные значения|собственным значением]] матрицы, кратность его равна числу [[Отношение связности, компоненты связности|компонент связности]] графа.
|proof=Собственным значением матрицы называют значения \lambda, которые удовлетворяют уравнению:

\begin{vmatrix}
k_{1, 1} - \lambda &k_{1, 2} & \cdots & k_{1, |V|} \\
k_{2, 1} & k_{2, 2} - \lambda & \cdots & k_{2, |V|} \\ 
\vdots & \vdots & \ddots & \vdots \\
k_{|V|, 1} & k_{|V|, 2} & \cdots & k_{|V|, |V| - \lambda}
\end{vmatrix} = 0

Прибавим к первой строке все остальные строки (это не изменит значение определителя):

\begin{vmatrix}
k_{1, 1} + k_{2, 1} + \cdots + k_{|V|, 1} - \lambda & k_{1, 2} + k_{2, 2} + \cdots + k_{|V|, 2} - \lambda & \cdots & k_{1, |V|} + k_{2, |V|} + \cdots + k_{|V|, |V|} - \lambda \\
k_{2, 1} & k_{2, 2} - \lambda & \cdots & k_{2, |V|} \\ 
\vdots & \vdots & \ddots & \vdots \\
k_{|V|, 1} & k_{|V|, 2} & \cdots & k_{|V|, |V|} - \lambda
\end{vmatrix}

Так как сумма элементов каждого столбца равна 0, получим:

\begin{vmatrix}
 - \lambda &-\lambda & \cdots & - \lambda \\
k_{2, 1} & k_{2, 2} - \lambda & \cdots & k_{2, |V|} \\ 
\vdots & \vdots & \ddots & \vdots \\
k_{|V|, 1} & k_{|V|, 2} & \cdots & k_{|V|, |V|} - \lambda
\end{vmatrix} = 0

- \lambda
\begin{vmatrix}
 1 & 1 & \cdots & 1 \\
k_{2, 1} & k_{2, 2} - \lambda & \cdots & k_{2, |V|} \\ 
\vdots & \vdots & \ddots & \vdots \\
k_{|V|, 1} & k_{|V|, 2} & \cdots & k_{|V|, |V|} - \lambda
\end{vmatrix}= 0. 

Следовательно, 0 является собственным значением.

'''Доказательство кратности:'''

Пусть дан граф G c n компонентами связности. Перенумеруем его вершины так, чтобы сначала шли вершины первой компоненты связности, затем второй и т.д. Тогда матрица Кирхгофа примет блочно-диагональный вид, и i-тый блок этой матрицы будет являтся матрицей Кирхгофа для i-той компоненты связности.

Из свойства блочно-диагональной матрицы \det K = \det K_{1} \cdot \det K_{2} \cdot \ldots \cdot \det K_{n}, где K_{i} — матрица Кирхгофа для i-той компоненты связности, и свойства, доказанного выше, 

\det K_{i} = - \lambda \cdot det X_{i} \quad \Rightarrow \quad \det K = (-1)^{n} \cdot \lambda^{n} \cdot \det X_{1} \cdot \det X_{2} \cdot \ldots \cdot \det X_{n} 
}}

==См. также==

*[[Связь матрицы Кирхгофа и матрицы инцидентности]]
*[[Подсчет числа остовных деревьев с помощью матрицы Кирхгофа]]

==Источники информации==

*Асанов М., Баранский В., Расин В.: Дискретная математика: Графы, матроиды, алгоритмы. стр. 18
*[http://ru.wikipedia.org/wiki/%CC%E0%F2%F0%E8%F6%E0_%CA%E8%F0%F5%E3%EE%F4%E0 Википедия — Матрица Кирхгофа]

[[Категория: Алгоритмы и структуры данных]]
[[Категория: Остовные деревья]]
[[Категория: Свойства остовных деревьев ]]