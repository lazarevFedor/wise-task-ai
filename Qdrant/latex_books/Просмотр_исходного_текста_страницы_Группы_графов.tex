==Вершинная и рёберная группы графов==
{{Определение
|definition=
'''Автоморфизмом''' (англ. ''Automorphism'') графа G называется изоморфизм графа G на себя
}}

Каждый автоморфизм \alpha графа G есть [[:группа#Группа_подстановок|подстановка]] множества вершин V, сохраняющая смежность. Конечно, подстановка \alpha переводит любую вершину графа в вершину той же степени. Очевидно, что последовательное выполнение двух автоморфизмов есть также автоморфизм;

{{Определение
|definition=
Автоморфизмы графа G образуют [[:группа#Группа_подстановок|группу подстановок]] \Gamma (G) , действующую на множестве вершин V(G). Эту [[:группа|группу]] называют '''группой''' или иногда '''вершинной группой графа''' G (англ. ''point-group'').
}}

{{Определение
|definition=
Вершинная группа графа G индуцирует другую группу подстановок \Gamma_1 (G) , называемую '''реберной группой графа''' G (англ. ''line-group'') {{---}} она действует на множестве ребер E(G). 
}}

[[Файл:fordm.png|right]]

Для иллюстрации различия групп \Gamma и \Gamma_1 рассмотрим граф K_4 - x, показанный на рисунке; его вершины помечены v_1 , v_2, v_3, v_4 а ребра x_1, x_2, x_3, x_4, x_5 . Вершинная группа \Gamma (K_4 - x) состоит из четырех подстановок
(v_1)(v_2)(v_3)(v_4); (v_1)(v_3)(v_2v_4); (v_2)(v_4)(v_1v_3); (v_1v_3)(v_2v_4).

Тождественная подстановка вершинной группы индуцирует тождественную подстановку на множестве ребер, в то время как подстановка (v_1)(v_3)(v_2v_4) индуцирует подстановку на множестве ребер, в которой ребро x_5 остается на месте, x_1 меняется с x_4, а x_2 с x_3. Таким образом, реберная группа \Gamma_1 (K_4 - x) состоит из следующих подстановок, индуцируемых указанными выше элементами вершинной группы: 
(x_1)(x_2)(x_3)(x_4)(x_5); (x_1x_4)(x_2x_3)(x_5); (x_1x_2)(x_3x_4)(x_5); (x_1x_3)(x_2x_4)(x_5).

Понятно, что реберная и вершинная группы графа K_4 - x изоморфны. Но они, конечно, не могут быть идентичными, так как степень группы \Gamma_1 (K_4 - x) равна 5, а степень группы \Gamma (K_4 - x) равна 4. 

{{Теорема
|statement=
Реберная и вершинная группы графа G изоморфны тогда и только тогда, когда граф G имеет не более одной изолированной вершины, а граф K_2 не является его компонентой.
|proof=
 Пусть подстановка \alpha' группы \Gamma_1(G) индуцируется подстановкой \alpha группы \Gamma(G). Из определения операции умножения в группе \Gamma_1(G) вытекает, что 
\alpha'\beta'=\alpha\beta

для \forall \alpha,\beta \in \Gamma(G). Поэтому отображение \alpha\rightarrow\alpha ' является групповым гомоморфизмом группы \Gamma(G) на \Gamma_1(G). Следовательно, \Gamma(G)\cong\Gamma_1(G) тогда и только тогда, когда ядро этого отображения тривиально. 

 \Rightarrow 

:Для доказательства необходимости предположим, что \Gamma(G)\cong\Gamma_1(G). Тогда из неравенства \alpha\not=i(i — тождественная подстановка) следует, что \alpha'\not=i. Если в графе G существуют две различные изолированные вершины v_1 и v_2, то можно определить подстановку \alpha\in\Gamma(G), положив \alpha(v_1) = v_2, \alpha(v_2) = v_1, \alpha(v) = v для \forall v \not= v_1,v_2 . Тогда \alpha\not=i, но \alpha'=i. Если K_2 {{---}} компонента графа G, то, записав ребро графа K_2 в виде x = v_1v_2 и определив подстановку \alpha\in\Gamma(g) точно так же, как выше, получим \alpha\not=i, но \alpha'=i. 

 \Leftarrow 

:Чтобы доказать достаточность, предположим, что граф G имеет не больше одной изолированной вершины и K_2 не является его компонентой. Если группа \Gamma(G) тривиальна, то очевидно, что группа \Gamma_1(G) оставляет на месте каждое ребро и, следовательно, \Gamma_1(G) {{---}} тривиальная группа. Поэтому предположим, что существует подстановка \alpha\in\Gamma(G), для которой \alpha(u)=v\not=u. Тогда степени вершин u и v равны. Поскольку вершины u и v не изолированы, их степени не равны нулю. Здесь возникает два случая. 

:''Случай 1.'' Вершины u и v смежны. Пусть x=uv. Так как K_2 не является компонентой графа G, то степени обеих вершин u и v больше единицы. Следовательно, существует такое ребро y \not= x инцидентное вершине u, что ребро \alpha'(y) инцидентно вершине v. Отсюда \alpha'(y) \not= y, и тогда \alpha'\not=i. 

:''Случай 2.'' Вершины u и v не смежны. Пусть x — произвольное ребро, инцидентное вершине u. Тогда \alpha'(x) \not= x, следовательно, \alpha'\not=i.
}}

==Операции на группах подстановок==

Пусть A — группа подстановок порядка m = |A| и степени d, действующая на множестве X = \{x_1,x_2,\ldots,x_d\}, а B {{---}} другая группа подстановок порядка n = |B| и степени e, действующая на множестве Y = \{y_1,y_2,\ldots,y_e\}. Например, пусть A = C_3 {{---}} циклическая группа порядка 3, действующая на множестве X={1, 2, 3}. Эта группа состоит из трех подстановок (1)(2)(3), (123) и (132). Если взять в качестве B симметрическую группу S_2 порядка 2, действующую на множестве Y = \{a,b\}, то получим две подстановки (a)(b) и (ab). Проиллюстрируем на этих двух группах подстановок действие нескольких бинарных операций. 

===Сумма подстановок===
A + B {{---}} это группа подстановок, действующая на объединении X \cup Y непересекающихся множеств X и Y элементы которой записываются в виде \alpha + \beta и представляют собой упорядоченные пары подстановок \alpha из A и \beta из B. Каждый элемент z, принадлежащий множеству X \cup Y преобразуется подстановкой \alpha + \beta по правилу 

(\alpha + \beta)(z) = 
\begin{cases}
\alpha z, z \in X, \\
\beta z, z \in Y.
\end{cases}

Таким образом, группа C_3 + S_2 содержит 6 подстановок, каждую из которых можно записать в виде суммы подстановок \alpha \in C_3 и \beta\in S_2, как, например, (123)(ab)=(123)+(ab). (степень равна 5.)

===Произведение групп===
A \times B {{---}} это группа подстановок, действующая на множестве X\times Y, элементы которой записываются в виде \alpha\times\beta и представляют собой упорядоченные пары подстановок \alpha из A и \beta из B. Элемент (x,y) множества X\times Y преобразуется подстановкой \alpha\times\beta естественным образом: 

(\alpha\times\beta)(x,y)=(\alpha x,\beta y)

Подстановкой в группе C_3\times S_2, которая соответствует подстановке (123)+(ab) будет (1a\ 2b\ 3a\ 1b\ 2a\ 3b), где для краткости символ (1,a) заменен на 1a. (Порядок и степень равны 6.)

===Композиция групп===
A[B] группы A относительно группы B также действует на множестве X\times Y. Для любой подстановки \alpha из A и любой последовательности (\beta_1,\beta_2,\ldots,\beta_d), содержащей d (не обязательно различных) подстановок из B, существует единственная подстановка из A[B], которая записывается в виде (\alpha;\beta_1,\beta_2,\ldots,\beta_d), такая, что для всякой пары (x_i , y_i) из X\times Y выполняется равенство 

(\alpha;\beta_1,\beta_2,\ldots,\beta_d)(x_i , y_j) = (\alpha x_i,\beta_i y_j).

Композиция C_3 [S_2 ] имеет степень 6 и порядок 24. Любую подстановку из C_3 [S_2 ] можно записать в таком виде, как она действует на множестве X\times Y. Вводя опять обозначение 1a для упорядоченной пары (1,a) и используя формулу выше можно представить подстановку ((123);(a)(b),(ab),(a)(b)) в виде (1a\ 2a\ 3b\ 1b\ 2b\ 3a). Заметим, что группа S_2 [C_3 ] имеет порядок 18 и поэтому не изоморфна группе C_3 [S_2 ] . 

===Степенная группа=== 
(обозначается B^A) действует на множестве Y^X всех функций, отображающих X в Y. Будем всегда предполагать, что степенная группа действует на множестве, состоящем более чем из одной функции. Для каждой пары подстановок \alpha из A и \beta из B существует единственная подстановка из B^A (записывается \beta^\alpha), которая действует на любую функцию f из Y^X в соответствии со следующим соотношением, определяющим образ каждого элемента x\in X при отображении \beta^\alpha f: 

(\beta^\alpha f)(x)=\beta f(\alpha x).

Степенная группа S_{2}^{C_3} имеет порядок 6 и степень 8. Применяя формулу выше, видим, что подстановка этой группы, полученная из подстановок \alpha = (123) и \beta = (ab), имеет один цикл длины 2 и один цикл длины 6. 

==См. также==
*[[Группа]]
*[[wikipedia:ru:Автоморфизм_графа | Википедия {{---}} Автоморфизм_графа]]

==Источники информации==
* Харари Ф. Теория графов. М.: Мир, 1973. (Изд. 3, М.: КомКнига, 2006. — 296 с.)

[[Категория: Дискретная математика и алгоритмы]]
[[Категория: Основные определения теории графов]]
[[Категория: Теория групп]]