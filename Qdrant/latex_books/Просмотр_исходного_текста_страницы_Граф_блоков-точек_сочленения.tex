{{Определение
|definition=
Пусть [[Основные определения: граф, ребро, вершина, степень, петля, путь, цикл|граф]] G связен. Обозначим A_1...A_n {{---}} блоки, а a_1...a_m {{---}} [[Точка сочленения, эквивалентные определения|точки сочленения]] G.
Построим двудольный граф T, поместив A_1...A_n и a_1...a_m в различные его доли. Если точка сочленения принадлежит блоку, проведем между ними ребро. Полученный граф T называют '''графом блоков-точек сочленения''' ''(англ. block cutpoint graph)'' графа G.
}}
[[Файл:block_cut_vertex_1.png|thumb|250px|Граф G]]
[[Файл:block_cut_vertex_2.png|thumb|135px|Граф T]]

{{Лемма
|id=lemma1
|statement=
В определении, приведенном выше, T {{---}} [[Дерево, эквивалентные определения|дерево]].
|proof=
Достаточно показать, что в T нет циклов.
Пусть A_i, a_k, A_j: a_k \in A_i, A_j {{---}} последовательные вершины T, лежащие на цикле. Тогда существует последовательность точек сочленения и блоков, соединяющая A_i и A_j и не содержащая a_k. По ней можно проложить путь в G (можем переходить из блока в блок по точке сочленения и из одной части блока в другую) и замкнуть его в вершине a_k, получив цикл. Получается, что некоторые рёбра из A_i и A_j принадлежат одному и тому же циклу, что противоречит тому, что они находятся в разных блоках.
}}

==См. также==
* [[Граф компонент реберной двусвязности]]

==Источники информации==
* Асанов М. О., Баранский В. А., Расин В. В. '''Дискретная математика: графы, матроиды, алгоритмы''' — НИЦ РХД, 2001. — 288 с. — ISBN 5-93972-076-5

[[Категория:Алгоритмы и структуры данных]]
[[Категория:Связность в графах]]