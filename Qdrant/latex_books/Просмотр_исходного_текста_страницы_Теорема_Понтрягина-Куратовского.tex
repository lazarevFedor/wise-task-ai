Теорему доказал в 1927 году известный советский математик Лев Семенович Понтрягин, но не опубликовал.
Независимо от Понтрягина в 1930 году доказательста нашел и впервые напечатал польский математик Казимир Куратовский.
Первые доказательства теоремы Понтрягина-Куратовского были очень сложными. Сравнительно простое доказательство нашел в 1997 г. петербургский школьник Юрий Макарычев. 

__TOC__

{{Теорема
|statement =
Граф [[Укладка_графа_на_плоскости| планарен]] тогда и только тогда, когда он не содержит подграфов, [[Укладка графа на плоскости #def_hmp| гомеоморфных]] K_{5} или K_{3, 3} .
|proof =
Заметим, что из планарности графа следует планарность гомеоморфного графа и наоборот. В самом деле, пусть G_1 {{---}} плоский граф.
Если добавить на нужных ребрах вершины степени 2 и удалить некотрые вершины степени 2 в G_1 , получим укладку гомеоморфного графа G_2 . Таким образом, доказательство необходимости следует из [[Непланарность_K5_и_K3,3| непланарности K_5 и K_{3, 3}]].

Докажем достаточность. От противного: пусть существует непланарный граф, который не содержит подграфов, гомеоморфных K_{5} или K_{3, 3} . Пусть G {{---}} такой граф с наименьшим возможным числом рёбер, не содержащий изолированных вершин. 
=== G связен ===
Если G не [[Отношение_связности,_компоненты_связности|связен]], то в силу минимальности G его компоненты связности планарны и, следовательно, сам граф G планарен.
=== G {{---}} обыкновенный граф ===
В самом деле, пусть в графе G есть петля или кратное ребро e . Тогда в силу минимальности G граф G - e планарен. Добавляя ребро e к графу G - e получим, что граф G планарен.
=== G {{---}} [[Отношение_вершинной_двусвязности|блок]] ===
Пусть, от противного, в графе есть [[Точка_сочленения,_эквивалентные_определения|точка сочленения]] v . Через G_1 обозначим подграф графа G , порождённый вершинами одной из компонент связности графа G - v и вершинной v , а через 
 G_2 подграф графа G , порождённый вершинами остальных компонент связности графа G - v и вершиной v . 

В силу минимальности G , G_1 и G_2 {{---}} планарны.

[[Файл:New.nb.pic.1.png|400px|рис. 1]]

Возьмём укладку графа G_1 на плоскости такую, что вершина v лежит на границе внешней грани. Ее можно получить, взяв любую укладку G_1 на плоскости, по ней построив укладку на шаре, используя обратную стереографическую проекцию [http://en.wikipedia.org/wiki/Stereographic_projection Wikipedia {{---}} Stereographic projection] , потом повернуть сферу так, чтоб v оказалась на внешней грани стереографической проекции повернутого шара.

Затем во внешней грани графа G_1 возьмём укладку графа G_2 такую, что вершина v будет представлена на плоскости в двух экземплярах.

[[Файл:nb.pic.2.png|400px|рис. 2]]

Соединим два экземпляра вершины v пучком жордановых линий, не допуская лишних пересечений с укладками графов G_1 и G_2 , состоящим из такого количества линий, какова степень вершины v в графе G_2 . Далее отбросим вхождение вершины v в граф G_2 , заменяя инцидентные ей рёбра на жордановы линии, полученные из линий указанного пучка и рёбер.

[[Файл:nb.pic.3.png|400px|рис. 3]]

Таким образом мы получили укладку графа G на плоскости, что невозможно.
=== В G нет мостов === 
Граф G не равен K_2 и в нем нет точек сочленения, следовательно в G нет [[Мост,_эквивалентные_определения|мостов]]. 
=== В G' существует цикл, содержащий вершины a и b ===
Пусть e = ab {{---}} произвольное ребро графа G , G' = G - e . 
# граф G' планарен в силу минимальности графа G .
# граф G' связен в силу отсутствия в графе G мостов.

Пусть a и b лежат в одном блоке B графа G' .
# Если |VB| \geqslant 3, то существует цикл графа G', содержащий вершины a и b .
# Если |VB| = 2 , то в B имеется ребро e' = ab , но тогда в G имеются кратные рёбра e и e' , что невозможно.
#Если вершины a и b лежат в разных блоках графа G' , что существует точка сочленения v , принадлежащая любой простой (a, b) {{---}} цепи графа G' . Через G'_1 обозначим подграф графа G' , порождённый вершиной v и вершинами компоненты связности графа G' - v , содержащей a , а через G'_2 {{---}} подграф графа G' , порождённый вершиной v и вершинами остальных компонент связности графа G' - v (в этом множестве лежит вершина b ). Пусть G''_1 = G'_1 + e_1 , где e_1 = vb {{---}} новое ребро.

[[Файл:nb.pic.4.png|400px|рис. 4]]

Заметим, что в графе G''_1 рёбер меньше, чем в графе G . Действительно, вместо ребра e в G''_1 есть ребро e_1 и часть рёбер из графа G осталась в графе G''_2 . Аналогично, в графе G''_2 рёбер меньше, чем в графе G . 
Теперь в силу минимальности графа G графы G''_1 и G''_2 планарны. Возьмем укладку графа G''_1 на плоскости такую, что ребро e_1 = av лежит на границе внешней грани(ее существование доказывается аналогично существованию такой укладки для вершины графа). Во внешней грани графа G''_1 возьмем укладку графа G''_2 такую, что ребро e_2 = vb лежит па границе внешней грани.

[[Файл:nb.pic.5.png|400px|рис. 5]]

Отметим, что опять вершина v представлена на плоскости в двух экземплярах. Очевидно, добавление ребра e = ab не меняет планарности графа G''_1 U G''_2. Склеим оба вхождения вершины v точно так же, как это мы сделали в предыдущем пункте доказательства.

[[Файл:nb.pic.6.png|400px|рис. 6]]

Сотрем затем ранее добавленные ребра e_1 и e_2 . В результате мы получим укладку графа G на плоскости, что невозможно. Утверждение доказано.

===Вспомогательные определения и утверждение об одновременно разделяющейся внутренней части===
Среди всех укладок графа G' на плоскости и среди всех циклов C, содержащих a и b, зафиксируем такую укладку и такой цикл, что внутри области, ограниченной циклом C, лежит максимальное возможное число граней графа G'. Зафиксируем один из обходов по циклу C (на рисунках будем рассматривать обход по часовой стрелке по циклу C). Для вершин u и v цикла C через C[u,v] будем обозначать простую (u,v) {{---}} цепь, идущую по циклу C от u до v в направлении обхода цикла. Конечно, C[u,v] \ne C[v,u]. Положим C(u,v) = C[u,v] \setminus {u,v}, т.е. C(u,v) получено из C[u,v] отбрасыванием вершин u и v.
{{Определение
|definition =
Внешним графом (англ. ''Outside graph'') (относительно цикла C) будем называть подграф графа G', порождённый всеми вершинами графа G', лежащими снаружи от цикла C.
}}
{{Определение
|definition =
Внешними компонентами (англ. ''Outside components'') будем называть компоненты связности внешнего графа.
}}
В силу связности графа G' для любой внешней компоненты должны существовать рёбра в G', соединяющие её с вершинами цикла C.
{{Определение
|definition =
Внешними частями (англ. ''Outside parts'') будем называть внешние компоненты вместе со всеми рёбрами, соединяющими компоненту с вершинами цикла C, и инцидентными им вершинами , либо рёбра графа G', лежащие снаружи от цикла C и соединяющие две вершины из C, вместе с инцидентными такому ребру вершинами.
}}
[[Файл:nb.pic.7.png|440px|рис. 7]]
{{Определение
|definition =
Внутренним графом (англ. ''Inside graph'') (относительно цикла C) будем называть подграф графа G', порождённый всеми вершинами графа G', лежащими внутри цикла C.
}}
{{Определение
|definition =
Внутренними компонентами (англ. ''Inside components'') будем называть компоненты связности внутреннего графа.
}}
В силу связности графа G' для любой внутренней компоненты должны существовать рёбра в G', соединяющие её с вершинами цикла C.
{{Определение
|definition =
Внутренними частями (англ. ''Inside parts'') будем называть внутренние компоненты вместе со всеми рёбрами, соединяющими компоненту с вершинами цикла C, и инцидентными им вершинами, либо рёбра графа G', лежащие внутри цикла C и соединяющие две вершины из C, вместе с инцидентными такому ребру вершинами
}}
Будем говорить, что внешняя (внутренняя) часть ''встречает цикл'' C в своих точках прикрепления к циклу C.
{{Лемма
|about=1
|statement =
Любая внешняя часть встречает цикл C точно в двух точках, одна из которых лежит в C(a,b), а другая {{---}} в C(b,a).
|proof = 
Если внешняя часть встречает цикл C точно в одной точке v, то v является точкой сочленения графа G, что невозможно.

[[Файл:nb.pic.8.png|420px|рис. 8]]

Таким образом, внешняя часть встречает цикл C не менее чем в двух точках. Если внешняя часть встречает цикл C в двух точках из C[a,b] (случай C[b,a] рассматривается аналогично), то в G' имеется цикл, содержащий внутри себя больше граней, чем цикл C, и проходящий через a и b, что невозможно.

[[Файл:nb.pic.9.png|420px|рис. 9]]

Итого, внешняя часть встречает цикл C хотя бы в двух точках, никакие две из которых не лежат в C[a,b] и C[b,a]. То есть ровно одна лежит в C(a,b) и ровно одна {{---}} в C(b,a).
}}
{{Определение
|definition =
Ввиду леммы 1 будем говорить, что любая внешняя часть является (a,b) {{---}} разделяющей частью (англ. ''separating part''), поскольку она встречает и C(a,b), и C(b,a).
}} 
Аналогично можно ввести понятие (a,b) {{---}} разделяющей внутренней части. Заметим, что внутренняя часть может встречать цикл C, вообще говоря, более чем в двух точках, но не менее чем в двух точках.
{{Лемма
|about=2
|statement = 
Существует хотя бы одна (a,b) {{---}} разделяющая внутренняя часть.
|proof =
Пусть, от противного, таких частей нет. Тогда, выходя из a внутри области, ограниченной C, и двигаясь вблизи от C по направлению обхода C и вблизи от встречающихся внутренних частей, можно уложить ребро e = ab внутри цикла C, т.е. G {{---}} планарный граф, что невозможно.

[[Файл:nb.pic.10.png|280px|рис. 10]]

}}
{{Лемма
|about=3
|statement =
Существует внешняя часть, встречающая C(a,b) в точке c и C(b,a) {{---}} в точке d, для которой найдётся внутренняя часть, являющаяся одновременно (a,b) {{---}} разделяющей и (c,d) {{---}} разделяющей. 

[[Файл:nb.pic.11.png|240px|рис. 11]]

|proof =
Пусть, от противного, лемма 3 неверна. Упорядочим (a,b) {{---}} разделяющие внутренние части в порядке их прикрепления к циклу C при движении по циклу от a до b и обозначим их соответственно через In_{1},In_{2},\dots Пусть u_{1} и u_{2} {{---}} первая и последняя вершины из C(a,b), в которых In_{1} встречает цикл C, а v_{1} и v_{2} {{---}} первая и последняя вершины из C(b,a), в которых In_{1} встречает цикл C (возможно, вообще говоря, u_{1} = u_{2} или v_{1} = v_{2}). Поскольку лемма 3 неверна, для любой внешней части обе её вершины, в которых она встречает C, лежат либо на C[v_{2},u_{1}], либо на C[u_{2},v_{1}]. Тогда снаружи цикла C можно провести жорданову кривую P, не пересекая рёбер графа G', соединяющую v_{2} с u_{1}.

[[Файл:nb.pic.12.png|310px|рис. 12]]

Поскольку на участках C(u_{1},u_{2}) и C(v_{1},v_{2}) нет точек прикрепления внешних частей, используя жорданову кривую P, внутреннюю часть In_{1} можно перебросить ("вывернуть" наружу от цикла C) во внешнюю область цикла C, т.е. уложить её снаружи от цикла C и сделать её внешней частью.
Аналогично все остальные (a,b) {{---}} разделяющие внутренние части можно перебросить во внешнюю область от цикла C. После этого точно так же, как в доказательстве леммы 2, ребро e = ab можно уложить внутри цикла C, так как не останется (a,b) {{---}} разделяющих внутренних частей. Следовательно, мы получим укладку графа G, что невозможно.
}}

== Разбор случаев взаимного положения вершин ''a, b, c, d, u1, u2, v1, v2'' ==
* Пусть пара вершин \ v_1 и \ v_2 является (a, b) {{---}} разделяющей. 
*: Тогда, в частности, v_2 \ne a и v_1 \ne b. 
*: В этом случае граф G содержит подграф, гомеоморфный \ K_{3,3} (отметим, что в In существует простая (v_1, v_2) {{---}} цепь).

;: [[Файл:Case_1.png|270px|рис. 7.1]]

* Пусть пара вершин v_1 и v_2 не является (a, b) {{---}} разделяющей. 
*: Тогда v_1, v_2 лежат на C[a, b] или на C[b, a]. 
*: Без ограничения общности будет считать, что v_1 и v_2 лежат на C[a, b].
*: 
*# Пусть v_1 и v_2 лежат на C(a, b), т.е. v_1 \ne b и v_2 \ne a. 
*## Пусть u_2 лежит на C(d, a).
*##: Тогда в графе G имеется подграф, гомеоморфный K_{3,3}.
*##: [[Файл:Сase_2.1.1.png|200px|рис. 7.3]]
*## Пусть u_2 = d.
*##:Тогда во внешней части In имеется вершина w и три простые цепи от w соответственно до d, v_1, v_2, которые в качестве общей точки имеют только точку w. 
*##:В этом случае в графе G имеется подграф, гомеоморфный K_{3,3}.
*##:[[Файл:Сase_2.1.2.png|200px|рис. 7.4]]
*## Пусть u_2 лежит на C(b, d).
*##:Тогда в графе G есть подграф, гомеоморфный K_{3,3}.
*##:[[Файл:Сase_2.1.3.png|200px|рис. 7.5]]
*#:
*#:Теперь рассмотрим случаи (2 и 3), когда хотя бы одна из вершин v_1 и v_2 не лежит на С(a, b). 
*#:Без ограничения общности будем считать, что это вершина v_1, т.е v_1 = b(поскольку v_1 лежит на C[a, b]).
*#:
*# Пусть v_2 \ne a. 
*##: Пусть u_2 лежит на C(d, a).
*##: Тогда в графе G есть пограф, гомеоморфный K_{3,3}.
*##:[[Файл:Сase_2.2.1.png|200px|рис. 7.6]]
*## Пусть u_2 = d.
*##:Тогда в графе G имеется подграф, гомеоморфный K_{3,3}.
*##:[[Файл:Сase_2.2.2.png|220px|рис. 7.7]]
*## Пусть u_2 лежит на C(b, d).
*##:Тогда в графе G имеется подграф, гомеоморфный K_{3,3}. 
*##:[[Файл:Сase_2.2.3.png|200px|рис. 7.8]]
*#:
*#:
*# Пусть v_2 = a. 
*#:[[Файл:Сase_2.3(a).png|200px|рис. 7.9]]
*#:Рассмотрим теперь пару вершин u_1 и u_2. 
*#:Будем считать, что u_1 = c и u_2 = d, поскольку все другие случаи расположения вершин u_1 и u_2 так же, как были рассмотрены все случаи расположения v_1 и v_2. 
*#:Пусть P_1 и P_2 {{---}} соответственно кратчайшие простые (a, b) {{---}} цепь и (c, d)-цепь по внутренней части In.
*#:[[Файл:Сase_2.3(b).png|200px|рис. 7.10]]
*#:Заметим, что P_1 и P_2 имеют общую точку.
*#: 
*## Пусть цепи P_1 и P_2 имеют более одной общей точки.
*##: Тогда в графе G есть подграф, гомеоморфный K_{3,3}.
*##: [[Файл:Сase_2.3.1.png|200px|рис. 7.11]]
*## Пусть цепи P_1 и P_2 имеют точно одну общую точку w.
*##: Тогда в графе G есть подграф, гомеоморфный K_5.
*##: [[Файл:Сase_2.3.2.png|200px|рис. 7.12]]

Таким образом, доказано, что в графе G имеется подграф, гомеоморфный K_{3,3} или K_5, что противоречит нашему первому предположению.
}}

== См. также ==
* [[Двойственный граф планарного графа]]
* [[Теорема Фари]]

== Примечания ==

==Источники информации==
* [https://ru.wikipedia.org/wiki/Планарный_граф Википедия {{---}} Планарный граф]
* [http://en.wikipedia.org/wiki/Kuratowski's_theorem Wikipedia {{---}} Kuratowski's theorem]
* [http://acm.math.spbu.ru/~sk1/download/books/TheoremKuratowski.pdf "Вокруг критерия Куратовского планарности графов" (стр. 118)] 
* Асанов М., Баранский В., Расин В. {{---}} Дискретная математика {{---}} Графы, матроиды, алгоритмы

[[Категория: Алгоритмы и структуры данных]]
[[Категория: Укладки графов ]]