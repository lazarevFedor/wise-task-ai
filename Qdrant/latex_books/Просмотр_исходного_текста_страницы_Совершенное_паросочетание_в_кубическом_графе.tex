==Лемма о сравнимости по модулю 2==
{{Лемма
|id = lemma1
|statement = 
Пусть G {{---}} k-[[Основные определения теории графов#defRegularGraph |регулярный граф]], U \in V(G), |U| нечётно, m {{---}} число рёбер, соединяющих вершины множества U с вершинами из V(G) \setminus U. Тогда m \equiv k \pmod 2
|proof = 
[[Файл:Сравнимость.png|300px|thumb|left|Иллюстрация к лемме]]
m = (\sum\limits_{v \in U} d_G(v)) - 2e(G(U)), где e(G(U)) {{---}} количество рёбер, соединяющих вершину из U с другой вершиной из U.

тогда m = k|U| - 2e(G(U)). 

2e(G(U)) чётно, поэтому m \equiv k|U| \pmod 2. Так как |U| нечётно, m \equiv k \pmod 2.
}}

==Теорема Петерсона (Petersen)==
{{Определение
|id=cube_graf_def
|definition= '''Кубический граф''' (англ. ''Cubic graph'') {{---}} [[Основные определения: граф, ребро, вершина, степень, петля, путь, цикл| граф]], в котором все вершины имеют степень три. Другими словами, кубический граф является 3-регулярным.}}

{{Теорема
|id=th1 
|author=Петерсон
|statement = Пусть G{{---}}[[Отношение связности, компоненты связности#connected_graph | связный]] кубический граф, в котором не более 2 [[Мост, эквивалентные определения | мостов]]. Тогда в G есть [[Паросочетания: основные определения, теорема о максимальном паросочетании и дополняющих цепях#perfect_matching | совершенное паросочетание]].
|proof = 
Предположим, что совершенного паросочетания в G нет, тогда можно выбрать [[Теорема Татта о существовании полного паросочетания#Tutt_set | множество Татта]] S \subset V(G). 

Пусть U_1, \ldots, U_n {{---}} все нечётные компоненты связности графа G \setminus S. m_i {{---}} количество ребёр G, связывающих вершины U_i с вершинами S. 

По предыдущей лемме, все m_i нечётны. Так как не более чем два ребра графа G {{---}} мосты, то не более, чем два числа из m_1, \ldots, m_n равны 1, остальные числа не менее 3.

Так как S {{---}} множество Татта, то odd(G \setminus S) > |S|. Так как количество вершин кубического графа G чётно, мы имеем S \neq \emptyset, odd(G \setminus S) \equiv S \pmod 2, следовательно, n = odd(G \setminus S) \geqslant |S| + 2. Тогда

\sum\limits_{v \in S} d_G(v) \geqslant \sum\limits_{i = 1}^n m_i \geqslant 3n - 4 \geqslant 3(|S| + 2) - 4 = 3|S|
 + 2 > 3|S| = \sum\limits_{v \in S} d_G(v), что, очевидно, невозможно. 

Найдено противоречие, следовательно, множество Татта выбрать невозможно, следовательно, в G есть совершенное паросочетание.
}}

[[Файл:Петерсен 3 моста.png|300px|thumb|left|Кубический граф с тремя мостами, в котором не существует совершенного паросочетания.]]
Заметим, что утверждение теоремы не может быть усилено до большего числа мостов, так как для случая с тремя мостами существует контрпример.

==Теорема Фринка (Frink)==
{{Теорема
|id=th2. 
|author=Фринк
|statement=

Пусть G = (V, E) {{---}} двусвязный кубический граф. 
Возьмём ребро p = (c, d). Пусть вершины a и b смежны с вершиной c, а вершины e и f смежны с вершиной d (рисунок 1 (a)). 
Как минимум одно из двух сокращений графа G, состоящее из удаления вершин c, d и пересоединения вершин a, b, e, f рёбрами (a, e), (b, f) или (a, f), (b, e) (рисунок 1 (b), (c), рисунок 2) сохранит двусвязность графа.

|proof=
Обозначим компоненты графа G(V - \{c, d\}) как A, B, E, F, которые содержат вершины a, b, e, f соответственно. Так как G не имеет мостов (соответственно p не является мостом) должно существовать ребро, соединяющее одну из компонент A или B, с одной из компонент E или F. Без потери общности предположим, что A соединено с E. Заметим, что рёбра (b, c), (d, f) так же не являются мостами, значит возможны три случая (с учётом изоморфизма) (рисунок 3):
* компонента B соединена с F,
* компонента B соединена с E и компонента E соединена с F,
* компонента A соединена с B и компонента E соединена с F.
Во всех трёх случаях если G(V - \{c, d\}) расширить рёбрами (a, f), (b, e) (получим граф G'), добавленные рёбра будут лежать на некотором [[Основные определения: граф, ребро, вершина, степень, петля, путь, цикл| цикле]] в G' (рисунок 4). Так же для любой пары вершин u, v \in \{a, b, e, f\} существует цикл в G', содержащий данные вершины. Чтобы доказать, что G' двусвязен, нужно показать, что каждое ребро r из G' лежит на некотором цикле в G'. Пусть цикл C в G содержит r (такой цикл существует, так как G двусвязен). Если C не проходит через вершины c, d тогда C так же является циклом в G', иначе построим цикл C' графа G' из C следующим образом: 
* если путь x - c - d - y \in C , x \in \{a, b\} , y \in \{e, f\} , удалим этот путь и добавим любой другой из x в y в G' , не содержащий r (такой путь всегда существует, так как x и y принадлежат некоторому циклу в G' ),
* если путь a - c - b \in C , удалим этот путь и добавим любой другой из a в b в G' , не содержащий r,
*если путь e - d - f \in C , удалим этот путь и добавим любой другой из e в f в G' , не содержащий r.

C' это набор циклов (так как C' получен из C путём преобразования некоторых путей) и содержит r. Из этого следует, что каждое ребро графа G' лежит на некотором цикле, то есть граф не содержит мостов. Значит G' двусвязен.
}}

{|align="center"
 |-valign="center"
 |[[Файл:Frinks_algorithm1.png|thumb|500px|Рисунок 1. Сокращение двусвязного кубического графа. (a) Нужно удалить вершины c, d. (b) первый тип сокращения {{---}} вершина a соединена с e, вершина b соединена с f. (c) второй тип сокращений {{---}} вершина a соединена с f, вершина b соединена с e.]]
 |[[Файл:Frinks_algorithm2.PNG|thumb|500px|Рисунок 2. Особые случаи сокращения графа. (a) ребро (c, d), которое нужно удалить, кратное. Сокращение удаляет вершины c, d из графа и соединяет a и e новым ребром. (b) ребро (c, d) инцидентно двум двойным рёбрам (a, c) и (d, e). Сокращение удаляет вершины c, d и добавляет новое кратное ребро (a, e). (c) Ребро (c, d) инцидентно одному ребру (d, e). Сокращение удаляет вершины c, d и добавляет два новых ребра (a, e) и (b, e). (d) Ребро (c, d) тройной кратности. Сокращение удаляет вершины c, d. ]]
 |}

{|align="center"
 |-valign="center"
 |[[Файл:Frinks_algorithm3.PNG|thumb|500px|Рисунок 3. Все возможные соединения двусвязных компонент графа G[V - \{c,d\}]. (a) Компонента A соединена с компонентой E, компонента B соединена с компонентой F. (b) Компонента E соединена с компонентами A, B, F. (c) Компонента A соединена с компонентами B, E, компонента E соединена с компонентой F. ]]
 |[[Файл:Frinks_algorithm4.PNG|thumb|500px|Рисунок 4. Возможные соединения двусвязный компонент A, B, E, F после удаления ребра (c, d) и добавления рёбер (a, f) и (b, e).]]
 |}

==Алгоритм поиска совершенного паросочетания (Frink's algorithm)==
Будем сокращать данный граф G вышеизложенным способом (на каждой итерации можем выбирать любое ребро) пока не удалим все вершины.
Когда все вершины закончились, создадим пустое совершенное паросочетание M и начнём обратный процесс для всех сокращений, то есть будем восстанавливать граф (начиная с последних удалённых вершин). Каждый такой шаг будет приводить к одному из четырёх базовых случаев, представленных в рисунке 5 или к одному из специальных случаев из рисунка 6. Восстановление для всех специальных случаев, а так же для первых трёх базовых выполняется по строгому алгоритму, т.е. разрешимо за O(1). Единственный проблемный случай, когда оба ребра принадлежат совершенному паросочетанию. В этой ситуации необходимо найти альтернативный цикл, содержащий как минимум одно из этих рёбер и обновить паросочетание с этим циклом. Эти действия сводят четвёртый базовый случай к одному из первых трёх.

{|align="center"
 |-valign="center"
 |[[Файл:Frinks_algorithm5.PNG|thumb|400px|Рисунок 5. Базовые случаи восстановления графа.]]
 |[[Файл:Frinks_algorithm6.PNG|thumb|400px|Рисунок 6. Особые случаи восстановления графа.]]
 |}

==Псевдокод алгоритма Фринка==
*G {{---}} двусвязный кубический граф, 
*M {{---}} совершенное паросочетание G,
*функция \mathtt{bridgeless} возвращает true если у графа нет моста или false в противном случае,
*функция \mathtt{alternatingCycle} принимает три параметра: граф, совершенное паросочетание и ребро. Возвращает альтернативный цикл, включающий в себя данное ребро и обновляет совершенное паросочетание, 
*функция \mathtt{reductions} сокращает граф,
*функция \mathtt{simpleReversion} восстанавливает граф,
*функция \mathtt{reductedGraph} принимает три параметра: граф, удаляемые вершины, добавляемые рёбра. Возвращает новый граф, у которого удалены выбранные вершины, вместе c инцидентными рёбрами и добавлены другие рёбра, переданные в параметрах. При этом исходный граф не меняется.

 '''function''' frinkMatching(G):
 '''if''' |V| = 0 
 '''return''' \varnothing
 v - w = E[0]
 R = reductions(G, v - w)
 '''if''' bridgeless(reductedGraph(G, \{v, w\}, R[0])) 
 r = R[0]
 '''else'''
 r = R[1]
 M = frinkMatching(reductedGraph(G,\{v, w\}, r))
 '''if''' |r \cap M| = 2 
 C = alternatingCycle(G, M, r[0])
 M = M \oplus C
 M = (M - r)\ \cup simpleReversion(G, v, w, r, M)
 '''return''' M

==Время работы алгоритма Фринка==
Операция сокращения должна на каждом шаге проверять граф на наличие мостов за O(n), кроме того, при возникновении четвёртого базового случая требуется найти альтернативный цикл за O(n). В алгоритме O(n) операций сокращения и восстановления графа, причем каждая из этих операций требует O(n) времени. Таким образом, весь этот алгоритм исполняется за время O(n^2).

==См. также==

* [[Использование обхода в глубину для поиска цикла]]
* [[Использование обхода в глубину для поиска мостов]]
* [[Паросочетания: основные определения, теорема о максимальном паросочетании и дополняющих цепях]]

== Источники информации ==
* [https://www.lektorium.tv/course/22771 Лекториум {{---}} Дополнительные главы теории паросочетаний]
* [https://ru.wikipedia.org/wiki/%D0%9A%D1%83%D0%B1%D0%B8%D1%87%D0%B5%D1%81%D0%BA%D0%B8%D0%B9_%D0%B3%D1%80%D0%B0%D1%84 Wikipedia {{---}} Кубический граф]
* Piotr Stanczyk {{---}} THEORY AND PRACTICE OF COMPUTING MAXIMUM MATCHINGS IN GRAPHS стр. 21-28.
* Карпов В. Д. - Теория графов, стр 42

[[Категория: Алгоритмы и структуры данных]]
[[Категория: Задача о паросочетании]]