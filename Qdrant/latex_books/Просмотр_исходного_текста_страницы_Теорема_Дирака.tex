9м топ остальным по лицу хлоп

==Альтернативное доказательство==

{{Теорема
|about=Дирак {{---}} альтернативное доказательство
|statement=
Пусть G {{---}} неориентированный граф и \delta {{---}} минимальная степень его вершин. Если n \geqslant 3 и \delta \geqslant n/2, то G {{---}} [[Гамильтоновы графы|гамильтонов граф]].
|proof=
Для \forall k верна импликация d_k \leqslant k , поскольку левая её часть всегда ложна. Тогда по [[Теорема Хватала | теореме Хватала]] G {{---}} гамильтонов граф.
}}

{{Теорема
|about = Вывод из [[Теорема Оре|теоремы Оре]]
|statement = 
Пусть G {{---}} неориентированный граф и \delta {{---}} минимальная степень его вершин. Если n \geqslant 3 и \delta \geqslant n/2, то G {{---}} [[Гамильтоновы графы|гамильтонов граф]].
|proof = 
Возьмем любые неравные вершины u, v \in G . Тогда \displaystyle \deg u + \deg v \geqslant \frac n 2 + \frac n 2 = n . По теореме Оре G {{---}} гамильтонов граф.
}}

==См. также==
* [[Гамильтоновы графы]]
* [[Теорема Хватала]]
* [[Теорема Оре]]
* [[Теорема Поша]]

== Источники информации ==
* [[wikipedia:en:Dirac's_Theorem|Wikipedia {{---}} Dirac's Theorem]]
* Graham, R.L., Groetschel M., and Lovász L., eds. (1996). ''Handbook of Combinatorics'', Volumes 1 and 2. Elsevier (North-Holland), Amsterdam, and MIT Press, Cambridge, Mass. ISBN 0-262-07169-X.

[[Категория: Алгоритмы и структуры данных]]
[[Категория: Обходы графов]]
[[Категория: Гамильтоновы графы]]