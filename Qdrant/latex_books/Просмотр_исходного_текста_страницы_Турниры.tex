{{Определение
|definition = '''Турнир''' (англ. ''Tournament'') — [[ориентированный граф]], между любой парой различных вершин которого есть ровно одно ориентированное ребро.
}} 
Турниром из n вершин можно изобразить исход игры между n людьми, где каждый играет с каждым. Тогда ребро будет ориентировано от выигравшего человека к проигравшему.

[[Файл:Tournament_1_3.png|415px|thumb|left|Турниры из трех вершин]]

==Свойства турниров==
===Оценка количества турниров в графе===
Если в турнире опустить ориентацию ребер, то мы получим полный граф. А так как существует два варианта ориентации каждого ребра, то количество турниров в графе из n вершин равно 2^{\frac{n\cdot(n-1)}{2}}.
===Транзитивность===

[[Файл:Tournament_transitive.png|300px|thumb|right|Транзитивный турнир с 8 вершинами]]
Турнир, в котором (a, b)\land(b, c) \Rightarrow (a, c), называется транзитивным. В транзитивном турнире вершины могут быть полностью упорядочены в порядке достижимости.
{{Теорема
|id=theorem1
|statement=
Пусть T=\langle V, E\rangle — турнир, | V| = n. Тогда следующие утверждения эквивалентны:
#T транзитивен;
#T не содержит циклов длины 3;
#T ациклический;
# множества, составленные из \deg^{-} или \deg^{+} для каждой вершины T, есть \{ 0, 1, 2,..., n - 1\} ;
#T содержит ровно один гамильтонов путь.
|proof= 
1 \Rightarrow 2: Пусть существует цикл длины 3: (u, v), (v, w), (w, u). Однако по транзитивности должно существовать ребро (u, w), т.е. между u, w есть 2 противоположно направленных ребра, что невозможно по определению турнира.

2 \Rightarrow 3: Пусть в графе содержится цикл длины k \neq 3. Это не может быть цикл длины 2 (противоречит определению турнира). Обозначим его вершины в порядке обхода v_1, v_2, \ldots, v_k, k \geqslant 4. Заметим, что т.к. нет циклов длины 3, выполнена транзитивность (в противном случае существовали бы ребра (u, v), (v, w), (w, u)). Докажем по индукции, что существует ребро (v_1, v_{k - 1}). 

'''База индукции''' k = 3: (v_1, v_2) , (v_2, v_3) \in E \Rightarrow (v_1, v_3) \in E (по транзитивности). 

'''Переход индукции''' Пусть доказано для всех i , что (v_1, v_i) \in E, также известно, что (v_i, v_{i+1}) \in E, тогда по транзитивности (v_1, v_{i+1}) \in E.

Таким образом, в транзитивном турнире содержится цикл длины 3 — противоречие (см. предыдущий пункт).

3 \Rightarrow 4: Обозначим множество значений степеней исхода как Deg^{+}(T). Докажем индукцией по n.

'''База индукции''' n = 1: верно, т.к. есть одна вершина степени 0 

'''Переход индукции''' Пусть доказано для n - 1. В ациклическом графе существует сток t, deg^{+}t = 0. Рассмотрим граф T-t. Deg^{+}(T - t) = \{0, 1, \ldots, n - 2\} . Т.к. из каждой v \in V \setminus \{t\} ведет одно ребро в t, Deg^{+}(T)=\{deg^{+}t\} \cup \{x + 1 \mid x \in Deg^{+}(T -t)\} = \{0, 1, \ldots, n - 1\}. Для степеней захода можно доказать аналогично, рассмотрев исток вместо стока.

4 \Rightarrow 5: По [[Теорема Редеи-Камиона|теореме Редеи-Камиона]], в любом турнире есть гамильтонов путь, докажем индукцией по n, что этот путь единственный.

'''База индукции''' n = 1: верно, путь из одной вершины.

'''Переход индукции''' Рассмотрим вершину s: deg^{-}(s) = 0. Она будет первой в гамильтоновом пути (иначе мы в нее не зайдем). Рассмотрим граф T - s. Т.к. s была соединена со всеми его вершинами, их степени меньше на 1 соответствующих степеней в исходном турнире, значит Deg^{-}(T-s)=\{0,1, \ldots, n - 2\}, следовательно в T-s существует единственный гамильтонов путь v_1, v_2, \ldots v_{n -1} (по предположению). Пусть существуют 2 гамильтонова пути, начинающиеся на s, но тогда существуют 2 пути в T-s {{---}} противоречие.

5 \Rightarrow 1: Пусть P=v_1, v_2, \ldots, v_n — единственный гамильтонов путь. Пусть найдется m — наименьший индекс такой, что в вершину v_m идет ребро из вершины с большим индексом, а v_k — вершина с наибольшим индексом, из которой ребро ведет в v_m. Возможно несколько случаев:
# m \neq 1, k \neq n: Из v_{m -1} ведет ребро в v_{m+1} (по минимальности m), а из v_m ведет ребро в v_{k +1} (по максимальности k). Тогда будет существовать еще один гамильтонов путь P_1 = v_1, \ldots, v_{m-1}, v_{m+1}, \ldots, v_{k}, v_m, v_{k+1}, \ldots, v_n.
# m \neq 1, k = n: P_1 = v_1, \ldots, v_{m-1}, v_{m+1}, \ldots, v_{n}, v_m.
# m = 1, k \neq n: P_1 = v_2, \ldots, v_{k}, v_1, v_{k+1}
# m = 1, k = n: P_1 = v_2, \ldots, v_n, v_1
'''Замечание''' Может достигаться равенство m + 1 = n, в этом случае нужно исключить из пути 2 последовательных вхождения v_n. 
Во всех случаях получаем противоречие с единственностью гамильтонова пути, значит не существует такого m, т.е (v_i, v_j) \in E \Leftrightarrow i . Значит \forall i, j, k: 1 \leqslant i, j, k \leqslant n (v_i, v_j) \in E \land (v_j, v_k) \in E \Rightarrow i .
}}

===Теория Рамсея===
Транзитивные турниры играют существенную роль в [[Теория_Рамсея | теории Рамсея]], изучающей условия, при которых в произвольно формируемых математических объектах обязан появиться некоторый порядок. В частности, любой турнир с n вершинами содержит транзитивный подтурнир с 1+\lfloor\log_2 n\rfloor вершинами. Для его построения выберем любую вершину v как часть этого подтурнира и построим подтурнир рекурсивно на множестве либо входящих соседей вершины v, либо на множестве исходящих соседей, в зависимости от того, какое множество больше. 

===Конденсация===
{{Утверждение
|statement = Конденсация любого турнира является транзитивным турниром. 
|proof = Рассмотрим 2 компоненты сильной связности U, V, найдутся u \in U, v \in V: (u, v) \in E, либо (v, u) \in E , значит в конденсации есть либо ребро (U,V), либо (V,U). Т.к. мы рассмотрели произвольную пару вершин конденсации турнира, она является турниром. Конденсация любого орграфа ациклична, а по доказанной [[#theorem1|теореме]], это означает, что она транзитивна. 
}}
Таким образом, даже если турнир не является транзитивным, сильно связанные компоненты турнира могут быть [[Отношение порядка|вполне упорядочены]]. В самом деле, по [[#theorem1|теореме]], в турнире существует гамильтонов путь, значит вершины могут быть упорядочены по своим позициям в этом пути.

===Сильно связные турниры===
{{Определение|definition = Турнир называется [[Отношение связности, компоненты связности#sc_def |сильно связным]], если из любой вершины существуют пути до всех других.}}
{{Определение
|definition = Турнир называется [[Гамильтоновы графы | гамильтоновым]], если он содержит гамильтонов цикл.
}}

[[Файл:Tournament_2.png|380px|thumb|right|Негамильтонов турнир]]

Не все турниры гамильтоновы. Определение не исключает существование вершины с \deg^{-} или \deg^{+} равной нулю — в первую нельзя войти, а из второй — выйти. Однако отсутствие таких вершин не означает, что турнир гамильтонов (пример — на рисунке справа).

[[Теорема Редеи-Камиона]] устанавливает два следующих факта:
# Все турниры полугамильтоновы.
# Турнир гамильтонов тогда и только тогда, когда он сильно связен.

==См. также==
* [[Гамильтоновы графы]]
* [[Теорема Редеи-Камиона]]
* [http://epubs.siam.org/doi/abs/10.1137/0403002 Поиск гамильтонова цикла за O(n\cdot log(n))]

==Источники информации==
* Асанов М. О., Баранский В. А., Расин В. В. '''Дискретная математика: графы, матроиды, алгоритмы''' — НИЦ РХД, 2001. — ISBN 5-93972-076-5
* [[wikipedia:Tournament_(graph_theory) | Wikipedia {{---}} Турнир]]
* [http://www-math.ucdenver.edu/~wcherowi/courses/m4408/gtln12.html]

[[Категория: Алгоритмы и структуры данных]]
[[Категория: Обходы графов]]
[[Категория: Гамильтоновы графы]]