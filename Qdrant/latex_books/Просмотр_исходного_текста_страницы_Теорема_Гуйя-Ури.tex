{{Определение
|definition=
Ориентированный сильно связный граф называется '''орсвязаными'''.}}

== Лемма о длине цикла в ориентированном графе ==
{{Лемма
|about=о длине цикла в ориентированном графе
|statement= Пусть G {{---}} произвольный ориентированный граф и для каждой вершины v \in V(G) выполняется \deg^{out}(v) \geqslant \delta. Если \delta \geqslant 2, то в графе G существует простой цикл C длины хотя бы \delta + 1.
|proof=
Рассмотрим путь максимальной длины P = v_0 v_1 \dots v_s. Из последней вершины v_s выходит хотя бы \delta - 1 ребро в вершины, отличные от v_{s - 1}. Так как путь P максимальный, то продлить его нельзя, а значит, что из v_s выходят ребра только в вершины, содержащиеся в пути P. Пусть v_m \in P {{---}} вершина с наименьшим номером, в которую входит ребро из v_s. Тогда во множество \{v_m \dots v_{s - 1}\} входят не менее \delta ребер, выходящих из v_s. То есть в это множестве хотя бы \delta вершин. Значит, в цикле v_m \dots v_{s - 1} v_s не менее \delta + 1 вершины.
}}

== Теорема Гуйя-Ури ==
{{Теорема
|author=Гуйя-Ури, Ghouila-Houri
|statement=
Если G {{---}} сильно связный ориентированный граф c n вершинами и для каждой v \in V(G) выполняется 

\Bigg\{
\begin{matrix}
 \deg^{in}(v) \geqslant n/2 \\
 \deg^{out}(v) \geqslant n/2 \\

\end{matrix} 
, 
тогда G {{---}} гамильтонов.
|proof=
Будем доказывать теорему от противного. Предположим, что это не так. Очевидно, что условие теоремы выполняется при n = 2 и n = 3. Тогда существует орсвязный граф G с n \geqslant 4, который удовлетворяет условию и при этом не является гамильтоновым. Пусть C {{---}} максимальный цикл в G длины k. По лемме о длине цикла и по предположению о том, что граф не является гамильтоновым, получаем соотношение 1 + n/2 \leqslant k . Теперь рассмотрим P = v_0 \dots v_l {{---}} путь максимальной длины l \geqslant 0 в G такой, что никакая вершина этого пути не принадлежит циклу C. Тогда k + l + 1 \leqslant n. Следовательно, l \leqslant n - k - 1 \leqslant n - (1 + n/2) - 1 \leqslant n/2 - 2. Таким образом, l \leqslant n/2 - 2. Это значит, что в вершину v_0 входят как минимум два ребра, выходящие из вершин, лежащих на C, а из вершины v_l выходят как минимум два ребра, которые входят в вершины, принадлежащие C (так как если бы эти вершины не лежали на данном цикле, путь P можно было бы продлить).

Пусть A {{---}} множество вершин, принадлежащих C, ребра из которых приходят в вершину v_0, а a {{---}} их количество. Тогда a \geqslant 2. Для каждой такой вершины следующая за ней в цикле C l + 1 вершина не содержит входящих ребер, начало которых принадлежит v_l, иначе граф G содержал бы цикл длины > k. Заметим, что среди вершин множества A должна существовать такая вершина y, что следующая за ней l + 1 вершина в цикле C не является ни отцом v_0, ни сыном v_l.

Рассмотрим оставшуюся a - 1 вершину множества A, отличную от y. В следующую за каждой из них, очевидно, не может приходить ребро из v_l. Следовательно, как минимум (a - 1) + (l + 1) = a + l вершин C не являются сыновьями v_l, в противном случае, опять же, граф содержал бы цикл длины > k.

Так как P {{---}} самый длинный путь в G, ни одна вершина которого не принадлежит C, каждая вершина, ребро из которой приходит в v_0, лежит либо на P, либо на C. Так как \deg^{in}(v_0) \geqslant n/2, то a + l \geqslant n/2, следовательно \deg^{out}(v_l) \leqslant (n - 1) - (a + l) \leqslant (n - 1) - n/2 = n/2 - 1. Получаем противоречие с условием. Таким образом, предположение неверно, а значит, теорема доказана.
}}

== См. также ==
* [[Теорема Дирака]]

== Источники информации ==
* Gary Chartrand, Linda Lesniak, Ping Zhang (2010). ''Graphs & Digraphs, Fifth Edition'', chapter 4. ISBN 9781439895184.
* Д. В. Карпов. ''Теория графов''.

[[Категория: Алгоритмы и структуры данных]]
[[Категория: Обходы графов]]
[[Категория: Гамильтоновы графы]]