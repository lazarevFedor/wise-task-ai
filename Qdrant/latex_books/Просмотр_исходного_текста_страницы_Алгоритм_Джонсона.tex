'''Алгоритм Джонсона''' (англ. ''Johnson's algorithm'') находит кратчайшие пути между всеми парами вершин во взвешенном ориентированном графе с любыми весами ребер, но не имеющем отрицательных циклов.

== Алгоритм ==

=== Описание ===

Алгоритм Джонсона позволяет найти кратчайшие пути между всеми парами вершин в течение времени O(V^2\log(V) + VE) . Для разреженных графов этот алгоритм ведет себя асимптотически быстрее [[Алгоритм Флойда|алгоритма Флойда]]. Этот алгоритм либо возвращает матрицу кратчайших расстояний между всеми парами вершин, либо сообщение о том, что в графе существует цикл отрицательной длины.

В этом алгоритме используется метод '''изменения веса''' (англ. ''reweighting''). Суть его заключается в том, что для заданного графа G строится новая весовая функция \omega_\varphi , неотрицательная для всех ребер графа G и сохраняющая кратчайшие пути. Такая весовая функция строится с помощью так называемой [[Амортизационный_анализ#.D0.9C.D0.B5.D1.82.D0.BE.D0.B4_.D0.BF.D0.BE.D1.82.D0.B5.D0.BD.D1.86.D0.B8.D0.B0.D0.BB.D0.BE.D0.B2|потенциальной]] функции.

Пусть \varphi : V \rightarrow \mathbb R — произвольное отображение из множества вершин в вещественные числа. Тогда новой весовой функцией будет \omega_\varphi(u, v) = \omega(u, v) + \varphi(u) - \varphi(v) .

Такая потенциальная функция строится добавлем фиктивной вершины s в G , из которой проведены ориентированные ребра нулевого веса во все остальные вершины графа, и запуском [[Алгоритм Форда-Беллмана|алгоритма Форда-Беллмана]] из нее ( \varphi(v) будет равно длине кратчайшего пути из s в v ). На этом же этапе мы сможем обнаружить наличие отрицательного цикла в графе.

Теперь, когда мы знаем, что веса всех ребер неотрицательны, и кратчайшие пути сохранятся, можно запустить [[Алгоритм Дейкстры|алгоритм Дейкстры]] из каждой вершины и таким образом найти кратчайшие расстояния между всеми парами вершин.

=== Сохранение кратчайших путей ===
Утверждается, что если какой-то путь P был кратчайшим относительно весовой функции \omega , то он будет кратчайшим и относительно новой весовой функции \omega_\varphi .

{{Лемма
|statement=
Пусть P,\; Q {{---}} два пути a \rightsquigarrow b\; и \omega(P) Тогда \forall \varphi: \; \omega_\varphi(P) 
|proof=

:Рассмотрим путь P: \;u_0 \rightarrow u_1 \rightarrow u_2 \rightarrow \ldots \rightarrow u_k 

:Его вес с новой весовой функцией равен \omega_\varphi(P) = \omega_\varphi(u_0u_1) + \omega_\varphi(u_1u_2) + \ldots + \omega_\varphi(u_{k-1}u_k) .

:Вставим определение функции \omega_\varphi : \omega_\varphi(P) = \varphi(u_0) + \omega(u_0u_1) - \varphi(u_1) + \ldots + \varphi(u_{k-1}) + \omega(u_{k-1}u_k) - \varphi(u_k) 

:Заметим, что потенциалы все промежуточных вершин в пути сократятся. \omega_\varphi(P) = \varphi(u_0) + \omega(P) - \varphi(u_k)

:По изначальному предположению: \omega(P) . С новой весовой функцией веса соответствующих путей будут: 

:\omega_\varphi(P) = \varphi(a) + \omega(P) - \varphi(b)

:\omega_\varphi(Q) = \varphi(a) + \omega(Q) - \varphi(b)

:Отсюда, \omega_\varphi(P) 
}}

=== Теорема о существовании потенциальной функции ===
{{Теорема
|statement= 

В графе G нет отрицательных циклов \Leftrightarrow существует потенциальная функция \phi:\; \forall uv \in E\; \omega_\varphi(uv) \geqslant 0 

|proof=

\Leftarrow : Рассмотрим произвольный C — цикл в графе G

:По лемме, его вес равен \omega(C) = \omega_\varphi(C) + \varphi(u_0) - \varphi(u_0) = \omega_\varphi(C) \geqslant 0

\Rightarrow : Добавим фиктивную вершину s в граф, а также ребра s \to u весом 0 для всех u .

:Обозначим \delta(u,v) как минимальное расстояние между вершинами u,\; v, введем потенциальную функцию \phi 

: \phi(u) = \delta(s,u)

:Рассмотрим вес произвольного ребра uv \in E : \omega_\phi(uv) = \phi(u) + \omega(uv) - \phi(v) = \delta(s, u) + \omega(uv) - \delta(s, v).

:Поскольку \delta(s, u) + \omega(uv) {{---}} вес какого-то пути s \rightsquigarrow v , а \delta(s, v) {{---}} вес кратчайшего пути s \rightsquigarrow v, то \delta(s, u) + \omega(uv) \geqslant \delta(s, v) \Rightarrow \delta(s, u) + \omega(uv) - \delta(s, v) = \omega_\varphi(uv) \geqslant 0 .

}}

=== Псевдокод ===

Предварительно построим граф G' = (V',\;E'), где V' = V \cup \{s\}, s \not\in V, а E' = E \cup \{(s,\;v): \omega(s, v) = 0,\ v \in V \}
 '''function''' Johnson(G): '''int[][]'''
 '''if''' BellmanFord(G',\;\omega,\;s) == ''false''
 print "Входной граф содержит цикл с отрицательным весом"
 '''return''' \varnothing
 '''else''' '''for''' v \in V'
 \varphi(v) = \delta(s,\;v) // \delta(s,\;v) вычислено алгоритмом Беллмана — Форда
 '''for''' (u,\;v) \in E'
 \omega_\varphi(u,\;v) = \omega(u,\;v) + \varphi(u) - \varphi(v)
 '''for''' u \in V
 Dijkstra(G,\;\omega_\varphi,\;u)
 '''for''' v \in V
 d_{uv} \leftarrow \delta_\varphi(u,\;v) + \varphi(v) - \varphi(u)
 '''return''' d

Итого, в начале алгоритм Форда-Беллмана либо строит потенциальную функцию такую, что после перевзвешивания все веса ребер будут неотрицательны, либо выдает сообщение о том, что в графе присутствует отрицательный цикл.

Затем из каждой вершины запускается алгоритм Дейкстры для составления искомой матрицы. Так как все веса ребер теперь неотрицательны, алгоритм Дейкстры будет работать корректно. А поскольку перевзвешивание таково, что кратчайшие пути относительно обеих весовых функций совпадают, алгоритм Джонсона в итоге корректно найдет все кратчайшие пути между всеми парами вершин.

== Сложность ==
Алгоритм Джонсона работает за O(VE + VD), где O(D) — время работы [[Алгоритм Дейкстры| алгоритма Дейкстры]]. Если в алгоритме Дейкстры неубывающая очередь с приоритетами реализована в виде [[Фибоначчиевы кучи| фибоначчиевой кучи]], то время работы алгоритма Джонсона есть O(V^2\log V + V E). В случае реализации очереди с приоритетами в виде двоичной кучи время работы равно O(V E \log V).

== См. также ==
* [[Алгоритм Дейкстры]]
* [[Алгоритм Форда-Беллмана]]
* [[Алгоритм Флойда]]

== Источники информации ==
* ''Кормен Т., Лейзерсон Ч., Ривест Р.'' Алгоритмы: построение и анализ. 2-е изд. — М.: Издательский дом «Вильямс», 2007. — С. 1296.
* [http://rain.ifmo.ru/cat/view.php/vis/graph-paths/johnson-2001 Визуализатор алгоритма]

[[Категория: Алгоритмы и структуры данных]]
[[Категория: Кратчайшие пути в графах ]]