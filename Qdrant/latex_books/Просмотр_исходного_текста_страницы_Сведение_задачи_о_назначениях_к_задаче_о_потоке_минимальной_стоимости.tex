{{Задача
|definition = Дана квадратная матрица A_{N\times N}. Нужно выбрать в ней N элементов так, чтобы в каждой строке и в каждом столбце был выбран только один элемент, а сумма значений этих элементов была наименьшей.
}}
Очевидно, что решение данной задачи эквивалентно решению следующей: 
{{Задача
|definition = Имеется N заказов и N станков. Про каждый заказ известна стоимость его изготовления на каждом станке. На каждом станке можно выполнять только один заказ. Требуется распределить все заказы по станкам так, чтобы минимизировать суммарную стоимость.
}}

== Алгоритм ==
[[Файл:pic1.PNG|thumb|right|270px|Пример построенного графа для матрицы A = \begin{pmatrix}
1 & 2 \\
3 & 4 \end{pmatrix}]]
Сведем задачу о назначениях к задаче нахождения потока минимальной стоимости.
Построим ориентированный граф, состоящий из двух частей G следующим образом:
* Имеется исток S и сток T.
* В первой части находятся N вершин, соответствующие строкам матрицы или заказам. 
* Во второй N вершин, соответствующие столбцам матрицы или станкам.
* Между каждой вершиной i первой части и каждой вершиной j второй части проведём ребро с пропускной способностью 1 и стоимостью A_{ij}.
* От истока S проведём рёбра ко всем вершинам i первой части с пропускной способностью 1 и стоимостью 0.
* От каждой вершины второй части j к стоку T проведём ребро с пропускной способностью 1 и стоимостью 0.

Найдём в полученном графе G максимальный [[Поток минимальной стоимости|поток минимальной стоимости]]. 

== Доказательство ==
Понятно, что величина потока будет равна N. Заметим, что для каждой вершины i из первой части найдётся только одна вершина j из второй части, такая, что поток f(i, j) = 1. Поскольку найденный поток имеет минимальную стоимость, то сумма стоимостей выбранных рёбер будет наименьшей из возможных. Поэтому это взаимно однозначное соответствие между вершинами первой части и вершинами второй части является решением задачи.

== Асимптотика == 
Асимптотика этого решения равна асимптотике алгоритма, выбранного для поиска потока.

== Источники ==
* [http://e-maxx.ru/algo/assignment_mincostflow Задача о назначениях. Решение с помощью min-cost-flow]
* Ravindra Ahuja, Thomas Magnanti, James Orlin. Network flows (1993)

[[Категория:Алгоритмы и структуры данных]]
[[Категория: Задача о потоке минимальной стоимости]]