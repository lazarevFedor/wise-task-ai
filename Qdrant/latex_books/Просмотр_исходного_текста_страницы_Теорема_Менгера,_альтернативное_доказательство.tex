{{Теорема
|about=
Теорема Менгера для вершинной k связности
|statement=
Наименьшее число вершин, [[k-связность|разделяющих]] две несмежные вершины s и t, равно наибольшему числу непересекающихся простых (s-t) цепей.
|proof=

Очевидно, что если k вершин разделяют s и t, то сущесвует не более k непересекающихся простых (s-t) цепей.
Теперь покажем, что если k вершин графа разделяют s и t, то существует k непересекающихся простых (s-t) цепей. Для k=1 это очевидно. 
Пусть, для некоторого k>1 это неверно. Возьмем h {{---}} наименьшее такое k и F {{---}} граф с наименьшим числом вершин, для которого при выбранном h теорема не верна. Будем удалять из F ребра, пока не получим G такой, что в G s и t разделяют h вершин, а в G-x h-1 вершина, где x{{---}} произвольное ребро графа G.

Из определения G следует, что для всякого его ребра x существует множество S(x) из h-1 вершин, которое в G-x разделяет s и t. Далее, граф G-S(x) содержит по крайней мере одну (s-t) цепь, так как граф G имеет h вершин, разделяющих s и t в G. Каждая такая (s-t) цепь должна содержать ребро x=uv, поскольку она не является цепью в G-x. Поэтому u,v \notin S(x), и если u \neq s,t то S(x) \cup {u} разделяет s и t в G.

{{Лемма
|id=lemma_1
|about=
I
|statement=
В графе G нет вершин, смежных одновременно с s и t
|proof=
Если в G есть вершина w, смежная как с s, так и с t, то в графе G-w для разделения s и t требуется h - 1 непересекающихся (s-t) цепей. Добавляя w, получаем в графе G h непересекающихся (s-t) цепей, что противоречит предположению о графе F
}}

{{Лемма
|id=lemma_2
|about=
II
|statement=
Любой набор W, содержащий h вершин и разделяющий s и t является смежным с s или t. 
|proof=
Пусть W {{---}} произвольный набор h вершин, разделяющих s и t в G. 
Цепь, соединяющую s с некоторой вершиной w_i \in W и не содержащую других вершин из W будем называть (s-W) цепью. Аналогично назовем (W-t) цепь. Обозначим наборы всех (s-W) и (W-t) цепей P_s и P_t соответственно. Тогда каждая (s-t) цепь начинается с элемента из P_s и заканчивается элементом из P_t, поскольку любая цепь содержит вершину из W. Общие вершины цепей из P_s и P_t принадлежат набору W, так как по крайней мере одна цепь из каждого набора P_s и P_t содержит (любую) вершину w_i, и если бы существовала некоторая вершина, не принадлежащая набору W, но содержащаяся сразу и в (s-W) и в (W-t) цепи, то нашлась бы (s-t) цепь, не имеющая вершин из W. Наконец, выполняется либо равенство P_s-W={s}, либо равенство P_t - W={t}, поскольку в противном случае либо P_s вместе с ребрами \{w_1t,w_2t...\}, либо P_t вместе с ребрами \{sw_1,sw_2...\} образуют связные графы с меньшим числом вершин, чем у G, в которых s и t не смежны, и, следовательно, в каждом из них имеется h непересекающихся (s-t) цепей. Объединяя (s-W) и (W-t) части этих цепей, образуем в графе G h непересекающихся (s-t) цепей. Мы пришли к противоречию. Утверждение доказано.
}}

Пусть P=\{s, u_1, u_2 ... t\} {{---}} кратчайшая (s-t) цепь в G, u_1u_2=x. Заметим, что из [[#lemma_1 | леммы (I)]] u_1 \neq t Образуем множество S(x)=\{v_1, ... , v_{h-1}\}, разделяющее в G-x вершины s и t. Из [[#lemma_1 | леммы (I)]] следует, что u_1t \notin G. Используя [[#lemma_2 | лемму (II)]] и беря W=S(x)\cup {u_1}, получаем \forall i \; sv_i \in G. Таким образом в силу [[#lemma_1 | леммы (I)]] \forall i \; v_it \notin G. Однако, если выбрать W=S(x) \cup {u_2}, то в силу [[#lemma_2 | леммы (II)]] получим su_2 \in G, что противоречит выбору P как кратчайшей (s-t) цепи. Из полученного противоречия следует, что графа G, удовлетворяющего указанным условиям не существует, а значит не существует и графа F, для которого теорема не верна.
}}

{{Теорема
|about=
Теорема Менгера для k-связности (альтернативная формулировка)
|statement=
Две несмежные вершины k-отделимы тогда и только тогда, когда они k-соединимы.
}}

{{Теорема
|about=
Теорема Менгера для k-реберной связности
|statement=
Пусть G {{---}} конечный, неориентированный граф, \lambda(G) = k \Leftrightarrow для всех пар вершин x, y \in G существует k реберно непересекающихся путей из x в y.
|proof=
Аналогично теореме для вершинной связности.
}}
==См. также==
*[[Теорема Менгера]]

==Источники информации==
* Харари, Ф. Теория графов. — М.: Книжный дом «ЛИБРОКОМ», 2009

[[Категория:Алгоритмы и структуры данных]]
[[Категория:Связность в графах]]