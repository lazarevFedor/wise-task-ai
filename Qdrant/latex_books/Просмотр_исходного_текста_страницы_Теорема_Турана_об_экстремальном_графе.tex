==Теорема Турана==
[[Файл:Turan example.png|200px|thumb|right|Пример графа Турана при n = 8, r = 4]]
'''Теорема Ту́рана''' (англ. ''Turán's theorem'') {{---}} классическая теорема экстремальной теории графов[https://ru.wikipedia.org/wiki/%D0%AD%D0%BA%D1%81%D1%82%D1%80%D0%B5%D0%BC%D0%B0%D0%BB%D1%8C%D0%BD%D0%B0%D1%8F_%D1%82%D0%B5%D0%BE%D1%80%D0%B8%D1%8F_%D0%B3%D1%80%D0%B0%D1%84%D0%BE%D0%B2 Экстремальная теория графов].
Она послужила образцом для большого количества подобных теорем, которые изучают, как наличие тех или иных подструктур влияет на некоторые глобальные параметры ([[Раскраска графа|хроматическое число]]).

Впервые теорему сформулировал венгерский математик Пал Туран в 1941 году.

{{Определение
|definition=
K_n {{---}} полный граф на n вершинах.
}}

{{Определение
|definition=
ex(n, K_r) {{---}} максимальное количество ребер, которое может иметь граф на n вершинах, не включая в себя K_r как подграф.
}}

{{Определение
|definition=
'''Граф Турана''' T^{r-1}(n) {{---}} полный (r - 1)-[[Двудольные графы|дольный]] граф на n > r-1 вершинах, доли которого по мощности отличаются не более чем на 1. Если количество вершин не превосходит количество долей (n \leqslant r - 1), то T^{r-1}(n) = K_n.
}}

{{Определение
|definition=
t_{r-1}(n) {{---}} количество ребер в T^{r-1}(n).
}}

{{Лемма
|statement=
Если G {{---}} (r - 1)-дольный граф с максимальным количеством ребер, то G = T^{r-1}(n).
|proof=
Докажем от противного. Пусть существует (r - 1)-дольный граф с максимальным числом ребер, который не является графом Турана.
Обозначим его G_m.
Очевидно, что G_m является полным (r - 1)-дольным.
Так как G_m \ne T^{r-1}(n) , то в G_m существуют доли V_1 и V_2, что |V_1| - |V_2| > 1.
Но тогда возьмем вершину a \in V_1 и перекинем ее в V_2. Тогда количество вершин, которые не могут быть соседями a уменьшилось с размером ее доли. Остальной граф не изменился, поэтому общее количество ребер увеличилось.
Это противоречит предположению, что граф G_m максимален по числу ребер.

Значит лемма доказана.
}}
{{Теорема
|statement=
Для всех натуральных чисел r, n, где r > 1, любой граф G \nsubseteq K_r с n вершинами и ex(n, K_r) ребрами есть T^{r-1}(n).
|proof=
[[Файл:Turan theorem induction step.png|300px|thumb|left|Шаг индукции]]
Применим индукцию по n.

'''База:'''

При n \leqslant r - 1 имеем G = K_n = T^{r-1}(n), что и утверждалось. База доказана.

'''Шаг индукции:'''

Пусть теперь n \geqslant r.
Поскольку G реберно-максимален и не содержит подграфа K_r, то G содержит подграф K^{r-1}.
Обозначим любой из них как K.
Тогда по индукционному предположению G - K имеет не более t_{r-1}(n - r + 1) ребер, а любая вершина G - K имеет не более r - 2 соседей в K.
Следовательно мы можем оценить количество ребер в G:

|E(G)| \leqslant \underbrace{t_{r-1}(n - r + 1)}_{G-K} + \underbrace{(n - r + 1)(r - 2)}_{(G-K) \rightleftarrows (K)} + \underbrace{{r-1 \choose 2}}_{K} = t_{r-1}(n); (1)

Равенство справа следует непосредственно из графа Турана T^{r-1}(n).

Поскольку G экстремален для K_r, то в (1) имеет место равенство.
Таким образом, любая вершина из G - K имеет ровно r - 2 соседа в K {{---}} точно так же, как и вершины x_1,\cdots, x_{r-1} из самого K.

При i = 1,\cdots, r-1 пусть V_i = \{v \in V(G) \mid vx_i \not\in E(G)\} есть множество всех вершин G, чьи r - 2 соседей в K отличны от x_i.
Так как каждая вершина G - K имеет ровно r - 2 соседа в K, то все V_i не зависимы.
При этом они в объединении дают V(G) поскольку K_r \nsubseteq G.
Следовательно, граф G является (r-1)-дольным.
Тогда по лемме из предположения об экстремальности G следует, что G = T^{r-1}(n).

}}

==См. также==
*[[Раскраска графа]]
*[[Двудольные графы]]
==Примечания==

==Источники информации==
*''Дистель, Рейнград.'' Теория графов: Пер. с англ. — Новосибирск: Изд-во Ин-та математики, 2002. — 166-170 стр. — ISBN 5-86134-101-X.
*[https://ru.wikipedia.org/wiki/%D0%AD%D0%BA%D1%81%D1%82%D1%80%D0%B5%D0%BC%D0%B0%D0%BB%D1%8C%D0%BD%D0%B0%D1%8F_%D1%82%D0%B5%D0%BE%D1%80%D0%B8%D1%8F_%D0%B3%D1%80%D0%B0%D1%84%D0%BE%D0%B2 Экстремальная теория графов]

[[Категория: Раскраски графов]]
[[Категория: Алгоритмы и структуры данных]]