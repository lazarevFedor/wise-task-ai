Вагнер опубликовал теорему в 1937, после публикации в 1930 [[Теорема_Понтрягина-Куратовского|теоремы Куратовского]], согласно которой граф планарен тогда и только тогда, когда он не содержит подграфов [[Укладка_графа_на_плоскости|гомеоморфных]] K_{5} и K_{3, 3} . [[Теорема_Понтрягина-Куратовского|Теорема Куратовского]] сильнее теоремы Вагнера — [[Укладка_графа_на_плоскости|гомеоморфный]] подграф может быть преобразован в минор того же типа путём стягивания всех, кроме одного, рёбер в каждом пути, полученном при разбиении ребра путем добавления вершины, а вот обратное преобразование из минора в [[Укладка_графа_на_плоскости|гомеоморфный]] подграф того же типа не всегда возможно. 
__TOC__

{{Определение 
|definition = 
'''Минором графа''' (англ. ''Graph minor'') G будем называть граф H, если H может быть образован из G удалением рёбер и вершин или стягиванием рёбер. 
}} 

Миноры графов часто изучаются в более общем контексте миноров [[Примеры_матроидов:_графовый_матроид|графовых матроидов]]. В этом контексте обычно полагается, что графы связны, могут иметь петли и кратные рёбра (то есть, рассматриваются [[Основные_определения_теории_графов|псевдографы]], а не простые графы). Стягивание петли и удаление разрезающего ребра запрещены. При таком подходе удаление ребра оставляет ранг графа неизменным, а стягивание ребра всегда уменьшает ранг на единицу.

== Пример ==
В следующем примере граф H является минором графа G:

G [[Файл:Olddddd.png]]

H[[Файл:Olllddd.png‎]]

== Теорема ==

{{Теорема 
|statement = 
Граф [[Укладка_графа_на_плоскости| планарен]] тогда и только тогда, когда его миноры не содержат ни K_{5} ни K_{3, 3} . 
|proof = 
Иначе говоря в соответствии с [[Теорема_Понтрягина-Куратовского|теоремой Понтрягина-Куратовского]], теорему можно переформулировать: « ''В графе G есть миноры содержащие K_{5} или K_{3, 3} тогда и только тогда, когда существует подграф гомеоморфный K_{5} или K_{3, 3} '' » 

\Rightarrow

Разделим доказательство на две части 
# Если в G существует минор содержащий K_{3, 3} , тогда в G существует подграф гомеоморфный K_{3, 3} . 
# Если в G существует минор содержащий K_{5} , тогда в G существует подграф гомеоморфный либо K_{3, 3} , либо K_{5} . 

=====''Доказательство первой части'' =====
 

В силу определения минора, если в G существует минор содержащий K_{3, 3} ,значит существуют множества вершин U_{1} , U_{2} , U_{3} , W_{1} , W_{2} , W_{3} попарно не пересекающиеся, образующие индуцированный связанный подграф G, такие что для каждого i и j существует {u_{i, j} \in U_{i}} и {w_{i, j} \in W_{j}} и ({u_{i,j}} , {w_{i,j}}) принадлежит множеству ребер исходного графа. Следовательно для каждого i существует поддерево в G , у которого три листа w_{1} \in W_{1}, w_{2} \in W_{2}, w_{3} \in W_{3}, а все остальные вершины подграфа принадлежат U_{i} . Ситуация с j симметрична. 
Вследствие [[Лемма_о_рукопожатиях|леммы о рукопожатиях]] дерево с тремя вершинами гомеоморфно K_{1, 3} . Таким образом, в G существует подграф гомеоморфный шести копиям K_{1, 3} соединенные три на три, т.е. получаем K_{3, 3} . 

=====''Доказательство второй части'' =====

В силу определения минора, если в G существует минор содержащий K_{5} ,значит существуют множества вершин U_{1} \ldots U_{5} попарно не пересекающиеся, образующие индуцированный связанный подграф G, такие что для всех {i \ne j} существует {u_{i; \left \lbrace i,j \right \rbrace} \in U_{i}} и {u_{j; \left \lbrace i,j \right \rbrace} \in U_{j}} , такие что ( {u_{i; \left \lbrace i,j \right \rbrace}, u_{j; \left \lbrace i,j \right \rbrace}} ) принадлежит множеству ребер исходного графа. Следовательно для любого i существует поддерево T_{i} в G с четырьмя листьями, по одному листу в каждом U_{j} ({i \ne j}) и с остальными вершинами внутри U_{i} . 
Вследствие [[Лемма_о_рукопожатиях|леммы о рукопожатиях]] дерево с четырьмя вершинами гомеоморфно либо K_{1, 4} , либо двум связным копиям K_{1, 3} . Значит в G есть подграф гомеоморфный пяти копиям K_{1, 4} , соединенные друг с другом. Т.е. получаем K_{5} . В противном случае подграф гомеоморфный K_{3, 3} может быть получен с помощью следующих процедур: 
#Берем одну из T_{i} гомеоморфную двум соединенным копиям K_{1, 3} . Назовем их T_{i, r} и T_{i, b} . 
#Покрасим в красный вершины T_{i, r} , за исключением двух вершин которые будут окрашены в синий. 
#Покрасим в синий вершины T_{i, b} , за исключением двух вершин которые окрашены в красный. 
#Покрасим в синий вершины T_{j} , которые включают в себя T_{i, r} . 
#Покрасим в красный вершины T_{j} , которые включают в себя T_{i, b} . 
#Удалим ребра соединяющие одноцветные вершины из разных T_{j} . 
Такое «''обрезание''» приведет к тому что T_{j} будут иметь по три вершины, каждая содержится в таком подграфе, что она окрашено в другой цвет чем остальные вершины. 
Граф
сформированный из красных и синих вершин вместе с оставшимися ребрами изоморфен K_{3, 3} .

\Leftarrow

Пусть существует подграф гомеоморфный K_{5} или K_{3, 3} . В силу гомеоморфизма, заметим, что данные подграфы можно получить только путем стягивания ребер между вершинами такими, что хотя бы одна их них должна иметь степень 2. Удалим все ребра и вершины графа, которые не входят в этот подграф. Таким образом, мы получили минор содержащий K_{5} или K_{3, 3} соответственно. 
}}

==См. также==
* [[Теорема_Понтрягина-Куратовского|Теорема Понтрягина-Куратовского]]

== Источники информации ==
* [https://en.wikipedia.org/wiki/Graph_minor Graph minor — Wikipedia]
* Geir Agnarsson, Raymond Greenlaw. Graph Theory: Modeling, Applications and Algorithms, 2006 — глава 7.5 стр. 210

[[Категория: Алгоритмы и структуры данных]]
[[Категория: Укладки графов ]]