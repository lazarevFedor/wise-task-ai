{{Определение
|definition=
Пусть [[Основные_определения_теории_графов#.D0.9D.D0.B5.D0.BE.D1.80.D0.B8.D0.B5.D0.BD.D1.82.D0.B8.D1.80.D0.BE.D0.B2.D0.B0.D0.BD.D0.BD.D1.8B.D0.B5_.D0.B3.D1.80.D0.B0.D1.84.D1.8B|неориентированный граф]] G имеет n вершин: v_1, v_2, \ldots, v_n . Пусть d_i = \deg v_i \mbox{ } (i = \overline{1, n}) и вершины графа упорядочены таким образом, что d_1 \leqslant d_2 \leqslant \ldots \leqslant d_n . Последовательность d_1, d_2, \ldots, d_n называют '''последовательностью степеней''' графа G .
}}

{{Лемма
|about=
О добавлении ребра в граф
|statement=
Пусть неориентированный граф G' получен из неориентированного графа G добавлением одного нового ребра e . Тогда последовательность степеней графа G мажорируется последовательностью степеней графа G' .
|proof=
''Замечание'': Если в неубывающей последовательности d_1, d_2, \ldots, d_n увеличить на единицу число d_i , а затем привести последовательность к неубывающему виду, переставив число d_i + 1 на положенное место j , то исходная последовательность будет мажорироваться полученной. Если j = i, то утверждение леммы, очевидно, выполняется. Пусть j \neq i.
[[Файл: Hvatal_1.png|270px|thumb|center|Исходная последовательность степеней d ]]

* Рассмотрим элементы с номерами s = \overline{1, i - 1} . Они не изменились, следовательно мажорируются собой.
* Рассмотрим элементы с номерами s = \overline{i, j - 1} . s -й элемент полученной последовательности равен s + 1 -му элементу исходной. d_s \leqslant d_{s + 1} \Rightarrow d_s \leqslant d'_s = d_{s + 1} .
* Расмотрим j-ый элемент. Имеем d'_j \ge d'_{j-1} = d_{j} .
* Рассмотрим элементы с номерами s = \overline{j + 1, n} . Они не изменились, следовательно мажорируются собой.
[[Файл: Hvatal_2.png|290px|thumb|center|Новая последовательность степеней d' ]]
При добавлении в граф ребра e = uv, \mbox{ } (u \neq v) , степени вершин u и v увеличатся на единицу. Для доказательства леммы, дважды воспользуемся замечанием.
Значит, последовательность степеней полученного графа мажорирует последовательность степеней исходного.
}}

{{Теорема
|about=
Хватал
|statement=
Пусть:
* G — [[Отношение_связности,_компоненты_связности#.D0.A1.D0.BB.D1.83.D1.87.D0.B0.D0.B9_.D0.BD.D0.B5.D0.BE.D1.80.D0.B8.D0.B5.D0.BD.D1.82.D0.B8.D1.80.D0.BE.D0.B2.D0.B0.D0.BD.D0.BD.D0.BE.D0.B3.D0.BE_.D0.B3.D1.80.D0.B0.D1.84.D0.B0|связный граф]],
* n = |VG| \geqslant 3 — количество вершин,
* d_1 \leqslant d_2 \leqslant \ldots \leqslant d_n — его последовательность степеней.
Тогда если \forall k \in \mathbb N верна импликация: 
 d_k \leqslant k 
то граф G [[Гамильтоновы_графы#.D0.9E.D1.81.D0.BD.D0.BE.D0.B2.D0.BD.D1.8B.D0.B5_.D0.BE.D0.BF.D1.80.D0.B5.D0.B4.D0.B5.D0.BB.D0.B5.D0.BD.D0.B8.D1.8F|гамильтонов]].
|proof=
Для доказательства теоремы, докажем 3 леммы.
{{Лемма
|about=
1
|statement=
 d_k \leqslant k \Leftrightarrow |\{ v \in VG \mid d_v \leqslant k \}| \geqslant k. 
|proof=
 \Rightarrow Пусть:
* d_1 \leqslant d_2 \leqslant \ldots \leqslant d_k ,
* d_k \leqslant k ,
* |\{ d_1, d_2, \ldots, d_k \}| = k .
 \{ d_1, d_2, \ldots, d_k \} \subseteq \{ v \in VG \mid d_v \leqslant k \} \Rightarrow |\{ v \in VG \mid d_v \leqslant k \}| \geqslant k .

 \Leftarrow Из условия:
* |\{ v \in VG \mid d_v \leqslant k \}| = k + p ,
* p \geqslant 0 .
Расположим вершины в неубывающем порядке их степеней. 
 d_1 \leqslant d_2 \leqslant \ldots \leqslant d_k \leqslant \ldots \leqslant d_{k + p} \leqslant k \Rightarrow d_k \leqslant k .
}}

{{Лемма
|about=
2
|statement=
\ d_{n - k} \geqslant n - k \Leftrightarrow |\{ v \in VG \mid d_v \geqslant n - k \}| \geqslant k + 1. 
|proof=
 \Rightarrow Пусть:
* d_{n - k} \geqslant n - k ,
* d_{n - k} \leqslant d_{n - k + 1} \leqslant \ldots \leqslant d_n , 
* |\{ d_{n - k}, d_{n - k + 1}, \ldots , d_n \}| = k + 1 .
 \{ d_{n - k}, d_{n - k + 1}, \ldots , d_n \} \subseteq \{ v \in VG \mid d_v \geqslant n - k \} \Rightarrow \{ v \in VG \mid d_v \geqslant n - k \} \geqslant k + 1 .

 \Leftarrow :
* |\{ v \in VG \mid d_v \geqslant n - k \}| = k + 1 + p, (p \geqslant 0),
Расположим вершины в неубывающем порядке их степеней.
 d_n \geqslant d_{n - 1} \ldots \geqslant d_{n - k} \geqslant \ldots \geqslant d_{n - k - p} \geqslant n - k \Rightarrow d_{n - k} \geqslant n - k . 
}}

{{Лемма
|about=
3
|statement=
Если импликация (*) верна для некоторой последовательности степеней d , то она верна и для неубывающей последовательности d' , мажорирующей d .
|proof=
# Если d'_k > k , то первый аргумент импликации всегда ложен, следовательно импликация верна вне зависимости от второго аргумента. Значит, в этом случае импликация (*) верна для последовательности d' .
# Если d'_k \leqslant k, \mbox{ } d'_{n - k} \geqslant d_{n - k} \geqslant n - k , то оба аргумента импликации всегда истинны. Значит, и в этом случае импликация (*) верна для последовательности d' .
Значит, импликация (*) выполняется и для последовательности d' .
}}

Приведем доказательство от противного.

Пусть существует граф с числом вершин n \geqslant 3 , удовлетворяющий (*) , но негамильтонов.
Будем добавлять в него ребра до тех пор, пока не получим максимально возможный негамильтонов граф G (то есть добавление еще одного ребра сделает граф G гамильтоновым). 
По лемме о добавлении ребра и лемме №3 импликация (*) остается верной для графа G .
Очевидно, что граф \ K_n гамильтонов при k \geqslant 3 .
Будем считать G максимальным негамильтоновым остовным подграфом графа K_n .

Выберем две несмежные вершины u и v графа G , такие что \deg u + \deg v — максимально.
Будем считать, что \deg u \leqslant \deg v .
Добавив к G новое ребро e = uv , получим гамильтонов граф G + e .
Рассмотрим гамильтонов цикл графа G + e : в нём обязательно присутствует ребро e .
Отбрасывая ребро e , получим гамильтонову (u, v) -цепь в графе G : u = u_1 \rightarrow u_2 \rightarrow \ldots \rightarrow u_n = v .

Пусть S = \{ i \mid e_i = u_1 u_{i + 1} \in EG\}, T = \{ i \mid f_i = u_i u_n \in EG\} .
[[Файл: Hvatal_3.png|330px|thumb|center|Множество S обозначено красным цветом, множество T обозначено синим цветом]]

{{Утверждение
|statement=
 S \cap T = \emptyset .
|proof=
Предположим, что j \in S \cap T . Тогда получим гамильтонов цикл графа G : u_1 \xrightarrow{e_j} u_{j + 1} \rightarrow \ldots \rightarrow u_n \xrightarrow{f_j} u_j \rightarrow u_{j - 1} \rightarrow \ldots \rightarrow u_1 , что противоречит условию, что граф негамильтонов.
[[Файл: Hvatal_4.png|270px|thumb|center|]]
Значит, S \cap T .
}}

Из определений S и T следует, что S \cup T \subseteq \{1, 2, ..., n - 1 \} \Rightarrow 2 \deg u \leqslant \deg u + \deg v = |S| + |T| = |S \cup T| . Значит, \deg u .

Так как S \cap T = \emptyset , ни одна вершина u_j не смежна с v = u_n (для j \in S ). В силу выбора u и v , получим, что \deg u_j \leqslant \deg u . Пусть k = \deg u = |S| . Значит, \exists k вершин, степень которых не превосходит k .

По лемме №1: d_k \leqslant k . В силу импликации (*) : d_{n - k} \geqslant n - k .

По лемме №2, \exists k + 1 вершин, степень которых не меньше n - k .

Так как k = \deg u , то вершина u может быть смежна максимум с k из этих k+1 вершин. Значит, существует вершина w , не являющаяся смежной с u и для которой \deg w \geqslant n - k . Тогда получим, что \deg u + \deg w \geqslant k + (n - k) = n > \deg u + \deg v , что противоречит выбору u и v .

Значит, предположение неверно.
}}

==См. также==
* [[Гамильтоновы графы]]
* [[Теорема Дирака]]
* [[Теорема Оре]]

== Источники информации ==
* Асанов М., Баранский В., Расин В.: ''Дискретная математика: Графы, матроиды, алгоритмы''

[[Категория: Алгоритмы и структуры данных]]
[[Категория: Обходы графов]]
[[Категория: Гамильтоновы графы]]