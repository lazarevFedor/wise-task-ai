{{Определение
|definition=
Блокирующий поток (англ. ''blocking flow'') {{---}} такой [[Определение сети, потока|поток]] f в данной сети G, что любой s \leadsto t путь содержит насыщенное этим потоком ребро. Иными словами, в данной сети не найдётся такого пути из истока в сток, вдоль которого можно беспрепятственно увеличить поток.
}}

Блокирующий поток не обязательно максимален (пример: см. рис. 1). [[Теорема Форда-Фалкерсона]] говорит о том, что поток будет максимальным тогда и только тогда, когда в остаточной сети не найдётся s \leadsto t пути; в блокирующем же потоке ничего не утверждается о существовании пути по рёбрам, появляющимся в остаточной сети.

Более того, величина блокирующего потока может быть сколь угодно мала по сравнению с величиной максимального потока в сети (пример: см. рис. 2). В примере поток является блокирующим и имеет величину 1, в то время как максимальный можно делать сколь угодно большим, увеличивая количество вершин по той же схеме.

Блокирующий поток используется в [[Схема алгоритма Диница|алгоритме Диница]]. Его поиск с помощью удаляющего обхода занимает O(VE) времени.
{|align="center"
 |-valign="top"
 |[[Файл:Блокпоток.png|240px|thumb|right|Рис. 1. Пропускные способности всех рёбер равны единице, по красным рёбрам течёт единичный поток.]]
 |[[Файл:Блокирующийпоток.png|240px|thumb|right|Рис. 2. Пропускные способности всех рёбер равны единице, по красным рёбрам течёт единичный поток.]]
 |}

== См. также ==
* [[Алгоритм поиска блокирующего потока в ациклической сети]]
* [[Схема алгоритма Диница]]

== Источники ==
* [http://www.e-maxx.ru/algo/dinic Алгоритм Диница. Необходимые определения.]
* [[wikipedia:Dinic's_algorithm | Wikipedia {{---}} Dinic's algorithm]]
* [[wikipedia:Алгоритм_Диница | Википедия {{---}} алгоритм Диница ]]

[[Категория: Алгоритмы и структуры данных]]
[[Категория: Задача о максимальном потоке ]]