'''Теория Рамсея''' — раздел математики, изучающий условия, при которых в произвольно формируемых математических объектах обязан появиться некоторый порядок.
==Числа Рамсея==
{{Определение
|id=def1
|definition='''Клика''' (англ. ''clique'') в [[Основные определения теории графов#Неориентированные графы|неориентированном графе]] G(V, E) {{---}} подмножество [[Основные определения теории графов#Неориентированные графы|вершин]] C \subseteq V, такое что для любых двух различных вершин в C существует [[Основные определения теории графов#def_edge_und|ребро]], их соединяющее. Другими словами, клика графа G(V, E) {{---}} [[Основные определения теории графов#defFullGraph|полный]] подграф графа G(V, E). }}
{{Определение
|id=def2
|definition='''Число Рамсея''' r(n, m) (англ. ''Ramsey's number'') {{---}} наименьшее из таких чисел x \in \mathbb N, что при любой раскраске ребер полного графа на x вершинах в два цвета найдется клика на n вершинах с ребрами цвета 1 или клика на m вершинах с ребрами цвета 2. }}
Существует и другое определение для чисел Рамсея.
{{Определение
|id=def15
|definition='''Число Рамсея''' r(n, m) — это наименьшее из всех таких чисел x \in \mathbb N, что для любого графа G на x вершинах либо в G найдется K_n, либо в \overline G найдется граф K_m. 
}}
[[Файл:RamseyTheoryK5.png|200px|thumb|upright|Раскраска K_5 без одноцветных треугольников]]
Несложно доказать, что данные определения эквивалентны. Достаточно показать, что раскрашенному в два цвета графу K_n, можно однозначно поставить в соответствие граф G на n вершинах. Довольно часто определение для чисел Рамсея дается через задачу "о друзьях и незнакомцах"[https://en.wikipedia.org/wiki/Theorem_on_friends_and_strangers| Theorem on friends and strangers]. Пусть на вечеринке каждые два человека могут быть либо друзьями, либо незнакомцами, в общем виде задачи требуется найти, какое минимальное количество людей нужно взять, чтобы хотя бы n человек были попарно знакомы, или хотя бы m человек были попарно незнакомы. Если мы переформулируем данную задачу в терминах графов, то как раз получим определение числа Рамсея r(n, m), представленное ранее.

===Пример===
Чтобы получить лучшее представление природы чисел Рамсея, приведем пример. Докажем, что r(3,3) = 6. Представим, что ребра K_6 раскрашены в два цвета: красный и синий. Возьмем вершину v. Данной вершине, как и всем другим, инцидентны 5 рёбер, тогда, согласно принципу Дирихле, хотя бы три из них одного цвета. Для определенности положим, что хотя бы 3 ребра, соединяющие вершину v с вершинами r, s, t, синие. Если хотя бы одно из ребер rs, rt, st синее, то в графе есть синий треугольник (полный граф на трёх вершинах), иначе, если они все красные, есть красный треугольник. Таким образом, r(3,3) \leqslant 6 .
Чтобы доказать, что r(3,3) = 6 , предъявим такую раскраску графа K_5, где нет клики на трех вершинах ни синего, ни красного цвета. Такая раскраска представлена на рисунке справа. Понятно, что предъявлять отдельные раскраски для K_4, K_3 не нужно, так как достаточно взять соответствующие подграфы раскрашенного K_5.

===Теорема Рамсея. Оценки сверху===
{{Теорема|id=ter1|about=1, Теорема Рамсея 
|statement= Для любых n,m \in \mathbb N существует число r(n,m), при этом r(n,m) \leqslant r(n,m-1)+r(n-1,m), а также если числа r(n,m-1) и r(n-1,m) четные, то неравенство принимает вид r(n,m) \leqslant r(n,m-1)+r(n-1,m) - 1 .
|proof=

# Докажем с помощью метода математической индукции по n+m. '''База:''' r(n,\;1) = r(1,\;n) = 1, так как граф, состоящий из одной вершины, можно считать полным графом любого цвета. '''Индукционный переход:''' Пусть n>1 и m>1. Рассмотрим полный чёрно-белый граф из r(n-1,\;m)+r(n,\;m-1) вершин. Возьмём произвольную вершину v и обозначим через M и N множества вершин, инцидентных v в чёрном и белом подграфе соответственно. Так как в графе r(n-1,\;m)+r(n,\;m-1)=|M|+|N|+1 вершин, согласно принципу Дирихле, либо |M|\geqslant r(n-1,\;m), либо |N|\geqslant r(n,\;m-1). Пусть |M|\geqslant r(n-1,\;m). Тогда либо в M существует белый K_m, что доказывает теорему, либо в M есть чёрный K_{n-1}, который вместе с v образует чёрный K_n, в этом случае теорема также доказана. Случай |N|\geqslant r(n,\;m-1) рассматривается аналогично.
# Предположим, p=r(n-1,\;m) и q=r(n,\;m-1) оба чётны. Положим s=p+q-1 и рассмотрим чёрно-белый граф из s вершин. Если d_i степень i-й вершины в чёрном подграфе, то, согласно [[Лемма о рукопожатиях|лемме о рукопожатиях]], \sum\limits_{i=1}^s d_i — чётно. Поскольку s нечётно, должно существовать чётное d_i. Не умаляя общности, положим, что d_1 чётно. Обозначим через M и N вершины, инцидентные вершине 1 в чёрном и белом подграфах соответственно. Тогда |M|=d_1 и |N|=s-1-d_1 оба чётны. Согласно принципу Дирихле, либо |M|\geqslant p-1, либо |N|\geqslant q. Так как |M| чётно, а p-1 нечётно, первое неравенство можно усилить, так что либо |M|\geqslant p, либо |N|\geqslant q. Далее проводим рассуждения, аналогичные тем, что присутствуют в первом пункте теоремы. Таким образом, r(n,m) \leqslant r(n,m-1)+r(n-1,m) - 1.
}}
{{Утверждение|id=u1|about=1|statement=Для натуральных чисел m,n выполняется равенство r(n,m) \leqslant C_{n+m-2}^{n-1}
|proof=
Очевидно, C^{n-1}_{n+m-2}=1 при n=1 или m=1, как и соответствующие числа Рамсея. Индукцией по n и m при n,m \geqslant 2 получаем 
r(n,m) \leqslant r(n,m-1)+r(n-1,m) \leqslant C^{n-1}_{n+m-3}+C^{n-2}_{n+m-3}=C^{n-1}_{n+m-2} 
}}

===Оценки снизу===

{{Теорема|id=ter2|about=2, Теорема Эрдеша
|statement=Для любого натурального числа k \geqslant 2 выполняется неравенство r(k,k) \geqslant 2^{k/2}
|proof=
Так как r(2,2)=2, достаточно рассмотреть случай k \geqslant 3.
Пусть g(n, k) доля среди помеченных графов на n вершинах тех, что содержат клику на k вершинах. Всего графов на наших вершинах, очевидно 2^{C^2_n} (каждое из возможных рёбер C^2_n можно провести или не провести).

Посчитаем графы с кликой на k вершинах следующим образом: существует C^k_n способов выбрать k вершин для клики в нашем множестве, после чего все рёбра между ними будем считать проведенными, а остальные ребра выбираются произвольно. Таким образом, каждый граф с кликой на k вершинах будет посчитан, причём некоторые даже более одного раза. Количестве графов с кликой оказывается не более, чем C^k_n\cdot 2^{C^2_n-C^2_k}. Следовательно,

g(n,k) \leqslant \dfrac{C^k_n\cdot 2^{C^2_n-C^2_k}}{2^{C^2_n}}=\dfrac{n!}{(n-k)!\cdot k! \cdot 2^{C^2_k}}=\dfrac{(n-k+1)\cdot(n-k+2)\cdot\ldots \cdot(n-1)\cdot n}{ k! \cdot 2^{C^2_k}} (*)

Подставив n в неравенство (*) мы получаем

g(n,k) при k \geqslant 3

Предположим, что r(k,k)=n и разобьём все графы на n вершинах на пары \langle G, \overline G \rangle. Так как g(n,k), то существует пара \langle G, \overline G \rangle, в которой ни G, ни \overline G не содержат подграфа на k вершинах. Рассмотрим раскраску рёбер K_n в два цвета, в которой ребра цвета 1 образуют граф G. В такой раскраске нет клики на k вершинах ни цвета 1, ни цвета 2, получили противоречие. Значит n было выбрано неверно. Из этого следует r(k,k) \geqslant 2^{k/2}.
}}

===Свойства чисел Рамсея===
Следующими свойствами удобно пользоваться при подсчете значений чисел Рамсея r(n,m) на практике.
* r(n,m) = r(m,n)
* r(1,m) = 1
* r(2,m) = m

===Значения чисел Рамсея===
Задача нахождения точных значений чисел Рамсея чрезвычайно трудна, их известно довольно мало. Далее приведена таблица Станислава Радзишевского, в которой присутствуют практически все известные числа Рамсея или же промежутки, в которых они находятся.

{| class="wikitable" align="center" style="color: blue; background-color:#ffffcc;" cellpadding="10"
|+
!colspan="11"|Числа Рамсея
|-align="center"
! width="6%" |n,\ m
! width="6%" |1 
! width="6%" |2 
! width="6%" |3 
! width="6%" |4 
! width="6%" |5 
! width="6%" |6 
! width="6%" |7 
! width="6%" |8 
! width="6%" |9 
! width="6%" |10
|-align="center"
! 1 
| 1 
| 1 
| 1 
| 1 
| 1 
| 1 
| 1 
| 1 
| 1 
| 1 
|-align="center"
! 2 
| 1 
| 2 
| 3 
| 4 
| 5 
| 6 
| 7 
| 8 
| 9 
| 10
|-align="center"
! 3
| 1
| 3
| 6
| 9
| 14
| 18
| 23
| 28
| 36
| [40, 42]
|-align="center"
! 4
| 1
| 4
| 9
| 18
| 25
| [36, 41]
| [49, 61]
| [59, 84]
| [73, 115]
| [92, 149]
|-align="center"
! 5
| 1
| 5
| 14
| 25
| [43, 48]
| [58, 87]
| [80, 143]
| [101, 216]
| [133, 316]
| [149, 442]
|-align="center"
! 6
| 1
| 6
| 18
| [36, 41]
| [58, 87]
| [102, 165]
| [115, 298]
| [134, 495]
| [183, 780]
| [204, 1171]
|-align="center"
! 7
| 1
| 7
| 23
| [49, 61]
| [80, 143]
| [115, 298]
| [205, 540]
| [217, 1031]
| [252, 1713]
| [292, 2826]
|-align="center"
! 8
| 1
| 8
| 28
| [56, 84]
| [101, 216]
| [127, 495]
| [217, 1031]
| [282, 1870]
| [329, 3583]
| [343, 6090]
|-align="center"
! 9
| 1
| 9
| 36
| [73, 115]
| [133, 316]
| [183, 780]
| [252, 1713]
| [329, 3583]
| [565, 6588]
| [580, 12677]
|-align="center"
! 10
| 1
| 10
| [40, 42]
| [92, 149]
| [149, 442]
| [179, 1171]
| [289, 2826]
| [343, 6090]
| [581, 12677]
| [798, 23556]
|}

===Числа Рамсея для раскрасок в несколько цветов===
Теперь обобщим числа Рамсея на произвольное количество цветов.
{{Определение
|id=def4 
|definition=
'''Число Рамсея''' r(n_1,\ldots,n_k) — это наименьшее из всех таких чисел x \in \mathbb N, что при любой раскраске рёбер полного графа на x вершинах в k цветов для некоторого i \in [1 \ldots k] обязательно найдётся клика на n_i вершинах с рёбрами цвета i. k,n_1,\ldots,n_k \in \mathbb N
}}

{{Теорема
|id=ter3|about=3,Теорема Рамсея для нескольких цветов
|statement=\forall k, n_1, \ldots n_k \in \mathbb N существует число Рамсея r(n_1,\ldots,n_k), при этом r(n_1,\ldots,n_k)\leqslant r(n_1,\ldots, n_{k-2}, r(n_{k-1},\;n_k)).
|proof=
Возьмем граф из r(n_1,\ldots, n_{k-2}, r(n_{k-1}, n_k)) вершин и окрасим его рёбра в k цветов. Пока что будем считать цвета k-1 и k одним цветом. Тогда граф будет (k-1)-цветным. Согласно определению числа Рамсея r(n_1,\ldots,n_{k-2},r(n_{k-1},n_k)), такой граф либо содержит K_{n_i} для некоторого цвета i, такого что 1\leqslant i\leqslant k-2, либо K_{r(n_{k-1},n_k)}, окрашенный общим цветом k-1 и k. В первом случае доказательство завершено. Во втором случае вернём прежние цвета и заметим, что, по определению числа Рамсея, полный r(n_{k-1},n_k) — вершинный граф содержит либо K_{n_{k-1}} цвета k-1, либо K_{n_k} цвета k. Таким образом любое число Рамсея для раскраски в k цветов ограничено некоторым числом Рамсея для меньшего количества цветов, следовательно, r(n_1,\ldots,n_k) существует для любых k, n_1, \ldots n_k, \in \mathbb N , и теорема доказана.
}}

==Числа Рамсея больших размерностей==
{{Определение
|id=def5
|definition=
Пусть m,k,n_1,\ldots ,n_k \in \mathbb N, причём n_1,\ldots ,n_k \geqslant m. '''Число Рамсея''' r_m(n_1,\ldots ,n_k) — наименьшее из всех таких чисел x \in \mathbb N, что при любой раскраске m-элементных подмножеств x-элементного множества M в k цветов для некоторого i \in [1\ldots k] обязательно найдётся такое множество W_i, что |W_i|=n_i и все m-элементные подмножества множества W_i имеют цвет i. Число m называют '''размерностью''' числа Рамсея r_m(n_1,\ldots ,n_k).
}}
Заметим, что числа Рамсея размерности 2 — это определённые ранее числа Рамсея для клик.

{{Теорема
|id=ter4|about=4, Теорема Рамсея для чисел больших размерностей
|statement=Пусть m,k,n_1,\ldots,n_k {{---}} натуральные числа, причем k \geqslant 2, а n_1,\ldots ,n_k \geqslant m. Тогда существует число Рамсея r_m(n_1,\ldots n_k). 
|proof=
# Мы будем доказывать теорему по индукции. Начнем со случая k=2. Приступая к доказательству для числа r_m(n_1,n_2) мы будем считать доказанным утверждение теоремы для чисел Рамсея всех меньших размерностей и чисел Рамсея размерности m с меньшей суммой n_1+n_2. В качестве базы будем использовать случай чисел Рамсея размерности 2 разобранный выше. Итак, мы докажем, что r_m(n_1,n_2)-1 \leqslant p=r_{m-1}(r_m(n_1-1,n_2),r_m(n_1,n_2-1)). Для каждого множества M через M^k обозначим множество всех k-элементных подмножеств M. Рассмотрим (p+1)-элементное множество M и выделим в нём элемент a. Пусть M_0=M \setminus \{ a \}. Пусть \rho:M^m\rightarrow \{1, 2 \} — произвольная раскраска в два цвета. Рассмотрим раскраску \rho': M_0^{m-1} \rightarrow \{1, 2\} , определённую следующим образом: для каждого множества B \in M_0^{m-1} пусть \rho'(B) = \rho(B \cup \{ a \}). Так как |M_0|=p, либо существует r_m(n_1 — 1,n_2)-элементное подмножество M_1 \subset M_0, \rho'(B)=1 на всех B \in M_1^{m-1}, либо существует r_m(n_1,n_2-1)-элементное подмножество M_2 \subset M_0, \rho'(B)=2 на всех B \in M_2^{m-1}. Случаи аналогичны, рассмотрим первый случай и множество M_1. По индукционному предположен из |M_1|=r_m(n_1-1,n_2) следует, что либо существует n_1-1-элементное подмножество N_1 \subset M_1, \rho(A)=1 на всех A \in N^m_1, либо существует n_2-элементное подмножество N_2 \subset M_1, \rho(A)=2 на всех A \in N_2^m. Во втором случае искомое подмножество найдено (это N_2), рассмотрим первый случай и множество N=N_1 \cup \{a\}. Пусть A \in N^m. Если A \not\ni a, то A \in N_1^m и следовательно \rho(A)=1. Если же A \ni a, то множество A \setminus \{a\} \in N_1^{m-1} \subset M_1^{m-1} и поэтому \rho(A)=\rho'(A \setminus \{a \})=1. Учитывая, что |N|=n_1, мы нашли искомое подмножество и в этом случае.
# При k>2 будем вести индукцию по k с доказанной выше базой k=2. При k>2 мы докажем неравенство r_m(n_1,\ldots ,n_k) \leqslant q=r_m(r_m(n_1,\ldots ,n_{k-1}),n_k). Для этого мы рассмотрим множество M на q вершинах и произвольную раскраску \rho:M^m \rightarrow [1 \ldots k] в kцветов. Рассмотрим раскраску \rho':M^m \rightarrow \{0,k\}, в которой цвета 1,\ldots,k-1 раскраски \rho склеены в цвет 0. Тогда существует либо такое подмножество M_0 \subset M, что |M_0|=r_m(n_1,\ldots ,n_{k-1}) и \rho'(A)=0 на всех A \in M_0^m, либо существует такое n_k-элементное подмножество M_k \subset M, что \rho(A)=\rho'(A)=k на всех A \in M^m_k. Во втором случае M_k — искомое подмножество, а в первом случае заметим, что на любом подмножестве A \in M_0^m из \rho'(A)=0 следует \rho(A) \in [1 \ldots k-1]. Исходя из размера множества M_0 по индукционному предположению получаем, что найдется искомое подмножество множества M для одного из цветов 1,\ldots ,k-1, таким образом неравенство доказано, а из этого следует и существование числа Рамсея r_m(n_1,\ldots ,n_k). 
}}

==Числа Рамсея для произвольных графов==
Еще один способ обобщения классической теории Рамсея — замена клик на произвольные графы-шаблоны.
{{Определение
|id=def8
|definition=
Пусть H_1,H_2 — графы. '''Число Рамсея''' r(H_1,H_2) — это наименьшее из всех таких чисел x \in \mathbb N, что при любой раскраске рёбер полного графа на x вершинах в два цвета обязательно найдется подграф, [[Основные определения теории графов#isomorphic_graphs|изоморфный]] H_1 с рёбрами цвета 1 или подграф изоморфный H_2 с рёбрами цвета 2. 
}}
Существует и другое определение чисел Рамсея для произвольных графов.
{{Определение
|id=def16
|definition=
Пусть H_1,H_2 — графы. '''Число Рамсея''' r(H_1,H_2) — это наименьшее из всех таких чисел x \in \mathbb N, что для любого графа G на x вершинах либо в G найдется подграф изоморфный H_1, либо в \overline G найдется подграф изоморфный H_2. 
}}
Несложно показать, что эти определения эквивалентны (аналогично определениям для классических чисел Рамсея). Из результатов классической теории Рамсея становится понятно, что числа r(H_1,H_2) существуют.

{{Лемма
|id=l1|about=1|statement=Пусть m>1, а граф H таков, что v(H) \geqslant (m-1)(n-1)+1 и \alpha(H) \leqslant m-1, где v(H) {{---}} количество вершин в графе H. Тогда граф H содержит в качестве подграфа любое [[Основные определения теории графов#defTree|дерево]] на n вершинах.
|proof=
Зафиксируем m и проведем индукцию по n. 

'''База:''' для n=1 очевидно. 

'''Индукционный переход:''' Пусть верно для n-1, докажем для n. Рассмотрим произвольное дерево T_n на n вершинах, пусть дерево T_{n-1} получено из T_n удалением висячей вершины. Пусть U — максимальное независимое множество вершин графа H. Тогда |U|=\alpha(H) \leqslant m-1, следовательно v(H-U) \geqslant (m-1)(n-2)+1 и очевидно \alpha(H-U) \leqslant m-1.
По индукционному предположению, граф H-U содержит в качестве подграфа дерево T_{n-1}. Пусть a — вершина этого дерева, присоединив к которой висячую вершину, мы получим дерево T_n. Заметим, что множество U \cup \{a\} не является независимым ввиду максимальности U. Следовательно, вершина a смежна хотя бы с одной вершиной x \in U. Отметим, что x не принадлежит множеству вершин графа T_{n-1} и, присоединив вершину x к вершине a дерева T_{n-1}, получим дерево T_n в качестве подграфа графа H.
}}
{{Теорема
|id=ter5 
|author=5, Теорема Хватала
|statement=r(T_n,K_m)=(m-1)(n-1)+1, где T_n — дерево на n вершинах.
|proof=
Сперва докажем, что r(T_n,K_m) \geqslant (m-1)(n-1)+1. Для этого предъявим раскраску рёбер графа K_{(m-1)(n-1)}, в которой нет ни одного связного подграфа на n вершинах с рёбрами цвета 1 и нет клики на m вершинах с рёбрами цвета 2. Разобьём вершины графа на m-1 клику по n-1 вершине и покрасим все рёбра этих клик в цвет 1. Тогда любой связный подграф с рёбрами цвета 1 содержит не более n-1 вершины, в частности, нет подграфа с рёбрами цвета 1, изоморфного T_n. Рёбра цвета 2 (то есть, все оставшиеся рёбра) образуют (m-1)-дольный граф, в котором, очевидно, нет клики на m вершинах.
Теперь воспользуемся вторым [[#def16|определением]] числа Рамсея r(H_1, H_2). Рассмотрим произвольный граф G на {(m-1)(n-1)+1} вершинах. Предположим, что в графе G не существует клики на m вершинах. Тогда m>1 и \alpha( \overline G) \leqslant m-1. По [[#l1|лемме 1]], граф \overline G содержит в качестве подграфа любое дерево на n вершинах, в частности, дерево, изоморфное T_n.
}}

==Индуцированная теорема Рамсея==
{{Определение
|id=def9
|definition=Граф H называется '''индуцированным подграфом''' (англ. ''induced subgraph'') графа G если две вершины в H соединены ребром тогда и только тогда, когда они смежны в G. }}

{{Определение
|id=def10
|definition=Пусть H — граф. Граф G будем называть '''рамсеевским графом''' (англ. ''Ramsey’s graph'') для H, если при любой раскраске рёбер графа G в два цвета существует одноцветный по рёбрам индуцированный подграф графа G изоморфный H.}}

{{Определение
|id=def11
|definition='''Индуцированным числом Рамсея''' (англ. ''induced Ramsey’s number'') r_{ind}(H) для графа H будем называть минимальное число x \in \mathbb N, такое что существует рамсеевский граф на x вершинах для графа H.}}
Заметим, что при замене произвольного графа H на клику мы получаем частный случай классической теоремы Рамсея. 

{{Теорема
|id=ter6 
|about=6, Индуцированная теорема Рамсея
|statement=Для любого графа H существует рамсеевский граф G. 
}}

Доказательство [https://math.la.asu.edu/~andrzej/teach/mat598/lec8.pdf Induced Ramsey Theorem Proof] данной теоремы было приведено независимо различными математиками, однако благодаря ему получилось предоставить только очень грубые оценки значений индуцированных чисел Рамсея. В данный момент проблема нахождения сколько-нибудь точных границ индуцированных чисел Рамсея является нерешенной задачей математики.

==Особенности теории==
Результаты, полученные в теории Рамсея, обладают двумя главными характеристиками. Во-первых, они не позволяют получить сами структуры: теоремы лишь доказывают, что они существуют, но алгоритма для их нахождения не предлагают. Единственным способ найти нужную конструкцию зачастую является перебор. Во-вторых, чтобы искомые структуры существовали, обычно требуется, чтобы объекты, их содержащие, состояли из очень большого числа элементов. Зависимость числа элементов объекта от размера конструкции обычно, как минимум, экспоненциальная.

==См. также==
*[[Раскраска графа]]
*[[Раскраска двудольного графа в два цвета]]
*[[Теорема Турана об экстремальном графе]]

== Примечания==

== Источники информации ==
* [[wikipedia:Ramsey's theorem|Wikipedia — Ramsey's theorem]]
* [[wikipedia:Ramsey theory|Wikipedia — Ramsey theory]]
* [http://people.maths.ox.ac.uk/~gouldm/ramsey.pdf Ramsey Theory]
*[https://vtechworks.lib.vt.edu/bitstream/handle/10919/32873/Dickson_JO_T_2011.pdf?sequence=1&isAllowed=y An Introduction to Ramsey Theory on Graphs]
*[http://www.combinatorics.org/ojs/index.php/eljc/article/view/DS1| Small Ramsey Numbers by Stanisław Radziszowski]
[[Категория:Дискретная математика и алгоритмы]]
[[Категория:Дискретная математика]]
[[Категория:Теория графов]]
[[Категория: Раскраски графов]]