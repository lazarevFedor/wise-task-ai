Для [[Укладка графа на плоскости|планарного графа]] можно дать оценку сверху на [[Раскраска_графа#chromatic_number_difinition|хроматическое число]].

== Раскраска в 6 цветов == 
{{Лемма
|id=5deg_vertex_lemma 
|statement=В любом планарном графе G существует вершина [[Основные определения теории графов#def_undirected_graph_2 | степени]] не больше 5.
|proof=
Предположим это не так. Для любой вершины u_i графа G верно \mathrm{deg} \; u_i \geqslant 6 . Если сложить это неравенство для всех i , получим 2E \geqslant 6V . Но по [[Формула_Эйлера#EulerFormulaCons|следствию из теоремы Эйлера]] E \leqslant 3V-6 . Пришли к противоречию.
}}

{{Теорема 
|statement=
Пусть граф G — планарный. Тогда \chi (G) \leqslant 6.
|proof=
Докажем по индукции.

'''''База индукции'''''

Если граф содержит не более 6 вершин, то очевидно, что \chi (G) \leqslant 6.

'''''Индукционный переход'''''

Предположим, что для планарного графа с N вершинами существует раскраска в 6 цветов. Докажем то же для графа с N+1 вершиной.

По только что доказанной лемме в G найдётся вершина степени не больше 5. Удалим её; по предположению индукции получившийся граф можно раскрасить в 6 цветов. 

Вернём удалённую вершину и покрасим её в цвет, не встречающийся среди смежных ей вершин (ведь "занято" максимум 5 цветов). Индукционный переход доказан.
}}

== Раскраска в 5 цветов == 
{{Теорема
|about=
Хивуд
|statement=
Пусть граф G — планарный. Тогда \chi (G) \leqslant 5.
|proof=
[[Файл:Planar chromatic number 1.png|230px|thumb|right|u и смежные ей вершины]]
Начало доказательства такое же, как в предыдущей теореме, трудность возникает в индукционном переходе. Покажем что для случая с 5-ю цветами всё равно можно вернуть удалённую вершину так, чтобы раскраска осталась правильной.

Обозначим за u — возвращаемую вершину, v^{(k)} — вершину, покрашенную в k цвет.

Если среди вершин, смежных u , есть две вершины одного цвета, значит остаётся по меньшей мере один свободный цвет, в который мы и покрасим u .

Иначе, уложим полученный после удаления u граф на плоскость, вернём вершину u (пока бесцветную) и пронумеруем цвета в порядке обхода смежных вершин по часовой стрелке. 

Попробуем покрасить u в цвет 1. Чтобы раскраска осталась правильной, перекрасим смежную ей вершину v_1^{(1)} в цвет 3. Если среди смежных ей вершин есть вершины v_i^{(3)} , покрасим их в цвет 1, и так далее. Рассмотрим две необычные ситуации, которые могут наступить во время обхода:
#мы дойдём до уже однажды перекрашенной вершины (и хотим перекрасить её обратно, что не получится сделать). Видно что такая ситуация невозможна, поскольку мы меняли цвета вершин по схеме 1 \leftrightarrow 3, и если по завершении обхода мы получили две смежные вершины одного цвета, значит и до перекрасок в этом месте были две вершины одинакового цвета, а по предположению граф без u был раскрашен правильно.
#дойдём до вершины, смежной u , исходно имевшей цвет 3, которую перекрасить в 1 нельзя ( u теперь имеет цвет 1). 

Если этот процесс был успешно завершён, то получили правильную раскраску.
Если же в соответствии со вторым вариантом перекраска не удалась, это означает, что в графе есть цикл u v_1^{(1)} v_2^{(3)} v_3^{(1)} \ldots v_{k-1}^{(1)} v_k^{(3)} u .

Тогда попытаемся таким же образом перекрасить u в цвет 2, а смежную ей w_1^{(2)} в цвет 4 (со последующими перекрасками). Если удастся — раскраска получена.

Если нет, то получили ещё один цикл u w_1^{(2)} w_2^{(4)} w_3^{(2)} ... w_{k-1}^{(2)} w_k^{(4)} u . Но граф планарный, значит два полученных цикла пересекаются помимо вершины u по крайней мере ещё в одной, что невозможно, ведь вершины v_i первого цикла и w_j второго — разных цветов. Значит такой случай наступить не мог.

}}

{| cellpadding="10"
| || || || || Успешное перекрашивание || || || || || || Цикл 1—3, перекрасить не удаётся ||
|-
| || || || || [[Файл:Planar chromatic number 2.png|264px]] || || || || || || [[Файл:Planar chromatic number 4.png|228px]] 
|-
| || || || || [[Файл:Planar chromatic number 3.png|264px]] || || || || || || [[Файл:Planar chromatic number 5.png|228px]]
|}

Заметим, что не удаётся составить подобное доказательство для раскраски в четыре цвета, поскольку здесь наличие двух вершин одного цвета среди смежных u не исключает того, что при их (смежных вершин) раскраске использовались все возможные цвета.

== Раскраска в 4 цвета ==
{{Теорема
|about=
Проблема четырех красок
|statement='''Теорема о четырёх красках''' — утверждение о том, что всякую расположенную на сфере карту можно раскрасить четырьмя красками так, чтобы любые две области, имеющие общий участок границы, были раскрашены в разные цвета. При этом области могут быть как односвязными, так и многосвязными (в них могут присутствовать «дырки»), а под общим участком границы понимается часть линии, то есть стыки нескольких областей в одной точке общей границей для них не считаются.
}}
[[Файл:Map of Russia(four colour).png|230px|thumb|right|карта России раскрашенная в 4 цвета]]
Теорема о четырёх красках была доказана в 1976 году Кеннетом Аппелем и Вольфгангом Хакеном из Иллинойского университета. Это была первая крупная математическая теорема, доказанная с помощью компьютера. Первым шагом доказательства была демонстрация того, что существует определенный набор из 1936 карт, ни одна из которых не может содержать карту меньшего размера, которая опровергала бы теорему. Аппель и Хакен использовали специальную компьютерную программу, чтобы доказать это свойство для каждой из 1936 карт. Доказательство этого факта заняло сотни страниц. После этого Аппель и Хакен пришли к выводу, что не существует наименьшего контрпримера к теореме, потому что иначе он должен бы содержать, хотя не содержит, какую-нибудь из этих 1936 карт. Это противоречие говорит о том, что вообще не существует контрпримера. Изначально доказательство не было принято всеми математиками, поскольку его невозможно было проверить вручную. В дальнейшем оно получило более широкое признание, хотя у некоторых долгое время оставались сомнения.

Чтобы развеять оставшиеся сомнения, в 1997 году Робертсон, Сандерс, Сеймур и Томас опубликовали более простое доказательство, использующее аналогичные идеи, но по-прежнему проделанное с помощью компьютера. Кроме того, в 2005 году доказательство было проделано Джорджсом Гонтиром с использованием специализированного программного обеспечения (Coq v7.3.1)

== Эквивалентные формулировки ==
В теории графов утверждение теоремы четырёх красок имеет следующие формулировки:
* Хроматическое число планарного графа не превосходит 4.
* Рёбра произвольной триангуляции сферы можно раскрасить в три краски так, что все стороны каждого треугольника были раскрашены в разные цвета.

== Ложное доказательство ==
Ошибочным мнением считается, что решением проблемы четырех красок является - доказательство того, что невозможно начертить карту, на которой было бы всего лишь пять стран и каждая из этих стран примыкала бы к четырем остальным странам. Нетрудно доказать, что такую карту начертить нельзя. Можно предположить, что отсюда автоматически следует решение проблемы четырех красок для всех карт, но такое заключение неверно.
{| cellpadding="0"
| [[Файл:False disproof left.png|230px]] || [[Файл:False disproof right.png|230px]]
|- 

|}
Карта(слева) окрашена пятью цветами, и нужно изменить как минимум 4 из 10 регионов, чтобы получить окраску в четыре цвета(справа)

== Источники информации ==
* [http://matica.org.ua/lektsii-po-diskretnoy-matematike/3-08-6-raskraski-planarnich-grafov matica.org {{---}} Раскраска планарного графа ]
* [[wikipedia:ru:Проблема четырёх красок | Википедия {{---}} Проблема четырёх красок]]
* [[wikipedia:en:Four color theorem | Wikipedia {{---}} Four color theorem]]
[[Категория: Алгоритмы и структуры данных]]
[[Категория: Раскраски графов]]