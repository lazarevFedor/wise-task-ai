{{Теорема
|about=
о декомпозиционном барьере
|statement=
Существуют положительные вещественные числа c_{1} и c_{2}, такие что для любых натуральных V и E, удовлетворяющих неравенствам c_{1}V \leqslant E \leqslant c_{2}V^2, существует [[Определение сети, потока|сеть]] G с V вершинами и E ребрами, такая что для любого максимального потока f в G, любая его остаточная декомпозиция должна содержать \Omega (E) слагаемых (т.е. путей или циклов), причем каждый из путей (циклов) в декомпозиции должен иметь длину \Omega (V).
|proof=
[[Файл:DecompositionBarierExample.png|300px|thumb|right|Пример для V = 16, в который надо добавить нужное количество ребер]]
Возьмем c_{1} = \dfrac{11}{10} и c_{2} = \dfrac{1}{9}. Константа c_1 выбрана таким образом, чтобы между A и B было \Omega(E) ребер, а константа c_2 выбрана такой, потому что в рассматриваемой сети нельзя провести большее количество ребер. Чтобы получить искомую сеть, строится сеть, изображенная на рисунке, после чего добавляется нужное количество ребер из A в B. Пропускные способности ребер из A в B равны 1, остальных — +\infty (или просто достаточно большое число, например, V^2).
Теперь докажем саму теорему:
* Максимальный поток по модулю равен потоку через разрез, который разделяет A и B (т.е. пересекает все ребра с пропускной способностью 1). Поток по каждому пути в декомпозиции не превышает 1, а значит, этих путей не меньше, чем ребер между A и B, а их \Omega (E).
* По построению сети, любой путь из s в t содержит хотя бы \left(\dfrac{V}{3} + 3\right) ребер, что является \Omega (V).
}}

'''Следствие:''' Алгоритмы, которые могут выписать декомпозицию потока вместе с поиском самого потока ([[Схема алгоритма Диница|Алгоритм Диница]], [[Алгоритм Эдмондса-Карпа]], [[Алгоритм Форда-Фалкерсона, реализация с помощью поиска в глубину| Алгоритм Форда-Фалкерсона]] и подобные) не могут работать быстрее чем за O(VE), так как декомпозиция может быть сама по себе большой.

==См. также==
*[[Алгоритм Форда-Фалкерсона, реализация с помощью поиска в глубину| Алгоритм Форда-Фалкерсона]]
*[[Алгоритм Эдмондса-Карпа]]
*[[Схема алгоритма Диница| Алгоритм Диница]]
*[[Алгоритм поиска блокирующего потока в ациклической сети]]

==Источники информации==
* [https://youtu.be/PMqO0UCezqo?t=1h39m57s Андрей Станкевич: Лекториум, дополнительные главы алгоритмов, лекция 11]

[[Категория: Алгоритмы и структуры данных]]
[[Категория: Задача о максимальном потоке ]]