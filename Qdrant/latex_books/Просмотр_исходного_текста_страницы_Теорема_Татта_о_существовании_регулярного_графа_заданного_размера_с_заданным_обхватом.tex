{{Определение
|id = defMain. 
|definition = '''Обхват''' ''(англ. girth)'' графа G (обозначается g(G)) {{---}} это длина наименьшего простого цикла в графе G
}}

==Лемма о существовании вершины на заданном расстоянии==
{{Лемма
|statement= Пусть k, g \in \mathbb{N} , причём k \geqslant 3, G{{---}}граф, |V(G)| > \dfrac{k(k-1)^{g-1} - 2}{k - 2} , \forall v \in V(G) : d_G(v) \leqslant k; x, y \in V(G), d_G(x), d_G(y) \leqslant k - 1, тогда существует такая вершина z, что dist(x, z) \geqslant g - 1 и dist(y, z) \geqslant g.
|proof = 
[[Файл:Лемма к Татту.png|300px|thumb|left|Иллюстрация к теореме для k = 4. У вершины x(чёрной) не более k - 1 = 3 соседей (синих вершин), у каждой из k - 1 синих вершин не более k - 1 нерассмотренных соседей (красных вершин), то есть красных вершин не более (k - 1)^2, и так далее]]
Так как d_G(x), d_G(y) \leqslant k - 1 , а степени остальных вершин графа не более k, то на расстоянии не более g - 1 от y находится не более чем 1 + (k - 1) + \ldots + (k - 1)^{g - 1} = \sum\limits_{n=0}^{g - 1} (k - 1)^n = \dfrac{(k-1)^{g} - 1}{k - 2} вершин графа, а на расстоянии не более g - 2 от x находится не более чем 1 + (k - 1) + \ldots + (k - 1)^{g - 2} = \sum\limits_{n=0}^{g - 2} (k - 1)^n =\dfrac{(k-1)^{g - 1} - 1}{k - 2} вершин. Так как \dfrac{(k-1)^{g - 1} - 1}{k - 2} + \dfrac{(k-1)^{g} - 1}{k - 2} = \dfrac{k(k-1)^{g-1} - 2}{k - 2}, а |V(G)| > \dfrac{k(k-1)^{g-1} - 2}{k - 2}, то существует такая вершина z, что dist(x, z) \geqslant g - 1 и dist(y, z) \geqslant g.
}}

==Теорема==

{{Теорема
|id = thMain. 
|author = В. Татт
|statement = Пусть k, g, n \in \mathbb{N} , причём k, n \geqslant 3, n > \dfrac{k(k-1)^{g-1} - 2}{k - 2}, kn чётно. Тогда существует k-[[Основные определения теории графов#defRegularGraph | регулярный граф]] G c обхватом g(G) = g и количеством вершин |V| = n
|proof =
Пусть G_{set}(g, n, k) {{---}} семейство всех графов с n вершинами, обхватом g и максимальной степенью вершин не более k. При n > g очевидно, что G_{set}(g, n, k) \neq \emptyset: например, этому множеству принадлежит граф, состоящий из простого цикла на g вершинах и n - g изолированных вершин. 

Пусть v_{ {{---}} количество вершин степени меньше k в графе G, а dist_{ {{---}} максимальное из расстояний между парами вершин степени менее k в графе G. (при v_{, положим dist_{). Выберем в G_{set}(g, n, k) граф следующим образом: сначала выберем все графы с максимальным количеством рёбер, затем из них выберем графы с максимальным v_{, и, наконец, из оставшихся выберем граф G c максимальным dist_{. Если таких графов несколько, выберем любой.

Докажем, что G {{---}} регулярный граф степени k.

Предположим, что это не так и рассмотрим пару его максимально удаленных вершин степени менее k: пусть это будут вершины x и y (если вершина степени менее k в графе всего одна, то x =
 y). 

# [[Файл:Татт 1.png|300px|thumb|right|Расположение вершины z относительно вершин x и y]] Если dist_G(x, y) \geqslant g - 1, то соединим их ребром и получим граф G' = G \cup xy, G'\in G_{set}(g, n, k), при этом |E(G')| > |E(G)| (так как в графе G' есть все те рёбра, которые есть в G, и ребро xy). Значит, граф G не может быть выбран из множества G_{set}(g, n, k), так как у него не максимальное количество рёбер.
# Так d_G(x), d_G(y) \leqslant k - 1 , а степени остальных вершин графа не более k, то по лемме существует такая вершина z, что dist(x, z) \geqslant g - 1 и dist(y, z) \geqslant g. 
## d_G(z) . В таком случае, d_G(z) , что невозможно, согласно пункту 1. В таком случае:
## d_G(z) = k \geqslant 3, следовательно, существует ребро zu \in E(G), через которое проходят не все простые циклы длины g графа G, тогда g(G \setminus zu) = g(G) = g

[[Файл:Татт 2.png|300px|thumb|left|Получение графа G' из графа G]]

Пусть G' = G \setminus zu \cup zx. Тогда из

 dist_G(y, u) \geqslant dist_G(y, z) - 1 \geqslant g - 1 > dist_G(x, y) = dist_{.

следует, что d_G(u) = k.

g(G') = g(G) = g, |E(G')| = |e(G)| - 1 + 1 = |E(G)| . Тогда

 d_{G'}(x) = d_G(x) + 1 \leqslant k, d_{G'}(u) = d_G(u) - 1 = k - 1 ~~~ \textbf{(2)} .

Степени всех остальных вершин в G и G' совпадают. Тогда G' \in G_{set}(g, n, k). Из \textbf{(2)} следует, что v_{. Тогда ввиду выбора графа G должно быть v_{, что возможно лишь при d_{G'}(x) = k и d_G(x) = k - 1. Так как kn чётно, вершина x не может быть единственной вершиной степени менее k в графе G, поэтому x \neq y. 

Докажем, что dist_{G'}(y, u) > dist_{G}(y, x). Действительно, рассмотрим путь P: y \leadsto u, который реализует расстояние между y и u в G'. Если P проходит только по рёбрам G, то, учитывая \textbf{(1)}, получаем

 dist_{G'}(y, u) = dist_G(y, u) \geqslant g - 1 > dist_G(y, x) 

Значит, P проходит по ребру zx. Следовательно, P содержит путь по рёбрам графа G от y до одной из вершин x или z и ребро xz. Тогда

 dist_{G'}(y, u) = \min(dist_G(y, x), dist_G(y, z)) + 1 > dist_G(y, x) ,

так как dist_G(y, z) \geqslant g > dist_G(y, x). Таким образом

 dist_{ dist_G(y, x) dist_{

Получили противоречие с принципом выбора графа G, что доказывает, что G {{---}} k-регулярный.

}}

==См. также==
* [[Основные определения теории графов]]

==Источники информации==
* Карпов В. Д. - Теория графов, стр 108

[[Категория: Алгоритмы и структуры данных]]
[[Категория: Обходы графов]]