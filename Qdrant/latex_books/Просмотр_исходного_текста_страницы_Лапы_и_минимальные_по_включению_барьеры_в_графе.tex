{{Определение
|id = claw
|neat = 1 
|definition = '''Лапой''' (англ. ''claw'') называется индуцированный подграф графа G, [[ Основные определения теории графов#isomorphic_graphs | изоморфный]] [[ Основные определения теории графов#defBiparateGraph | двудольному ]] графу K_{1, 3}.
}} [[ Файл:Lapa.png|180px|thumb|right|Лапа ]]
 
 

 
 
 
{{Определение
|id = claw_center
|neat = 1 
|definition ='''Центром лапы''' (англ. ''claw center'') называется вершина [[ Основные определения теории графов#def_graph_degree_1 | степени ]] три в лапе.
}}
 
 

{{Определение
|id = minimum_barrier
|neat = 1 
|definition = '''Минимальным по включению [[ Декомпозиция Эдмондса-Галлаи#barrier | барьером ]] '''(англ.''minimal barrier'') называется барьер, который перестанет быть барьером при исключении из него любой вершины.
}}

 
 
 

{{Теорема
|id = theorem_about_claw
|statement = Пусть B {{---}} минимальный по включению барьер графа G, тогда каждая вершина B {{---}} центр лапы в G.
|proof = Предположим, что x\in B не является центром лапы. Тогда x смежна не более чем с двумя [[Отношение связности, компоненты связности#def2 | компонентами связности]] графа G \setminus B. 
Пусть B' = B\setminus \{ x \}. 
Найдём соотношение между [[ Теорема Татта о существовании полного паросочетания#odd | \mathrm{odd} ]](G\setminus B')\ и \mathrm{odd}(G\setminus B)\ . 
Для этого рассмотрим всевозможные случаи количества компонент связности в графе G \setminus B, с которыми смежна x, и посмотрим на их четности (компоненты в B, с которыми смежна x, нас не интересуют). 
# x смежна с двумя компонентами связности графа G \setminus B.[[ Файл:GraphsForLaps.png|300px|thumb|right|x смежна с двумя компонентами связности графа G \setminus B ]] 
#:* Одна компонента чётная, другая {{---}} нечетная. Тогда \mathrm{odd}(G\setminus B')\ = \mathrm{odd}(G\setminus B)\ - 1 . 
#:* Обе компоненты чётные: \mathrm{odd}(G\setminus B')\ = \mathrm{odd}(G\setminus B)\ + 1 . 
#:* Обе компоненты нечётные: \mathrm{odd}(G\setminus B')\ = \mathrm{odd}(G\setminus B)\ - 1 . 
#x смежна с одной компонентой связности графа G \setminus B. 
#:* Эта компонента чётная: \mathrm{odd}(G\setminus B')\ = \mathrm{odd}(G\setminus B)\ + 1 . 
#:* Эта компонента нечётная: \mathrm{odd}(G\setminus B')\ = \mathrm{odd}(G\setminus B)\ - 1 . 
# x не смежна ни с какой компонентой связности графа G \setminus B. 
#: \mathrm{odd}(G\setminus B')\ = \mathrm{odd}(G\setminus B)\ + 1 . 
Для любого из случаев выполнено: \mathrm{odd}(G\setminus B')\ \geqslant \mathrm{odd}(G\setminus B)\ - 1 . 
B {{---}} барьер \Leftrightarrow \mathrm{odd}(G\setminus B) - |B| = \mathrm{def}(G) . 
Тогда \mathrm{odd}(G\setminus B')\ \geqslant |B| - 1 + \mathrm{def}(G) . 
То есть \mathrm{odd}(G\setminus B') - |B'|\ \geqslant \mathrm{def}(G) . 
Тогда возможны два случая:
# Если выполняется равенство \mathrm{odd}(G\setminus B') - |B'|\ = \mathrm{def}(G) , то, по определению, B' является барьером. 
#: Но |B'| , а значит, B не является минимальным по включению барьером \Rightarrow противоречие условию теоремы. 
# Если \mathrm{odd}(G\setminus B') - |B'|\ > \mathrm{def}(G), то есть \mathrm{odd}(G\setminus B') - |B'|\ > \mathrm{odd}(G\setminus B) - |B|\. 
#: Тогда, по [[ Декомпозиция Эдмондса-Галлаи#Th_Berge| теореме Бержа]], \mathrm{def}(G) \ne \mathrm{odd}(G\setminus B) - |B|\ \Rightarrow противоречие. 
В обоих случаях мы пришли к противоречию, значит, наше предположение неверно и \forall x\in B является центром лапы в G.
}}

{{Утверждение 
|id = proposal1 
|author = D.P.Sumner, M.Las Vergnas
|about = следствие из теоремы
|statement = Пусть G {{---}} связный граф, не содержащий лапы, v(G) чётно. Тогда G имеет [[ Паросочетания: основные определения, теорема о максимальном паросочетании и дополняющих цепях#perfect_matching | совершенное паросочетание ]].
|proof= Пусть B {{---}} минимальный по включению барьер графа G. Тогда, по предыдущей теореме имеем B = \varnothing .
По условию G {{---}} связный граф с чётным числом вершин \Rightarrow \mathrm{odd}(G\setminus \varnothing )\ = 0 . 
B {{---}} барьер и он пуст \Leftrightarrow \mathrm{def}(G) = \mathrm{odd}(G\setminus \varnothing) - |\varnothing|\ = 0 . Значит, количество вершин, не покрытых [[ Паросочетания: основные определения, теорема о максимальном паросочетании и дополняющих цепях#maximal_matching | максимальным паросочетанием ]], равно 0, то есть в G существует совершенное паросочетание.
}}

==См. также==

* [[ Декомпозиция Эдмондса-Галлаи ]]
* [[ Паросочетания: основные определения, теорема о максимальном паросочетании и дополняющих цепях ]]
* [[ Теорема Татта о существовании полного паросочетания ]]
* [[ Теорема Самнера — Лас Вергнаса ]]

== Источники информации ==
* Карпов Д. В. {{---}} Теория графов, стр 55
* Ловас Л., Пламмер М. {{---}} Прикладные задачи теории графов. Теория паросочетаний в математике, физике, химии, стр 165-166

[[ Категория: Алгоритмы и структуры данных ]]
[[ Категория: Задача о паросочетании ]]