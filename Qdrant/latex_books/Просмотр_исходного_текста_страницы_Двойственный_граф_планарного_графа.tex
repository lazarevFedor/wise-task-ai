{{Определение
|neat=neat
|definition=ГрафНа самом деле, ''двойственный граф'' — '''псевдограф''', поскольку в нём могут быть петли и кратные рёбра. G' называется '''двойственным''' (англ. ''dual graph'') к [[Укладка графа на плоскости|планарному графу]] G, если:
# Вершины G' соответствуют граням G.
# Между двумя вершинами в G' есть ребро тогда и только тогда, когда соответствующие грани в G имеют общее ребро.
}}
[[Файл:Dual_graph_2.png|180px|thumb|right|Граф (белые вершины) и двойственный ему (серые вершины).]]

Чтобы для данного плоского графа G построить двойственный G', необходимо поместить по вершине G' в каждую грань G (включая внешнюю), а затем, если две грани в G имеют общее ребро, соединить ребром соответствующие им вершины в G' (если грани имеют несколько общих рёбер, соответствующие вершины следует соединить несколькими параллельными рёбрами). В результате всегда получится плоский псевдограф.

Например, существуют графы, двойственные себе: — K_1 и K_4. Далее мы убедимся, что среди полных графов только они обладают таким свойством.

== Свойства ==
[[Файл:Treenflower new.png|250px|thumb|right|Дерево и двойственный к нему «цветок».‎]]
* Если G' — ''двойственный'' к двусвязному графу G, то G — ''двойственный'' к G'.
* У одного и того же графа может быть несколько ''двойственных'', в зависимости от конкретной укладки (см. картинку).
* Поскольку любой трёхсвязный планарный граф допускает только одну укладку на сфереХарари, Ф. Теория графов. — М.: Книжный дом «ЛИБРОКОМ», 2009. — Теорема 11.5 — С. 130. — ISBN 978­-5­-397­-00622­-4, у него должен быть единственный ''двойственный граф''.
* [[Мост, эквивалентные определения|Мост]] переходит в петлю, а петля — в мост. Частный случай: полный граф K_2
* Мультиграф, ''двойственный'' к дереву, — цветок.

== Самодвойственные графы ==
{{Определение
|definition=Планарный граф называется '''самодвойственным''' (англ. ''self-dual graph''), если он изоморфен своему двойственному графу.
}}

{|align="center"
 |-valign="top"
 |[[Файл:Wheel8_new2.png|500px|thumb|left|Колесо и двойственный ему граф {{---}} тоже колесо.]]
 |[[Файл:K4_new.png|250px|thumb|right|K_4 (он же колесо).]]
 |}

{{Утверждение
|neat=neat
|statement=K_1 и K_4 — самодвойственные графы. Среди полных графов других самодвойственных нет.
|proof=Проверить, что K_1 и K_4 полны и самодвойственны несложно. Докажем, что других нет.Поскольку грани графа переходят в вершины, количество вершин и граней в исходном графе должно совпадать, т.е. V = F.Подставив в [[Формула Эйлера|формулу Эйлера]] имеем: 2V = E + 2 \Leftrightarrow V = \dfrac{E}{2} + 1.В полном графе E = \dfrac{V \cdot (V - 1)}{2}.Получаем квадратное уравнение: V^2 - 5V + 4 = 0.Его решения: V_1 = 1 и V_2 = 4.Таким образом, чтобы ''полный'' граф был ''самодвойственным'', в нём должна быть ровно '''одна''' или '''четыре''' вершины.
}}

{{Утверждение
|neat=neat
|statement=Все колёса самодвойственны.
|proof=Это утверждение очевидно.Достаточно убедиться, что два варианта укладки колеса (вершина с большой степенью внутри или вершина с большой степенью снаружи) двойственны друг другу.
}}

== См. также ==
*[[Формула Эйлера]]
*[[Укладка графа на плоскости]]
*[[Гамма-алгоритм]]

== Примечания ==

== Источники информации==
* [https://ru.wikipedia.org/wiki/%D0%94%D0%B2%D0%BE%D0%B9%D1%81%D1%82%D0%B2%D0%B5%D0%BD%D0%BD%D1%8B%D0%B9_%D0%B3%D1%80%D0%B0%D1%84 Википедия — Двойственный граф]
* [https://ru.wikipedia.org/wiki/%D0%9F%D0%BB%D0%B0%D0%BD%D0%B0%D1%80%D0%BD%D1%8B%D0%B9_%D0%B3%D1%80%D0%B0%D1%84 Википедия — Планарный граф]

[[Категория: Алгоритмы и структуры данных]]
[[Категория: Укладки графов]]