Особенности [[укладка графа на плоскости | планарных графов]] позволяют реализовать поиск [[разрез, лемма о потоке через разрез | минимального разреза]] в планарном графе с лучшей асимптотикой, чем в произвольном графе, где поиск минимального разреза требует поиска максимального потока, то есть работает за O(V^2E). Лучший известный алгоритм для планарных графов работает с асимптотикой O(V\log V).

==Алгоритм за O(n^2 log n)==

===Идея алгоритма===
Рассмотрим граф G^d, [[двойственный граф планарного графа | двойственный]] данному графу G. Рассмотрим кратчайший путь \Pi между вершинами \xi^s и \xi^t, принадлежащими граням \varphi_s и \varphi_t, соответствующим вершинам s и t в исходном графе. Определим \Pi-левые и \Pi-правые ребра, \Pi-левые будут лежать "слева" по пути из \xi^s в \xi^t, а \Pi-правые "справа". Формальное определение будет дано в следующем разделе. Тогда минимальный цикл в G^d, ограничивающий \varphi_t, будет \xi_i-циклом {{---}} циклом, содержащим ровно одно \Pi-левое и одно \Pi-правое ребро, причем \Pi-левое ребро входит в вершину \xi_i. 

[[Файл:Xi_cycles.png|600px|thumb|center|Кси-циклы]]

Каждый такой цикл соответствует разрезу в G. Тогда алгоритм можно записать так:
# Построим граф G^d, двойственный исходному графу G
# Найдем кратчайший путь \Pi между вершинами \xi^s и \xi^t
# Для каждой вершины \xi_i\in\Pi найдем кратчайший \xi_i-цикл
# Найдем среди этих циклов цикл минимальной длины и построим соответствующий ему разрез в G

===Корректность и асимптотика===
Рассмотрим неориентированный граф G=(V, E). Будем считать его трёхсвязным, если это не так, [[триангуляция полигонов (ушная + монотонная) | триангулируем]] его, используя ребра нулевой пропускной способности, это не изменит значение минимального (s-t)- разреза. Минимальный разрез исходного графа будет состоять из ребер минимального разреза нового графа, которые есть и в исходном графе.

G трёхсвязен, поэтому существует единственный [[двойственный граф планарного графа | двойственный]] ему граф G^d=(X, A), при этом G^d также трёхсвязен. Будем обозначать множество граней G как F, множество граней G^d как \Phi. Между элементами каждой из пар множеств V-\Phi, E-A и F-X существует взаимно однозначное соответствие. Будем обозначать как \alpha^d\in E элемент, соответствующий \alpha\in A. Тогда длину ребра \alpha определим как l(\alpha)=c(\alpha^d).

Будем обозначать грани G^d, соответствующие вершинам s и t, как \varphi_s и \varphi_t соответственно. Далее будем считать, что \varphi_s {{---}} внешняя грань G^d.

{{Лемма
|statement=
C {{---}} минимальный (s-t)- разрез, тогда C^d=\{\alpha \mid \alpha^d\in C\} {{---}} цикл минимальной длины, ограничивающий \varphi_t.
|proof=
Формальное доказательство леммы громоздкое и здесь приведено не будет, однако сама лемма достаточно интуитивно понятна."The following lemma is intuitive; however its formal proof is tedious, and therefore, omitted.", Maximum flow in planar networks, Itai & Shiloach, 1979, p. 147 
}}

Пусть \xi^s\in\varphi_s, \xi^t\in\varphi_t, \Pi=(\xi^s=\xi_1, \ldots, \xi_k=\xi^t) {{---}} кратчайший путь между \xi^s и \xi^t в G^d, \alpha_i {{---}} ребро между \xi_{i-1} и \xi_i для i=2,\ldots,k. Будем называть ребро \xi-\xi_i\in A \Pi-левым, если оно находится между \alpha_i и \alpha_{i+1} при обходе ребер, входящих в \xi_i, по часовой стрелке от \alpha_i, и \Pi-правым, если оно находится между \alpha_{i+1} и \alpha_i.
Для того, чтобы это определение имело смысл для первой и последней вершин пути, можно добавить две вершины \xi_0 и \xi_{k+1} и два ребра \alpha_1=\xi_0-\xi_1 и \alpha_{k+1}=\xi_k-\xi_{k+1}. Важно, что никакое ребро не является одновременно \Pi-левым и \Pi-правым.

[[Файл:Виды_ребер.png|600px|thumb|center|П-левые и П-правые ребра]]

Будем называть \xi_i-циклом простой цикл, который содержит ровно одно \Pi-левое и одно \Pi-правое ребро, при этом \Pi-левое ребро входит в вершину \xi_i. Любой \xi_i-цикл ограничивает \varphi_t.

{{Лемма
|statement=
Пусть C {{---}} кратчайший ограничивающий \varphi_t цикл. Тогда существует \xi_i-цикл такой же длины.
|proof=
\Pi {{---}} кратчайший путь между \xi^s и \xi^t, поэтому любой содержащийся в \Pi путь между \xi_i и \xi_j также кратчайший. При этом любой ограничиващий \varphi_t цикл обязан пересекать \Pi. Цикл, не являющийся \xi_i-циклом, не будет кратчайшим, так как его можно будет сократить вдоль пути \Pi.
}}

Для поиска минимального \xi_i-цикла построим из неориентированного графа G ориентированный граф \vec{G}^d следующим образом: все \Pi-левые ребра ориентируем из вершин \xi\in\Pi, все \Pi-правые ребра ориентируем в вершины \xi\in\Pi, а остальные ребра u\leftrightarrow v заменим на два ребра u\rightarrow v и v\rightarrow u. 

{{Лемма
|statement=
Пусть \xi_i\in\Pi, P_i {{---}} кратчайший нетривиальный путь из \xi_i в \xi_i в \vec{G}^d. Тогда соответствующий путь в G^d является кратчайшим \xi_i-циклом.
|proof=
Из определения \vec{G}^d следует, что если нетривиальный путь из \xi_i в \xi_i содержит больше одного \Pi-правого или \Pi-левого ребра, то в нем есть самопересечение, то есть он не является простым, следовательно, не является кратчайшим \xi_i-циклом.
}}

Таким образом, нахождение минимального \xi_i-цикла эквивалентно нахождению кратчайшего нетривиального пути из \xi_i в \xi_i в \vec{G}^d. Это можно сделать за O(m\log n), например, с помощью [[алгоритм Дейкстры | алгоритма Дейкстры]]. Так как в планарных графах m=O(n), это равно O(n\log n). Максимум n таких поисков дадут итоговую асимптотику O(n^2\log n).

==См. также==
* [[Двойственный граф планарного графа]]
* [[Разрез, лемма о потоке через разрез]]
* [[Укладка графа на плоскости]]

==Примечания==

==Источники информации==
* [http://verona.dei.unipd.it/~prin08/primo_meeting/italiano.pdf Improved Algorithms for Min Cuts and Max Flows in Undirected Planar Graphs]
* [https://pdfs.semanticscholar.org/b838/35db99723f922b7b286d10952b08c68970a7.pdf Maximum flow in planar networks, Itai & Shiloach, 1979]

[[Категория: Алгоритмы и структуры данных]]
[[Категория: Укладки графов]]