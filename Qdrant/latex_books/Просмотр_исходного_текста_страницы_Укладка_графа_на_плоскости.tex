{| class="standard" border=0
|[[Файл:planar_graph.png|250px|thumb|left|Пример планарного графа. Синим контуром обозначены грани, за исключением внешней грани (всего 5 граней). Обратите внимание, что внутри грани могут содержаться другие ребра и вершины.]]
|
{{Определение
|definition=
[[Основные_определения_теории_графов|Граф]] '''обладает укладкой''' в пространстве L, если он изоморфен графу, вершинами которого являются некоторые точки пространства, а ребрами {{---}} жордановы кривые Жордановыми кривыми, неформально говоря, называют кривые без самопересечений, которые можно «нарисовать одним росчерком пера»., соединяющие соответствующие вершины, причем
# Кривая, являющаяся ребром не проходит через другие вершины графа, кроме вершин, которые она соединяет;
# Две кривые, являющиеся ребрами, пересекаются лишь в вершинах, инцидентных одновременно обоим этим ребрам.
Соответствующий граф, составленный из точек пространства и жордановых кривых из L, называют '''укладкой''' исходного графа.
}}

{{Определение
|id= defplanar
|definition=
Граф называется '''планарным''' ''(англ. planar graph)'', если он обладает укладкой на плоскости. '''Плоским''' ''(англ. plane graph, planar embedding of the graph)'' называется граф уже уложенный на плоскости.
}}
|}
{| class="standard" border=0
|
{{Определение
|definition=
Плоский граф разбивает плоскость на несколько областей, называемых '''гранями''' ''(англ. faces)''. Одна из граней не ограничена, ее называют '''внешней''' ''(англ. external)'' гранью, а остальные {{---}} '''внутренними''' ''(англ. unbounded)'' гранями.
}}

Для плоских графов есть простое соотношение, называемое [[Формула_Эйлера|формулой Эйлера]]: V - E + F = 2, где V {{---}} вершины, E {{---}} ребра, F {{---}} грани.

Это свойство позволяет в некоторых случаях просто доказывать [[Непланарность K5 и K3,3|непланарность некоторых графов, например непланарность K_5 и K_{3,3}]].

Понятно, что любой граф, содержащий подграф K_5 или K_{3,3} непланарен. Оказывается, верно и обратное утверждение, но для его формулировки потребуется вспомогательное определение:
|
[[Файл:K33.png|200px|thumb|Полный двудольный граф K_{3,3}. Этот граф непланарен, и его не получится изобразить на плоскости без пересечений ребер.]]
|}
{{Определение
|id=def_hmp
|definition= 
[[Файл:Gomeomorf.png|350px|right]]
Введем отношение R следующим образом: два графа находятся в отношении R, если один можно свести к другому заменой вершины степени 2 на ребро между вершинами смежными ей, или наоборот, добавлением вершины степени два на ребро (см. картинку).

Отношением '''гомеоморфизма''' (или '''топологической эквивалентности''') назовем [[Транзитивное_замыкание|транзитивное замыкание]] отношения R: R*. 
}}

Граф планарен тогда и только тогда, когда он не содержит подграфов, гомеоморфных K_5 и K_{3,3}: [[Теорема Понтрягина-Куратовского| теорема Понтрягина-Куратовского]].

{{Теорема
|statement= 
В трехмерном евклидовом пространстве любой граф укладывается.
|proof=
Все вершины произвольного графа G помещаем в различных точках координатной оси OX. Рассмотрим пучок плоскостей, проходящих через ось OX, и зафиксируем |E| различных таких плоскостей. Теперь каждое ребро (u, v) изобразим полуокружностью, проходящей в соответствующей плоскости через вершины u, v. Ясно, что различные ребра не будут пересекаться кроме как в общих вершинах. 
}}

==См. также==
* [[Формула_Эйлера|Формула Эйлера]]
* [[Локализация_в_ППЛГ_методом_полос_%28персистентные_деревья%29|Локализация в ППЛГ методом полос (персистентные деревья)]]

==Примечания==

==Источники информации==
* Асанов М, Баранский В., Расин В. - Дискретная математика - Графы, матроиды, алгоритмы
* Харари, Ф. Теория графов. — М.: Книжный дом «ЛИБРОКОМ», 2009. — С. 126. — ISBN 978­-5­-397­-00622­-4.

[[Категория: Алгоритмы и структуры данных]]
[[Категория: Укладки графов ]]