==Теорема Форда-Фалкерсона==
Задача о потоке минимальной стоимости состоит в нахождении среди всех [[Определение сети, потока|потоков]] данной величины наименее затратного.

{{Лемма
|about=
о представлении потоков
|statement=
Пусть f и g {{---}} потоки в сети G . Тогда g можно представить как сумму f + f', где f' {{---}} поток в остаточной сети G_f.
|proof= 
Рассмотрим произвольное ребро (u, v) из G . f'(u, v) = g(u, v) - f(u, v) \leqslant c(u, v) - f(u, v) = c_f(u, v) . Таким образом, поток через каждое ребро не превосходит пропускной способности остаточной сети. 
Антисимметричность и закон сохранения потока проверяются аналогично [[Лемма о сложении потоков|лемме о сложении потоков]].
}}

{{Теорема
|statement=
Пусть: G {{---}} сеть с истоком s и стоком t , f {{---}} поток минимальной стоимости в сети G среди потоков величины a , P {{---}} путь минимальной стоимости s \leadsto t в остаточной сети.
Тогда:
\forall \delta : 0 \leqslant \delta \leqslant c_f(P) поток f + \delta \cdot f_P {{---}} поток минимальной стоимости среди потоков величины a + \delta, где \delta \cdot f_P {{---}} поток величины \delta, проходящий по пути P.

|proof=
Пусть g {{---}} поток минимальной стоимости величины a + \delta в G. Представим g = f + f', где f' {{---}} поток в остаточной сети G_f. Тогда разность g - f будет потоком в сети G_f и по [[Лемма о сложении потоков|лемме о сложении потоков]] его величина будет равна \delta.

По [[Теорема о декомпозиции|теореме о декомпозиции]] g - f можно представить как сумму элементарных потоков вдоль путей P_i : s \leadsto t и циклов C_i. В этом представлении нет отрицательных циклов, иначе прибавление его к f даст поток меньшей стоимости. Если есть положительный цикл, то вычтем его из g и получим поток меньшей стоимости. Таким образом, p(C_i) = 0 для всех циклов.

Тогда p(g - f) = \sum\limits_{P_i} p(P_i)\cdot c_f(P_i) \geqslant p(P) \cdot \sum\limits_{P_i}c_f(P_i) \geqslant p(P) \cdot \delta. 

Отсюда p(g) \geqslant p(f) + p(P) \cdot \delta \geqslant p(g) и поток f + \delta \cdot f_P {{---}} минимальный.

}}

==Алгоритм==
На основании теоремы построим алгоритм. На каждой итерации алгоритма будем находить путь минимальной стоимости из s в t в остаточной сети и дополнять поток вдоль него. Выбирать алгоритм для поиска кратчайших путей следует с учетом того, что в ходе алгоритма появляются ребра отрицательного веса.

===Описание===
* Начало.
* '''Шаг 1'''. Для каждого ребра зададим поток равный 0.
* '''Шаг 2'''. Построим остаточную сеть G_f.
* '''Шаг 3'''. Если существует путь s \leadsto t в остаточной сети G_f {{---}} перейдем к '''шагу 4''', иначе к '''шагу 6'''.
* '''Шаг 4'''. Найдем путь s \leadsto t c минимальной стоимостью: путь P.
* '''Шаг 5'''. Дополним поток f вдоль пути P.
* '''Шаг 6'''. Поток минимальной стоимости найден, т.к в остаточной сети не осталось ни одного пути.
* Конец.

===Реализация===

Пусть задана структура Edge:
 '''struct''' edge:
 '''int''' from 
 '''int''' to 
 '''double''' c // пропускная способность ребра
 '''double''' flow // поток через ребро
 '''double''' price // стоимость перемещения единицы потока через ребро

Будем использовать структуру для хранения ребер графа G.

 '''Edge[]''' findMinCostMaxFlow(G: (V, E), '''int''' s, '''int''' t): 
 '''for''' edge '''in''' E:
 edge.flow = 0
 '''while''' \exists путь s \leadsto t в остаточной сети G_f:
 P = путь s \leadsto t с наименьшей стоимостью.
 maxFlow = \displaystyle \min_{edge \in P} edge.c - edge.flow
 '''for''' edge '''in''' P:
 edge.flow += maxFlow
 '''return''' E

===Асимптотика===
Каждая итерация выполняется за время работы поиска кратчайшего пути, обозначим его F(V, E). В сетях с целочисленной пропускной способностью итераций будет не более |f|. 

Итого получаем время работы O(F(V, E) \cdot |f|).

==См. также==
*[[Поток минимальной стоимости| Поток минимальной стоимости]]
*[[Использование потенциалов Джонсона при поиске потока минимальной стоимости| Использование потенциалов Джонсона при поиске потока минимальной стоимости]]

==Источники информации==
*[https://ru.wikipedia.org/wiki/Теорема_Форда_—_Фалкерсона Wikipedia {{---}} Теорема Форда-Фалкерсона]
* Ravindra Ahuja, Thomas Magnanti, James Orlin. Network flows (1993)

[[Категория:Алгоритмы и структуры данных]]
[[Категория: Задача о потоке минимальной стоимости]]