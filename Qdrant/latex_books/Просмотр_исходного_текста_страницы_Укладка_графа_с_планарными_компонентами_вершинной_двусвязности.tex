{{Теорема
|about=об укладке графа с планарными компонентами вершинной двусвязности
|statement=Если [[Отношение вершинной двусвязности|компоненты вершинной двусвязности]] [[Основные определения: граф, ребро, вершина, степень, петля, путь, цикл|графа]] G [[Укладка графа на плоскости|планарны]], то и сам граф G планарен.
|proof=

Докажем вспомогательную лемму.

{{Лемма
|id=l1
|about=I
|statement=Пусть графы G_1 и G_2 планарны. Граф G получается из G_1 и G_2 совмещением вершин v_1 \in G_1 и v_2 \in G_2. Тогда G планарен.
|proof=

Предварительно заметим, что в доказательстве используются утверждения [[Укладка графа с планарными компонентами реберной двусвязности#l1|леммы I]] и [[Укладка графа с планарными компонентами реберной двусвязности#l2|леммы II]] из статьи [[Укладка графа с планарными компонентами реберной двусвязности]]. Итак, уложим G_2 на сфере и уложим G_1 на плоскости так, чтобы ребро e_1 \in G_1 смежное с v_1 (если таковое имеется) оказалось на границе внешней грани (по [[#l2|лемме II]] это возможно). Если такого ребра e_1 не существует, значит вершина v_1 изолирована, в таком случае возьмем любую укладку G_1 на плоскости и переместим точку, соответствующую v_1 во внешнюю грань. Иначе сожмем часть плоскости, содержащую укладку G_1 так, чтобы она вмещалась в одну из граней укладки G_2 смежную с v_1. Рассмотрим множество U вершин смежных с v_1. Уберем кривые, соответствующие ребрам, инцидентным v_1. Ясно, что после этого множество вершин U лежит на внешней границе укладки G_1. Соединим теперь каждую вершину из U c v_2 непересекающимися жордановыми линиями так, чтобы они не задевали укладок G_1 и G_2 (рис. 1). Таким образом мы совместили вершины v_1 и v_2 в вершине v_2, а значит получили укладку графа G на сфере, следовательно G - планарен.
[[Файл: Planar_vertex_biconnected_1.png|300px|center|thumb|рис. 1.]]
}}

Докажем утверждение теоремы для одной из компоненты связности графа G. Ясно, что имея укладки на плоскости каждой из компонент связности графа, мы можем получить укладку на плоскости и всего графа.
Итак пусть граф G связен. Если G = K_1, то G очевидно планерен, поэтому предположим, что |EG| \geqslant 1 , а значит имеется по-крайней мере один блок в G. Рассмотрим связный подграф T графа блоков и точек сочленений графа G такой, что \forall v - т.с. G имеем \deg(v) \geqslant 2. Из [[Граф блоков-точек сочленения#lemma1|леммы]] и из связности T получаем, что T &mdash; двудольное [[Дерево, эквивалентные определения|дерево]]. 

Докажем индукцией по числу вершин в графе T, что подграф G' графа G состоящий из блоков графа G принадлежащих графу T планарен (далее будем говорить, что G' соответствует T).

'''База индукции.''' 

Если |VT| = 1, то граф T тривиальный. Его единственная вершина &mdash; это блок графа G, который по утверждению теоремы планарен.

'''Индукционный переход.''' 

Пусть утверждение верно для |VT| . Рассмотрим T, для которого |VT| = m > 1, и соответствующий T подграф G' графа G. Докажем, что G' планарен. 

Положим G_1 &mdash; это блок графа G' являющийся висячей вершиной дерева T (вспомним, что в дереве, в котором более одной вершины, всегда есть есть висячие вершины, и то, что висячими вершинами в графе блоков и т.с. не могут быть т.с.), a v {{---}} т.с. в G' смежная с G_1 в T. G_1 планарен по утверждению теоремы, т.к. блоки графа G' совпадают с блоками графа G. Заметим, что \deg(v) > 1, т.к. v {{---}} т.с., следовательно не висячая. Рассмотрим два случая:

#\deg(v) = 2 в T (рис. 2). Обозначим за T' T\backslash \{u,v\}. Поскольку степень ни одной из т.с. G' принадлежащих T (кроме удаленной v) не уменьшилась, значит T' удовлетворяет условиям на T из предположения индукции. Заметим, что VT' = VT - 2 = m - 2 . Заметим также, что T' связен, т.к. u и v по очереди были висячими вершинами T и T\backslash \{u\}.[[Файл: Planar vertex biconnected 2.png|270px|center|thumb|рис. 2. Красные {{---}} точки сочленений. Голубые {{---}} блоки.]]
#\deg (v) > 2 в T (рис. 3). Обозначим за T' T\backslash \{u\}. Поскольку степень ни одной из т.с. G' принадлежащих T (кроме v, для нее степень уменьшилась ровно на 1) не уменьшилась, а для вершины v в T' верно, что \deg(v) >= 2, то T' удовлетворяет условиям на T из предположения индукции. Заметим, что VT' = VT - 1 = m - 1 . Заметим также, что T' связен, т.к. u была висячей вершиной в T.[[Файл: Planar vertex biconnected 3.png|270px|center|thumb|рис. 3. Красные {{---}} точки сочленений. Голубые {{---}} блоки.]]

Рассмотрим подграф G_2 графа G' соответствующий дереву T'. Поскольку T' связен, степени вершин в T' соответствующих т.с. графа G' удовлетворяют предположению индукции и, очевидно, также как и T граф T' является подграфом графа блоков и точек сочленений G, получим, что G_2 планарен по предположению индукции, т.к. VT' . 

Из определения ребер дерева блоков и точек сочленений получаем, что графы G_1 и G_2 имеют единственную общую точку {{---}} точку сочленения v. Поскольку множество блоков G' принадлежащих T состоит из G_1 и множества блоков T', то G' = G_1\cup G_2. G_1, G_2, G' удовлетворяют условию [[#l1|леммы I]], поэтому получим укладку G из укладок G_1 и G_2 так, как это сделано в доказательстве леммы. Получаем, что G' планарен. А значит предположение индукции верно.

Рассматривая в качестве T граф T_G блоков и точек сочленений G. По [[Граф блоков-точек сочленения|лемме]] T_G {{---}} дерево, следовательно каждая его вершина имеет степень как минимум 1. Поскольку граф G содержит хотя бы один блок. Если это единственный блок, то T_G не содержит ни одной точки сочленения. Если граф G содержит хотя бы 2 блока и, следовательно хотя бы одну точку сочленения, то T_G {{---}} дерево, содержащее хотя бы одно ребро. Поскольку в графе блоков и точек сочленений точки сочленений не могут быть висячими вершинами, то каждая из т.с. графа G принадлежащих T_G имеет степень как минимум 2. Получаем, что T_G удовлетворяет условиям на T из предположения индукции, а значит G планарен.
}}

'''Замечание.''' В доказательстве теоремы непосредственно указывается способ получения укладки графа G из имеющихся укладок его блоков.

==См. также==
*[[Укладка_графа_с_планарными_компонентами_реберной_двусвязности|Укладка графа с планарными компонентами реберной двусвязности]]

==Источники информации==

* Асанов М. О., Баранский В. А., Расин В. В. '''Дискретная математика: графы, матроиды, алгоритмы''' — НИЦ РХД, 2001. — 288 с. — ISBN 5-93972-076-5

* H. Whitney '''Non-separable and planar graphs''' — Trans. Amer. Math. Soc., 1932.

[[Категория: Алгоритмы и структуры данных]]
[[Категория: Укладки графов ]]