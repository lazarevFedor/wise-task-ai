== Алгоритм построения кодов Прюфера ==
Кодирование Прюфера переводит [[Количество помеченных деревьев#Помеченное дерево|помеченные деревья порядка n]] в последовательность чисел от 1 до n по алгоритму:
Пока количество вершин больше двух:
# Выбирается лист v с минимальным номером.
# В код Прюфера добавляется номер вершины, смежной с v.
# Вершина v и инцидентное ей ребро удаляются из дерева.

Полученная последовательность называется '''кодом Прюфера''' ''(англ. Prüfer code)'' для заданного дерева.

{{Лемма
|statement=
Номер вершины v встречается в коде Прюфера тогда и только тогда, когда v не является листом, причём встречается этот номер к коде дерева в точности \deg v - 1 раз.
|proof=
# Вершина с номером n не может быть удалена, следовательно на последнем шаге у неё была смежная вершина, и число n встретилось в коде.
# Если вершина не является листом, то у неё на некотором шаге была смежная вершина - лист, следовательно номер этой вершины встречается в коде.
# Если вершина является листом с номером меньше n, то она была удалена до того, как был удален её сосед, следовательно её номер не встречается в коде.

Таким образом, номера всех вершин, не являющихся листьями или имеющих номер n, встречаются в коде Прюфера, а остальные - нет.
}}

{{Лемма
|statement=
По любой последовательности длины n - 2 из чисел от 1 до n можно построить помеченное дерево,
для которого эта последовательность является кодом Прюфера.
|proof=
Доказательство проведем по индукции по числу n
''База индукции:''

n = 1 - верно.

''Индукционный переход:''

Пусть для числа n верно, построим доказательство для n+1:

Пусть у нас есть последовательность: A = [a_1, a_2, ..., a_{n - 2}].
Выберем минимальное число v не лежащее в A. По предыдущей лемме v - вершина, которую мы удалили первой. Соединим v и a_1 ребром. Выкинем из последовательности A число a_1. Перенумеруем вершины, для всех a_i > v заменим a_i на a_i - 1. А теперь мы можем применить предположение индукции.
}}

{{Теорема
|statement=
Кодирование Прюфера задаёт биекцию между множествами помеченных деревьев порядка n и последовательностями длиной n - 2 из чисел от 1 до n
|proof=
# Каждому помеченному дереву приведенный алгоритм сопоставляет последовательность.
# Каждой последовательности, как следует из предыдущей леммы, соотвествует помеченное дерево.
}}

Следствием из этой теоремы является [[Количество помеченных деревьев|формула Кэли]].

== Пример построения кода Прюфера ==
[[Файл: Prufer.png|500px]]

== Алгоритм декодирования кодa Прюфера ==

В массиве вершин исходного дерева V найдём вершину v_{min} с минимальным номером, не содержащуюся в массиве с кодом Прюфера P, т.е. такую, что она является листом или концом уже добавленного в граф ребра, т.е. она стала листом в процессе построения кода Прюфера (по первому пункту построения). Вершина p_1 была добавлена в код Прюфера как инцидентная листу с минимальным номером (по второму пункту построения), поэтому в исходном дереве существует ребро {p_1, v_{min}}, добавим его в список ребер. Удалим первый элемент из массива Р, а вершину v_{min} - из массива V т.к. она больше не может являться листом (по третьему пункту построения). Будем выполнять вышеуказанные действия, пока массив P не станет пустым. В конце работы алгоритма в массиве V останутся две вершины, составляющие последнее ребро дерева (это следует из построения).

=== Реализация ===
 # P - код Прюфера
 # V - вершины
 '''function''' buildTree(P, V):
 '''while'' '''not'' P.empty():
 u = P[0]
 v = min(x '''\in''' V: P.count(x) == 0)
 G.push({u, v})
 P.erase(0)
 V.erase(indexOf(v))
 G.push({v[0], v[1]})
 '''return''' G

== Пример декодирования кода Прюфера ==
[[Файл: backprufer.png|700px]]

==См. также==
*[[Связь матрицы Кирхгофа и матрицы инцидентности]]
*[[Матрица Кирхгофа]]
*[[Количество помеченных деревьев]]
*[[Подсчет числа остовных деревьев с помощью матрицы Кирхгофа]]

== Источники информации ==
* [http://www.intuit.ru/department/algorithms/graphsuse/11/2.html Университет INTUIT | Представление с помощью списка ребер и кода Прюфера]

[[Категория: Алгоритмы и структуры данных]]
[[Категория: Остовные деревья ]]
[[Категория: Свойства остовных деревьев ]]