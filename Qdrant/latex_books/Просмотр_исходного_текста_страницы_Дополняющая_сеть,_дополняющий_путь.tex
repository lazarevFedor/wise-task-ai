{{Определение
|definition=
'''Остаточной пропускной способностью''' (англ. ''residual capacity'') ребра (u, v) называется величина дополнительного [[Определение сети, потока#flow|потока]], который мы можем направить из u в v , не превысив [[Определение сети, потока#flow|пропускную способность]] c(u, v) . Иными словами c_f(u, v) = c(u, v) - f(u, v) .
}}

{{Определение
|id=residual_network
|definition=
Для заданной [[Определение сети, потока#flow_network|транспортной сети]] G=(V,E) и потока f, '''остаточной сетью''', ('''дополняющая сеть''', англ. ''residual network'') в G, порожденной потоком f, является сеть G_f=(V,E_f), где E_f=\{(u,v) \in V\times V \mid c_f(u, v) > 0\}
}}

{{Определение
|definition=
Для заданной транспортной сети G=(V,E) и потока f '''дополняющим путем''' (англ. ''augmenting path'') p является простой путь из [[Определение сети, потока#flow_network| истока в сток]] в остаточной сети G_f=(V,E_f).
}}

{|border="0" cellpadding="5" width=30% align=center
|[[Файл:Flow-network.png|thumb|340px|center|Граф с некоторым потоком]]
|[[Файл:Residual-network.png|thumb|340px|center|Остаточная сеть этого графа]]
|
|}

== Источники информации==
* ''Кормен Т., Лейзерсон Ч., Ривест Р.'' Алгоритмы: построение и анализ.[http://wmate.ru/ebooks/?dl=380&mirror=1] — 2-е изд. — М.: Издательский дом «Вильямс», 2007. — С. 1296. ISBN — 978-0-2625-3196-2
* [http://ru.wikipedia.org/wiki/Транспортная_сеть Википедия {{---}} Транспортная сеть]
* [http://en.wikipedia.org/wiki/Flow_network Wikipedia {{---}} Flow Network]

[[Категория:Алгоритмы и структуры данных]]
[[Категория:Задача о максимальном потоке]]