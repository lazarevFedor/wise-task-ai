{{Определение
|definition ='''Фактором ''' ''(англ. factor)'' [[Основные определения теории графов|графа]] G называется [[Остовные деревья: определения, лемма о безопасном ребре|остовный подграф]] графа G, имеющий хотя бы одно ребро.
}}

{{Определение
|definition = Граф G — сумма факторов G_i, если графы G_i не имеют попарно общих рёбер, а G является их объединением. Такое разложение называется '''факторизацией ''' ''(англ. factorization)'' графа G.
}}

{{Определение
|definition = '''k-фактор''' — [[Основные определения теории графов|регулярный]] остовный подграф степени k. Если граф G представляет собой сумму k-факторов, то их объединение называется k-факторизацией, а сам граф G назыается k-факторизуемым.
}}

{{Определение
|definition = Пусть задана функция f : V(G) \rightarrow \mathbb{N}, такая что \forall~v \in V(G):f(v)\leq \text{deg}(v). Тогда остовный подграф G_f в котором степень каждой вершины v равна f(v) называется '''f-фактором'''.
}}

[[Файл:1-A-general-graph-G-with-a-3-regular-factor-2-A-general-graph-G-with-an-f-factor (1).png|700px|thumb|right| Примеры факторов в графе: (1) {{---}} 3-фактор, (2) {{---}} f-фактор (значения f(v) указаны возле вершин)]]

== Задача о поиске произвольного f-фактора ==

Сведем задачу о поиске f-фактора к задаче о поиске наибольшего паросочетания.

Пусть дан граф G и функция f : V(G) \rightarrow \mathbb{N}. Построим граф G^* следующим образом. 
# Для каждого ребра (u,w)\in E(G) добавим в граф G^* по одной новой вершине в множества S(u) и S(w), и соединим их ребром (e(u),e(w)). В результате каждой вершине v \in V(G) будет соответствовать множество S(v) \subset V(G^*) такое что |S(v)|=deg(v); Каждому ребру (u,w) \in E(G) будет соответствовать ребро (e(u),e(w)), причем ни для каких двух ребер из E(G) концы соответствующих им ребер в G^* не пересекаются.
# Для каждой вершины v\in V(G) добавим в G^* новые deg(v)-f(v) вершин, образующие множество T(v). Каждую вершину из T(v) свяжем ребром с каждой вершиной из S(v). В результате для каждой вершины v \in V(G) Множество S(v)\cup T(v) образует полный двудольный граф. 

[[Файл:A-general-graph-G-with-an-f-factor-and-the-corresponding-simple-graph-G-with-a.png|700px|thumb|centre| Граф G и соответствующий ему граф G^*]]

{{Теорема
|statement = 
Граф G имеет f-фактор тогда и только тогда, когда соответствующий графу G и функции f граф G^* имеет совершенное паросочетание.
|proof = 
\Rightarrow

Пусть граф G имеет f-фактор G_f. Построим паросочетание M для графа G^* следующим образом:
# Для каждого ребра (u,w)\in G_f добавим в M соответствующее ему ребро из G^* . Теперь для каждой вершины v \in V(g) f(v) вершин из множества S(v) покрыты M .
# Для каждой вершины v \in V(g) пусть R(v)\subset S(v) {{---}} множество вершин еще не покрытых M. R(v)\cup T(v) является полным двудольным графом, причем размер каждой из долей равен deg(v)-f(v), следовательно этот граф имеет совершенное паросочетание M_v. Добавим каждое ребро из M_v в M.
В результате каждая вершина в G^* покрыта M, следовательно M является совершенным паросочетанием.

\Leftarrow 

Пусть G^* имеет совершенное паросочетание M. Для каждой вершины v\in V(G) T(v) является независимым множеством и полностью покрыто M, следовательно множество R(v)\subset S(v) покрыто ребрами, концы которых лежат в T(v), а значит каждая вершина из S(v)\setminus R(v) покрыта ребром, второй конец которого принадлежит S(w) : w \in V(G), причем |S(v)\setminus R(v)| = deg(v)-(deg(v)-f(v))=f(v). Поэтому если мы добавим в G_f все ребра соответствующие ребрам из M покрывающим S(v)\setminus R(v) : v \in V(G), то есть все ребра из M концы которых лежат в множествах S, то степень каждой вершины v \in G_f будет равна f(v), а значит G_f будет являться f-фактором.
}}

Из доказательства напрямую следует, что для нахождения f-фактора графа G достаточно найти совершенное паросочетание в графе G^*. Т.к. G^* в общем случае не является двудольным, для решения этой задачи можно воспользваться [https://ru.wikipedia.org/wiki/%D0%90%D0%BB%D0%B3%D0%BE%D1%80%D0%B8%D1%82%D0%BC_%D1%81%D0%B6%D0%B0%D1%82%D0%B8%D1%8F_%D1%86%D0%B2%D0%B5%D1%82%D0%BA%D0%BE%D0%B2 Алгоритмом Эдмондса для поиска наибольшего паросочетания] работающим за время O(E \cdot V^2). Построение графа G^* можно выполнить за время O(E+V). Поэтому итоговая асимптотика {{---}} O(E \cdot V^2)

== 1-факторизация ==
{{Теорема 
|statement=
[[Основные определения теории графов|Полный]] граф K_{2n} 1-факторизуем.
|proof=
[[Файл: Факторизация K6.png|thumb|360px|right|1-факторизация графа K_6]]
Нам нужно только указать разбиение множества рёбер E графа на (2n - 1) 1-фактора. Для этого обозначим вершины графа G через v_1, v_2, \dots, v_{2n} и определим множества рёбер X_i = (v_iv_{2n}) \cup (v_{i - j}v_{i + j}; j = 1, 2, \dots, n - 1), i = 1, 2, \dots, 2n - 1 , где каждый из индексов i - j и i + j является одним из чисел 1, 2, \dots, 2n - 1; здесь сумма и разность берутся по модулю 2n - 1. Легко видеть, что набор X_i даёт необходимое разбиение множества X, а сумма подграфов G_i, порождённых множествами X_i, является 1-факторизацией графа K_{2n}.
}}
== 2-факторизация ==

{{Утверждение
|statement =
Если граф 2-факторизуем, то каждый его фактор должен быть объединением непересекающихся (по вершинам) [[Основные определения теории графов|циклов]].
|proof =
Начнём обход 2-фактора с какой-то вершины. Пойдём по одному из её рёбер и попадаем в смежную ей вершину. Далее идём по рёбрам, по которым мы ещё не ходили. Мы входим в вершину по одному ребру и выходим по другому, так как степень каждой вершины равна 2, пока не вернёмся в первую вершину. Это цикл, так как в каждой вершине мы были только один раз. Если есть вершины, которые мы не посетили, то снова начинаем обход с любой из таких вершин. В вершины прежних циклов попасть нельзя, так как мы уже проходили по рёбрам этих вершин. Значит, циклы не пересекаются по вершинам.
}}

{{Теорема
|id=regular2factor
|author=J. Petersen, 1981
|about = О наличии 2-фактора в регулярном графе чётной степени.
|statement = Пусть G {{---}} [[Основные определения теории графов#defRegularGraph |регулярный граф]] чётной степени. Тогда в G есть 2-фактор.
|proof = 
Пусть G {{---}} 2k-регулярный граф, пусть G [[Отношение связности, компоненты связности#connected_graph | связен]]. 

Согласно [[Эйлеровость графов#eulerTheorem | критерию эйлеровости]] граф G имеет эйлеров цикл v_0e_1 \cdots e_lv_l, где v_0 = v_l.

Будем строить граф H следующим образом: разделим каждую вершину графа G v на две, назовём их v^- и v^+. Заменим каждое ребро в эйлеровом обходе v_iv_{i+1} на ребро v_i^-v_{i+1}^+

[[Файл:2-фактор(1).png|300px|thumb|right|Пример регулярного графа чётной степени. В нём есть эйлеров цикл 1{{---}}2{{---}}3{{---}}4{{---}}1]]

[[Файл:2-фактор(2).png|300px|thumb|right|Соответствующий ему граф H]]

Получившийся граф является k-регулярным, и по [[Рёберная раскраска двудольного графа#lem2 | лемме о существовании совершенного паросочетания в регулярном двудольном графе]] в нём есть совершенное паросочетание, то есть 1-фактор.

Объединим вершины v^- и v^+ обратно в вершину v. Так как в графе H каждой вершине было инцидентно 1 ребро, то после объединения в графе G каждой вершине будет инцидентно 2 ребра.

Если G несвязен, то аналогичные рассуждения можно провести для каждой его [[Отношение связности, компоненты связности#def2 | компоненты связности]], и, таким образом, найти 2-фактор в каждой его компоненте связности. Тогда каждой вершине каждой его компоненты связности будет инцидентно 2 ребра, значит, каждой вершине G будет инцидентно 2 ребра, значит, в G существует 2-фактор. 
}}

[[Файл:Факторизация K7 разбиение.png|300px|thumb|right| Пример графа, имеющего 3 различных 2-фактора, то есть разбиваемого на 3 рёберно непересекающихся [[Гамильтоновы графы#defCycle|гамильтоновых цикла]]]]

Заметим, что если 2-фактор связен, то он является [[Гамильтоновы графы|гамильтоновым циклом]].

{{Теорема
|statement = 
Граф K_{2n+1} можно представить в виде суммы n гамильтоновых циклов.
|proof = 
[[Файл: Факторизация K7.png|thumb|360px|right|2-факторизация графа K_7. Рёбра, отмеченные пунктиром, не пересекают другие рёбра при правильной [[Укладка графа на плоскости|укладке графа]].]]Для того чтобы в графе K_{2n+1} построить n гамильтоновых циклов, непересекающихся по рёбрам, перенумеруем сначала его вершины v_1, v_2, \dots, v_{2n+1}. На множестве вершин v_1, v_2, \dots, v_{2n} зададим n непересекающихся простых цепей P_i=v_i v_{i-1} v_{i+1} v_{i-2} \dots v_{i+n-1}v_{i-n} следующим образом: j-ой вершине цепи P_i является вершина v_k, где k=i+(-1)^{j+1}\dfrac{j}{2}, все индексы приводятся к числам 1, 2, \dots, 2n по модулю 2n. Гамильтонов цикл Z_i можно получить, соединив вершину v_{2n+1} с концевыми вершинами цепи P_i. 
}}

== Замечания ==
* Факторизация графов используется как один из способов построения покрывающих наборов, используемых при создании тестов для программ с большим количеством параметров.
* 1-факторизация k-регулярного графа является рёберной [[Раскраска графа|k-раскраской графа]].
== См. также ==
* [[Гамильтоновы графы]]
* [[Остовные деревья: определения, лемма о безопасном ребре|Остовные деревья]]
* [[Основные определения теории графов]]

== Источники информации ==
* Харари Фрэнк '''Теория графов''' Пер. с англ. и предисл. В. П. Козырева. Под ред. Г.П.Гаврилова. Изд. 2-е. — М.: Едиториал УРСС, 2003. — 296 с. — ISBN 5-354-00301-6
* [http://en.wikipedia.org/wiki/Graph_factorization Wikipedia — Graph factorization]
* Factors and Factorizations of Graphs: Proof Techniques in Factor Theory / Jin Akiyama, Mikio Kano. — Springer Science & Business Media, 2011. ISBN 3642219187, 9783642219184.

[[Категория: Алгоритмы и структуры данных]]
[[Категория: Основные определения теории графов]]