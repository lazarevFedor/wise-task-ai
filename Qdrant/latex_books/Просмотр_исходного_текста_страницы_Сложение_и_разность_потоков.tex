{{Лемма
|about =
о сложении потоков
|statement=
Пусть G = (V, E) {{---}} [[Определение_сети,_потока#flow_network|транспортная сеть]] с источником s и стоком t, а f {{---}} [[Определение_сети,_потока#flow|поток]] в G. Пусть G_f {{---}} [[Дополняющая_сеть,_дополняющий_путь#residual_network|остаточная сеть]] в G, порожденная потоком f, а f' {{---}} поток в G_f. Тогда сумма потоков f + f', определяемая уравнением (f + f')(u, v) = f(u,v) + f'(u,v), является потоком в G, и [[Определение_сети,_потока#flow|величина]] этого потока равна |f + f'| = |f| + |f'|.
|proof=
Необходимо проверить, выполняются ли ограничения антисимметричности, пропускной способности и сохранения [[Определение_сети,_потока#flow|потока]].
 
# Для подтверждения антисимметричности заметим, что для всех (u,v) \in V справедливо:
#: (f + f')(u, v) = f(u,v) + f'(u,v) = -f(v,u) - f'(v,u) = -(f(v,u) + f'(v,u)) = -(f + f')(v,u)
#:
# Покажем соблюдение ограничений пропускной способности.
#: Заметим, что f'(u,v) \leqslant c_f(u,v) для всех u,v \in V и c_f(u, v) = c(u, v) - f(u, v) . Тогда 
#: (f + f')(u,v) = f(u,v) + f'(u,v) \leqslant f(u,v) + (c(u,v) - f(u,v)) = c(u,v) . 
#:
# Заметим, что для всех u \in V - \{s,t\} справедливо равенство: 
#: \sum\limits_{v\in V} (f + f')(u, v) = \sum\limits_{v\in V} (f(u,v) + f'(u,v)) = \sum\limits_{v\in V} f(u,v) + \sum\limits_{v\in V} f'(u,v) = 0 + 0 = 0 
#: |f + f'| = \sum\limits_{v\in V} (f + f')(s, v) = \sum\limits_{v\in V} (f(s,v) + f'(s,v)) = \sum\limits_{v\in V} f(s,v) + \sum\limits_{v\in V} f'(s,v) = |f| + |f'|
}}

{{Лемма
|about =
о разности потоков
|statement=
Пусть G = (V, E) {{---}} транспортная сеть с источником s и стоком t, а h и f {{---}} [[Определение_сети,_потока#flow|потоки]] в G . Пусть G_f {{---}} [[Дополняющая_сеть,_дополняющий_путь#residual_network|остаточная сеть]] в G, порожденная потоком f. Тогда разность потоков h - f, определяемая уравнением (h - f)(u, v) = h(u,v) - f(u,v), является потоком в G_f, и величина этого потока равна |h - f| = |h| - |f|.
|proof=
Антисимметричность и правило сохранения потока для h - f проверяются аналогично лемме о сложении потоков.

Покажем соблюдение ограничений пропускной способности.

(h - f)(u,v) = h(u,v) - f(u,v) \leqslant c(u,v) - f(u,v) = c_f(u,v) . 

Теперь покажем, что [[Определение_сети,_потока#flow|величина]] потока h - f равна разности величин потоков h и f.

 |h - f| = \sum\limits_{v\in V} (h - f)(s, v) = \sum\limits_{v\in V} (h(s,v) - f(s,v)) = \sum\limits_{v\in V} h(s,v) - \sum\limits_{v\in V} f(s,v) = |h| - |f|
}}

== Источники информации ==
* ''Кормен Т., Лейзерсон Ч., Ривест Р.'' Алгоритмы: построение и анализ.[http://wmate.ru/ebooks/?dl=380&mirror=1] — 2-е изд. — М.: Издательский дом «Вильямс», 2007. — С. 1296.

[[Категория:Алгоритмы и структуры данных]]
[[Категория:Задача о максимальном потоке]]