==Двумерный случай==
{{Теорема
|about=
формула Эйлера
|statement=
Для произвольного [[Укладка графа на плоскости|плоского]] связного графа G с V вершинами, E ребрами и F [[Укладка графа на плоскости|гранями]] справедливо следующее соотношение:
V - E + F = 2
|proof=

Воспользуемся методом математической индукции по количеству граней графа.

'''База индукции''': 
F = 2. Граф G представляет собой многоугольник с n вершинами (рис. 1). Тогда V = E = n, значит, равенство V - E + F = 2 выполняется.

'''Индукционный переход''': 

Покажем, что если теорема верна для графов с F гранями, то она будет верна и для графов с F + 1 гранями. Пусть G {{---}} плоский граф, имеющий V вершин, E ребер и F граней, и для него справедлива формула Эйлера. Добавим новую грань (пунктирная линия на рис.2), проводя по внешней грани F_{\infty} некоторую элементарную цепь, соединяющую две вершины максимального цикла графа G. Если эта цепь имеет r ребер, то необходимо добавить r - 1 новых вершин и одну новую грань. Ясно, что формула Эйлера останется справедливой и для нового графа, так как V' - E' + F' = (V + r - 1) - (E + r) + (F + 1) = V - E + F = 2
}}

{|align="center"
 |-valign="top"
 |[[Файл:Eulerformul1.png|300px|thumb|рис. 1]]
 |[[Файл:Eulerformul2.png|300px|thumb|рис. 2]]
|}

{{Теорема
|id=EulerFormulaCons
|about=
следствие из формулы Эйлера
|statement=
Пусть G связный [[Укладка графа на плоскости|планарный]] обыкновенный граф с V вершинами (V \geqslant 3), E ребрами и F гранями. Тогда E \leqslant 3V - 6
|proof=
Поскольку G не содержит петель и кратных ребер, то каждая грань граничит хотя бы с тремя ребрами. Пусть, двигаясь вдоль i-й грани мы пройдем l_i ребер. Очевидно, что \sum \limits_{i=1}^{F}l_i = 2E. Поскольку l_i \geqslant 3 \hspace{3pt} (i = 1..F), получаем 3F \leqslant 2E. Из формулы Эйлера 3E - 3V + 6 = 3F \leqslant 2E, то есть E \leqslant 3V - 6.
}}

==Трехмерный случай==

Покажем, что в трехмерном случае так же имеет место формула Эйлера.

{{Теорема
|about=
формула Эйлера для многогранников
|statement=
Для любого выпуклого многогранника имеет место равенство V - E + F = 2, где V {{---}} число вершин, E {{---}} число ребер и F {{---}} число граней данного многогранника. 
|proof=
[[Файл:Hypercube.gif|350px|thumb|right|Пример невыпуклого многоугольника для которого V - E + F = 0. Многоугольник получен путем вырезания куба внутри куба.]]
Для доказательства соотношения Эйлера представим поверхность выпуклого многогранника сделанной из эластичного материала. Удалим (вырежем) одну из его граней и оставшуюся поверхность растянем на плоскости. Получим планарный граф, содержащий F' = F - 1 внутренних граней, V вершин и E ребер. 

Тогда справедливо уже доказанное соотношение: V - E + F ' = 1 .

Подставляем F' = F - 1 и получаем V - E + F = 2.
}}

{{Теорема
|about=
следствие из формулы Эйлера для многогранников
|statement=
В любом выпуклом многограннике имеется или треугольная грань, или трехгранный угол. Более того, число треугольных граней плюс число трехгранных углов больше или равно восьми.
|proof=
Обозначим через V_{i} число вершин выпуклого многогранника, в которых сходится i ребер. Тогда для общего числа вершин V имеет место равенство V = V_{3} + V_{4} + V_{5} + \dots

Аналогично, обозначим через F_{i} число граней выпуклого многогранника, у которых имеется i ребер. Тогда для общего числа граней F имеет место равенство F = F_{3} + F_{4} + F_{5} + \dots 

Посчитаем число ребер E многогранника. Имеем: 3V_{3} + 4V_{4} + 5V_{5} + \dots = 2E, 3F_{3} + 4F_{4} + 5F_{5} + \dots = 2E. 

По теореме Эйлера выполняется равенство 4V - 4E + 4F = 8. Подставляя вместо V, E и F их выражения, получим:

4V_{3} + 4V_{4} + 4V_{5} + \dots - (3V_{3} + 4V_{4} + 5V_{5} + \dots) - (3F_{3} + 4F_{4} + 5F_{5} + \dots) + 4F_{3} + 4F_{4} + 4F_{5} + \dots = 8.
Следовательно, V_{3} + F_{3} = 8 + V_{5} + \dots + F_{5} + \dots , значит, число треугольных граней плюс число трехгранных углов больше или равно восьми.}}

==Источники информации==
* Асанов М,, Баранский В., Расин В. {{---}} Дискретная математика {{---}} Графы, матроиды, алгоритмы (стр. 104-107)
* О.Оре {{---}} Графы и их применение (стр. 131-135)
*[https://ru.wikipedia.org/wiki/%D0%A2%D0%B5%D0%BE%D1%80%D0%B5%D0%BC%D0%B0_%D0%AD%D0%B9%D0%BB%D0%B5%D1%80%D0%B0_%D0%B4%D0%BB%D1%8F_%D0%BC%D0%BD%D0%BE%D0%B3%D0%BE%D0%B3%D1%80%D0%B0%D0%BD%D0%BD%D0%B8%D0%BA%D0%BE%D0%B2 Википедия {{---}} Теорема Эйлера для многоугольников]
* [http://www.geometry2006.narod.ru/Lecture/Mnogogr/Mnogogr.htm Выпуклые многогранники]

[[Категория: Алгоритмы и структуры данных]]
[[Категория: Укладки графов ]]