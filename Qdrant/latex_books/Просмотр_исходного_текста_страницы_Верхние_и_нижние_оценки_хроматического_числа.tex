== Верхняя оценка длиной максимального нечетного цикла ==
{{Лемма 
|about = оценка хроматического числа длиной максимального нечётного цикла
|statement= Пусть G(V,E) {{---}} произвольный связный неориентированный граф и \Delta(G) {{---}} длина максимального простого цикла графа G, \Delta \geqslant 3. Тогда, \chi(G) \leqslant\Delta(G) + 1. 
|proof=
Опишем на графе следующий алгоритм раскраски:
*Из произвольной вершины v запустим алгоритм поиска в глубину. Пусть T {{---}} дерево обхода глубина графа G с корнем в вершине v.
*Произвольную вершину u, покрасим в цвет dist(v,u) \bmod (\Delta + 1), где dist(v,u){{---}} расстояние между вершинами u,v в графe T.
Докажем от противного, что после выполнения описанного алгоритма граф G будет правильно раскрашен.
Предположим, что после выполнения алгоритма покраски в графе существует ребро, соединяющее вершины a, b одного цвета. Пусть color(v) {{---}} цвет вершины после выполнения алгоритма раскраски. Заметим, что для произвольной вершины графа p, dist(v,p) = color(p) + n(\Delta + 1) , n \geqslant 0 . Тогда, dist(v,a) - dist(v,b) = k(\Delta + 1). Поскольку в дереве dfs между вершинами находящимися на одинаковом расстоянии от корня нет перекрестных ребер, то k \geqslant 1. То есть, вершины a, b лежат на простом цикле длины по крайней мере \Delta + 2. Получается противоречие с условием потому, что длина максимального простого цикла получается больше чем \Delta.
Таким образом в графе G после выполнения алгоритма раскраски нет вершин одного цвета соединенных ребром и при этом каждая вершина покрашена в один из \Delta + 1, то есть G правильно раскрашен в \Delta + 1 цвет, следовательно \chi(G) \leqslant \Delta(G) + 1
 

}}

==Нижняя оценка числом независимости ==
{{Определение
 
|definition=
Подмножество S вершин графа G называется '''независимым''', если любые две вершины из S не смежны в G 
}}

{{Определение
 
|definition=
'''Число независимости''' \alpha(G) графа G(V,E) {{---}} \max \{|S|:S \in V и S независимо в G\}
}}
{{Лемма 
|about = нижняя оценка
|statement= Пусть G(V,E) {{---}} произвольный связный неориентированный граф с n вершинами .Тогда, n/\alpha \leqslant \chi(G). 
|proof=
Пусть, V_1,V_2...V_\chi множеств вершин окрашенных в соответствующие цвета при правильно покраски графа G.Каждое из V_i {{---}} независимое множество (поскольку вершины множества покрашены в один цвет при правильной покраски графа G, следовательно, они попарно не смежны внутри множества ).
Заметим, что для произвольного i, |V_i| \leqslant \alpha (т.к V_i независимое множество). То есть, \sum\limits^{\chi}_{i = 1}|V_i| = n \leqslant \chi \alpha , следовательно n / \alpha \leqslant \chi.
}}

== Верхняя оценка количеством ребер ==
{{Лемма 
|about = верхняя оценка
|statement= Пусть G(V,E) {{---}} произвольный связный неориентированный граф с m ребрами.Тогда, \chi(G) \leqslant \dfrac{1}{2} +\sqrt{2m + \dfrac{1}{4}}. 
|proof=
Пусть, V_1,V_2...V_\chi множеств вершин окрашенных в соответствующие цвета при правильно покраски графа G. Заметим, что между любыми двумя различными множествами существует хотя бы одно ребро (в противном случаи эти множества можно было бы покрасить в один цвет).
Тогда, \dfrac{1}{2}\chi(\chi-1) \leqslant m \Rightarrow (\chi - \dfrac{1}{2})^2 \leqslant 2m + \dfrac{1}{4} \Rightarrow \chi(G) \leqslant \dfrac{1}{2} +\sqrt{2m + \dfrac{1}{4}} .
}}

== Нижняя оценка количеством ребер и количеством вершин ==
{{Лемма 
|about = нижняя оценка Геллера
|statement= Пусть G(V,E) {{---}} произвольный связный неориентированный граф с n вершинами и m ребрами. Тогда, \dfrac{n^2}{n^2 - 2m} \leqslant \chi(G) . 
|proof=
Пусть, V_1,V_2...V_\chi множеств вершин окрашенных в соответствующие цвета при правильно покраски графа G.
m \leqslant \dfrac{1}{2}n(n - 1) - \dfrac{1}{2}\sum\limits^{\chi}_{i = 1}|V_i|(|V_i| - 1) \Rightarrow \dfrac{n^2}{n^2 - 2m} \leqslant \dfrac{n^2}{n^2 -n(n - 1) + \sum\limits^{\chi}_{i = 1}|V_i|(|V_i| - 1)} = \dfrac{n^2}{n + \sum\limits^{\chi}_{i = 1}|V_i|(|V_i| - 1)} = \dfrac{n^2}{\sum\limits^{\chi}_{i = 1}|V_i| + \sum\limits^{\chi}_{i = 1}|V_i|(|V_i| - 1)} = \dfrac{n^2}{\sum\limits^{\chi}_{i = 1}|V_i|^2} = \dfrac{(\sum\limits^{\chi}_{i = 1}|V_i|)^2}{\sum\limits^{\chi}_{i = 1}|V_i|^2} \leqslant \chi.
}}

==См. также==
*[[Хроматическое_число_планарного_графа|Хроматическое число планарного графа]]

== Источники информации ==
* [http://www.ucdenver.edu/academics/colleges/CLAS/Departments/math/students/alumni/Documents/Student%20Theses/Mitchell_MSThesis.pdf Множество разных оценок для хроматических чисел]
* [http://geometr.freehostia.com/sravnenie_summ.html Сравнение квадрата суммы и суммы квадратов действительных чисел]

[[Категория: Алгоритмы и структуры данных]]
[[Категория: Раскраски графов]]