[[Файл:MST-example.png|right|thumb|200px|Пример минимального остовного дерева.]]
==Необходимые определения==
Рассмотрим связный неориентированный взвешенный [[Основные определения теории графов|граф]] G =( V, E ) , где V {{---}} множество [[Основные определения теории графов| вершин]], E {{---}} множество [[Основные определения теории графов|ребер]]. Вес ребра определяется, как функция w : E \to \mathbb{R} . 
{{Определение
|id = spanning_tree
|definition =
'''Остовное дерево''' (англ. ''spanning tree'') графа G = ( V, E ) {{---}} ациклический связный подграф данного связного неориентированного графа, в который входят все его вершины.
}}{{Определение
|definition =
'''Минимальное остовное дерево''' (англ. ''minimum spanning tree'') графа G = ( V, E ) {{---}} это его ациклический связный подграф, в который входят все его вершины, обладающий минимальным суммарным весом ребер.
}}
Заметим, что граф может содержать несколько минимальных остовных деревьев. 
Для формулировки и доказательства леммы о безопасном ребре рассмотрим следующие определения.

Пусть G' {{---}} подграф некоторого минимального остовного дерева графа G = ( V, E ) .
{{Определение
|definition =
Ребро ( u, v ) \notin G' называется '''безопасным''' (англ. ''safe edge''), если при добавлении его в G' , G' \cup \{ ( u, v ) \} также является подграфом некоторого минимального остовного дерева графа G .
}}{{Определение
|definition =
'''Разрезом''' (англ. ''cut'') неориентированного графа G = ( V, E ) называется разбиение V на два непересекающихся подмножества: S и T = V \setminus S . Обозначается как \langle S, T \rangle .
}}{{Определение
|definition =
Ребро ( u, v ) \in E '''пересекает''' (англ. ''crosses'') разрез \langle S, T \rangle , если один из его концов принадлежит множеству S , а другой {{---}} множеству T .
}}

==Лемма о безопасном ребре==
{{Теорема
|statement=
Рассмотрим связный неориентированный взвешенный граф G = ( V, E ) с весовой функцией w : E \to \mathbb{R}. Пусть G' = ( V, E' ) {{---}} подграф некоторого минимального остовного дерева G , \langle S, T \rangle {{---}} разрез G , такой, что ни одно ребро из E' не пересекает разрез, а ( u, v ) {{---}} ребро минимального веса среди всех ребер, пересекающих разрез \langle S, T \rangle . Тогда ребро e = ( u, v ) является безопасным для G'.
|proof=
[[Файл:Лемма_о_безопасном_ребре.png‎|right|thumb|300px]]
Достроим E' до некоторого минимального остовного дерева, обозначим его T_{min}. Если ребро e \in T_{min}, то лемма доказана, поэтому рассмотрим случай, когда ребро e \notin T_{min}. Рассмотрим путь в T_{min} от вершины u до вершины v. Так как эти вершины принадлежат разным долям разреза, то хотя бы одно ребро пути пересекает разрез, назовем его e'. По условию леммы w(e) \leqslant w(e'). Заменим ребро e' в T_{min} на ребро e. Полученное дерево также является минимальным остовным деревом графа G, поскольку все вершины G по-прежнему связаны и вес дерева не увеличился. Следовательно E' \cup \{e\} можно дополнить до минимального остовного дерева в графе G, то есть ребро e {{---}} безопасное.
}}

==Cм. также==
*[[Алгоритм Прима]]
*[[Алгоритм Краскала]]
*[[Алгоритм Борувки]]

==Источники информации==
* Кормен Т., Лейзерсон Ч., Ривест Р., Штайн К. {{---}} Алгоритмы. Построение и анализ : Вильямс, 2-е издание, 2005, С. 644-649

[[Категория: Алгоритмы и структуры данных]] 
[[Категория: Остовные деревья]]
[[Категория: Построение остовных деревьев]]