== Паросочетание в двудольном графе==

{{Определение
|id=matching_def
|definition= '''Паросочетание''' (англ. ''matсhing'') M в двудольном графе — произвольное множество рёбер двудольного графа такое, что никакие два ребра не имеют общей вершины.}}
{{Определение
|definition= Вершины двудольного графа, инцидентные рёбрам паросочетания M, называются '''покрытыми''' (англ. ''matched''), а неинцидентные — '''свободными''' (англ. ''unmatched'').}} 
{{Определение
|definition= '''Числом рёберного покрытия''' (англ. ''edge covering number'') называется размер минимального рёберного покрытии графа G и обозначается через \rho(G).}}
{{Определение
|definition= Число рёбер в наибольшем паросочетании графа G называется '''числом паросочетания''' (англ. ''matching number'').}}
{{Определение
|id = maximal_matching
|definition= '''Максимальное паросочетание (по включению)''' (англ. ''maximal matching'') — это такое паросочетание M в графе G, которое не содержится ни в каком другом паросочетании этого графа, то есть к нему невозможно добавить ни одно ребро, которое бы являлось несмежным ко всем рёбрам паросочетания.}}
Другими словами, паросочетание M графа G является максимальным, если любое ребро в G имеет непустое пересечение по крайней мере с одним ребром из M.
{{Определение
|id = maximal_matching_size
|definition= '''Максимальное паросочетание (по мощности)''' (англ. ''maximum cardinality matching'') — это паросочетание M в графе G максимальное по мощности.}}

{{Определение
|id = perfect_matching
|definition= Паросочетание M графа G называется '''совершенным (или полным)''' (англ. ''perfect matching''), если оно покрывает все вершины графа.}}
{{Определение
|definition= '''Чередующаяся цепь''' (англ. ''alternating path'') — путь в двудольном графе, для любых двух соседних рёбер которого верно, что одно из них принадлежит паросочетанию M, а другое нет.}}
{{Определение
|definition= '''Дополняющая цепь (или увеличивающая цепь)''' (англ. ''augmenting path'') — чередующаяся цепь, у которой оба конца свободны.}}
{{Определение
|definition= '''Уменьшающая цепь''' (англ. ''reduce path'') — чередующаяся цепь, у которой оба конца покрыты.}}
{{Определение
|definition= '''Сбалансированная цепь''' (англ. ''balanced path'') — чередующаяся цепь, у которой один конец свободен, а другой покрыт.}}

== Свойства ==

В любом графе без изолированных вершин, число паросочетания и число рёберного покрытия в сумме дают число вершин. Если существует совершенное паросочетание, то оба числа равны |V|/2.

== Пример максимального и полного паросочетания, чередующейся цепи ==

{|align="center"
 |-valign="center"
 |[[Файл: Maximal matching.jpg|thumb|210px|красные рёбра являются рёбрами максимального паросочетания]]
 |[[Файл: Perfect_matching.jpg|thumb|245px|красные рёбра являются рёбрами полного паросочетания.]]
 |[[Файл: Alternating_path.jpg|thumb|210px|Пусть красные рёбра принадлежат паросочетанию M, а синие не принадлежат, тогда чередующаяся цепь: 1-8-4-6-3-7]]
 |}

== Теорема о максимальном паросочетании и дополняющих цепях ==

{{Теорема
|id=theorem1
|statement=
Паросочетание M в двудольном графе G является максимальным тогда и только тогда, когда в G нет дополняющей цепи.
|proof=
\Rightarrow

Пусть в двудольном графе G с максимальным паросочетанием M существует дополняющая цепь. Тогда пройдя по ней и заменив вдоль неё все рёбра, входящие в паросочетание, на невходящие и наоборот, мы получим большее паросочетание. То есть M не являлось максимальным. Противоречие.

\Leftarrow

Рассмотрим паросочетание M в графе G и предположим, что M {{---}} не наибольшее. Докажем, что тогда имеется увеличивающая цепь относительно M. Пусть M' {{---}} другое паросочетание и |M'|>|M|. Рассмотрим подграф H графа G, образованный теми рёбрами, которые входят в одно и только в одно из паросочетаний M, M'. Иначе говоря, множеством рёбер графа H является симметрическая разность M\oplus M'. В графе H каждая вершина инцидентна не более чем двум рёбрам (одному из M и одному из M' ), т.е. имеет степень не более двух. В таком графе каждая компонента связности {{---}} путь или цикл. В каждом из этих путей и циклов чередуются рёбра из M и M'. Так как |M'|>|M|, имеется компонента, в которой рёбер из M' содержится больше, чем рёбер из M. Это может быть только путь, у которого оба концевых ребра принадлежат M'. Заметим, что относительно M этот путь является увеличивающей (дополняющей) цепью.
}}

==См. также==
* [[Теорема Холла]]
* [[Связь максимального паросочетания и минимального вершинного покрытия в двудольных графах]]
* [[Связь вершинного покрытия и независимого множества]]

== Источники информации ==
* [http://en.wikipedia.org/wiki/Matching_%28graph_theory%29 Wikipedia {{---}} Matching]
* [https://ru.wikipedia.org/wiki/%D0%9F%D0%B0%D1%80%D0%BE%D1%81%D0%BE%D1%87%D0%B5%D1%82%D0%B0%D0%BD%D0%B8%D0%B5 Википедия {{---}} Паросочетание]
* Асанов М. О., Баранский В. А., Расин В. В. — Дискретная математика: Графы, матроиды, алгоритмы. стр. 227-232 '''ISBN 978-5-8114-1068-2'''

[[Категория: Алгоритмы и структуры данных]]
[[Категория: Задача о паросочетании]]