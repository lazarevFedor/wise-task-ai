== Вспомогательная лемма ==
{{Лемма 
|statement= Пусть G(V,E) {{---}} произвольный [[Отношение связности, компоненты связности|связный]] неориентированный граф и \Delta(G) {{---}} максимальная степень вершин G. Если в таком графе существует вершина w степени \deg w , то \chi(G) \leqslant \Delta(G).
|proof=
 [[Файл:Brooks_1.png‎|400px|thumb|Алгоритм раскраски на пятом шаге]]
Запустим алгоритм [[Обход в ширину|обхода в ширину]] из вершины w. Пронумеруем вершины v_1,...,v_n, где v_i вершина рассмотренная на i-ом шаге алгоритма bfs. Далее начнем красить вершины в обратном порядке в один из \Delta цветов так, чтобы никакое ребро графа не соединяло вершины одного цвета. На i-ом шаге покраски, для вершины v_{n - i+1} есть не более \Delta(G) - 1 уже покрашенных соседей (т.к \deg(v_{n - i+1}) \leqslant \Delta(G) и предок данной вершины в дереве bfs еще не покрашен, а если предка нет, то это вершина и есть w), следовательно вершину v_{n-i+1} можно покрасить по крайней мере в один из свободных цветов. Поскольку на каждом шаге алгоритм отработает корректно, следовательно граф можно правильно раскрасить в не более чем \Delta цветов, то есть \chi(G) \leqslant \Delta(G).

}}

== Теорема ==
{{Теорема
|about= Брукса
|statement=Пусть G(V,E) {{---}} связный неориентированный граф и G не является K_m или C_{2m+1}, ни для какого m, тогда \chi(G) \leqslant \Delta(G), где \Delta(G) {{---}} максимальная степень вершин G 

|proof=
Для доказательства теоремы рассмотрим несколько случаев:
#\Delta(G) \leqslant 2, тогда:
#*Если \Delta = 0, G = K_1
#*Если \Delta = 1, G = K_2
#*Если \Delta = 2, то:
#*# G {{---}} либо дерево либо четный цикл и тогда \chi(G) = 2
#*# G нечетный цикл
#\Delta(G) \geqslant 3, тогда:
##Если G не является вершинно двусвязным графом, тогда в графе G \exists v \in V {{---}} [[Отношение связности, компоненты связности|точка сочленения]]. Пусть G_1,G_2 {{---}} две компоненты связности, полученные при удалении вершины v. Тогда, по выше доказанной лемме эти компоненты можно правильно раскрасить в не более чем \Delta цветов. Поскольку количество соседей вершины v в каждой из компонент не более \Delta - 1, то G можно правильно раскрасить в не более чем \Delta цветов.
##Если G является вершинно двусвязным графом. Тогда, \exists v,u \in V :(u,v) \notin E и при удалении вершин v,u граф теряет связность. Пусть G_1,G_2 {{---}} два подграфа G:(G_1 \cap G_2 = \{v,u\}) \land (G_1 \cup G_2 = G). Рассмотрим два случая.
### Если сумма степеней вершин u,v в каждом из подграфов G_1,G_2 меньше 2(\Delta-1). Тогда, в одном из данных подграфах \deg u \leqslant \Delta - 2 или \deg v \leqslant \Delta - 2 . То есть, эти подграфы можно правильно раскрасить в не более чем \Delta цветов так, чтобы вершины u,v были бы разных цветов. А из этого следует, что граф G тоже можно правильно раскрасить в не более чем \Delta цветов.
### Если сумма степеней вершин u,v в одном из подграфов G_1,G_2 равна 2(\Delta-1). Тогда, степени обеих вершин в одном из подграфов равны \Delta - 1, рассмотрим например, что в подграфе G_1:
###* Если вершины u,v смежны с вершиной p \in G_2, тогда мы можем правильно раскрасить G_2, где степени вершин u,v равны 1, в не более чем \Delta цветов так, чтобы вершины u,v были одного цвета. Следовательно, можно покрасить граф G в не более чем \Delta цветов.
###*[[Файл:Brooks_2.png‎|400px|thumb|Алгоритм раскраски. Третий случай, пятый шаг]]Если вершины u,v смежны с вершинами u_1,v_1 \in G_2 соответственно, тогда вместо вершин \{u,v\} рассмотрим вершины \{u,v_1\}. Заметим, что при удалении этих вершин граф потеряет связность и между ними нет ребра. При этом, сумма степеней новой пары вершин в каждой из компонент, полученных после их удаления, меньше 2(\Delta-1). Поэтому, если для этой пары вершин провести рассуждения аналогичные тем, которые проводились для вершин v,u, получится, что граф G можно правильно раскрасить в не более чем \Delta цветов.
##Если G является k-связным графом, где k > 2. Тогда, рассмотрим w \in V : \deg w = \Delta. У вершины w должны существовать две соседние вершины u,v : uv \notin E , в противном случае G = K_n. Пусть G_- = G - u - v . Заметим, что G_- связный граф, запустим для G_- алгоритм обхода в ширину из вершины w. Пронумеруем вершины v_1,...,v_{n-2}, где v_i вершина рассмотренная на i-ом шаге алгоритма bfs. Теперь пусть v_{n-1} = v, и v_n = u. Покрасим v_n,v_{n-1} в один цвет, далее начнем красить вершины в обратном порядке, начиная с v_{n-2} в один из \Delta цветов так, чтобы никакое ребро графа не соединяло вершины одного цвета. Заметим, что так всегда можно сделать, поскольку на i-ом шаге покраски, где i \neq n, для вершины v_{n - i+1} есть не более \Delta(G) - 1 уже покрашенных соседей. Следовательно, вершину v_{n-i+1} можно покрасить по крайней мере в один из свободных цветов. Вершину w мы тоже сможем правильно раскрасить в один из \Delta цветов потому, что ее \Delta соседей покрашено в не более чем \Delta - 1 цветов. Таким образом граф G можно правильно раскрасить в не более чем \Delta цветов.
}}

== См. также ==
*[[Раскраска графа]]

== Источники информации ==
*[http://myweb.facstaff.wwu.edu/sarkara/brooks.pdf Brooks’ Theorem]
*[http://en.wikipedia.org/wiki/Brooks'_theorem Wikipedia: Brooks' theorem]
*[http://ru.wikipedia.org/wiki/Теорема_Брукса Википедия: Теорема Брукса]

[[Категория: Алгоритмы и структуры данных]]
[[Категория: Раскраски графов]]