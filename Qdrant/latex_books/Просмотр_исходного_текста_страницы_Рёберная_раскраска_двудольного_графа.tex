== Основные определения ==

{{Определение
|id = edge_colouring
|definition = '''Рёберной раскраской''' (англ. ''Edge colouring'') графа G(V, E) называется отображение \varphi из множества рёбер E во множество красок \{c_{1} \ldots c_{t}\}, что для для любых двух различных рёбер e_{i}, e_{j}, инцидентных одной вершине, верно \varphi (e_{i}) \neq \varphi (e_{j}). 
}}

{{Определение
|id = chromativ_index
|definition = '''Хроматическим индексом''' (англ. ''Chromatic index'') \chi '(G) графа G(V, E) называется такое минимальное число '''t''', что существует рёберная раскраска графа в '''t''' цветов.
}}

== Некоторые оценки хроматического индекса ==
{{Лемма
|id = lem1
|about = о нижней оценке хроматического индекса
|statement= \forall\ G(V, E) : \Delta (G) \leqslant \chi '(G), где \Delta (G) {{---}} максимальная степень вершины в графе
|proof= Действительно, давайте рассмотрим вершину максимальной степени в графе. Ей инцидентно ровно \Delta(G) рёбер. При этом, чтобы все они имели попарно различные цвета, они все должны иметь различные цвета, иначе найдётся пара различных рёбер, инцидентных одной вершине и имеющих одинаковый цвет.
}}

Заметим, что в теории графов доказывается более строгое неравенствоhttp://math.uchicago.edu/~may/REU2015/REUPapers/Green.pdf, ограничивающее \chi '(G). А именно то, что \forall\ G(V, E) : \Delta (G) \leqslant \chi '(G) \leqslant \Delta (G) + 1.

== Рёберная раскраска двудольного графа ==
{{Лемма
|id = lem2
|about = о совершенном паросочетании
|statement= В [[Основные определения теории графов#defBiparateGraph | двудольном]] k-[[Основные определения теории графов#defRegularGraph |регулярном]] графе с одинаковыми по размеру долями существует совершенное паросочетание.
|proof= 
Возьмём L {{---}} произвольное подмножество левой доли. Рассмотрим подграф образованный L и множеством всех их соседей из правой доли R. Все вершины левой доли нашего подграфа будут иметь степень k, а степени вершин правой доли '''не превосходят''' k.

Посчитаем количество рёбер m_{L} в данном подграфе. В силу его двудольности, это число будет равняться сумме степеней вершин одной из долей. m_{L} = \underset{{v\in L}}{\sum} deg(v) = |L|\cdot k = \underset{{u\in R}}{\sum} deg(u) \leqslant |R|\cdot k. Из этого мы получаем, что |L|\leqslant |R|.

Значит в данном графе выполняется [[Теорема Холла | Теорема Холла]]. Из чего следует, что в нём есть совершенное паросочетание.
}}

{{Теорема
|statement= Существует рёберная раскраска двудольного графа G в \Delta(G) цветов. Иными словами, для двудольного графа \chi '(G) = \Delta(G)
|proof= 
В доказательство рассмотрим следующий алгоритм поиска такой раскраски:

# Для начала сделаем доли графа одинаковыми по размеру, дополнив меньшую из долей необходимым количеством [[Основные определения теории графов#isolated_vertex | изолированных вершин]];
# Следующим жадным алгоритмом сделаем его \Delta(G)-регулярным: пока граф не регулярный возьмём вершину левой доли степени меньше \Delta(G) и аналогичную вершину правой доли. Соединим их ребром;
# Мы получили регулярный двудольный граф с равными долями. По лемме о совершенном паросочетании в таком графе есть совершенное паросочетание. Найдём его, например [[Алгоритм Куна для поиска максимального паросочетания | алгоритмом Куна]], и удалим из графа;
# Заметим, что граф всё ещё остался регулярным, так как степень каждой вершины уменьшилась на 1. Будем повторять процесс, пока в графе есть рёбра;
# В итоге мы разобьём рёбра графа на \Delta(G) совершенных паросочетаний; 
# В конце нам остаётся каждое паросочетание покрасить в свой цвет и удалить рёбра, которых не было в изначальном графе;

Докажем, что жадный алгоритм из пункта 2 всегда выполняет поставленную задачу.

Предположим, что это не так, и, не нарушая общности, у нас остались вершины в правой доле степени меньше \Delta(G), а в левой таких вершин нет. Давайте посчитаем количество рёбер m в графе. Из левой доли исходит |L| \cdot \Delta(G) рёбер. В правую же приходит не более |R| \cdot \Delta(G) рёбер, но так как существует вершина степени меньше \Delta(G), то неравенство строгое. Получается |L| \cdot \Delta(G) = m . Но в нашем графе |L| = |R|. Следовательно \Delta(G) , что приводит нас к противоречию.

Таким образом мы нашли раскраску двудольного графа в \Delta(G) цветов и предъявили алгоритм её получения. А по лемме о нижней оценки, меньше цветов использовать нельзя. Следовательно \chi '(G) = \Delta(G)

Заметим, что наш жадный алгоритм может проводить кратные рёбра в графе. Однако ни лемма о совершенном паросочетании, ни [[Теорема Холла | Теорема Холла]] не используют в своём доказательстве отсутствие таковых.

}}

==См. также==
* [[Теорема Холла]]
* [[Алгоритм Куна для поиска максимального паросочетания]]
* [[Раскраска двудольного графа в два цвета]]

==Примечания==

==Источники информации==
* [https://ru.wikipedia.org/wiki/Рёберная_раскраска Википедия {{---}} Рёберная раскраска]

[[Категория: Раскраски графов]]
[[Категория: Алгоритмы и структуры данных]]