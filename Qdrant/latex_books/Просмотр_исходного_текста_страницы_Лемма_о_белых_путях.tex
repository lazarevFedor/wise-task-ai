{{Лемма
|statement =
Не существует такого момента выполнения [[Обход в глубину, цвета вершин|поиска в глубину]], в который бы существовало ребро из черной вершины в белую.
|proof =
Пусть в процессе выполнения процедуры dfs нашлось ребро из черной вершины v в белую вершину u. Рассмотрим момент времени, когда мы запустили dfs(v). В этот момент вершина v была перекрашена из белого в серый, а вершина u была белая. Далее в ходе выполнения алгоритма будет запущен dfs(u), поскольку обход в глубину обязан посетить все белые вершины, в которые есть ребро из v. По алгоритму вершина v будет покрашена в черный цвет тогда, когда завершится обход всех вершин, достижимых из нее по одному ребру, кроме тех, что были рассмотрены раньше нее. Таким образом, вершина v может стать черной только тогда, когда dfs выйдет из вершины u, и она будет покрашена в черный цвет. Получаем противоречие.
}}

== Лемма о белых путях ==
{{Лемма
|statement =
Пусть дан граф G. Запустим dfs(G). Остановим выполнение процедуры dfs от какой-то вершины u графа G в тот момент, когда вершина u была выкрашена в серый цвет (назовем его первым моментом времени). Заметим, что в данный момент в графе G есть как белые, так и черные, и серые вершины. Продолжим выполнение процедуры dfs(u) до того момента, когда вершина u станет черной (второй момент времени).
Тогда вершины графа G\setminus u, бывшие черными и серыми в первый момент времени, не поменяют свой цвет ко второму моменту времени, а белые вершины либо останутся белыми, либо станут черными, причем черными станут те, что были достижимы от вершины u по белым путям.
|proof =
Черные вершины останутся черными, потому что цвет может меняться только по схеме белый \to серый \to черный. Серые останутся серыми, потому что они лежат в стеке рекурсии и там и останутся. 
Далее докажем два факта:
{{Утверждение
|statement=
Если вершина была достижима по белому пути в первый момент времени, то она стала черной ко второму моменту времени.
|proof =
Если вершина v была достижима по белому пути из u, но осталась белой, это значит, что во второй момент времени на пути из u в v встретится ребро из черной вершины в белую, чего не может быть по лемме, доказанной выше.
}}
{{Утверждение
|statement=
Если вершина стала черной ко второму моменту времени, то она была достижима по белому пути в первый момент времени.
|proof =
Рассмотрим момент, когда вершина v стала черной: в этот момент существует cерый путь из u в v, а это значит, что в первый момент времени сущестовал белый путь из u в v, что и требовалось доказать.
}}
Отсюда следует, что если вершина была перекрашена из белой в черную, то она была достижима по белому пути, и что если вершина как была, так и осталась белой, она не была достижима по белому пути, что и требовалось доказать.
}}

[[Категория: Алгоритмы и структуры данных]]
[[Категория: Обход в глубину]]