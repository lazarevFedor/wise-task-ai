{{Определение
|id = maximum_barrier
|neat = 1 
|definition = '''Максимальным по включению [[ Декомпозиция Эдмондса-Галлаи#barrier | барьером ]] '''(англ.''maximal barrier'') называется барьер, не являющийся подмножеством любого другого барьера.
}}

{{Теорема
|id = theorem_about_maximum_barriers
|statement = Пересечение всех максимальных по включению барьеров графа G равно A(G).
|proof = Пусть H {{---}} пересечение всех максимальных по включению барьеров графа G. Чтобы доказать теорему, докажем, что A(G)\subset H и A(G)\supset H.
[[Файл: Max_barriers_a.png|170px|thumb|right|Рисунок 1]]
[[Файл: Max_barriers_b.png|170px|thumb|right|Рисунок 2]]

A(G)\subset H: 
Пусть B {{---}} максимальный по включению барьер, |A(G)\setminus B| = k > 0, B' = B \cup A(G) \Rightarrow |B'| = |B| + k.
Докажем, что B' {{---}} барьер и получим противоречие. Для этого достаточно доказать, что \mathrm{odd}(G\setminus B')\ \geqslant \mathrm{odd}(G\setminus B)\ + k, ведь в таком случае \mathrm{odd}(G\setminus B')\ \geqslant \mathrm{def}(G)\ + |B| + k \Rightarrow \mathrm{odd}(G\setminus B')\ - |B'| \geqslant \mathrm{def}(G). 
Пусть W {{---}} компонента связности графа G - B, содержащая t > 0 вершин из A(G) (см. рисунок 1), если такой компоненты нет, то k = 0 {{---}} противоречие. 
В силу [[ Декомпозиция Эдмондса-Галлаи#barier_struct1| леммы о связи барьера]] с D(G), B\cap D(G) = \varnothing \Rightarrow B'\cap D(G) = \varnothing. Поэтому, W содержит все компоненты связности графа G(D(G)), соединённые рёбрами с W\cap A(G). 
По [[ Декомпозиция Эдмондса-Галлаи#theorem_Gallai_Edmonds| теореме Эдмондса-Галлаи]] все эти компоненты связности нечетные и их хотя бы t + 1. 
Таким образом, при добавлении t вершин из W\cap A(G) в барьер может исчезнуть одна нечётная компонента связности (если |W| нечётно), а появляется хотя бы t + 1 нечётных компонент связности. 
Просуммировав прибавления по всем компонентам связности графа G - B, содержащим вершины из A(G), мы получим, что \mathrm{odd}(G\setminus B')\ \geqslant \mathrm{odd}(G\setminus B)\ + k, что и требовалось доказать.

A(G)\supset H:
Предположим противное: пусть существует вершина x\notin A(G), принадлежащая всем максимальным барьерам. По [[ Декомпозиция Эдмондса-Галлаи#barier_struct3| теореме о структуре барьера]] x\in C(G).
Рассмотрим максимальное паросочетание M графа G, пусть xy\in M.
Докажем, что B = A(G)\cup \{ y \} {{---}} барьер графа G. Так как |B| = |A(G)| + 1, достаточно доказать, что \mathrm{odd}(B)\ \geqslant \mathrm{odd}(A(G))\ + 1.
По [[ Декомпозиция Эдмондса-Галлаи#theorem_Gallai_Edmonds| теореме Эдмондса-Галлаи]] y\in C(G). Пусть W {{---}} компонента связности графа C(G), содержащая x и y (см. рисунок 2). Вершины W разбиваются на пары соединённых рёбрами из M, поэтому |W| чётно.
Множество W' = W\setminus \{ y \} содержит нечётное число вершин и является объединением нескольких компонент связности графа G - B, которых нет в G - A(G). Среди этих компонент связности есть нечётная, значит B {{---}} барьер графа G.
Пусть B' {{---}} максимальный барьер графа G, содержащий B.
В максимальном паросочетании M графа G все вершины барьера B' должны быть соединены рёбрами с вершинами различных нечётных компонент связности графа G - B', следовательно, x\notin B'.
Полученное противоречие показывает, что пересечение всех максимальных барьеров графа G может содержать только вершины из A(G). 
}}

==См. также==

* [[ Декомпозиция Эдмондса-Галлаи ]]
* [[ Лапы и минимальные по включению барьеры в графе ]]
* [[ Паросочетания: основные определения, теорема о максимальном паросочетании и дополняющих цепях ]]
* [[ Теорема Татта о существовании полного паросочетания ]]

== Источники информации ==
* Карпов Д. В. {{---}} Теория графов, стр 54-55

[[ Категория: Алгоритмы и структуры данных ]]
[[ Категория: Задача о паросочетании ]]