{{Определение
|definition=
(s,t)-разрезом (англ. ''s-t cut'') \langle S,T\rangle в сети G называется пара множеств S,T, удоволетворяющих условиям:

# s\in S, t\in T
# S = V\setminus T
}}

{{Определение
|definition=
'''Пропускная способность разреза''' (англ. ''capacity of the cut'') \langle S,T\rangle обозначается c(S,T) и вычисляется по формуле: c(S,T)=\sum\limits_{u\in S}\sum\limits_{v\in T}c(u,v).
}}

{{Определение
|definition=
'''Поток в разрезе''' (англ. ''flow in the cut'') \langle S,T\rangle обозначается f(S,T) и вычисляется по формуле: f(S,T)=\sum\limits_{u\in S}\sum\limits_{v\in T}f(u,v).
}}

{{Определение
|definition=
'''Минимальным разрезом''' (англ. ''minimum cut'') называется разрез с минимально возможной пропускной способностью
}}

{{Лемма
|about =
о величине потока
|statement =
Пусть \langle S,T\rangle — разрез в G. Тогда f(S,T)=|f|.
|proof =
f(S,T)=f(S,V)-f(S,S)=f(S,V)=f(S\setminus s,V)+f(s,V)=f(s,V)=|f|

*1-е равенство выполняется, так как суммы не пересекаются: f(S,V)=f(S,S)+f(S,T)

*2-е равенство выполняется из-за антисимметричности: f(S,S)=-f(S,S)=0

*3-е равенство выполняется, как и 1-е, из-за непересекающихся сумм

*4-е равенство выполняется из-за сохранения потока
}}

{{Лемма
|about =
закон слабой двойственности потока и разреза
|statement =
Пусть \langle S,T\rangle — разрез в G. Тогда f(S,T)\leqslant c(S,T).
|proof =
{c(S,T)-f(S,T)=\sum\limits_{u\in S}\sum\limits_{v\in T}c(u,v)-\sum\limits_{u\in S}\sum\limits_{v\in T}f(u,v)=
\sum\limits_{u\in S}\sum\limits_{v\in T}(c(u,v)-f(u,v))\geqslant 0}, из-за ограничений пропускных способностей f(u,v) \leqslant c(u,v).
}}

{{Лемма
|about =
о максимальном потоке и минимальном разрезе
|statement =
Если f(S,T)=c(S,T), то поток f — максимален, а разрез \langle S,T\rangle — минимален.
|proof =
[[Файл:Минимальный_разрез.png|мини|справа|300x100px|Потоки и разрезы]]
Из закона слабой двойственности следует, что f(S_1,T_1)\leqslant c(S_2,T_2) для любых двух разрезов \langle S_1,T_1\rangle и \langle S_2,T_2\rangle в сети G, так как f(S_1,T_1)=|f|=f(S_2,T_2)\leqslant c(S_2,T_2).
Значит, если расположить все величины потоков и разрезов на оси OX, то у потоков с разрезами может быть максимум 1 точка пересечения.
Очевидно, что эта точка определяет максимальный поток среди всех потоков и минимальный разрез среди всех разрезов сети G.}}

 
[[Файл:разрезы.png|мини|слева|800x600px|Среди всех разрезов сети разрез с минимальной пропускной способностью определяет максимальный поток в сети.]]

{|border="1" class="wikitable" style="width: 400px; height: 150px; float: слева;" 
|+ style="caption-side:bottom; "|''Минимальный разрез — 1 с пропускной способностью 60''

|-
| '''Разрез'''|| '''"Разрезанные" ребра'''|| '''Пропускная способность'''
|-

|1
| (1,2),(1,3),(1,4)
| 10+30+20=60

|-
| 2
|(1,3),(1,4),(2,3),(2,5) 
|30+10+40+30=110 

|-
|3
|(2,5),(3,5),(4,5) 
| 30+20+20=70

|}

== Источники информации ==
* ''Кормен, Томас Х., Лейзерсон, Чарльз И., Ривест, Рональд Л., Штайн Клиффорд'' '''Алгоритмы: построение и анализ''', 2-е издание. Пер. с англ. — М.:Издательский дом "Вильямс", 2010. — 1296 с.: ил. — Парал. тит. англ. — ISBN 978-5-8459-0857-5 (рус.)
* [http://ru.wikipedia.org/wiki/Разрез_графа Википедия: Разрез графа]
* [http://en.wikipedia.org/wiki/Cut_(graph_theory) Википедия: Разрез графа (англ.)]

[[Категория:Алгоритмы и структуры данных]]
[[Категория:Задача о максимальном потоке]]