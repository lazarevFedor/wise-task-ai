==Теорема о существовании простого пути в случае существования пути==
[[Файл:Simple way.png|thumb|250px|center|Ориентированный граф. Красным выделен вершинно-простой путь. Синим {{---}} реберно-простой путь.]]
{{Теорема
|statement=
Если между двумя [[Основные определения теории графов|вершинами графа]] существует [[Основные определения теории графов|путь]], то между ними существует [[Основные определения теории графов|вершинно-простой путь]].
|proof =
=== Конструктивное доказательство ===
Рассмотрим путь: v_0e_1v_1e_2v_2 \ldots e_nv_n между вершинами v_0 и v_n, причём v_0 \neq v_n. Возьмем v_i {{---}} вершина на данном пути. Если она лежит на данном пути более одного раза, то она принадлежит какому-то (не обязательно простому) циклу v_ie_{i+1}v_{i+1}e_{i+2} \ldots v_{j=i}. Удалим этот цикл. Получившаяся последовательность вершин и рёбер графа останется путём v_0 \ldots v_n, но не будет содержать найденный цикл. Начнём процесс с вершины v_0 и будем повторять его каждый раз для следующей вершины нового пути, пока не дойдём до последней. По построению, получившийся путь будет содержать каждую из вершин графа не более одного раза, а значит, будет вершинно-простым.

=== Неконструктивное доказательство ===
Выберем из всех путей между данными вершинами путь наименьшей длины.

{{Утверждение
|statement = Допустим, что выбранный путь не является простым}}
Тогда в нём содержатся две одинаковые вершины v_i = v_j, i . Удалим из исходного пути отрезок от e_{i+1} до v_j, включительно. Конечная последовательность также будет путём от v_0 до v_n и станет короче исходной. Получено противоречие с условием: взятый нами путь оказался не кратчайшим. Значит, предположение неверно, выбранный путь {{---}} простой.
}}
{{Утверждение
|statement = Данная теорема не верна для случая v_0 = v_n.
|proof = В данном случае мы не сможем найти вершинно-простой путь, так как путь начинается и заканчивается в одной и той же вершине.
}}

== Замечания ==
* Так как вершинно-простой путь всегда является рёберно-простым, данная теорема справедлива и для рёберно-простого пути.
* Теорема может быть сформулирована как для [[Основные определения теории графов|ориентированного]], так и для неориентированного графа.

== См. также ==
* [[Основные определения теории графов]]
* [[Теорема о существовании простого цикла в случае существования цикла]]

[[Категория: Алгоритмы и структуры данных]]
[[Категория: Основные определения теории графов]]