{{Теорема
|author=Формула Кэли
|statement=Число помеченных деревьев порядка n равно n^{n - 2}.
|proof=
Можно доказать формулу двумя способами.

''Первый способ.'' 

Так как между помеченными деревьями порядка n и последовательностями длины n - 2 из чисел от 1 до n существует биекция ([[Коды Прюфера|Код Прюфера]]), то количество помеченных деревьев совпадает с количеством последовательностей длины n - 2 из чисел от 1 до n = n^{n - 2}.

''Второй способ.''
С помощью [[Подсчет числа остовных деревьев с помощью матрицы Кирхгофа |матрицы Кирхгофа]] для полного графа на n вершинах. Число помеченных деревьев порядка n, очевидно, равно числу остовов в полном графе K_n, которое есть n^{n-2} по следствию теоремы Кирхгофа.
}}

==См. также==
*[[Матрица Кирхгофа]]
*[[Подсчет числа остовных деревьев с помощью матрицы Кирхгофа]]
*[[Связь матрицы Кирхгофа и матрицы инцидентности]]
*[[Коды Прюфера]]

== Источники информации==
*[http://rain.ifmo.ru/cat/view.php/theory/graph-general/cayley-2008 Дискретная математика: Алгоритмы. Формула Кэли]

[[Категория: Алгоритмы и структуры данных]]
[[Категория: Остовные деревья ]]
[[Категория: Свойства остовных деревьев ]]
[[Категория: Удалить]]