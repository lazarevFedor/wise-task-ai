{{Определение
|id = odd
|definition =\mathrm{odd}({G}) {{---}} число нечетных компонент связности в графе {G}, где '''нечетная компонента''' (англ. ''odd component'') {{---}} это [[Отношение связности, компоненты связности#def2|компонента связности]], содержащая нечетное число вершин. 
}}

{{Определение
|id = Tutt_set
|definition ='''Множество Татта''' графа {G} {{---}} множество S \subset {V_{G}}, для которого выполнено условие: \mathrm{odd}({G} \setminus S) > \left\vert S \right\vert
}}

==Критерий Татта==
Пусть {G'} {{---}} граф, полученный из {G}=\langle {V},{E} \rangle добавлением ребер, при этом в {G'} нет [[Теорема Холла#def1|полного паросочетания]], но оно появляется при добавлении любого нового ребра.

Так как новых вершин не добавлялось, то {G'}=\langle {V},{E'}\rangle 

Пусть U = \{ v \in {V}: \deg_{G'} (v) = n - 1 \}. 

Очевидно, что \left\vert U \right\vert \ne n, потому что {G'} {{---}} не полный граф. 
{{Лемма
|statement= {G'} \setminus U {{---}} объединение несвязных полных графов.
|proof=Пусть это не так, тогда существуют вершины x,y,z \in {V} \setminus U, такие что xy, yz \in {E'}, но xz \notin {E'}. Так как y \notin U, то \exists t \notin U: yt \notin {E'}.

По построению {G'} в графе {G'}+xz существует полное паросочетание M_1. Аналогично, в графе {G'}+yt существует полное паросочетание M_2. Так как в {G'} нет полного паросочетания, то xz \in M_1 и yt \in M_2. 

Возможны два случая:
# Вершины x,z и y,t лежат в разных полных подграфах графа {G'} \setminus U, обозначим их H_1 и H_2, соответственно. 
#: Покроем вершины подграфа H_1 паросочетанием M_2, при этом заметим, что ребро xz не входит в это паросочетание. Аналогично покроем паросочетанием M_1 вершины подрафа H_2 и ребро yt не войдет в это паросочетание. Если остались непокрытые вершины, то покроем их ребрами из любого паросочетания M_1 или M_2. Таким образом, мы получим полное паросочетание в графе {G'}, что противоречит его построению. 
#: [[Файл:Граф_для_теоремы_Татта.png|right|200px|thumb|К доказательству 2-ого пункта леммы.]]
# Вершины x,y,z и t лежат в одном подграфе графа {G'} \setminus U.
#: Построим граф H, такой что {V_{H}}={V} и {E_{H}}=M_1 \oplus M_2[http://ru.wikipedia.org/wiki/%D1%E8%EC%EC%E5%F2%F0%E8%F7%E5%F1%EA%E0%FF_%F0%E0%E7%ED%EE%F1%F2%FC Симметрическая разность] . Получим, что вершины x,y,z и t лежат на каком-то чередующемся цикле из ребер M_1 и M_2. Рассмотрим подробнее, почему это будет именно так. Ребро xz принадлежит паросочетанию M_1, значит вершина y и какая-то произвольная вершина v будут покрыты ребром паросочетания M_1, при этом эти ребра не принадлежат паросочетанию M_2, но ребра yt и vu, где u {{---}} произвольная вершина, принадлежат M_2 и не принадлежат M_1 и так далее. Таким образом и получается чередующийся цикл в графе H. В силу симметричности x и z можно считать, что вершины расположены в порядке tzxy. Тогда существует путь P_1=t..zx..y и полное паросочетание в нем, следовательно существует и путь P_2=t..zy..x, содержащий только ребра графа {G'}. Тогда на пути x..y возьмем ребра из паросочетания M_2, а на пути t..z - ребра из паросочетания M_1. Непокрытыми остались вершины z и y, которые мы покроем ребром yz. Вершины, не принадлежащие рассматриваемому циклу, покроем ребрами любого из паросочетаний M_1, M_2 (выберем ребра одного из них). Таким образом, получили полное паросочетание в графе {G'}, противоречие. 

В каждом из возможных случаев получили противоречие, значит, наше начальное предположение тоже неверно и {G'} \setminus U {{---}} объединение несвязных полных графов, лемма доказана.
}}

== Теорема Татта ==

{{Теорема
|statement=В графе {G} существует полное паросочетание \Leftrightarrow 
\forall S \subset {V} выполнено условие: \mathrm{odd}({G} \setminus S) \leqslant \left\vert S \right\vert (то есть в графе {G} нет ни одного множества Татта)
|proof =
\Rightarrow 
Рассмотрим M {{---}} полное паросочетание в графе {G} и множество вершин S \subset {V}.

Одна из вершин каждой нечетной компоненты связности графа {G} \setminus S соединена ребром паросочетания M с какой-то вершиной из S. Иначе мы не сможем покрыть паросочетанием все вершины этой компоненты связности и получим противоречие с тем, что полное паросочетание существует по условию теоремы. Таким образом, получаем, что \mathrm{odd}({G} \setminus S) \leqslant \left\vert S \right\vert.

\Leftarrow 
Пусть для графа {G} выполнено, что \mathrm{odd}({G} \setminus S) \leqslant \left\vert S \right\vert, но полного паросочетания в этом графе не существует.

Рассмотрим граф {G'} и множество вершин U (из леммы). Так как число нечетных компонент не увеличивается при добавлении новых ребер, то \forall S \subset {V} выполнено \mathrm{odd}({G'} \setminus S) \leqslant \mathrm{odd}({G} \setminus S) \leqslant \left\vert S \right\vert. По лемме, доказанной выше: {G'} \setminus U {{---}} объединение несвязных полных графов.

Очевидно, что в каждой четной компоненте связности графа {G'} \setminus U мы можем построить полное паросочетание. В каждой нечетной компоненте этого графа построим паросочетание, которое покрывает все вершины кроме одной, оставшуюся непокрытой вершину, соединим с какой-то вершиной множества U. При этом мы будем использовать различные вершины из U, это возможно, так как \mathrm{odd}({G'} \setminus U) \leqslant \left\vert U \right\vert. Если все вершины множества U оказались покрытыми, то мы получили полное паросочетание в графе {G'}. Противоречие, так как по построению в {G'} нет полного паросочетания.

Значит, в U осталось какое-то количество непокрытых вершин, при этом их четное число, потому что число вершин в {G'} четно, так как \mathrm{odd}({G'} \setminus \varnothing) \leqslant \left\vert \varnothing \right\vert = 0 и уже покрыто паросочетанием четное число вершин. Так как в множество U входят вершины, которые в {G'} смежны со всеми остальными, то мы сможем разбить оставшиеся вершины на пары и покрыть их паросочетанием.

Таким образом, получили в {G'} полное паросочетание, что противоречит тому, как мы задали этот граф изначально. Значит, начальное предположение не верно, и в {G} существует полное паросочетание.
 
}}

==См. также==
* [[Матрица Татта и связь с размером максимального паросочетания в двудольном графе]]
* [[Паросочетания: основные определения, теорема о максимальном паросочетании и дополняющих цепях]]
* [[Декомпозиция Эдмондса-Галлаи]]

==Примечания==

== Источники информации ==
*[http://logic.pdmi.ras.ru/~dvk/211/graphs_dk.pdf Д.В Карпов. Теория графов] (2 глава, стр. 29)
*[http://en.wikipedia.org/wiki/Tutte_theorem Wikipedia — Tutte theorem]

[[Категория:Алгоритмы и структуры данных]]
[[Категория:Задача о паросочетании]]