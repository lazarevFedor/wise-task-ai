{{Определение
|id = Heawood number
|definition =
'''Хроматическим числом поверхности поверхности S_n''' или '''n-ым числом Хивуда''' называется число \chi \left( S_n \right), равное максимальному хроматическому числу графа, который можно уложить на поверхность n-ого рода.
}}

==Теорема о нижней границе хроматического числа поверхности==

{{Теорема
|about=
Теорема Рингеля и Янгса
|statement=
Для любого положительного целого числа n хроматическое число поверхности n-ого рода \chi \left( S_n \right) \geqslant \left[ \dfrac{7 + \sqrt{1 + 48n}}{2} \right].
|proof=

Воспользуемся формулой Эйлера V + F - E = 2 - 2n. Давайте докажем нижнюю границу на E. Максимизируем число граней: каждая из них может быть треугольником. Тогда для E существует неулучшаемая нижняя граница:

E \geqslant 3 \left( V - 2 + 2n \right)

n \geqslant \dfrac{1}{6} E - \dfrac{1}{2} \left( V - 2 \right).

Рассмотрим полный граф K_p, тогда получаем, что

\gamma \left( K_p \right) \geqslant \dfrac{1}{6} \dfrac{p (p - 1)}{2} - \dfrac{p - 2}{2}

\gamma \left( K_p \right) \geqslant \left\{ \dfrac{(p - 3)(p - 4)}{12} \right\}, функция монотонно возрастает при p \geqslant 4, и для любого n наибольшее значение функция \left\{ \dfrac{(p - 3)(p - 4)}{12} \right\} достигается при p=\left[\dfrac{7 + \sqrt{1 + 48n}}{2} \right]. Поскольку \chi\left(K_p\right) = p, откуда получаем, что \chi \left( S_n \right) \geqslant \left[ \dfrac{7 + \sqrt{1 + 48n}}{2} \right].
}}

==Теорема о верхней границе хроматического числа поверхности==

{{Теорема
|about=
Гипотеза Хивуда
|statement=
Для любого положительного целого числа n хроматическое число поверхности n-ого рода \chi \left( S_n \right) \leqslant \left[ \dfrac{ 7 + \sqrt{1 + 48n} }{ 2 } \right].
|proof=

Пусть задан граф G с V вершина, E рёбрами и F гранями, также будем считать, что G {{---}} триангуляция (добавляя таким образом рёбра мы всё ещё получаем граф, который можно уложить на поверхности n-ого рода). Обозначим за d {{---}} среднюю степень вершины графа G, тогда должно быть справедливым следующее равенство:

dV = 2E = 3F 

Воспользуемся [[формула Эйлера | формулой Эйлера]] V - E + F = 2 - 2 n, откуда 

E = V + F + 2 (n - 1) и F = 2 V + 4 (n - 1)

и подставляя в первое равенство получаем

dV = 6V + 12(n - 1) 

d = 6 + \dfrac{12(n - 1)}{V}

Поскольку d \leqslant V - 1, то

V - 1\geqslant 6 + \dfrac{12(n - 1)}{V}

Найдём единственный положительный корень неравенства

V \geqslant \left[ \dfrac{7 + \sqrt{1 + 48n}}{2} \right]

Обозначим за H(n) = \left[ \dfrac{7 + \sqrt{1 + 48n}}{2} \right]. Если V \leqslant H(n), то тогда граф G очевидно можно раскрасить в H(n) цветов и неравенство верное. Допустим, что V > H(n), тогда

 d 

Значит в такое графе существует хотя бы одна вершина степени не больше H(n) - 2, стянем её с любой соседней и получим новый граф G' с V - 1 вершинами. Если V - 1 = H(n), то граф G' можно раскрасить в H(n) цветов, значит и сам граф G можно также раскрасить в H(n) цветов, если V - 1 > H(n), то опять найдём вершину степени H(n) - 2 и снова стянем её и будем продолжать так до тех пор, пока не получим желаемый граф.
}}

Из всего выше сказанного получаем, что \chi \left(S_n\right) в точности равно \left[ \dfrac{7 + \sqrt{1 + 48n}}{2} \right].

==Проблема четырёх красок==
Заметим, что теорема Хивуда не работает при n = 0, поэтому [[проблема четырёх красок]] не может быть доказана с помощью этой теоремы, однако при подстановке n = 0 получаем \chi \left( S_0 \right) = 4.

==См. также==
* [[Хроматическое число планарного графа]]
* [[Проблема четырёх красок]]
* [[Формула Эйлера]]

==Источники информации==
* [https://en.wikipedia.org/wiki/Heawood_conjecture Wikipedia {{---}} Heawood conjecture]
* [https://oeis.org/A000934 Последовательность чисел Хивуда]
* Ф.Харари «Теория графов» {{---}} М.: Мир, 1973 г. {{---}} стр. 162 - 164