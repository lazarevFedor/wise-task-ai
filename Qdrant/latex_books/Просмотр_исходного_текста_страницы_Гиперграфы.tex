{{Определение
|definition=
'''Гиперграфом''' (англ. ''hypergraph'') H называют такую пару H = (X, E) , где X - множество вершин, а E - семейство подмножеств X , называемых '''гиперребрами''' (англ. ''hyperedges'') 
}} 

Обычные графы, у которых ребра могут соединять только две вершины, являются частным случаем гиперграфа, у которых все гиперребра содержат только две вершины.

[[Файл:Hypergraph.jpg|thumb|450px|Рис. 1: Гиперграф с множеством вершин V = \{ v_1, v_2, v_3, v_4, v_5, v_6, v_7 \} и гиперребрами E = \{ \{ v_1, v_2, v_3 \} , \{ v_2, v_3 \} , \{ v_3, v_5, v_6 \} , \{ v_4 \} \}]]

==Основные понятия гиперграфов==

{{Определение
|definition=
'''Путем''' (англ. ''path'') между двумя гиперребрами e_i и e_j гиперграфа H называется последовательность гиперребер e_{u_1}, e_{u_2} , \ldots ,e_{u_k} таких что :
# e_{u_1} = e_i и e_{u_k} = e_j
# \forall v: 1 \leqslant v \leqslant k-1, e_v \cap e_{v+1} \ne \emptyset
}}

{{Определение
|definition=
Гиперграф H называется '''связным''' (англ. ''connected'') тогда и только тогда, когда существует путь между каждой парой гиперребер.
}}

[[Файл:Connected_hypergraph.jpg‎|thumb|450px|center|Рис. 2: Связный гиперграф]]

Пусть E - набор гиперребер, e_1 и e_2 - элементы E и q = e_1 \cap e_2. 
{{Определение
|definition=
q называется '''сочленением''' (англ. ''articulation'') E , если при его удалении из всех гиперребер E, множество разрывается.
}}

На рис.2 q = e_4 \cap e_6 = \{ x_{12}, x_{13}\} является сочленением E.

==Матрица инцидентности ==

Пусть дан гиперграф H = (X, E) , где X = \{ x_1, x_2, \ldots , x_n \} и E = \{ e_1, e_2, \ldots , e_m \}. Любой гиперграф может задаваться матрицей инцидентности (смотри [[Матрица_инцидентности_графа|матрицу инцидентности в обычном графе)]] A = (a_{ij}) размером n \times m, где

 a_{ij} = \left \{ 
\begin{array}{ll}
 0 & x_i \in e_j \\
 1 & \mathrm{otherwise}
 \end{array}
 \right.

Так например, для гиперграфа на рис.1 мы можем построить матрицу инцидентности по таблице отношения принодлежности вершины к гиперребру:

{| class="wikitable" align="left" style="color: black; background-color:#ffffcc;" cellpadding="10"
|+
!
!e_1
!e_2
!e_3
!e_4
|-align="center"
!v_1
|✓|||| ||
|-align="center"
!v_2
|✓||✓|| ||
|-align="center" 
 !v_3
|✓||✓||✓|| 
|-align="center" 
 !v_4
| |||| ||✓
|-align="center" 
 !v_5
|||||✓|| 
|-align="center" 
 !v_6
|||||✓|| 
|-align="center" 
 !v_7
|||||||
|}

 A = 
\begin{pmatrix}
1 & 0 & 0 & 0\\
1 & 1 & 0 & 0\\
1 & 1 & 1 & 0\\
0 & 0 & 0 & 1\\
0 & 0 & 1 & 0\\
0 & 0 & 1 & 0\\
0 & 0 & 0 & 0\\
\end{pmatrix}

==Цикл в гиперграфе==

{{Определение
|definition=
'''Простым циклом''' длины s в гиперграфе H = (V, E) называется последовательность ( A_0, v_0, A_1, \ldots , A_{s - 1}, v_{s - 1}, A_s) , где A_0 , \ldots , A_{s - 1} - различные ребра H , ребро A_s совпадает с A_0 , а v_0, \ldots , v_{s - 1} - различные вершины H , причем v_i \in A_i \cap A_{i+1} для всех i = 0, \ldots , s - 1. 
}}

[[Файл:Cycle_hyper.jpg|thumb|450px|center|Рис. 3: Простейший случай цикла в гиперграфе]]

Универсальным способом задания гиперграфа является кенигово представление.

{{Определение
|definition=
'''Кенигово представление''' гиперграфа H = (V, E) - обыкновенный двудольный граф '''K(H)''' , отражающий отношение инцидентности различных элементов гиперграфа, с множеством вершин V \cup E и долями V, E.
}}

Первым, кто дал определение ацикличности гипергафа является Клауд Берж:

{{Теорема
|statement=
Гиперграф H не содержит циклов в том случае, если его кенигово представление - ацикличный граф, сожержит в противном случае.
}}

Таким образом, если у нас есть цикл в графе кенигова представления, значит и сам гиперграф имеет цикл.

[[Файл:Cycle_example.png|thumb|center|500px|Рис. 4: Пример гиперграфа, содержащего цикл]]

===Алгоритм нахождения цикла в гиперграфе===

Поскольку гиперграф может задаваться кениговым представлением, тогда произведём серию поисков в глубину в двудольном графе. Т.е. из каждой вершины, в которую мы ещё ни разу не приходили, запустим поиск в глубину, который при входе в вершину будет красить её в серый цвет, а при выходе - в чёрный. И если поиск в глубину пытается пойти в серую вершину, то это означает, что мы нашли цикл (если граф неориентированный, то случаи, когда поиск в глубину из какой-то вершины пытается пойти в предка, не считаются).

==Ацикличность гиперграфов==

{{Определение
|definition=
'''Редукцией''' (англ. ''reduction'') гиперграфа H = (V, E) называется такой гиперграф H' = (V, E') , который получается из исходного путем удаления всех гиперребер, которые полностью содержатся в других гиперреберах.
}}

{{Определение
|definition=
Гиперграф называется '''уменьшенным''' (англ. ''reduced'') , если он эквивалентен своей редукции, то есть не имеет гиперребер внутри других гиперребер.
}}

Пусть M - множество вершин гиперграфа H = (V, E). Множество '''частичных ребер''' (англ. ''partial edges''), порожденных множеством M, определяется как множество, полученное путем пересечения гиперребер из множества E с M. Таким образом, получаем множество : \{ e \cap M : e \in E \} - \{ \emptyset \} и берем его редукцию.

Множество частичных ребер, порожденное из гиперграфа H множеством M, называется '''вершинно-порожденным''' (англ. ''node-generated'') множеством частичных ребер.

{{Определение
|definition=
'''Блоком''' (англ. ''block'') уменьшенного гиперграфа называется связное, вершинно - порожденное множество частичных ребер без сочленения.
}}

{{Определение
|definition=
Множество частичных ребер называется '''тривиальным''' (англ. ''trivial''), если оно содержит одно гиперребро.
}}

{{Определение
|definition=
Уменьшенный гиперграф называется ''' \alpha - ацикличным''' (англ. '' \alpha -acyclity'') , если всего его блоки тривиальны, иначе называют ''' \alpha -цикличным''' (англ. '' \alpha-cyclity'').
}}

''Пример''

[[Файл:Alpha-acyclity-1.png|thumb|left|500px|Рис. 5: \alpha-ацикличный гиперграф]]
[[Файл:Alpha-acyclity-2.png|thumb|center|500px|Рис. 6: Подмножество гиперребер \{ ABC, CDE, EFA\} ]]

Очень просто проверить что на рис. 3 представлен \alpha -ацикличный гиперграф. Он содержит четыре гиперребра - ABC, CDE, EFA, ACE. Сочленение для всего множества гиперребер является ABC \cap ACE = AC , так как после удаления вершин A и C гиперграф не будет связным (вершина B не будет ни с кем соединена). Заметим, что на рис. 6 подмножетсво гиперребер \{ ABC, CDE, EFA \} не имеет сочленения. Однако, это множество не является вершинно - порожденным , таким образом, нет никаких противоречий с предположением, что гиперграф на рис. 5 является \alpha -ацикличным.

Заметим, что \alpha -ацикличность имеет одно нелогичное свойство: при добавлении гиперребер к \alpha -цикличному гиперграфу он может стать \alpha -ацикличным (например, при добавлении гиперребра, которое охватывает все вершины, всегда будет делать гиперграф \alpha -ацикличным). Из-за этого свойства было введено более строгое определение, называемое \beta -ацикличностью.

{{Определение
|definition=
Гиперграф H = (V, E) является ''' \beta -ацикличным''' (англ. '' \beta -acyclity'') , если все его подгиперграфы \alpha -ацикличны.
}}

Так например гиперграф на рис. 5 является \alpha -ацикличным, но не является \beta -ацикличным, так как его подгиперграф на рис. 6 является \alpha -цикличным.

== См. также ==
* [[Основные_определения_теории_графов|Основные определения теории графов]]

== Источники информации ==
* [https://en.wikipedia.org/wiki/Claude_Berge wikipedia.com — Клауд Берж]
* [https://en.wikipedia.org/wiki/Hypergraph wikipedia.com — Гиперграфы]
* [http://www.sciencedirect.com/science/article/pii/S0012365X09003446?np=y sciencedirect.com — Ацикличность в гиперграфах]

[[Категория:Дискретная математика и алгоритмы]]
[[Категория:Основные определения теории графов]]