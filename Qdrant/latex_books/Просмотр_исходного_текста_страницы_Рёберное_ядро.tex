{{Определение|
definition=
'''Рёберное ядро''' (англ. ''core'') C_1(G) графа G {{---}} это подграф [[Основные определения теории графов#finite graph|графа]] G, порожденный объединением таких независимых множеств Y \subset E(G), что |Y| = \alpha_{0}(G), где \alpha_{0}(G) {{---}} число вершинного покрытия.
}}

{{Определение|
definition= 
Множество [[Основные определения теории графов#def_graph_edge_1|ребер]] (вершин) называется '''независимым''' (англ. ''independent''), если никакие его два элемента не смежны.
}}
{{Определение
|id=def_3
|definition=
'''Вершинным покрытием''' (англ. ''vertex cover'') графа G называется такое множество V его вершин, что у любого ребра в G хотя бы одна из вершин лежит в V.
}}
{{Определение|
definition=
'''Числом вершинного покрытия''' (англ. ''point-covering number'') называется число вершин в наименьшем вершинном покрытии графа G.
}}
==Критерий существования реберного ядра==
{{Определение|
definition=
Наименьшее вершинное покрытие M графа G с множеством вершин V называется '''внешним''' (англ. ''external vertex cover''), если для любого подмножества M' \subseteq M выполняется неравенство |M'| \leqslant |U(M')|, где U(M') = \{v \mid \:v \in V(G) \setminus M, \: vu \in E(G), \: u \in M'\}. 
}}
{{Теорема|
statement=
Для произвольного графа G следующие утверждения эквивалентны:
(1) G имеет не пустое рёберное ядро. 
(2) G имеет внешнее наименьшее вершинное покрытие.
(3) каждое наименьшее вершинное покрытие для G является внешним.
|proof=
Обозначим минимальное вершинное покрытие G как M. Пусть U = V(G) \setminus M. 
Докажем (1) \Rightarrow (3). Предположим, что в G существует наименьшее вершинное покрытие M, которое не является внешним.
Это значит что \exists M' : \: M' = \{u_1, \dots, u_r \}, где r \leqslant \alpha_0(G),
такое что |M'| > |U(M')|. Пусть U(M') = \{u_1, \dots, u_t\}, \: t . Так же, пусть X {{---}} максимальное независимое множество ребер в G. Поскольку никакие две вершины U не смежны, каждое ребро из X соединено, по крайней мере, с одной вершиной из M. Если какое-нибудь ребро из X соединено более чем с одной ввершиной из M, то |X| и C_1(G) = \varnothing . Так что будем считать, что каждое ребро из X смежно ровно с одной вершиной из M. Из этого сдедует, что |X| \leqslant t - r + \alpha_0(g) . И снова C_1(G) = \varnothing.
Следствие (3) \Rightarrow (2) {{---}} очевидно. 
Докажем (2) \Rightarrow (1).
Пусть M = \{v_1, \dots, v_s\} {{---}} наименьшее внешнее вершинное покрытие. Пусть Y_i = \{u \mid u \in U, uv_i \in E(G) \}. Тогда для любого k: \:\: 1 \leqslant k \leqslant s, объединение любых k различных множеств Y_i содержит, по меньшей мере k вершин. 
Следовательно, по [[Теорема Холла|теореме о свадьбах (Холла)]], существует множество s различных вершин \{y_1, \dots, y_s\}, \: y_j \in Y_j. Следовательно существует набор независимых ребер y_1v_1, \dots, y_sv_s. А значит C_1(G) не может быть пустым.
}}
[[Файл:EdgeCore.png|thumb|500px|рис. 1. a) граф H, б) реберное ядро графа H ]]
В качестве примера рассмотрим граф H изображенный на рис. 1 а). Этот граф имеет два наименьших вершинных покрытия: M_1 = \{B, E, F\} и M_2 = \{B, E, G\}.
Пусть M_1' = M_1 то U(M_1') = \{A, C, D, G\}. Пусть M_1'' = \{E, F\}. Тогда U(M_1'') =\{C, D, G\}.
Отсюда |M_1'| \leqslant |U(M_1')| и |M_1''| \leqslant |U(M_1'')|. И это верно для любого подмножества M_1. Значит, M_1 {{---}} внешнее покрытие. Значит и M_2 {{---}} внешнее покрытие.

==Реберное ядро в двудольном графе==
Здесь и далее будем рассматривать [[Двудольные графы|двудольный граф]] G, в котором обозначим S {{---}} множество вершин левой доли, T {{---}} множество вершин правой доли.
{{Определение |
definition= G {{---}} '''полунесводимый граф''' (англ. ''semi-irreducible graph''), если G имеет ровно одно вершинное покрытие M, такое что или M \cap S или M \cap T {{---}} пусто
}}
{{Определение|
definition= 
G {{---}} '''несводимый''' граф (англ. ''irreducible graph''), если он имеет ровно два наименьших вершинных покрытия M_1 и M_2, таких что либо M_1 \cap S \cup M_2 \cap T = \varnothing , либо M_2 \cap S \cup M_1 \cap T = \varnothing
}}
{{Определение|
definition=
G {{---}} '''сводимый граф''' (англ. ''reducible graph'') если он не является ни полунесводимым, ни несводимым.
}}

{{Теорема|
id=th2|
statement=
Если оба конца ребра w \in E(G) покрыто некоторым минимальным вершинным покрытием, то w \notin C_1(G).
|proof=
Сошлемся на теорему 3 (Theorem 3)A. L. Dulmage and N. S. Mendelsohn, 1958, pp. 519. аналогичного результата для двудольных графов. То же самое доказательство можно перенести на произвольный граф.
}}
{{ Утверждение
|about=Следствие 1
|statement=Eсли G имеет минимальное вершинное покрытие, которое не является независимым, то G \neq C_1(G).
}}

{{ Утверждение
|about=Следствие 2
|id=proposal2
|statement=Если G {{---}} сводимый связный двудольный граф, то G \neq C_1(G).
}}

{{Теорема|
id=th3|
statement=
Если G имеет непустое реберное ядро, то C_1(G) \supset G, C_1(C_1(G)) = C_1(G), а компоненты C_1(G) являются несводимыми или полунесводимыми двудольными подграфами G
}}

{{Теорема
|id=th4
|statement=
G и его реберное ядро C_1(G) совпадают тогда и только тогда, когда G является двудольным и не является сводимым.
}}

=== Примеры ===
[[File:Bipartite_graph_1.png|thumb|130px|Двудольный граф G_1]]
[[File:Bipartite_graph_2.png|thumb|130px|Двудольный граф G_2]]

Рассмотрим двудольные графы G_1 и G_2, изображенные на рисунках 1 и 2. В графе G_1 пусть S_1 = \{v_3, v_6\} и T_1 = \{v_1, v_2, v_4, v_5, v_7 \}. Этот граф имеет единственное наименьшее вершинное покрытие M_1 = \{v_3, v_6\} и, поскольку M_1 \cap T_1 = \varnothing, он полунесводимый; следовательно, он совпадает со своим рёберным ядром. В графе G_2 пусть S_2 = \{u_1, u_4, u_5\} и T_2 = \{u_2, u_3, u_6\}. В нём два наименьших вершинных покрытия, именно M_2 = \{u_1,u_4, u_5\} и N_2 = \{u_2, u_3, u_6\}. Так как M_2 \cap T_2 = \varnothing и N_2 \cap S_2 = \varnothing, то G_2 {{---}} несводимый граф и, значит, совпадает со своим рёберным ядром.

== См. также ==
* [[NP-полнота задачи о независимом множестве]]
* [[Теория Рамсея]]
* [[Связь максимального паросочетания и минимального вершинного покрытия в двудольных графах]]

==Примечания==

== Источники информации ==

* [https://math.dartmouth.edu/archive/m38s12/public_html/sources/Hall1935.pdf P. Hall, On representatives of subsets, Journal of the London Mathematical Society 10 (1935) pp. 26-30.]

* [https://cms.math.ca/openaccess/cjm/v10/cjm1958v10.0517-0534.pdf A. L. Dulmage and N. S. Mendelsohn: Coverings of bipartite graphs, Canad J. Math., (1958), 517-534.]

[[Категория: Алгоритмы и структуры данных]]
[[Категория: Основные определения теории графов]]