'''Теорема Самнера — Лас Вергнаса''' даёт достаточное условие для существования совершенного паросочетания в графах чётного порядка.

==Подготовка к доказательству==
{{Определение
|definition=
'''Смежными листами''' (англ. ''coincident endpoints'') в неориентрированном графе называется такая пара вершин x, y, что \operatorname{deg}x = 1, \operatorname{deg}y = 1, причём обе вершины имеют общую смежную вершину (другими словами, расстояние между этими вершинами \rho(x, y) = 2). 
}}
Для доказательства основной теоремы потребуется доказать вспомогательную лемму:
{{Лемма
|statement=
Если G — связный граф, состоящий из n \geq 3 вершин и не содержащий ''смежных листов'', то найдутся такие две '''смежные''' вершины x, y, что граф G \backslash \{x, y\} также будет связен.
|proof=
:Лемма, очевидно выполняется для полных графов K_n. Таким образом, будем считать далее, что диаметр графа d \geq 2.
:Пусть a, y — вершины графа G, находящиеся на расстоянии \rho(a, y) = d, а D = a \dots xy — путь между этими вершинами длины d (вершины a и x могут совпадать).
:Предположим, что G^* = G \backslash \{x, y\} не связен. Обозначим за A компоненту связности G^* такую, что a \in A. Так как D является диаметром графа G, то все вершины графа G^* \backslash A смежны с x в графе G (иначе мы бы нашли пару вершин, расстояние между которыми было бы больше, чем d). После этого возможны несколько случаев:
:# Граф G^* \backslash A содержит компоненту B размера m \geq 2. Тогда для \forall b, c \in B, которые в B являются смежными, в G \backslash \{b, c\} будет существовать путь из каждой вершины до x, а значит, G \backslash \{b, c\} связен.
:# Граф G^* \backslash A содержит вершину e, смежную с y в графе G. Тогда по аналогичным причинам граф G \backslash \{e, y\} связен.
:# Граф G^* \backslash A не содержит компонент размера m \geq 2 (случай 1), а значит он содержит только изолированные вершины. Тогда все вершины G^* \backslash A связаны только с x в исходном графе G (они могли быть связаны максимум с еще одной вершиной y, а это было рассмотрено в случае 2). Но, так как граф не содержит смежных листов, то G^* \backslash A состоит из единственной вершины f. Если \operatorname{deg}y = 1, то f и y являлись бы смежными листами. Таким образом, y должен быть связан с вершиной из A. Следовательно, G \backslash \{f, x\} связен.
}}

==Теорема==
Докажем оригинальную версию теоремы, доказанную независимо Самнером (''Sumner'', 1974) и Лас Вергнасом (''Las Vergnas'', 1975). Напомним, что индуцированный подграф - это граф, образованный из подмножества вершин графа вместе со всеми рёбрами, соединяющими пары вершин из этого подмножества.

{{Теорема
|about= 
Самнера — Лас Вергнаса
|statement=
Пусть G — связный граф четного порядка 2n \geq 3, и k - такое число, что любой индуцированный связный подграф G четного порядка 2k содержит совершенное паросочетание (1 ). Тогда G также содержит совершенное паросочетание.
|proof=
:Докажем теорему по индукции.
:Теорема довольно просто проверяется для случаев n = 2, 3. Предположим, что теорема выполняется для n - 1 (n \geq 4). Пусть G — связный граф порядка 2n и предположим, что k - это такое число, что любой индуцированный связный подграф G четного порядка 2k содержит совершенное паросочетание.
:Случай k = n очевиден, поэтому можно считать, что k \leq n - 1. 
:Если граф содержит смежные листы a и b, то рассмотрим любой связный подграф графа G четного порядка 2k, содержащий эти вершины. Тогда хотя бы одна из вершин a, b будет не покрыта паросочетанием.
:Таким образом, граф G не содержит смежных листов. Тогда из леммы следует, что существуют смежные вершины x и y, что граф G \backslash \{x, y\} связен. 
:По нашему индукционному предположению, граф G \backslash \{x, y\} содержит совершенное паросочетание, а значит, добавив ребро xy, мы получим совершенное паросочетание для G.
}}

Также можно доказать более слабое, но полезное утверждение про графы без лап (индуцированных подграфов K_{1,3}).
{{Утверждение
|about=следствие из теоремы
|statement=
Пусть G — связный граф чётного порядка 2n, не содержащий лап. Тогда G содержит совершенное паросочетание.
|proof=
:Единственный связный граф порядка 4, который не содержит совершенного паросочетания — это K_{1,3}. Таким образом, это утверждение является частным случаем теоремы Самнера — Лас Вергнаса при k = 2, за исключением тривиального случая n = 1.
[[Файл:all_connected_graphs_4_vertices.png|thumb|550px|center|Все связные неориентированные графы, состоящие из 4 вершин, с точностью до изоморфизма]]
}}

==См. также==
* [[ Паросочетания: основные определения, теорема о максимальном паросочетании и дополняющих цепях ]]
* [[ Теорема Татта о существовании полного паросочетания ]]
* [[Лапы и минимальные по включению барьеры в графе]]

== Источники информации ==
* [[wikipedia:en:Claw-free_graph#Matchings | Википедия {{---}} Claw-free graph]]
* [[wikipedia:ru:Граф_без_клешней#Паросочетания | Википедия {{---}} Граф без клешней]]
* ''David P. Sumner''. Graphs with 1-factors. — 1974. — Т. 42, вып. 1. — стр. 8—12

[[Категория: Алгоритмы и структуры данных]]
[[Категория: Задача о паросочетании]]