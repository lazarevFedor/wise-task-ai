{{Теорема
|about=
Непланарность K_5
|statement=
Граф K_5 [[Укладка графа на плоскости|непланарен]].
|proof=
Граф K_5 имеет 5 вершин и 10 ребер. Если он планарен, то по [[Формула Эйлера#EulerFormulaCons|следствию из формулы Эйлера]] получаем 10 \leqslant 3 \cdot 5 - 6 = 9. Что невозможно.
}}
{{Теорема
|about=
Непланарность K_{3,3}
|statement=
Граф K_{3,3} непланарен.
|proof=
Граф K_{3,3} содержит V = 6, E = 9 и F [[Укладка графа на плоскости|граней]]. 
Пусть граф K_{3,3} планарен. Тогда по [[Формула Эйлера|формуле Эйлера]] F = E - V + 2 = 9 - 6 + 2 = 5. Пусть, двигаясь вдоль i-й грани мы пройдем l_i ребер. Очевидно, что \sum_{i=1}^{F}l_i = 2E. Поскольку граф двудольный, все его циклы имеют четную длину. Значит l_i \geqslant 4. Получаем 4F \leqslant 2E, то есть 2F \leqslant E. То есть 2\cdot5 = 10 \leqslant 9, что невозможно.
}}

==См. также==
* [[wikipedia:ru:Планарный граф | Википедия {{---}} Планарный граф]]
* [[wikipedia:ru:Домики и колодцы | Википедия {{---}} Домики и колодцы]]

==Источники информации==
* Асанов М. О., Баранский В. А., Расин В. В. {{---}} '''Дискретная математика: Графы, матроиды, алгоритмы: Учебное пособие. 2-е изд., испр. и доп.''' стр. 134 {{---}} СПб.: Издательство "Лань", 2010. {{---}} 368 с.: ил. {{---}} (Учебники для вузов. Специальная литература). ISBN 978-5-8114-1068-2

[[Категория: Алгоритмы и структуры данных]]
[[Категория: Укладки графов ]]