==Вершинная двусвязность==
{{Определение
|definition=
Два ребра [[Основные определения: граф, ребро, вершина, степень, петля, путь, цикл|графа]] называются '''вершинно двусвязными''' (англ. ''vertex biconnected''), если существуют вершинно непересекающиеся пути, соединяющие их концы.
}}
Заметим, что если имеется два различных двусвязных ребра, то они лежат на некотором вершинно простом цикле.

{{Определение
|definition=
'''Блоками''' (англ. ''block''), или компонентами вершинной двусвязности графа, называют его подграфы, множества ребер которых — классы эквивалентности вершинной двусвязности, а множества вершин {{---}} множества всевозможных концов ребер из соответствующих классов.
}}

{{Теорема
|statement=
Отношение вершинной двусвязности является отношением эквивалентности на ребрах.
|proof=
[[Файл: Vertex_biconnected.png|370px|thumb|right|К доказательству транзитивности]]
'''Рефлексивность:'''
В данном случае имеем 2 пустых пути, которые, очевидно, не пересекаются.

'''Симметричность:'''
Следует из симметричности определения.

'''Транзитивность:'''
Пусть имеем ребра: ef вершинно двусвязно с cd, cd вершинно двусвязно с ab, при этом все они различны. Ребра ef и cd лежат на вершинно простом цикле C. Будем считать, что существуют непересекающиеся пути P : a \leadsto c, Q : b \leadsto d (ситуация, когда они идут наоборот, разбирается аналогично). Пусть x {{---}} первая вершина на P, лежащая также на C, y {{---}} первая вершина на Q, лежащая на C. Проделав пути от a до x и от b до y, далее пойдем по циклу C в нужные (различные) стороны, чтобы достичь e и f. То есть ef вершинно двусвязно с ab.
}}

''Замечание.'' Рассмотрим следующее определение: вершины u и v называются вершинно двусвязными, если между ними существуют 2 пути, не пересекающихся по вершинам, за исключением концов. Это определение не может претендовать на корректность, так как в этом случае отношение вершинной двусвязности перестанет быть транзитивным.

==Точки сочленения==
{{main|Точка сочленения, эквивалентные определения}}
{{Определение
|definition=
'''Точка сочленения''' (англ. ''articulation points'') графа G {{---}} вершина, принадлежащая как минимум двум блокам G.
}}
{{Определение
|definition=
'''Точка сочленения''' графа G {{---}} вершина, при удалении которой в G увеличивается число компонент связности.
}}

== См. также ==
* [[Отношение рёберной двусвязности]]

==Источники информации==
* Харари, Ф. Теория графов. — М.: Книжный дом «ЛИБРОКОМ», 2009
* [[wikipedia:ru:Двусвязный_граф | Википедия {{---}} Двусвязный граф]]

[[Категория:Алгоритмы и структуры данных]]
[[Категория:Связность в графах]]