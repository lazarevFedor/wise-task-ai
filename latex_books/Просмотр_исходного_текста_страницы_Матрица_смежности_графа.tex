__NOTOC__
{{Определение
|definition ='''Матрицей смежности''' ''(англ. Adjacency matrix)'' A=||\alpha_{i,j}|| невзвешенного графа G=(V,E) называется матрица A_{[V\times{}V]}, в которой \alpha_{i,j} — количество рёбер, соединяющих вершины v_i и v_j, причём при i=j каждую петлю учитываем дважды, если граф не является ориентированным, и один раз, если граф ориентирован.
}}

{{Определение
|definition ='''Матрицей смежности''' ''(англ. Adjacency matrix)'' A=||\alpha_{i,j}|| взвешенного графа G=(V,E) называется матрица A_{[V\times{}V]}, в которой \alpha_{i,j} — вес ребра, соединяющего вершины v_i и v_j.
}}

====Примеры матриц смежности:====
{| border="1" cellpadding="5" cellspacing="0" style="text-align:center"
!style="background:#f2f2f2"|Взвешенность графа
!style="background:#f2f2f2"|Вид графа
!style="background:#f2f2f2"|Матрица смежности
|-
!style="background:#f2f2f2"|Не взвешенный граф
|style="background:#f9f9f9"|[[Файл: Adjacency matrix.png|180px]]
|style="background:#f9f9f9"|\begin{pmatrix}
0 & 1 & 0 & 0 & 1\\
1 & 0 & 1 & 1 & 1\\
0 & 1 & 0 & 1 & 0\\
0 & 1 & 1 & 0 & 1\\
1 & 1 & 0 & 1 & 0\\
\end{pmatrix}
|-
!style="background:#f2f2f2"|Взвешенный граф
!style="background:#f9f9f9"|[[Файл:weighted_graph.png|180px]]
|style="background:#f9f9f9"|\begin{pmatrix}
0 & 40 & \infty & \infty & 18\\
40 & 0 & 22 & 6 & 15\\
\infty & 22 & 0 & 14 & \infty \\
\infty & 6 & 14 & 0 & 20\\
18 & 15 & \infty & 20 & 0 \\
\end{pmatrix}
|}

==Оценка памяти и времени работы==

Матрица смежности занимает O(|V|^2) памяти. За O(1) можно определить вес ребра или его наличие между любыми двумя вершинами. Такой способ хранения графа хорошо подходит для плотных графов, в которых число рёбер между различными парами вершин \Omega(|V|^2).

== Свойства ==
{{Утверждение
|statement=Для графов без петель и кратных рёбер матрица смежности бинарна (состоит из нулей и единиц).
}}
{{Утверждение
|statement=Для графов без петель и кратных рёбер главная диагональ матрицы смежности целиком состоит из нулей.
}}

{{Утверждение
|about=о сумме элементов строки матрицы смежности для ориентированного графа
|statement=Сумма элементов i-й строки равна deg^- v_i, то есть \sum\limits_{j=1}^{n}\alpha_{i,j} = deg^- v_i.
Аналогично сумма элементов j-го стоблца равна deg^+ v_j, то есть \sum\limits_{i=1}^{n}\alpha_{i,j} = deg^+ v_j.
}}

{{Утверждение
|about=о сумме элементов строки матрицы смежности для неориентированного графа
|statement=Матрица смежности является симметричной.
|proof=
Сумма элементов i-й строки равна deg \; v_i, то есть \sum\limits_{j=1}^{n}\alpha_{i,j} = deg \; v_i. Вследствие симметричности суммы элементов i-й строки и i-го столбца равны.
}}

{{Теорема
|about=о поиске количества путей заданной длины с помощью матрицы смежности ориентированного графа
|statement= Пусть A_{[V\times{}V]}=\alpha_{i,j} — [[Матрица смежности графа|матрица смежности]] [[Основные определения: граф, ребро, вершина, степень, петля, путь, цикл|ориентированного графа]] G=(V,E) без петель и A^l=\gamma_{i,j}, где l\in\mathbb{N}. Тогда \gamma_{i,j} равно количеству путей v_i\leadsto{}v_j длины l.

|proof=Утверждение очевидно при l = 1. Пусть l > 1, и утверждение верно для l - 1. Тогда A^{l-1}=\varepsilon_{i,j}, где \varepsilon_{i,j} равно количеству путей v_i\leadsto{}v_j длины l-1. Следовательно, 
: \gamma_{i,j}=\sum\limits_{s=1}^{n}{\varepsilon_{i,s}\alpha_{s,j}}
равно числу путей v_i\leadsto{}v_j длины l, так как каждый такой маршрут состоит из путей v_i\leadsto{}v_s длины l-1 и ребра, ведущего из предпоследней вершины v_s пути в его последнюю вершину v_j.
}}

== См. также ==
* [[Матрица инцидентности графа]]

== Источники информации ==
* Харари Фрэнк '''Теория графов''' Пер. с англ. и предисл. В. П. Козырева. Под ред. Г.П.Гаврилова. Изд. 2-е. — М.: Едиториал УРСС, 2003. — 296 с. — ISBN 5-354-00301-6
* Асанов М. О., Баранский В. А., Расин В. В. '''Дискретная математика: графы, матроиды, алгоритмы''' — НИЦ РХД, 2001. — 288 с. — ISBN 5-93972-076-5

[[Категория: Алгоритмы и структуры данных]]
[[Категория: Основные определения теории графов]]