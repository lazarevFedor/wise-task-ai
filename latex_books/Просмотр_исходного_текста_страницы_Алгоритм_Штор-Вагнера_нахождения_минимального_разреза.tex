== Необходимые определения ==
G - неориентированный взвешенный граф с n вершинами и m рёбрами.
{{Определение |definition=
'''Разрезом''' называется такое разбиение множества V на два подмножества A и B, что:
* A, B \subset V;
* A, B \neq \emptyset;
* A \cap B = \emptyset;
* A \cup B = V.
}}

{{Определение |definition=
'''Весом разреза''' называется сумма весов рёбер, проходящих через разрез, т.е. таких рёбер, один конец которых принадлежит A, а второй конец - B.
* w(A, B) = \sum\limits_{uv \in E, u \in A, v \in B} w(u, v)
}}

Эту задачу называют "глобальным минимальным разрезом". Глобальный минимальный разрез равен минимуму среди разрезов минимальной стоимости по всевозможным парам исток-сток. Хотя эту задачу можно решить с помощью любого алгоритма нахождения максимального потока (запуская его O(n^2) раз для всевозможных пар истока и стока), однако ниже описан гораздо более простой и быстрый алгоритм, предложенный Матильдой Штор (Mechthild Stoer) и Франком Вагнером (Frank Wagner) в 1994 г.

В общем случае допускаются петли и кратные рёбра, все кратные рёбра можно заменить одним ребром с их суммарным весом а петли не влияют на решение. Поэтому будем считать, что кратных рёбер и петель во входном графе нет.

== Алгоритм == 
Идея алгоритма довольно проста. Будем n-1 раз повторять следующий процесс: находить минимальный разрез между какой-нибудь парой вершин s и t, а затем объединять эти две вершины в одну (создавать новую вершину, список смежности которой равен объединению списков смежности s и t). В конце концов, после n-1 итерации, останется одна вершина. После этого ответом будет являться минимальный среди всех n-1 найденных разрезов. Действительно, на каждой i-ой стадии найденный минимальный разрез \langle A,B \rangle между вершинами s_i и t_i либо окажется искомым глобальным минимальным разрезом, либо же, напротив, вершины s_i и t_i невыгодно относить к разным множествам, поэтому мы ничего не ухудшаем, объединяя эти две вершины в одну.

Следовательно нам необходимо для данного графа найти минимальный разрез между какой-нибудь парой вершин s и t. Для этого вводим некоторое множество вершин A, которое изначально содержит единственную произвольную вершину s. На каждом шаге находится вершина, наиболее сильно связанная с множеством A, т.е. вершина v \not\in A, для которой следующая величина w(v,A) = \sum\limits_{(v,u) \in E, \atop u \in A} w(v,u) максимальна (максимальна сумма весов рёбер, один конец которых v, а другой принадлежит A). Этот процесс завершится, когда все вершины перейдут в множество A.

 minCut(граф G):
 v[i] - список вершин, которые были сжаты в i-тую (сначала заполняется i);
 for i = 1..n-1
 A = Ø;
 fill(w, 0);
 for j = 1..n-1
 s = {s \in V | s \notin A, w[s] - max};
 if (j != n-1)
 A += s;
 пересчитываем связность w[i] для остальных вершин; 
 prev = s;
 else
 if (w[s] \cup prev;
 return minCut - список вершин в минимальном разрезе;

== Корректность алгоритма ==
{{Теорема
|statement=
Если добавить в множество A по очереди все вершины, каждый раз добавляя вершину, наиболее сильно связанную с A, то пусть предпоследняя добавленная вершина {{---}} s, а последняя {{---}} t. Тогда минимальный s-t разрез состоит из единственной вершины {{---}} t
|proof=
Рассмотрим произвольный s-t разрез C и покажем, что его вес не может быть меньше веса разреза, состоящего из единственной вершины t:
 
: w (\{t\}) \le w (C). 

Пусть v - вершина, которую мы хотим добавить в A, тогда A_v - состояние множества A в этот момент. Пусть C_v - разрез множества A_v \cup v, индуцированный разрезом C. Вершина v - активная, если она и предыдущая добавленная вершина в A принадлежат разным частям разреза C, тогда для любой такой вершины:

: w (v, A_v) \le w (C_v). 

t - активная вершина, для неё выполняется:

: w (t,A_t) \le w (C_t) 
: w (t,A_t) = w (\{t\}), w (C_t) = w (C)

Получили утверждение теоремы.
Для доказательства воспользуемся методом математической индукции.
Для первой активной вершины v это неравенство верно, так как все вершины A_v принадлежат одной части разреза, а v - другой. Пусть неравенство выполнено для всех активных вершин до v, включая v, докажем его для следующей активной вершины u.

: w (u,A_u) \equiv w (u,A_v) + w (u,A_u \setminus A_v) (*)

Заметим, что 

: w (u,A_v) \le w (v,A_v) (**)

вершина v имела большее значение w, чем u, так как была добавлена в A раньше.
По предположению индукции:

: w (v,A_v) \le w (C_v)

Следовательно из (**):

: w(u,A_v) \le w(C_v)

А из (*) имеем:

: w (u,A_u) \le w (C_v) + w (u,A_u \setminus A_v) 

Вершина u и A_u \setminus A_v находятся в разных частях разреза C, значит w (u,A_u \setminus A_v) равна сумме весов рёбер, которые не входят в C_v, но входят в C_u.

: w (u,A_u) \le w (C_v) + w (u,A_u \setminus A_v) \le w (C_u)

Что и требовалось доказать.
}}

== Асимптотика ==
#Нахождение вершины с наибольшей w за O (n), n-1 фаза по n-1 итерации в каждой. В итоге имеем O (n^3)
#Если использовать фибоначчиевы кучи для нахождения вершины с наибольшей w, то асимптотика составит O (nm + n^2 \log n)
#Если использовать двоичные кучи, то асимптотика составит O (nm \log n + n^2)

== Применение ==
Нахождение разреза минимальной стоимости является основой в одном из методов сегментации изображений (сегментацией изображения называется разбиение его на некоторые области, непохожие по некоторому признаку). 

Изображение представляется в виде взвешенного графа, вершинами которого являются точки изображения (как правило, пиксели, но, возможно, и большие области, от этого зависит качество сегментации, а также скорость её построения). Вес ребра представляет отражает "разницу" между точками (расстояние в некоторой метрике). Разбиение изображения на однородные области сводится к задаче поиска минимального разреза в графе. Специально для такого рода задач был предложен метод нахождения разреза минимальной стоимости [https://www.google.ru/url?sa=t&rct=j&q=&esrc=s&source=web&cd=2&ved=0CDQQFjAB&url=http%3A%2F%2Fwww.cs.berkeley.edu%2F~malik%2Fpapers%2FSM-ncut.pdf&ei=cP2-UuqhAuSJ4gTnhYCwAg&usg=AFQjCNFn9GZPlFjDUgDofCScu6Wm47qMWQ&sig2=Yufd8LreEQKHe3NGnFVm7A&bvm=bv.58187178,d.bGE&cad=rjt Normalized Cut (J. Shi, J. Malik (1997))]

== Источники ==
* [http://e-maxx.ru/bookz/files/stoer_wagner_mincut.pdf Mechthild Stoer, Frank Wagner. A Simple Min-Cut Algorithm]
* [http://e-maxx.ru/algo/stoer_wagner_mincut Алгоритм Штор-Вагнера]
* [http://cgm.computergraphics.ru/content/view/147 Методы сегментации изображения]

== Ссылки ==
*[[Алгоритм Каргера для нахождения минимального разреза]]
*[https://www.google.ru/url?sa=t&rct=j&q=&esrc=s&source=web&cd=2&ved=0CDQQFjAB&url=http%3A%2F%2Fwww.cs.berkeley.edu%2F~malik%2Fpapers%2FSM-ncut.pdf&ei=cP2-UuqhAuSJ4gTnhYCwAg&usg=AFQjCNFn9GZPlFjDUgDofCScu6Wm47qMWQ&sig2=Yufd8LreEQKHe3NGnFVm7A&bvm=bv.58187178,d.bGE&cad=rjt Метод Normalized Cut]

[[Категория: Алгоритмы и структуры данных]]
[[Категория: Задача о максимальном потоке]]