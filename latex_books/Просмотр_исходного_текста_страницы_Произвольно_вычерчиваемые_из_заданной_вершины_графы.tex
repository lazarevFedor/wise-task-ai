{{Определение
|definition=
[[Основные определения теории графов|Граф]] называется '''произвольно вычерчиваемым из вершины v''' (англ. ''Arbitrarily traceable graph''), если любая цепь с началом в вершине v может быть продолжена до эйлерового цикла графа G. }}
{{Утверждение
|statement=Любой произвольно вычерчиваемый из вершины v граф является [[Эйлеров цикл, Эйлеров путь, Эйлеровы графы, Эйлеровость орграфов|эйлеровым графом]].
}}

{{Теорема
|statement=
[[Эйлеров цикл, Эйлеров путь, Эйлеровы графы, Эйлеровость орграфов|Эйлеров граф]] G, содержащий хотя бы одно ребро, является произвольно вычерчиваемым из вершины v \Longleftrightarrow вершина v принадлежит всем циклам графа G.
|proof=
[[Файл:ATG_part1.jpg|200px|right]]
\Rightarrow 
Пусть в G \exists цикл C, v \notin C.
Рассмотрим G_1 = G/C (здесь и далее это означает удаление только ребер, не трогая вершины). При удалении цикла все степени вершин остались четными, потому что каждая вершина содержит четное количество ребер цикла, и следовательно G_1 {{---}} эйлеров. Тогда в G_1 \exists эйлеров цикл. Если начать обход по эйлерову циклу из v, то и закончится он в v. Если теперь вернуть цикл C, то мы никак не сможем его обойти, так как из вершины v больше нет не посещенных ребер \Rightarrow G не свободно вычерчиваемый из v.
[[Файл:ATG_part2.jpg|200px|right]]
\Leftarrow 
Пусть дан эйлеров граф G, вершина v принадлежит всем его циклам.
Рассмотрим произвольный путь P = v \leadsto w. Пусть G_1 = G/P. Возможны 2 случая:

# Если v = w, то P {{---}} цикл, значит степени всех вершин в G_1 остались четными \Rightarrow G_1 {{---}} эйлеров.
# Если v \neq w, то так как G эйлеров граф \exists эйлеров путь w \leadsto v \in G_1.

Покажем, что в обоих случаях эйлеров обход пройдет по всем ребрам G_1.

В G \exists единственная компонента связности, содержащая ребра. При удалении P их количество не могло увеличится, иначе должен быть цикл, не содержащий v(смотри рисунок). Значит в G_1 \exists единственная компонента связности содержащая ребра, причем G_1 либо полуэйлеров, либо эйлеров \Rightarrow в G_1 \exists эйлерова цепь Q = w \leadsto v \Rightarrow P+Q эйлеров цикл в графе G.
}}

== Строение ==
[[Файл:ATGexample.jpg|right|300px]]
Опираясь на теорему опишем строение всех графов, произвольно вычерчиваемых из вершины v. 
Возьмем произвольный [[Дерево, эквивалентные определения|лес]] H, не содержащий вершину v. Каждую вершину нечетной степени соединим некоторым нечетным числом кратных ребер с v, а каждую вершину четной степени - четным числом кратных ребер с v (не исключая 0), причем каждую изолированную вершину обязательно соединим с v.
Полученный граф G:
* связен,
* имеет только вершины четной степени,
* является произвольно вычерчиваемым из v, как эйлеров граф, у которого v принадлежит всем циклам.
Теперь докажем, почему таким образом можно получить все графы, произвольно вычерчиваемые из вершины v. Пусть какой-то такой граф нельзя получить методом описанным выше. Тогда уберем все ребра из вершины v и посмотрим на граф, который остался. Он не является лесом, иначе мы могли бы получить этот граф нашим методом. Но если он не является лесом, то в нем есть хотя бы один цикл, который не содержит v. А по теореме о произвольно вычерчиваемымых из вершины графах такого быть не может. Следовательно наше предположение ошибочно.

==См. также==
* [[Покрытие рёбер графа путями]]
* [[Алгоритм построения Эйлерова цикла]]

==Источники информации==
* Асанов М., Баранский В., Расин В. ''Дискретная математика: Графы, матроиды, алгоритмы.'', Ижевск: ННЦ "Регулярная и хаотическая динамика", 2001. ISBN 5-93972-076-5

[[Категория: Алгоритмы и структуры данных]]
[[Категория: Обходы графов]]
[[Категория: Эйлеровы графы]]