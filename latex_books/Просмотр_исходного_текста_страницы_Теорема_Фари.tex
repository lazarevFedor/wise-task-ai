Теорема была независимо доказана Клаусом Вагнером (Klaus Wagner) в 1936ом году и Иштваном Фари (István Fáry) в 1948ом году. Иногда ее называют теоремой Фари-Вагнера.

{{Определение
|id=def1
|definition='''Триангуляция графа''' (англ. ''triangulation'') {{---}} представление [[Укладка графа на плоскости#defplanar | планарного графа]] на плоскости в таком виде, что каждая его грань ограничена тремя ребрами (является треугольником).
}}

{{Определение
|id=def2
|definition='''Разделяющий треугольник''' (англ. ''separating triangle'') {{---}} цикл длины 3 в графе G, внутри и снаружи которого находятся вершины графа.
}}

Разделяющий треугольник изображён ниже. Относительно него существует три вида вершин: внешние, внутренние и лежащие на самом треугольнике.

[[File:Fary1.png|250px|Рисунок 1]]

{{Теорема
|about=Фари
|statement=Любой планарный граф имеет плоское представление, в котором все ребра представлены в виде отрезков прямых.
|proof=

Докажем теорему для плоской триангуляции графа G. Ее можно достичь, добавив в G необходимое количество ребер. Применим индукцию по числу вершин |V|.

База индукции, когда |V|=3, выполняется тривиальным образом.
Предположим, что графы с любым числом вершин меньше |V| \geqslant 4, мы можем нарисовать требуемым образом. 
Рассмотрим ребро vw, [[Матрица инцидентности графа#definc | инцидентное]] внутренней вершине глубочайшего разделяющего треугольника, то есть такого, который не содержит внутри себя других разделяющих треугольников. Если в графе нет разделяющих треугольников, то возьмём любое ребро. Тогда vw {{---}} граница двух граней vwp и vwq. 

[[File:Fary2.png|250px|Рисунок 2]]

Так как мы взяли вершины внутри самого глубого разделяющего треугольника, то у вершин v и w может быть только два общих соседа p и q.
Пусть (vp, vw, vq, vx_1, vx_2 ... vx_k) и (wq, wv, wp, wy_1, wy_2 ... wy_m) {{---}} обход по часовой стрелке ребер, исходящих соостветсвенно из v и w.
Пусть G' {{---}} граф, полученный из G стягиванием ребра vw в вершину s. Заменим пары параллельных ребер vq и wq на sq и vp, wp на sp. Получим вершину s, из которой исходят ребра (sp, sy_1, sy_2 ... sy_m, sq, sx_1, sx_2 ... sx_k) {{---}} по часовой стрелке.
 
[[File:Fary3.png|250px|Рисунок 3]]

Мы получили граф G', с меньшим числом вершин равным V - 1, то есть его можно уложить на плоскости требуемым образом: все ребра прямые (и сохранен обход по часовой стрелке ребер, инцидентных s).
Для любого \varepsilon > 0 обозначим C_{\varepsilon}(s) {{---}} круг радиуса \varepsilon, с вершиной s в центре. 
Для каждого соседа t вершины s в графе G' обозначим R_{\varepsilon}(t) объединение всех отрезков, проведённых из t в C_{\varepsilon}(s). 

Возьмем \varepsilon равным минимуму из всех расстояний от вершины s до инцидентных ей вершин и до отрезков, проходящих мимо нее.
 
[[File:Fary5.png|250px|Рисунок 4]]

Тогда получим, что все соседи t вершины s находятся снаружи C_{\varepsilon}(s) и только ребра G', инцидентные s, могут пересекать R_{\varepsilon}(t). 

[[File:Fary4.png|250px|Рисунок 5]]

Проведем линию L через вершину s так, чтобы вершина p лежала с одной ее стороны, а q {{---}} с другой (такая линия существует, иначе рёбра sp и sq накладывались бы друг на друга) и никакое из ребер \{sx_i : 1 и \{sy_i : 1 не лежало на L. 
Ребра sq и sq разбивают C_{\varepsilon}(s) на две дуги: первая пересекает ребра \{sx_i : 1 , а вторая {{---}} ребра \{sy_i : 1 . 
L пересекает C_{\varepsilon}(s) в двух точках. Расположим v и w в этих точках: v на дуге, пересекающей \{sx_i : 1 , а w с другой стороны.

[[File:Fary6.png|250px|Рисунок 6]]

Удалим s и инцидентные ей ребра, нарисуем прямые ребра G, инцидентные v и w. 

[[File:Fary7.png|250px|Рисунок 7]]

Получим, что vw лежит на L. Так как p и q лежат с разных сторон L, ребра, инцидентные v и w, не пересекаются. 
По выбору \varepsilon, ребра, инцидентные v и w, не пересекают и другие ребра G. Таким образом желаемая укладка графа G достигнута. 
Теперь мы можем удалить добавленные нами ребра, оставив в графе лишь исходные (уже прямые) ребра. 
}}

==См. также==
* [[Теорема Понтрягина-Куратовского]] 
* [[Укладка графа на плоскости]]

==Источники информации==
* [[wikipedia:Fáry's_theorem | Wikipedia {{---}} Fáry's theorem ]]
* [http://arxiv.org/abs/cs/0505047 Доказательство теоремы Фари]

[[Категория: Алгоритмы и структуры данных]]
[[Категория: Укладки графов ]]